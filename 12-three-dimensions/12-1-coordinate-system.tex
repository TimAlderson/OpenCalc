\section{The Coordinate System}\label{sec:3Dcoordinatesystem}

Throughout the text thus far we have focused investigating functions of the form $y=f(x)$, with
one independent and one dependent variable. Such functions can be
represented in two dimensions, using two numerical axes that allow us
to identify every point in the plane with two numbers. We now shift our focus to three-dimensional space; to identify every point in three
dimensions we require three numerical values. The obvious way to make
this association is to add one new axis, perpendicular to the $x$ and
$y$ axes we already understand. We could, for example, add a third
axis, the $z$ axis, with the positive $z$ axis coming straight out of
the page, and the negative $z$ axis going out the back of the
page. This is difficult to work with on a printed page, so more often
we draw a view of the three axes from an angle:

\begin{figure}[H]
$$\vbox{\beginpicture
\normalgraphs
\setcoordinatesystem units <7truemm,7truemm>
\setplotarea x from 0 to 5, y from 0 to 5
\axis left /
\axis bottom /
\plot 0 0 -3 -3 /
\put {$x$} [l] <4pt,0pt> at 5 0
\put {$y$} [b] <0pt,4pt> at 0 5
\put {$z$} [tr] <-4pt,-4pt> at -3 -3
\endpicture
}$$
\end{figure}

You must then imagine that the $z$ axis is perpendicular to the other
two. Just as we have investigated functions of the form $y=f(x)$ in
two dimensions, we will investigate three dimensions largely by
considering functions; now the functions will (typically) have the
form $z=f(x,y)$. Due to the fact that we are used to having the result of a
function graphed in the vertical direction, it is somewhat easier to
maintain that convention in three dimensions. To accomplish this, we
normally rotate the axes so that $z$ points up; the result is then:

\begin{figure}[H]
$$\vbox{\beginpicture
\normalgraphs
\setcoordinatesystem units <6truemm,6truemm>
\setplotarea x from 0 to 5, y from 0 to 6
\axis left /
\axis bottom /
\plot 0 0 -3 -3 /
\setdashes
\plot 4 0 3 -1 /
\plot 3 -1 -1 -1 /
\plot 3 -1 3 4 /
\setdots
\plot 0 5 4 5 3 4 -1 4 0 5 /
\plot -1 4 -1 -1 /
\plot 4 5 4 0 /
\put {$\bullet$} at 3 4
\put {$(2,4,5)$} [tr] <-4pt,-4pt> at 3 4
\put {$y$} [l] <4pt,0pt> at 5 0
\put {$z$} [b] <0pt,4pt> at 0 6
\put {$x$} [tr] <-4pt,-4pt> at -3 -3
\put {$4$} [b] <0pt,3pt> at 4 0
\put {$2$} [br] <-2pt,2pt> at -1 -1
\put {$5$} [r] <-3pt,0pt> at 0 5
\endpicture
}$$
\end{figure}

Note that if you imagine looking down from above, along the $z$ axis,
the positive $z$ axis will come straight toward you, the positive $y$
axis will point up, and the positive $x$ axis will point to your
right, as usual. Any point in space is identified by providing the
three coordinates of the point, as shown; naturally, we list the
coordinates in the order $(x,y,z)$. One useful way to think of
this is to use the $x$ and $y$ coordinates to identify a point in the
$x$-$y$ plane, then move straight up (or down) a distance given by
the $z$ coordinate.

It is now fairly simple to understand some ``shapes'' in three
dimensions that correspond to simple conditions on the coordinates. In
two dimensions the equation $x=1$ describes the vertical line through
$(1,0)$. In three dimensions, it still describes all points with
$x$-coordinate 1, but this is now a plane, as in
Figure~\ref{fig:plane}.

\begin{figure}[H]
\centerline{
\vbox{\beginpicture
\normalgraphs
\setcoordinatesystem units <3truecm,3truecm>
\setplotarea x from 0 to 1.1, y from 0 to 1.1
\put {\hbox{\epsfxsize7cm\epsfbox{images/plane.eps}}} at 0 0
\endpicture}}
\caption{The plane $x=1$. \label{fig:plane}}
\end{figure}

Recall the very useful distance formula in two dimensions which comes directly from the Pythagorean Theorem: the
distance between points $(x_1,y_1)$ and $(x_2,y_2)$ is
$\sqrt{(x_1-x_2)^2+(y_1-y_2)^2}$\index{distance formula}. What is the distance between two points
$(x_1,y_1,z_1)$ and $(x_2,y_2,z_2)$ in three dimensions?
Geometrically, we want the length of the long diagonal labelled $c$ in
the ``box'' in Figure~\ref{fig:3d distance formula}. Since $a$,
$b$, $c$ form a right triangle, $a^2+b^2=c^2$. $b$ is the vertical
distance between $(x_1,y_1,z_1)$ and $(x_2,y_2,z_2)$, so
$b=|z_1-z_2|$.  The length $a$ runs parallel to the $x$-$y$ plane, so
it is simply the distance between $(x_1,y_1)$ and $(x_2,y_2)$, that
is, $a^2=(x_1-x_2)^2+(y_1-y_2)^2$. Now we see that
$c^2=(x_1-x_2)^2+(y_1-y_2)^2+(z_1-z_2)^2$ and
$c=\sqrt{(x_1-x_2)^2+(y_1-y_2)^2+(z_1-z_2)^2}$. 

It is sometimes useful to give names to points, for example we might
let $P_1=(x_1,y_1,z_1)$, or more concisely we might refer to the point
$P_1(x_1,y_1,z_1)$, and subsequently use just $P_1$. Distance between
two points in either two or three dimensions is sometimes denoted by
$d$, so for example the formula for the distance between $P_1(x_1,y_1,z_1)$
and $P_2(x_2,y_2,z_2)$ might be expressed as
$$d(P_1,P_2)=\sqrt{(x_1-x_2)^2+(y_1-y_2)^2+(z_1-z_2)^2}.$$

\begin{formulabox}[Distance]
The distance between points  $P_1(x_1,y_1)$ and $P_2(x_2,y_2)$ in two dimensions is
\[
d(P_1,P_2) = \sqrt{(x_1-x_2)^2+(y_1-y_2)^2}
\]

\medskip

The distance between points $P_1(x_1,y_1,z_1)$ and $P_2(x_2,y_2,z_2)$ in three dimensions is \[
d(P_1,P_2)=\sqrt{(x_1-x_2)^2+(y_1-y_2)^2+(z_1-z_2)^2}
\]
\end{formulabox}

\begin{figure}[H]
\centerline{
\vbox{\beginpicture
\normalgraphs
\setcoordinatesystem units <7truemm,7truemm>
\setplotarea x from 0 to 7, y from 0 to 6
\axis left /
\axis bottom /
\plot 0 0 -3 -3 /
\setdashes
\plot 5 2 4 1 /
\plot 4 1 1 1 1 5 4 5 5 6 5 2 /
\plot 4 1 4 5 /
\plot 5 6 2 6 1 5 /
\setdots
\plot 2 6 2 2 1 1 /
\plot 2 2 5 2 /
\setsolid
\plot 4 1 2 2 4 5 /
\put {$a$} [tr] <-2pt,-2pt> at 3 1.5
\put {$b$} [l] <2pt,0pt> at 4 3
\put {$c$} [br] <-2pt,2pt> at 3 3.5
%\put {$(2,4,5)$} [l] <4pt,0pt> at 3 4
\put {$y$} [l] <4pt,0pt> at 7 0
\put {$z$} [b] <0pt,4pt> at 0 6
\put {$x$} [tr] <-4pt,-4pt> at -3 -3
\endpicture}}
\caption{Distance in three dimensions. \label{fig:3d distance formula}}
\end{figure}

In two dimensions, the distance formula immediately gives us the
equation of a circle: the circle of radius $r$ and center at $(h,k)$
consists of all points $(x,y)$ at distance $r$ from $(h,k)$, so the
equation is $r=\sqrt{(x-h)^2+(y-k)^2}$ or $r^2=(x-h)^2+(y-k)^2$. Now
we can get the similar equation $r^2=(x-h)^2+(y-k)^2+(z-l)^2$, which
describes all points $(x,y,z)$ at distance $r$ from $(h,k,l)$, namely,
the sphere with radius $r$ and center $(h,k,l)$.

%%%%%%%%%%%%%%%%%%%%%%%%%%%%%%%%%%%%%%%%%%%%
\Opensolutionfile{solutions}[ex]
\section*{Exercises for \ref{sec:3Dcoordinatesystem}}

\begin{enumialphparenastyle}

\begin{ex}
Sketch the location of the points $(1,1,0)$, $(2,3,-1)$,
and $(-1,2,3)$ on a single set of axes.
\end{ex}

\begin{ex}
Describe geometrically the set of points $(x,y,z)$ that
satisfy $z=4$.
\end{ex}

\begin{ex}
Describe geometrically the set of points $(x,y,z)$ that
satisfy $y=-3$.
\end{ex}

\begin{ex}
Describe geometrically the set of points $(x,y,z)$ that
satisfy $x+y=2$.
\end{ex}

\begin{ex}
The equation $x+y+z=1$ describes some collection of points
in $\ds \R^3$. Describe and sketch the points that satisfy $x+y+z=1$ and
are in the $x$-$y$ plane, in the $x$-$z$ plane, and in the 
$y$-$z$ plane.
\end{ex}

\begin{ex}
Find the lengths of the sides of the triangle with 
vertices $(1,0,1)$, $(2,2,-1)$, and $(-3,2,-2)$.
\begin{sol}
$3$, $\sqrt{26}$, $\sqrt{29}$
\end{sol}
\end{ex}

\begin{ex}
Find the lengths of the sides of the triangle with 
vertices $(2,2,3)$, $(8,6,5)$, and $(-1,0,2)$. Why do the results tell
you that this isn't really a triangle?
\begin{sol}
$\sqrt{14}$, $2\sqrt{14}$, $3\sqrt{14}$.
\end{sol}
\end{ex}

\begin{ex}
Find an equation of the sphere with center at $(1,1,1)$ and
radius 2.
\begin{sol}
$(x-1)^2+(y-1)^2+(z-1)^2=4$.
\end{sol}
\end{ex}

\begin{ex}
Find an equation of the sphere with center at $(2,-1,3)$ and
radius 5.
\begin{sol}
$(x-2)^2+(y+1)^2+(z-3)^2=25$.
\end{sol}
\end{ex}

\begin{ex}
Find an equation of the sphere with center $(3,-2,1)$ and
that goes through the point $(4,2,5)$.
\end{ex}

\begin{ex}
Find an equation of the sphere with center at $(2,1,-1)$ and
radius 4. Find an equation for the intersection of this sphere with
the $y$-$z$ plane; describe this intersection geometrically.
\begin{sol}
$(x-2)^2+(y-1)^2+(z+1)^2=16$,
$(y-1)^2+(z+1)^2=12$
\end{sol}
\end{ex}

\begin{ex}
Consider the sphere of radius 5 centered at $(2,3,4)$.  What is
the intersection of this sphere with each of the coordinate planes?
\end{ex}

\begin{ex}
Show that for all values of $\theta$ and $\phi$, the point
$(a\sin\phi\cos\theta,a\sin\phi\sin\theta,a\cos\phi)$ lies on the
sphere given by $x^2+y^2+z^2=a^2$.
\end{ex}

\begin{ex}
Prove that the midpoint of the line segment connecting
$(x_1,y_1,z_1)$ to $(x_2,y_2,z_2)$ is at 
$\ds\left({x_1+x_2\over 2},{y_1+y_2\over 2},{z_1+z_2\over 2}\right)$.
\end{ex}

\begin{ex}
Any three points $P_1(x_1,y_1,z_1)$, $P_2(x_2,y_2,z_2)$,
$P_3(x_3,y_3,z_3)$, lie in a plane and form a triangle.  The \dfont{triangle inequality}
says that $ d(P_1,P_3)\le d(P_1,P_2)+d(P_2,P_3)$.  Prove the triangle inequality
using either algebra (messy) or the law of cosines (less messy).
\end{ex}

\begin{ex}
Is it possible for a plane to intersect a sphere in exactly two
points?  Exactly one point? Explain.  
\end{ex}

\end{enumialphparenastyle}
