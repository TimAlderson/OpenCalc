\section{Other Coordinate Systems}\label{sec:3Dothercoordsystem}

Coordinate systems are tools that let us use algebraic methods to
understand geometry. While the \dfont{rectangular} (also called
\dfont{Cartesian}) coordinates that we
have been discussing are the most common, some problems are easier to
analyze in alternate coordinate systems.\index{Cartesian coordinates}

A coordinate system is a scheme that allows us to identify any point
in the plane or in three-dimensional space by a set of numbers. In
rectangular coordinates these numbers are interpreted, roughly
speaking, as the lengths of the sides of a rectangular ``box.''

In two dimensions you may already be familiar with an alternative,
called \dfont{polar coordinates}\index{polar coordinates}. In this system, each
point in the plane is identified by a pair of numbers $(r,\theta)$.
The number $\theta$ measures the counter-clockwise angle between the positive
$x$-axis and a vector with tail at the origin and head at the
point, as shown in Figure~\ref{fig:polar coordinates}; the number
$r$ measures the distance from the origin to the
point. Either of these may be negative; a negative $\theta$ indicates
the angle is measured clockwise from the positive
$x$-axis instead of counter-clockwise, and a negative $r$ indicates
the point at distance $|r|$ in the opposite of the direction given by
$\theta$. 

The relationship between polar and rectangular coordinates is given by\index{polar coordinates!convert to rectangular}
\begin{align*}
x = r \cos \theta \\
y = r \sin \theta 
\end{align*}
and
\begin{align*}
r = \sqrt{x^2 + y^2} \\
\tan \theta = \frac{y}{x} 
\end{align*}

\begin{example}{Rectangular to Polar Coordinates}{}
Convert the point $(x,y) = (1, \sqrt3)$ into polar coordinates.
\end{example}

\begin{solution}
First calculate $r$:
\[
r = \sqrt{x^2 + y^2} = \sqrt{ 1 + 3} = 2
\]
Now find $\theta$ such that $\tan \theta = \frac{\sqrt{3}}{1}$. The required $\theta$ is $\frac{\pi}{3}$. 

The polar coordinates are $(2, \frac{\pi}{3})$. 
\end{solution}

Figure~\ref{fig:polar coordinates} also shows the point with
rectangular coordinates $(1,\sqrt3)$ and polar coordinates 
$(2,\pi/3)$, 2 units from the origin and $\pi/3$ radians from the
positive $x$-axis.

\begin{figure}[H]
\centerline{
\vbox{\beginpicture
\normalgraphs
\setcoordinatesystem units <15truemm,15truemm>
\setplotarea x from 0 to 1.3, y from 0 to 2
\axis left /
\axis bottom /
\arrow <4pt> [0.35, 1] from 0 0 to 1 1.73
\put {$x$} [l] <3pt,0pt> at 1.3 0
\put {$y$} [b] <0pt,3pt> at 0 2
\put {$(r,\theta)$} [l] <3pt,0pt> at 1 1.73
\put {$r$} [br] <-2pt,2pt> at 0.5 0.86
\put {$\theta$} [bl] <2pt,0pt> at 0.173 0.1
\arrow <4pt> [0.35,1] from 0.5 -0.2 to 0 -0.2
\arrow <4pt> [0.35,1] from 0.5 -0.2 to 1 -0.2
\put {$r\cos\theta$} at 0.5 -0.4
\put {$r\sin\theta$} [l] <3pt,0pt> at 1 0.86
%\put {$\ds{\pi\over 3}$} [bl] <2pt,2pt> at 0.2 0
\circulararc 60 degrees from 0.2 0 center at 0 0
\setdashes
\plot 1 0 1 1.73 /
\setsolid
\setcoordinatesystem units <15truemm,15truemm> point at -3 0
\setplotarea x from 0 to 1.3, y from 0 to 2
\axis left /
\axis bottom /
\arrow <4pt> [0.35, 1] from 0 0 to 1 1.73
\put {$x$} [l] <3pt,0pt> at 1.3 0
\put {$y$} [b] <0pt,3pt> at 0 2
\put {$(2,\pi/3)$} [l] <3pt,0pt> at 1 1.73
\put {$\pi/3$} [bl] <2pt,0pt> at 0.173 0.1
\circulararc 60 degrees from 0.2 0 center at 0 0
\setdashes
\endpicture}}
\caption{Polar coordinates: the general case and
the point with rectangular coordinates $(1,\sqrt3)$. \label{fig:polar coordinates}}
\end{figure}

We can extend polar coordinates to three dimensions simply by adding a
$z$ coordinate; this is called \dfont{cylindrical coordinates}\index{cylindrical coordinates}.
Each point in three-dimensional space is represented by three
coordinates $(r,\theta,z)$ in the obvious way: this point is $z$ units
above or below the point $(r,\theta)$ in the $x$-$y$ plane,
as shown in Figure~\ref{fig:cylindrical coordinates}. The point
with rectangular coordinates $(1,\sqrt3, 3)$ and cylindrical
coordinates $(2,\pi/3,3)$ is also indicated
in Figure~\ref{fig:cylindrical coordinates}.

\begin{figure}[H]
\centerline{
\vbox{\beginpicture
\normalgraphs
\setcoordinatesystem units <10truemm,10truemm>
\setplotarea x from 0 to 2, y from -1 to 3.3
\axis bottom shiftedto y=0 /
\putrule from 0 0 to 0 3.3
\plot 0 0 -1 -1 /
\arrow <4pt> [0.35,1] from 0 0 to 1.5  -0.7
\put {$(r,\theta,z)$} [l] <3pt,0pt> at 1.5 2.3
\put {$\bullet$} at 1.5 2.3
\put {$z$} [b] <0pt,3pt> at 0 3.3
\put {$x$} [tr] <-3pt,-3pt> at -1 -1
\put {$y$} [l] <3pt,0pt> at 2 0
\put {$\theta$} [t] <0pt,-3pt> at 0 -0.3
\put {$z$} [l] <3pt,0pt> at 1.5 1
\put {$r$} [tr] <-2pt,-2pt> at 1 -0.467
\circulararc 110 degrees from -0.2 -0.2 center at 0 0
\setdashes
\plot 1.5 -0.7 1.5 2.3 /
\setcoordinatesystem units <10truemm,10truemm> point at -5 0
\setplotarea x from 0 to 2, y from -1 to 3.3
\axis bottom shiftedto y=0 /
\putrule from 0 0 to 0 3.3
\plot 0 0 -1 -1 /
\arrow <4pt> [0.35, 1] from 0 0 to 1.5 -0.7
\put {$(2,\pi/3,3)$} [l] <3pt,0pt> at 1.5 2.3
\put {$\bullet$} at 1.5 2.3
\put {$z$} [b] <0pt,3pt> at 0 3.3
\put {$x$} [tr] <-3pt,-3pt> at -1 -1
\put {$y$} [l] <3pt,0pt> at 2 0
\put {$\pi/3$} [t] <0pt,-3pt> at 0 -0.3
\circulararc 110 degrees from -0.2 -0.2 center at 0 0
%\put {$\ds{\pi\over 3}$} [bl] <2pt,2pt> at 0.2 0
\setdashes
\plot 1.5 -0.7 1.5 2.3 /
\endpicture}}
\caption{Cylindrical coordinates: the general case
and the point with rectangular coordinates $(1,\sqrt3, 3)$. \label{fig:cylindrical coordinates}}
\end{figure}

Some figures with relatively complicated equations in rectangular
coordinates will be represented by simpler equations in cylindrical
coordinates. For example, the cylinder in Figure~\ref{fig:cylinder}
has equation $x^2+y^2=4$ in rectangular coordinates, but equation
$r=2$ in cylindrical coordinates.

\begin{figure}[H]
\centerline{
\vbox{\beginpicture
\normalgraphs
\setcoordinatesystem units <3truecm,3truecm>
\setplotarea x from 0 to 1.1, y from 0 to 1.1
\put {\hbox{\epsfxsize7cm\epsfbox{images/cylinder.eps}}} at 0 0
\endpicture}}
\caption{The cylinder $r=2$. \label{fig:cylinder}}
\end{figure}

Given a point $(r,\theta)$ in polar coordinates, it is easy 
to see  (as in Figure~\ref{fig:polar coordinates}) that
the rectangular coordinates of the same point are
$(r\cos\theta,r\sin\theta)$, and so the point $(r,\theta,z)$ in
cylindrical coordinates is $(r\cos\theta,r\sin\theta,z)$ in
rectangular coordinates. This means it is usually easy to convert any
equation from rectangular to cylindrical coordinates: simply substitute\index{cylindrical coordinates!convert to rectangular}
\begin{align*}
  x&=r\cos\theta	\\
  y&=r\sin\theta
\end{align*}
and leave $z$ alone.
For example,
starting with $x^2+y^2=4$ and substituting $x=r\cos\theta$,
$y=r\sin\theta$ gives 
\begin{align*}
  r^2\cos^2\theta+r^2\sin^2\theta&=4	\\
  r^2(\cos^2\theta+\sin^2\theta)&=4	\\
  r^2&=4	\\
  r&=2.
\end{align*}
Of course, it's easy to see directly that this defines a cylinder as
mentioned above.

Cylindrical coordinates are an obvious extension of polar coordinates
to three dimensions, but the use of the $z$ coordinate means they are
not as closely analogous to polar coordinates as another standard
coordinate system. In polar coordinates, we identify a point by a
direction and distance from the origin; in three dimensions we can do
the same thing, in a variety of ways. The question is: how do we
represent a direction? One way is to give the angle of rotation,
$\theta$, from the positive $x$ axis, just as in cylindrical
coordinates, and also an angle of rotation, $\phi$, from the positive
$z$ axis. Roughly speaking, $\theta$ is like longitude and $\phi$ is
like latitude. (Earth longitude is measured as a positive or negative
angle from the prime meridian, and is always between 0 and 180
degrees, east or west; $\theta$ can be any positive or negative angle,
and we use radians except in informal circumstances. 
Earth latitude is measured north or south
from the equator; $\phi$ is measured from the north pole down.) This
system is called \dfont{spherical coordinates}\index{spherical coordinates}; 
the coordinates are listed in the order
$(\rho,\theta,\phi)$, where $\rho$ is the distance from the
origin, and like $r$ in polar and cylindrical coordinates it may be negative. 
The general case and an
example are pictured in Figure~\ref{fig:spherical coordinates}; the
length marked $r$ is the $r$ of cylindrical coordinates.

\begin{figure}[H]
\centerline{
\vbox{\beginpicture
\normalgraphs
\setcoordinatesystem units <10truemm,10truemm>
\setplotarea x from 0 to 2, y from -1 to 3.3
\axis bottom shiftedto y=0 /
\putrule from 0 0 to 0 3.3
\plot 0 0 -1 -1 /
\arrow <4pt> [0.35, 1] from 0 0 to 1.5 -0.7
\arrow <4pt> [0.35, 1] from 0 0 to 1.5 2.3
\put {$(\rho,\theta,\phi)$} [l] <3pt,0pt> at 1.5 2.3
\put {$z$} [b] <0pt,3pt> at 0 3.3
\put {$x$} [tr] <-3pt,-3pt> at -1 -1
\put {$y$} [l] <3pt,0pt> at 2 0
\put {$\theta$} [t] <0pt,-3pt> at 0 -0.3
\put {$\phi$} [bl] <2pt,2pt> at 0 0.4
\put {$\rho$} [tl] <2pt,2pt> at 0.75 1.15
\put {$r$} [tr] <-2pt,-2pt> at 1 -0.467
\circulararc 110 degrees from -0.2 -0.2 center at 0 0
\circulararc -32 degrees from 0 0.4  center at 0 0
\setdashes
\plot 1.5 -0.7 1.5 2.3 /
\setsolid
\setcoordinatesystem units <10truemm,10truemm> point at -5 0
\setplotarea x from 0 to 2, y from -1 to 3.3
\axis bottom shiftedto y=0 /
\putrule from 0 0 to 0 3.3
\plot 0 0 -1 -1 /
\arrow <4pt> [0.35, 1] from 0 0 to 1.5 -0.7
\arrow <4pt> [0.35, 1] from 0 0 to 1.5 2.3
\put {$(\sqrt{13},\pi/3,\arctan(2/3))$} [l] <3pt,0pt> at 1.5 2.3
\put {$z$} [b] <0pt,3pt> at 0 3.3
\put {$x$} [tr] <-3pt,-3pt> at -1 -1
\put {$y$} [l] <3pt,0pt> at 2 0
\put {$\pi/3$} [t] <0pt,-3pt> at 0 -0.3
\circulararc 110 degrees from -0.2 -0.2 center at 0 0
\circulararc -32 degrees from 0 0.4  center at 0 0
%\put {$\ds{\pi\over 3}$} [bl] <2pt,2pt> at 0.2 0
\setdashes
\plot 1.5 -0.7 1.5 2.3 /
\endpicture}}
\caption{Spherical coordinates: the general case and the point 
with rectangular coordinates $(1,\sqrt3 , 3)$. \label{fig:spherical coordinates}}
\end{figure}

As with cylindrical coordinates, we can easily convert equations
in rectangular coordinates to the equivalent in spherical coordinates,
though it is a bit more difficult to discover the proper substitutions.
Figure~\ref{fig:rectangular to spherical coordinates} shows
the typical point in spherical coordinates from 
Figure~\ref{fig:spherical coordinates},
viewed now so that the arrow marked $r$ in the original graph appears as
the horizontal ``axis'' in the left hand graph. From this diagram it
is easy to see that the $z$ coordinate is $\rho\cos\phi$, and that 
$r=\rho\sin\phi$, as shown. Thus, in converting from rectangular to
spherical coordinates we will replace $z$ by $\rho\cos\phi$. To see
the substitutions for $x$ and $y$ we now view the same point from
above, as shown in the right hand graph. The hypotenuse of the
triangle in the right hand graph
is $r=\rho\sin\phi$, so the sides of the triangle, as shown,
are $x=r\cos\theta=\rho\sin\phi\cos\theta$ and 
$y=r\sin\theta=\rho\sin\phi\sin\theta$. Therefore to convert from rectangular to spherical coordinates,
we make these substitutions:\index{spherical coordinates!convert to rectangular}
\begin{align*}
  x&=\rho\sin\phi\cos\theta	\\
  y&=\rho\sin\phi\sin\theta	\\
  z&=\rho\cos\phi.
\end{align*}

\begin{figure}[H]
\centerline{
\vbox{\beginpicture
\normalgraphs
\setcoordinatesystem units <10truemm,10truemm>
\setplotarea x from 0 to 2.3, y from -1 to 3.3
\axis bottom shiftedto y=0 /
\putrule from 0 0 to 0 3.3
\plot 0 0 2 3 /
\put {$(\rho,\theta,\phi)$} [l] <3pt,0pt> at 2 3
\put {$z$} [b] <0pt,3pt> at 0 3.3
\put {$\phi$} [bl] <2pt,2pt> at 0 0.4
\put {$\rho$} [tl] <2pt,2pt> at 1 1.5
\circulararc -32 degrees from 0 0.4  center at 0 0
\put {$\rho\sin\phi$} [t] <0pt,-3pt> at 1 0
\arrow <4pt> [0.35,1] from 0.3 -0.3 to 0 -0.3
\arrow <4pt> [0.35,1] from 1.7 -0.3 to 2 -0.3
\put {$\rho\sin\phi$} at 1 3
\arrow <4pt> [0.35,1] from 0.3 3 to 0 3
\arrow <4pt> [0.35,1] from 1.7 3 to 2 3
\put {$\rho\cos\phi$} [r] <-3pt,0pt> at 0 1.5
\arrow <4pt> [0.35,1] from -0.3 1.8 to -0.3 3
\arrow <4pt> [0.35,1] from -0.3 1.2 to -0.3 0
\setdashes
\plot 2 0 2 3 /
\setsolid
\setcoordinatesystem units <17truemm,17truemm> point at -3 0
\setplotarea x from 0 to 1.3, y from 0 to 2
\axis bottom shiftedto y=0 /
\axis left /
\plot 0 0 1 1.728 /
\arrow <4pt> [0.35, 1] from 0 0 to 1 1.728
%\put {\rotstart{-60 rotate}$\rho\sin\phi$\rotfinish} <3pt,-3pt> at 0.5 0.86
\put {$x$} [l] <3pt,0pt> at 1.3 0
\put {$y$} [b] <0pt,3pt> at 0 2
\put {$\theta$} [bl] <2pt,2pt> at 0.346 0.2 
\put {$\rho\sin\phi\cos\theta$} at 0.5 -0.4
\arrow <4pt> [0.35,1] from 0.5 -0.1 to 1 -0.1
\arrow <4pt> [0.35,1] from 0.5 -0.1 to 0 -0.1
\put {$\rho\sin\phi\sin\theta$} [l] <3pt,0pt> at 1 0.85
\arrow <4pt> [0.35,1] from 1.3 1.1 to 1.3 1.728
\arrow <4pt> [0.35,1] from 1.3 0.5 to 1.3 0
\circulararc 60 degrees from 0.4 0 center at 0 0
\setdashes
\plot 1 0 1 1.728 /
\endpicture}}
\caption{Converting from rectangular to spherical coordinates. \label{fig:rectangular to spherical coordinates}}
\end{figure}

As the cylinder had a simple equation in cylindrical coordinates, so
does the sphere in spherical coordinates. 

\begin{example}{}{}
Find an equation for the sphere of radius $2$ in spherical coordinates. 
\end{example}

\begin{solution}
If we start with the
Cartesian equation of the sphere and substitute, we get the spherical
equation: 
\begin{align*}
  x^2+y^2+z^2&=2^2	\\
  \rho^2\sin^2\phi\cos^2\theta+
     \rho^2\sin^2\phi\sin^2\theta+\rho^2\cos^2\phi&=2^2	\\
  \rho^2\sin^2\phi(\cos^2\theta+\sin^2\theta)+\rho^2\cos^2\phi&=2^2	\\
  \rho^2\sin^2\phi+\rho^2\cos^2\phi&=2^2	\\
  \rho^2(\sin^2\phi+\cos^2\phi)&=2^2	\\
  \rho^2&=2^2	\\
  \rho&=2
\end{align*}
Therefore, in spherical coordinates, a sphere of radius $2$ is expressed $\rho=2$.
\end{solution}

Although not as simple as with cylindrical coordinates, we can use spherical coordinates to describe the equation of a cylinder.

\begin{example}{Cylinder Equation in Spherical Coordinates}{}
Find an equation for the cylinder $x^2+y^2=4$ in spherical
coordinates.
\end{example}
\begin{solution}
Proceeding as in the previous example:
\begin{align*}
  x^2+y^2&=4	\\
  \rho^2\sin^2\phi\cos^2\theta+
     \rho^2\sin^2\phi\sin^2\theta=4	\\
  \rho^2\sin^2\phi(\cos^2\theta+\sin^2\theta)&=4	\\
  \rho^2\sin^2\phi&=4	\\
  \rho\sin\phi&=2	\\
  \rho&={2\over\sin\phi}
\end{align*}
\end{solution}


%%%%%%%%%%%%%%%%%%%%%%%%%%%%%%%%%%%%%%%%%%%%
\Opensolutionfile{solutions}[ex]
\section*{Exercises for \ref{sec:3Dothercoordsystem}}

\begin{enumialphparenastyle}

\begin{ex}
Convert the following points in rectangular coordinates to
cylindrical and spherical coordinates:
\begin{enumerate}
\item $(1,1,1)$
\item $(7,-7,5)$
\item $(\cos(1),\sin(1),1)$
\item $(0,0,-\pi)$
\end{enumerate}
\begin{sol}
\begin{enumerate}
	\item	$\ds (\sqrt2,\pi/4,1)$, $\ds (\sqrt3,\pi/4,\arccos(1/\sqrt3))$
	\item	$\ds (7\sqrt2,7\pi/4,5)$, $\ds (\sqrt{123},7\pi/4,\arccos(5/\sqrt{123})$
	\item	$(1,1,1)$, $\ds (\sqrt2,1,\pi/4)$
	\item	$(0,0,-\pi)$, $(\pi,0,\pi)$
\end{enumerate}
\end{sol}
\end{ex}

\begin{ex}
Find an equation for the sphere $x^2+y^2+z^2=4$ in
cylindrical coordinates.
\begin{sol}
	$r^2+z^2=4$
\end{sol}
\end{ex}

\begin{ex}
Find an equation for the $y$-$z$ plane in cylindrical
coordinates. 
\begin{sol}
	$r\cos\theta=0$
\end{sol}
\end{ex}

\begin{ex}
Find an equation equivalent to $x^2+y^2+2z^2+2z-5=0$ in
cylindrical coordinates.
\begin{sol}
	$r^2+2z^2+2z-5=0$
\end{sol}
\end{ex}

\begin{ex}
Suppose the curve $z=e^{-x^2}$ in the $x$-$z$ plane is
rotated around the $z$ axis. Find an equation for the resulting
surface in cylindrical coordinates.
\begin{sol}
	$z=e^{-r^2}$
\end{sol}
\end{ex}

\begin{ex}\label{ex:rotate z=x z axis cylindrical}
Suppose the curve $z=x$ in the $x$-$z$ plane is
rotated around the $z$ axis. Find an equation for the resulting
surface in cylindrical coordinates.
\begin{sol}
	$z=r$
\end{sol}
\end{ex}

\begin{ex}
Find an equation for the plane $y=0$ in
spherical coordinates.
\begin{sol}
	$\sin\theta=0$
\end{sol}
\end{ex}

\begin{ex}
Find an equation for the plane $z=1$ in
spherical coordinates.
\begin{sol}
	$1=\rho\cos\phi$
\end{sol}
\end{ex}

\begin{ex}
Find an equation for the sphere with radius 1 and center at
$(0,1,0)$ in spherical coordinates.
\begin{sol}
	$\rho=2\sin\theta\sin\phi$.
\end{sol}
\end{ex}

\begin{ex}
Find an equation for the cylinder $x^2+y^2=4$ in
spherical coordinates.
\begin{sol}
	$\rho\sin\phi=2$
\end{sol}
\end{ex}

\begin{ex}\label{ex:rotate z=x z axis spherical}
Suppose the curve $z=x$ in the $x$-$z$ plane is
rotated around the $z$ axis. Find an equation for the resulting
surface in spherical coordinates.
\begin{sol}
	$\cos\phi=1/\sqrt2$
\end{sol}
\end{ex}

\begin{ex}
Plot the polar equations $r=\sin(\theta)$ and $r=\cos(\theta)$
and comment on their similarities.  (If you get stuck on how to plot
these, you can multiply both sides of each equation by $r$ and convert
back to rectangular coordinates).
\end{ex}

\begin{ex}
Extend Exercises~\ref{ex:rotate z=x z axis cylindrical} 
and \ref{ex:rotate z=x z axis spherical} by rotating the curve $z=mx$
around the $z$ axis and converting to both cylindrical and spherical
coordinates.
\begin{sol}
	$z=mr$; $\cot\phi=m$ if $m\neq0$, $\phi=0$ if $m=0$
\end{sol}
\end{ex}

\begin{ex}
Convert the spherical formula $\rho=\sin \theta \sin \phi$ to
rectangular coordinates and describe the surface defined by the
formula (Hint: Multiply both sides by $\rho$.)
\begin{sol}
	A sphere with radius $1/2$, center at $(0,1/2,0)$
\end{sol}
\end{ex}

\begin{ex}
We can describe points in the first octant by $x >0$, $y>0$ and
$z>0$.  Give similar inequalities for the first octant in cylindrical
and spherical coordinates.
\begin{sol}
	$0<\theta<\pi/2$, $0<\phi<\pi/2$, $\rho>0$;
$0<\theta<\pi/2$, $r>0$, $z>0$
\end{sol}
\end{ex}

\end{enumialphparenastyle}
