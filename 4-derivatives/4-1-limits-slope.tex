\section{The Rate of Change of a Function}\label{sec:Slope}

Suppose that $y$ is a function of $x$, say $y = f(x)$.
It is often useful to know how sensitive the value of $y$
is to small changes in $x$.

\begin{example}{Small Changes in $x$}{SmallChanges}
Consider  $\ds y=f(x)=\sqrt{625-x^2}$ (the upper semicircle of radius
25 centered at the origin), and let's compute the changes of $y$ 
resulting from small changes of $x$ around $x=7$.
\end{example}

\begin{solution} 
When $x=7$, we find that $\ds y=\sqrt{625-49}=24$.
Suppose we want to know how much $y$ changes when $x$ increases a little,
say to 7.1 or 7.01.

In the case of a straight line $y=mx+b$, the slope $m=\Delta y/\Delta
x$ measures the change in $y$ per unit change in $x$. This can be
interpreted as a measure of ``sensitivity''; for example, if
$y=100x+5$, a small change in $x$ corresponds to a change one hundred
times as large in $y$, so $y$ is quite sensitive to changes in $x$.

Let us look at the same ratio $\Delta y/\Delta x$ for our function
$\ds y=f(x)=\sqrt{625-x^2}$ when $x$ changes from 7 to $7.1$.  Here $\Delta
x=7.1-7=0.1$ is the change in $x$, and
\begin{align*}
\Delta y =f(x+\Delta x)-f(x)&=f(7.1)-f(7)\cr
         &=\sqrt{625-7.1^2}-\sqrt{625-7^2}\cr
	 & \approx 23.9706-24=-0.0294.\cr
\end{align*}
Thus, $\Delta y/\Delta x\approx -0.0294/0.1=-0.294$. This means that $y$
changes by less than one third the change in $x$, so apparently $y$ is
not very sensitive to changes in $x$ at $x=7$. We say ``apparently''
here because we don't really know what happens between 7 and
$7.1$. Perhaps $y$ changes dramatically as $x$ runs through the values
from 7 to $7.1$, but at $7.1$ $y$ just happens to be close to its
value at $7$. This is not in fact the case for this particular
function, but we don't yet know why.
\end{solution}

The quantity $\Delta y/\Delta x\approx -0.294$ may be interpreted as 
the slope of the line through $(7,24)$ and
$(7.1,23.9706)$, called a \dfont{chord} of the circle.
In general, if we draw the chord from the point $(7,24)$ to a nearby
point on the semicircle $(7+\Delta x,\,f(7+\Delta x))$, the slope of this
chord is the so-called \dfont{difference quotient}
$$
\frac{f(7+\Delta x)-f(7)}{\Delta x}=
\frac{\sqrt{625-(7+\Delta x)^2}-24}{\Delta x}.
$$
For example, if $x$ changes only from 7 to 7.01, then the
difference quotient (slope of the chord) is approximately equal to
$(23.997081-24)/0.01=-0.2919$.  This is slightly different than for the
chord from $(7,24)$ to $(7.1,23.9706)$.

As $\Delta x$ is made smaller (closer to 0), $7+\Delta x$ gets closer to 7 and the chord joining
$(7,f(7))$ to $(7+\Delta x,f(7+\Delta x))$ shifts slightly, as shown
 in Figure~\ref{fig:chords}. The chord
gets closer and closer to the \dfont{tangent line} 
to the circle at the point $(7,24)$.  (The
tangent line is the line that just grazes the circle at that point,
i.e., it doesn't meet the circle at any second point.)  Thus, as
$\Delta x$ gets smaller and smaller, the slope $\Delta y/\Delta x$ of
the chord gets closer and closer to the slope of the tangent line.
This is actually quite difficult to see when $\Delta x$ is small,
because of the scale of the graph. The values of $\Delta x$ used for
the figure are $1$, $5$, $10$ and $15$, not really very small values.
The tangent line is the one that is uppermost at the right hand
endpoint.

\smallskip
\begin{figure}[!ht]
\centerline{\beginpicture
\normalgraphs
%\ninepoint
\setcoordinatesystem units <2truemm,2truemm>
\setplotarea x from 0 to 26, y from 0 to 26
\circulararc 90 degrees from 25 0 center at 0 0
\axis left ticks numbered from 5 to 25 by 5 /
\axis bottom ticks numbered from 5 to 25 by 5 /
\plot 0 26.04 25 18.75 /
\plot 0 26.2 25 18.3379 /
\plot 0 26.896 25 16.554 /
\plot 0 27.97 25 13.8 /
\plot 3 27.5384 25 8.02 /
\endpicture}
\caption{Chords approximating the tangent line. \label{fig:chords}}
\end{figure}

So far we have found the slopes of two chords that should be close to
the slope of the tangent line, but what is the slope of the tangent
line exactly? Since the tangent line touches the circle at just one
point, we will never be able to calculate its slope directly, using
two ``known'' points on the line. What we need is a way to capture
what happens to the slopes of the chords as they get ``closer and
closer'' to the tangent line.

Instead of looking at more particular values of $\Delta x$, let's see
what happens if we do some algebra with the difference quotient using
just $\Delta x$. The slope of a chord from $(7,24)$ to a nearby point
$\left(7+\Delta x,f(7+\Delta x)\right)$ is given by
\begin{align*}
\frac{f(7+\Delta x)-f(7)}{\Delta x}&=\frac{\sqrt{625-(7+\Delta x)^2} - 24}{\Delta x}\\
&= \left( \ds\frac{\sqrt{625-(7+\Delta x)^2} - 24}{\Delta x} \right) 
\left(\frac{\sqrt{625-(7+\Delta x)^2}+24}{\sqrt{625-(7+\Delta x)^2}+24} \right)\\
&=\frac{625-(7+\Delta x)^2-24^2}{\Delta x(\sqrt{625-(7+\Delta x)^2}+24)}\\
&=\frac{49-49-14\Delta x-\Delta x^2}{\Delta x(\sqrt{625-(7+\Delta x)^2}+24)}\\
&=\frac{\Delta x(-14-\Delta x)}{\Delta x(\sqrt{625-(7+\Delta x)^2}+24)}\\
&=\frac{-14-\Delta x}{\sqrt{625-(7+\Delta x)^2}+24}
\end{align*}
Now, can we tell by looking at this last formula what happens when
$\Delta x$ gets very close to zero? The numerator clearly gets very
close to $-14$ while the denominator gets very close to
$\ds \sqrt{625-7^2}+24=48$. The fraction is therefore very close to 
$-14/48 = -7/24 \cong -0.29167$. In fact, the slope of the tangent line is exactly $-7/24$.

What about the slope of the tangent line at $x=12$? Well, 12 can't be
all that different from 7; we just have to redo the calculation with
12 instead of 7. This won't be hard, but it will be a bit
tedious. What if we try to do all the algebra without using a specific
value for $x$? Let's copy from above, replacing 7 by  $x$.
\begin{align*}
\frac{f(x+\Delta x)-f(x)}{\Delta x}&=\frac{\sqrt{625-(x+\Delta x)^2} - \sqrt{625-x^2}}{\Delta x}\\
&=\frac{\sqrt{625-(x+\Delta x)^2} - \sqrt{625-x^2}}{\Delta x}
\frac{\sqrt{625-(x+\Delta x)^2}+\sqrt{625-x^2}}{\sqrt{625-(x+\Delta x)^2}+\sqrt{625-x^2}}\\
&=\frac{625-(x+\Delta x)^2-625+x^2}{\Delta x(\sqrt{625-(x+\Delta x)^2}+\sqrt{625-x^2})}\\
&=\frac{625-x^2-2x\Delta x-\Delta x^2-625+x^2}{\Delta x(\sqrt{625-(x+\Delta x)^2}+\sqrt{625-x^2})}\\
&=\frac{\Delta x(-2x-\Delta x)}{\Delta x(\sqrt{625-(x+\Delta x)^2}+\sqrt{625-x^2})}\\
&=\frac{-2x-\Delta x}{\sqrt{625-(x+\Delta x)^2}+\sqrt{625-x^2}}
\end{align*}

Now what happens when $\Delta x$ is very close to zero? Again it seems
apparent that the quotient will be very close to
$$\frac{-2x}{\sqrt{625-x^2}+\sqrt{625-x^2}}
=\frac{-2x}{2\sqrt{625-x^2}}=\frac{-x}{\sqrt{625-x^2}}.
$$
Replacing $x$ by 7 gives $-7/24$, as before, and now we can easily do
the computation for 12  or any other value of
$x$ between $-25$ and 25.

So now we have a single expression, $\ds {-x/ \sqrt{625-x^2}}$,
that tells us the slope of the tangent line for any value of
$x$. This slope, in turn, tells us how sensitive the value of $y$ is
to small changes in the value of $x$. 

The expression $\ds {-x/ \sqrt{625-x^2}}$ defines a new function
 called the \dfont{derivative} of the
original function (since it is derived from the original function).
 If the original is referred to as $f$ or $y$ then
the derivative is often written $f'$ or $y'$ (pronounced ``f
prime'' or ``y prime''). So in this case we might write $\ds f'(x)=-x/
\sqrt{625-x^2}$ or $y'=\ds {-x/ \sqrt{625-x^2}}$. At a particular point, say $x=7$, we write
$f'(7)=-7/24$ and we say that ``$f$ prime of 7 is $-7/24$'' or ``the derivative of
$f$ at 7 is $-7/24$.''

To summarize, we compute the derivative of $f(x)$ by forming the
difference quotient
$$
\frac{f(x+\Delta x)-f(x)}{\Delta x},
$$
which is the slope of a line, then we figure out what happens when
$\Delta x$ gets very close to 0. 

At this point, we should note that the idea of letting $\Delta x$ get closer and closer to 0
is precisely the idea of a limit that we discussed in the last chapter. The limit here is a limit
as $\Delta x$ approaches 0. Using limit notation, we can write
$f'(x)=\lim_{\Delta x\to 0}\frac{f(x+\Delta x)-f(x)}{\Delta x}$.

In the particular case of a circle, there's a simple way to find the
derivative.  Since the tangent to a circle at a point is perpendicular to
the radius drawn to the point of contact, its slope is the negative
reciprocal of the slope of the radius.  The radius joining $(0,0)$ to
$(7,24)$ has slope 24/7.  Hence, the tangent line has slope
$-7/24$. In general, a radius to the point $\ds (x,\sqrt{625-x^2})$ has
slope $\ds \sqrt{625-x^2}/x$, so the slope of the tangent line is
$\ds {-x/ \sqrt{625-x^2}}$, as before. It is {\bf NOT} always true that a
tangent line is perpendicular to a line from the origin---don't use
this shortcut in any other circumstance.

As above, and as you might expect, for different values of $x$ we
generally get different values of the derivative $f'(x)$. Could it be
that the derivative always has the same value? This would mean that
the slope of $f$, or the slope of its tangent line, is the same
everywhere. One curve that always has the same slope is a line; it
seems odd to talk about the tangent line to a line, but if it makes
sense at all the tangent line must be the line itself. It is not hard
to see that the derivative of $f(x)=mx+b$ is $f'(x)=m$.