\section{Implicit Differentiation}\label{sec:ImplicitDiff}
As we have seen, there is a close relationship 
between the derivatives of $\ds e^x$ and $\ln x$ because these functions
are inverses. Rather than relying on pictures for our understanding,
we would like to be able to exploit this relationship
computationally.  In fact this technique can help us find derivatives
in many situations, not just when we seek the derivative of an inverse
function. 

We will begin by illustrating the technique to find what we already
know, the derivative of $\ln x$. Let's write $y=\ln x$ and then
$\ds x=e^{\ln x}=e^y$, that is, $\ds x=e^y$. We say that this equation
defines the function $y=\ln x$ 
implicitly because while it is not an explicit expression $y=\ldots$, it is true
that if $\ds x=e^y$ then $y$ is in fact the natural logarithm
function. Now, for the time being, pretend that all we know of $y$ is
that $\ds x=e^y$; what can we say about derivatives? We can take the
derivative of both sides of the equation:
$$\frac{d}{dx}x=\frac{d}{dx}e^y.$$
Then using the chain rule on the right hand side:
$$1 = \left(\frac{d}{dx}y\right) e^y = y'e^y.$$
Then we can solve for $y'$:
$$y'=\frac{1}{e^y} = \frac{1}{x}.$$
There is one little difficulty here. To use the chain rule to compute 
$\ds d/dx(e^y)=y'e^y$ we need to know that the function $y$ {\it has} a
derivative. All we have shown is that {\it if} it has a derivative
then that derivative must be $1/x$. When using this method we will
always have to assume that the desired derivative exists, but
fortunately this is a safe assumption for most such problems. 
 
The example $y=\ln x$ involved an inverse function defined implicitly,
but other functions can be defined implicitly, and sometimes a single
equation can be used to implicitly define more than one
function. 

Here's a familiar example.

\begin{example}{Derivative of Circle Equation}{impcircles}
The equation $\ds r^2=x^2+y^2$
describes a circle of radius $r$. The circle is not a function
$y=f(x)$ because for some values of $x$ there are two corresponding
values of $y$. If we want to work with a function, we can break the
circle into two pieces, the upper and lower semicircles, each of which
is a function. Let's call these $y=U(x)$ and $y=L(x)$; in fact this is
a fairly simple example, and it's possible to give explicit
expressions for these: $\ds U(x)=\sqrt{r^2-x^2\ }$ and
$\ds L(x)=-\sqrt{r^2-x^2\ }$.  But it's somewhat easier, and quite useful,
to view both functions as given implicitly by $\ds r^2=x^2+y^2$: both
$\ds r^2=x^2+U(x)^2$ and $\ds r^2=x^2+L(x)^2$ are true, and we can think of 
$\ds r^2=x^2+y^2$ as defining both $U(x)$ and $L(x)$.

Now we can take the derivative of both sides as before, remembering
that $y$ is not simply a variable but a function---in this case, $y$
is either $U(x)$ or $L(x)$ but we're not yet specifying which one.
When we take the derivative we just have to remember to apply the
chain rule where $y$ appears.
\begin{eqnarray*}
\frac{d}{dx}r^2&=&\frac{d}{dx}(x^2+y^2)\\
0&=&2x+2yy'\\
y'&=&\frac{-2x}{2y}=-\frac{x}{y}
\end{eqnarray*}
Now we have an expression for $y'$, but it contains $y$ as well as
$x$. This means that if we want to compute $y'$ for some particular
value of $x$ we'll have to know or compute $y$ at that value of $x$ as
well. It is at this point that we will need to know whether $y$ is
$U(x)$ or $L(x)$. Occasionally it will turn out that we can avoid
explicit use of $U(x)$ or $L(x)$ by the nature of the problem. 
\end{example}

\begin{example}{Slope of the Circle}{SlopeCircle}
Find the slope of the circle $\ds 4=x^2+y^2$ at the point
$\ds (1,-\sqrt{3})$. 
\end{example}

\begin{solution} 
Since we know both the $x$ and $y$ coordinates of the
point of interest, we do not need to explicitly recognize that this
point is on $L(x)$, and we do not need to use $L(x)$ to compute
$y$ -- but we could. Using the calculation of $y'$ from above, 
$$y'=-\frac{x}{y}=-\frac{1}{-\sqrt{3}}=\frac{1}{\sqrt{3}}.$$
It is instructive to compare this approach to others.

We might have recognized at the start that $\ds (1,-\sqrt{3})$ is on the
function $\ds y=L(x)=-\sqrt{4-x^2}$. We could then take the derivative of
$L(x)$, using the power rule and the chain rule, to get
$$L'(x)=-{1\over 2}(4-x^2)^{-1/2}(-2x)={x\over\sqrt{4-x^2}}.$$
Then we could compute $\ds L'(1)=1/\sqrt{3}$ by substituting $x=1$.

Alternately, we could realize that the point is on $L(x)$, but use the
fact that $y'=-x/y$. Since the point is on $L(x)$ we can replace $y$
by $L(x)$ to get
$$y'=-\frac{x}{L(x)}=-\frac{x}{\sqrt{4-x^2}},$$
without computing the derivative of $L(x)$ explicitly. Then we
substitute $x=1$ and get the same answer as before.
\end{solution}

In the case of the circle it is possible to find the functions $U(x)$
and $L(x)$ explicitly, but there are potential advantages to using
implicit differentiation anyway. In some cases it is more difficult or
impossible to find an explicit formula for $y$ and implicit
differentiation is the only way to find the derivative.

\begin{example}{Derivative of Function defined Implicitly}{FunctionDefinedImplicitly}
Find the derivative of any function defined implicitly by 
$\ds yx^2+y^2=x$. 
\end{example}

\begin{solution} 
We treat $y$ as an unspecified function and use the
chain rule:
\begin{eqnarray*}
\frac{d}{dx}(yx^2+y^2)&=&\frac{d}{dx}x\\
(y\cdot 2x+y'\cdot x^2)+2yy'&=1\\
y'\cdot x^2+2yy'&=&-y\cdot 2x\\
y'&=&\frac{-2xy}{x^2+2y}
\end{eqnarray*}
\end{solution}

\begin{example}{Derivative of Function defined Implicitly}{FunctionDefinedImplicitly2}
Find the derivative of any function defined implicitly by 
$\ds yx^2+e^y=x$. 
\end{example}

\begin{solution} 
We treat $y$ as an unspecified function and use the
chain rule:
\begin{eqnarray*}
\frac{d}{dx}(yx^2+e^y)&=&\frac{d}{dx}x\\
(y\cdot 2x+y'\cdot x^2)+y'e^y &=& 1\\
y'x^2+y'e^y&=& 1-2xy\\
y'(x^2+e^y)&=& 1-2xy\\
y'&=&\frac{1-2xy}{x^2+e^y}
\end{eqnarray*}
\end{solution}

You might think that the step in which we solve for $y'$ could
sometimes be difficult---after all, we're using implicit
differentiation here because we can't solve the equation
$\ds yx^2+e^y=x$ for $y$, so maybe after taking the derivative we get
something that is hard to solve for $y'$. In fact, {\it this never
  happens.} All occurrences $y'$ come from applying the chain rule,
and whenever the chain rule is used it deposits a single $y'$
multiplied by some other expression. So it will always be possible to
group the terms containing $y'$ together and factor out the $y'$, just
as in the previous example. If you ever get anything more difficult
you have made a mistake and should fix it before trying to continue.

It is sometimes the case that a situation leads naturally to an
equation that defines a function implicitly. 

\begin{example}{Equation and Derivative of Ellipse}{Ellipse}
Discuss the equation and derivative of the ellispe.
\end{example}

\begin{solution} 
Consider all the points $(x,y)$ that have the property that
the distance from $(x,y)$ to $\ds (x_1,y_1)$ plus the distance 
from $(x,y)$ to $\ds (x_2,y_2)$ is $2a$ ($a$ is some constant). These
points form an ellipse, which like a circle is not a function but can be
viewed as two functions pasted together. Since we know how to write
down the distance between two points, we can write down an implicit
equation for the ellipse:
$$\sqrt{(x-x_1)^2+(y-y_1)^2}+\sqrt{(x-x_2)^2+(y-y_2)^2}=2a.$$
Then we can use implicit differentiation to find the slope of the
ellipse at any point, though the computation is rather messy.
\end{solution}

\begin{example}{Derivative of Function defined Implicitly}{FunctionDefinedImplicitly3}
Find $\ds\frac{dy}{dx}$ by implicit differentiation if
$$2x^3+x^2y-y^9=3x+4.$$
\vspace{-0.5cm}
\end{example}

\begin{solution} 
Differentiating both sides with respect to $x$ gives:
$$6x^2+\left(2xy+x^2\frac{dy}{dx}\right)-9y^8\frac{dy}{dx}=3,$$
$$x^2\frac{dy}{dx}-9y^8\frac{dy}{dx}=3-6x^2-2xy$$
$$\left(x^2-9y^8\right)\frac{dy}{dx}=3-6x^2-2xy$$
$$\frac{dy}{dx}=\frac{3-6x^2-2xy}{x^2-9y^8}.$$
\end{solution}

In the previous examples we had functions involving $x$ and $y$, and we
thought of $y$ as a function of $x$.  In these problems we
differentiated with respect to $x$. So when faced with $x$'s in the
function we differentiated as usual, but when faced with $y$'s we
differentiated as usual except we multiplied by a $\frac{dy}{dx}$ for
that term because we were using Chain Rule.

In the following example we will assume that both $x$ and $y$ are
functions of $t$ and want to differentiate the equation with respect
to $t$.  This means that every time we differentiate an $x$ we will be
using the Chain Rule, so we must multiply by $\frac{dx}{dt}$, and
whenever we differentiate a $y$ we multiply by $\frac{dy}{dt}$.

\begin{example}{Derivative of Function of an Additional Variable}{FunctionAdditionalVariable}
Thinking of $x$ and $y$ as functions of $t$, differentiate the following equation with respect to $t$:
$$x^2+y^2=100.$$
\vspace{-0.5cm}
\end{example}

\begin{solution} 
Using the Chain Rule we have:
$$2x\frac{dx}{dt}+2y\frac{dy}{dt}=0.$$
\end{solution}

\begin{example}{Derivative of Function of an Additional Variable}{FunctionAdditionalVariable2}
If $y=x^3+5x$ and $\ds\frac{dx}{dt}=7$, find $\ds\frac{dy}{dt}$ when $x=1$.
\end{example}

\begin{solution} 
Differentiating each side of the equation $y=x^3+5x$ with respect to $t$ gives:
$$\frac{dy}{dt}=3x^2\frac{dx}{dt}+5\frac{dx}{dt}.$$
When $x=1$ and $\frac{dx}{dt}=7$ we have:
$$\frac{dy}{dt}=3(1^2)(7)+5(7)=21+35=56.$$
\end{solution}

%%%%%%%%%%%%%%%%%%%%%%%%%%%%%%%%%%%%%%%%%
\subsection*{Logarithmic Differentiation}
Previously we've seen how to do the derivative of a number to a function $(a^{f(x)})'$, and also a function to a number $[(f(x))^n]'$.
But what about the derivative of a function to a function $[(f(x))^{g(x)}]'$?

In this case, we use a procedure known as \dfont{logarithmic differentiation}.

\begin{formulabox}[Steps for Logarithmic Differentiation]
\begin{itemize}
\item Take $\ln$ of both sides of $y=f(x)$ to get $\ln y=\ln f(x)$ and simplify using logarithm properties,
\item Differentiate implicitly with respect to $x$ and solve for $\ds\frac{dy}{dx}$,
\item Replace $y$ with its function of $x$ (i.e., $f(x)$).
\end{itemize}
\end{formulabox}

\begin{example}{Logarithmic Differentiation}{LogarithmicDifferentiation}
Differentiate $y=x^x$.
\end{example}

\begin{solution} 
We take $\ln$ of both sides:
$$\ln y=\ln x^x.$$
Using log properties we have:
$$\ln y = x\ln x.$$
Differentiating implicitly gives:
$$\frac{y'}{y}=(1)\ln x+x\frac{1}{x}.$$
$$\frac{y'}{y}=\ln x+1.$$
Solving for $y'$ gives:
$$y'=y(1+\ln x).$$
Replace $y=x^x$ gives:
$$y'=x^x(1+\ln x).$$

Another method to find this derivative is as follows:
\begin{eqnarray*}
\frac{d}{dx}x^x&=&\frac{d}{dx}e^{x\ln x}\\
&=&\left(\frac{d}{dx}x\ln x\right)e^{x\ln x}\\
&=&(x\frac{1}{x}+\ln x)x^x\\
&=&(1+\ln x)x^x
\end{eqnarray*}
\end{solution}

In fact, logarithmic differentiation can be used on more complicated
products and quotients (not just when dealing with functions to the
power of functions).

\begin{example}{Logarithmic Differentiation}{LogarithmicDifferentiation2}
Differentiate (assuming $x>0$):
$$y=\frac{(x+2)^3(2x+1)^9}{x^8(3x+1)^4}.$$
\vspace{-0.5cm}
\end{example}

\begin{solution} 
Using product \& quotient rules for this problem is a complete nightmare!
Let's apply logarithmic differentiation instead.
Take $\ln$ of both sides:
$$\ln y=\ln\left(\frac{(x+2)^3(2x+1)^9}{x^8(3x+1)^4}\right).$$
Applying log properties:
$$\ln y=\ln\left((x+2)^3(2x+1)^9\right)-\ln\left(x^8(3x+1)^4\right).$$
$$\ln y=\ln\left((x+2)^3\right)+\ln\left((2x+1)^9\right)-\left[\ln\left(x^8\right)+\ln\left((3x+1)^4\right)\right].$$
$$\ln y=3\ln(x+2)+9\ln(2x+1)-8\ln x-4\ln(3x+1).$$
Now, differentiating implicitly with respect to $x$ gives:
$$\frac{y'}{y}=\frac{3}{x+2}+\frac{18}{2x+1}-\frac{8}{x}-\frac{12}{3x+1}.$$
Solving for $y'$ gives:
$$y'=y\left(\frac{3}{x+2}+\frac{18}{2x+1}-\frac{8}{x}-\frac{12}{3x+1}\right).$$
Replace $y=\frac{(x+2)^3(2x+1)^9}{x^8(3x+1)^4}$ gives:
$$y'=\frac{(x+2)^3(2x+1)^9}{x^8(3x+1)^4}\left(\frac{3}{x+2}+\frac{18}{2x+1}-\frac{8}{x}-\frac{12}{3x+1}\right).$$
\end{solution}


%%%%%%%%%%%%%%%%%%%%%%%%%%%%%%%%%%%%%%%%%%%%%%%%%
\Opensolutionfile{solutions}[ex]
\section*{Exercises for Section \ref{sec:ImplicitDiff}}

\begin{enumialphparenastyle}

%%%%%%%%%%
\begin{ex} 
Find a formula for the derivative $y'$ at the point $(x,y)$:
\begin{multicols}{2}
\begin{enumerate}
	\item	$\ds y^2=1+x^2$
	\item	$\ds x^2+xy+y^2=7$
	\item	$\ds x^3+xy^2=y^3+yx^2$
	\item	$\ds 4\cos x \sin y = 1$
	\item	$\ds\sqrt{x} + \sqrt{y} = 9$
	\item	$\ds \tan(x/y) = x+ y$
	\item	$\ds \sin (x+y ) =xy$
	\item	$\ds\frac{1}{x} + \frac{1}{y} = 7$
\end{enumerate}
\end{multicols}
\begin{sol}
\begin{enumerate}
	\item	$\ds x/y$
	\item	$\ds -(2x+y)/(x+2y)$
	\item	$\ds (2xy-3x^2-y^2)/(2xy-3y^2-x^2)$
	\item	$\ds \sin(x)\sin(y)/(\cos(x)\cos(y))$
	\item	$\ds-\sqrt{y}/\sqrt{x}$
	\item	$\ds (y\sec^2(x/y)-y^2)/(x\sec^2(x/y)+y^2)$
	\item	$\ds (y-\cos(x+y))/(\cos(x+y)-x)$
	\item	$\ds -y^2/x^2$
\end{enumerate}
\end{sol}
\end{ex}

%%%%%%%%%%
\begin{ex} 
A hyperbola passing through $(8,6)$ consists of all points whose distance
from the origin is a constant more than its distance from the point (5,2).
Find the slope of the tangent line to the hyperbola at $(8,6)$.
\begin{sol}
	$1$
\end{sol}
\end{ex}

%%%%%%%%%%
\begin{ex} 
The graph of the equation $\ds x^2 - xy + y^2 = 9$ is an ellipse.
Find the lines tangent to this curve at the two
 points where it intersects the $x$-axis. Show that these lines are
 parallel.
\begin{sol}
	$y=2x\pm6$
\end{sol}
\end{ex}

%%%%%%%%%%
\begin{ex} 
Repeat the previous problem for the points at which the
 ellipse intersects the $y$-axis.
\begin{sol}
	$y=x/2\pm3$
\end{sol}
\end{ex}

%%%%%%%%%%
\begin{ex} 
	If $\ds y=\log_a x$ then $\ds a^y=x$. Use implicit
	differentiation to find $\ds y'$.
\end{ex}

%%%%%%%%%%
\begin{ex} 
Find the points on the ellipse from the previous two problems
 where the slope is horizontal and where it is vertical.
\begin{sol}
	$\ds (\sqrt3,2\sqrt3)$, $\ds (-\sqrt3,-2\sqrt3)$, $\ds (2\sqrt3,\sqrt3)$,
$\ds (-2\sqrt3,-\sqrt3)$ 
\end{sol}
\end{ex}

%%%%%%%%%%
\begin{ex} 
Find an equation for the tangent line to 
$\ds x^4 = y^2 + x^2$ at $\ds (2, \sqrt{12})$. 
(This curve is the \dfont{kampyle of Eudoxus}.)
\begin{sol}
	$\ds y=7x/\sqrt3-8/\sqrt3$
\end{sol}
\end{ex}

%%%%%%%%%%
\begin{ex} 
Find an equation for the tangent line to $\ds x^{2/3} +
y^{2/3} = a^{2/3}$ at a point $\ds (x_1 ,y_1)$ on the curve, 
with $\ds x_1 \neq 0$ and $\ds y_1 \neq 0$. (This curve is an {\dfont astroid}.)
\begin{sol}
	$\ds y=(-y_1^{1/3}x+y_1^{1/3}x_1+x_1^{1/3}y_1)/x_1^{1/3}$
\end{sol}
\end{ex}

%%%%%%%%%%
\begin{ex} 
Find an equation for the tangent line to $\ds (x^2 +y^2 )^2 =x^2
-y^2$ at a point $\ds (x_1 , y_1)$ on the curve, with $\ds x_1 \neq 0, -1, 1$.
(This curve is a \dfont{lemniscate}.)
\begin{sol}
	$\ds (y-y_1)/(x-x_1)=(2x_1^3+2x_1y_1^2-x_1)/(2y_1^3+2y_1x_1^2+y_1)$
\end{sol}
\end{ex}

%%%%%%%%%%
\begin{ex}  
Two curves are \dfont{orthogonal} if at each point of intersection,
the angle between their tangent lines is $\pi/2$. Two
families of curves, $\cal{A}$ and $\cal{B}$, are
\dfont{orthogonal trajectories} of each other if given any curve $C$
in $\cal{A}$ and any curve $D$ in $\cal{B}$ the curves $C$
and $D$ are orthogonal.
For example, the family of horizontal lines in the plane is
orthogonal to the family of vertical lines in the plane.
\begin{enumerate}
	\item	Show that $\ds x^2 -y^2 =5$ is orthogonal to $\ds 4x^2 +9y^2
	=72$. (Hint: You need to find the intersection points of the two
	curves and then show that the product of the derivatives at each
	intersection point is $-1$.)
	\item	Show that $\ds x^2 +y^2 = r^2$ is orthogonal to
	$y=mx$. Conclude that the family of circles centered at the origin is
	an orthogonal trajectory of the family of lines that pass through the
	origin.
	
	Note that there is a technical issue when $m=0$. The circles fail to
	be differentiable when they cross the $x$-axis. However, the circles
	are orthogonal to the $x$-axis. Explain why. Likewise, the vertical
	line through the origin requires a separate argument.
	\item	For $k\not= 0$ and $c \neq 0$ show that $\ds y^2 -x^2 =k$ is orthogonal to
	$yx =c$. In the case where $k$ and $c$ are both zero, the curves
	intersect at the origin. Are the curves $\ds y^2 -x^2 =0$ and $yx=0$
	orthogonal to each other?
	\item	Suppose that $m\neq 0$. Show that the family of curves
	$\ds \{y=mx+b \mid b\in \R \}$ is orthogonal to the
	family of curves $\ds \{y=-(x/m)+c \mid c \in \R\}$.
\end{enumerate}
\end{ex}

%%%%%%%%%%
\begin{ex} 
Differentiate the function $\ds y={(x-1)^8 (x-23)^{1/2}\over 27 x^6(4x-6)^8 }$
\end{ex}

\begin{ex}
Differentiate the function $\ds f(x)=(x+1)^{\sin x}$.
\end{ex}

\begin{ex}
Differentiate the function $\ds g(x)=\frac{e^x(\cos x+2)^3}{\sqrt{x^2+4}}$.
\end{ex}

\end{enumialphparenastyle}