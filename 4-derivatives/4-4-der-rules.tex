\section{Derivative Rules}\label{sec:DerivativeRules}
Using the definition of the derivative of a function is quite tedious.
In this section we introduce a number of different shortcuts that can be used to compute the derivative.
Recall that the \ifont{definition of derivative} is:

Given any number $x$ for which the limit 
$$f'(x)=\lim_{h\to 0}\frac{f(x+h)-f(x)}{h}$$
exists, we assign to $x$ the number $f'(x)$.

Next, we give some basic \ifont{derivative rules} for finding derivatives without having to use the limit definition directly.

\begin{theorem}{Derivative of a Constant Function}{Derivative of a Constant Function}
Let $c$ be a constant, then $\ds{\frac{d}{dx}(c)=0}$.
\end{theorem}

\begin{proof}
Let $f(x)=c$ be a constant function. By the definition of derivative:
$$f'(x)=\lim_{h\to 0}\frac{f(x+h)-f(x)}{h}=\lim_{h\to 0}\frac{c-c}{h}=\lim_{h\to 0}0=0.$$
\end{proof}

\begin{example}{Derivative of a Constant Function}{DerivativeConstantFunction}
 The derivative of $f(x)=17$ is $f'(x)=0$ since the derivative of a constant is $0$.
\end{example}

\begin{theorem}{The Power Rule}{The Power Rule}
If $n$ is a positive integer, then $\ds{\frac{d}{dx}(x^n)=nx^{n-1}}$.
\end{theorem}

\begin{proof}
We use the formula:
$$x^n-a^n=(x-a)(x^{n-1}+x^{n-2}a+\cdots+xa^{n-2}+a^{n-1})$$
which can be verified by mulitplying out the right side.
Let $f(x)=x^n$ be a power function for some positive integer $n$.
Then at any number $a$ we have:
$$f'(a)=\lim_{x\to a}\frac{f(x)-f(a)}{x-a}=\lim_{x\to a}\frac{x^n-a^n}{x-a}=\lim_{x\to a}(x^{n-1}+x^{n-2}a+\cdots+xa^{n-2}+a^{n-1})=na^{n-1}.$$
\end{proof}

It turns out that the Power Rule holds for any real number $n$ (though it is a bit more difficult to prove).

\begin{theorem}{The Power Rule (General)}{The Power Rule (General)}
If $n$ is any real number, then $\ds{\frac{d}{dx}(x^n)=nx^{n-1}}$.
\end{theorem}

\begin{example}{Derivative of a Power Function}{DerivativePowerFunction}
By the power rule, the derivative of $g(x)=x^4$ is $g'(x)=4x^3$.
\end{example}

\begin{theorem}{The Constant Multiple Rule}{The Constant Multiple Rule}
If $c$ is a constant and $f$ is a differentiable function, then $$\ds{\frac{d}{dx}[cf(x)]=c\frac{d}{dx}f(x)}.$$
\end{theorem}

\begin{proof}
For convenience let $g(x)=cf(x)$.
Then:
$$g'(x)=\lim_{h\to 0}\frac{g(x+h)-g(x)}{h}=\lim_{h\to 0}\frac{cf(x+h)-cf(x)}{h}$$
$$=\lim_{h\to 0}c\left[\frac{f(x+h)-f(x)}{h}\right]=c\lim_{h\to 0}\frac{f(x+h)-f(x)}{h}=cf'(x),$$
where $c$ can be moved in front of the limit by the Limit Rules.
\end{proof}

\begin{example}{Derivative of a Multiple of a  Function}{DerivativeMultipleFunction}
By the constant multiple rule and the previous example, the derivative of $F(x)=2\cdot(17+x^4)$ is
$$F'(x)=2(4x^3)=8x^3.$$
\end{example}

\begin{theorem}{The Sum/Difference Rule}{The Sum/Difference Rule}
If $f$ and $g$ are both differentiable functions, then
$$\ds{\frac{d}{dx}[f(x)\pm g(x)]=\frac{d}{dx}f(x)\pm\frac{d}{dx}g(x)}.$$
\end{theorem}

\begin{proof}
For convenience let $r(x)=f(x)\pm g(x)$.
Then:
$$\begin{array}{ccl}
r'(x)&=&\ds{\lim_{h\to 0}\frac{r(x+h)-r(x)}{h}}\\
\\
&=&\ds{\lim_{h\to 0}\frac{[f(x+h)\pm g(x+h)]-[f(x)\pm g(x)]}{h}}\\
\\
~&=&\ds{\lim_{h\to 0}\left[\frac{f(x+h)-f(x)}{h}\pm\frac{g(x+h)-g(x)}{h}\right]}\\
\\
&=&\ds{\lim_{h\to 0}\frac{f(x+h)-f(x)}{h}\pm\lim_{h\to 0}\frac{g(x+h)-g(x)}{h}}\\
\\
~&=&f'(x)+g'(x)\\\end{array}$$
\end{proof}

\begin{example}{Derivative of a Sum/Difference of Functions}{DerivativeSumDifferenceFunction}
By the sum/difference rule, the derivative of $h(x)=17+x^4$ is 
$$h'(x)=f'(x)+g'(x)=0+4x^3=4x^3.$$
\end{example}

\begin{theorem}{The Product Rule}{The Product Rule}
If $f$ and $g$ are both differentiable functions, then
$$\ds{\frac{d}{dx}[f(x)\cdot g(x)]=f(x)\frac{d}{dx}[g(x)]+g(x)\frac{d}{dx}[f(x)]}.$$
\end{theorem}

\begin{proof}
For convenience let $r(x)=f(x)\cdot g(x)$.
As in the previous proof, we want to separate the functions $f$ and $g$.
The trick is to add and subtract $f(x+h)g(x)$ in the numerator.
Then:
$$\begin{array}{ccl}
r'(x)&=&\ds{\lim_{h\to 0}\frac{f(x+h)g(x+h)-f(x+h)g(x)+f(x+h)g(x)-f(x)g(x)}{h}}\\
\\
~&=&\ds{\lim_{h\to 0}\left[f(x+h)\frac{g(x+h)-g(x)}{h}+g(x)\frac{f(x+h)-f(x)}{h}\right]}\\
\\
~&=&\ds{\lim_{h\to 0}f(x+h)\cdot\lim_{h\to 0}\frac{g(x+h)-g(x)}{h}+\lim_{h\to 0}g(x)\cdot\lim_{h\to 0}\frac{f(x+h)-f(x)}{h}}\\
\\
~&=&f(x)g'(x)+g(x)f'(x)\\\end{array}$$
\end{proof}

\begin{example}{Derivative of a Product of Functions}{DerivativeProductFunction}
Find the derivative of $\ds h(x)=(3x-1)(2x+3)$.
\end{example}

\begin{solution} 
One way to do this question is to expand the expression.
Alternatively, we use the product rule with $f(x)=3x-1$ and $g(x)=2x+3$.
Note that $f'(x)=3$ and $g'(x)=2$, so,
$$h'(x)=(3)\cdot(2x+3)+(3x-1)\cdot(2)=6x+9+6x-2=12x+7.$$
\end{solution}

\begin{theorem}{The Quotient Rule}{The Quotient Rule}
If $f$ and $g$ are both differentiable functions, then $$\ds{\frac{d}{dx}\left[\frac{f(x)}{g(x)}\right]
=\frac{g(x)\frac{d}{dx}[f(x)]-f(x)\frac{d}{dx}[g(x)]}{[g(x)]^2}}.$$
\end{theorem}

\begin{proof}
The proof is similar to the previous proof but the trick is to add and subtract the term $f(x)g(x)$ in the numerator.
We omit the details.
\end{proof}

\begin{example}{Derivative of a Quotient of Functions}{DerivativeQuotientFunction}
Find the derivative of $\ds h(x)=\frac{3x-1}{2x+3}$.
\end{example}

\begin{solution} 
By the quotient rule (using $f(x)=3x-1$ and $g(x)=2x+3$) we have:
$$h'(x)=\frac{\frac{d}{dx}(3x-1)\cdot (2x+3)-(3x-1)\cdot\frac{d}{dx}(2x+3)}{(2x+3)^2}$$
$$=\frac{3(2x+3)-(3x-1)(2)}{(2x+3)^2}=\frac{11}{(2x+3)^2}.$$
\end{solution}

\begin{example}{Second Derivative}{}
Find the second derivative of $f(x)=5x^3+3x^2$.
\end{example}

\begin{solution} 
We must differentiate $f(x)$ twice:
$$f'(x)=15x^2+6x,$$
$$f''(x)=30x+6.$$
\end{solution}


%%%%%%%%%%%%%%%%%%%%%%%%%%%%%%%%%%%%%%%%%%%%%%%%%
\Opensolutionfile{solutions}[ex]
\section*{Exercises for Section \ref{sec:DerivativeRules}}

\begin{enumialphparenastyle}

%%%%%%%%%%
\begin{ex} 
Find the derivatives of the following functions.
\begin{multicols}{3}
\begin{enumerate}
	\item	$\ds x^{100}$
	\item	$\ds x^{-100}$
	\item	$\ds {1\over x^5}$
	\item	$\ds x^\pi$
	\item	$\ds x^{3/4}$
	\item	$\ds x^{-9/7}$
	\item	$\ds 5x^3+12x^2-15$
	\item	$\ds -4x^5 + 3x^2 - 5/x^2$
	\item	$\ds 5(-3x^2 + 5x + 1)$
	\item	$\ds (x+1)(x^2+2x-3)$
	\item	$\ds (x+1)(x^2+2x-3)^{-1}$
	\item	$\ds x^3(x^3-5x+10)$
	\item	$\ds (x^2+5x-3)(x^5)$
	\item	$\ds (x^2+5x-3)(x^{-5})$
	\item	$\ds (5x^3+12x^2-15)^{-1}$
\end{enumerate}
\end{multicols}
\begin{sol}
\begin{multicols}{3}
\begin{enumerate}
	\item	$\ds 100x^{99}$
	\item	$\ds -100x^{-101}$
	\item	$\ds -5x^{-6}$
	\item	$\ds \pi x^{\pi-1}$
	\item	$\ds (3/4)x^{-1/4}$
	\item	$\ds -(9/7)x^{-16/7}$
	\item	$\ds 15x^2+24x$
	\item	$\ds -20x^4+6x+10/x^3$
	\item	$\ds -30x+25$
	\item	$\ds 3x^2+6x-1$
	\item	$\ds -\frac{x^2+2x+5}{(x^2+2x-3)^2}$
	\item	$\ds 3x^2(x^3-5x+10)+x^3(3x^2-5)$ 
	\item	$\ds x^4(7x^2+30x-15)$
	\item	$\ds\frac{-3x^2-20x+15}{x^6}$
	\item	$\ds -\frac{3x(5x+8)}{(5x^3+12x^2-15)^2}$ 
\end{enumerate}
\end{multicols}
\end{sol} 
\end{ex}

%%%%%%%%%%
\begin{ex} 
 Find an equation for the tangent line to $\ds f(x) = x^3/4 - 1/x$ at $x=-2$.
\begin{sol}
$y=13x/4+5$
\end{sol}
\end{ex}

%%%%%%%%%%
\begin{ex} 
Find an equation for 
the tangent line to $\ds f(x)= 3x^2 - \pi ^3$ at $x= 4$.
\begin{sol}
$\ds y=24x-48-\pi^3$
\end{sol}
\end{ex}

%%%%%%%%%%
\begin{ex} 
Suppose the position of an object at time $t$ is  given by
$\ds f(t)=-49 t^2/10+5t+10$. Find a function giving the speed of the object
at time $t$. The acceleration of an object is the rate at which its
speed is changing, which means it is given by the derivative of the
speed function. Find the acceleration of the object at time $t$.
\begin{sol}
$-49t/5+5$, $-49/5$
\end{sol}
\end{ex}

%%%%%%%%%%
\begin{ex} 
Let $\ds f(x) =x^3$ and $c= 3$. Sketch the graphs of $f$,
$cf$, $f'$, and $(cf)'$ on the same diagram.
\end{ex}

%%%%%%%%%%
\begin{ex} 
The general polynomial $P$ of degree $n$ in the variable $x$
has the form $\ds P(x)= \sum _{k=0 } ^n a_k x^k = a_0 + a_1 x + \ldots
+ a_n x^n$. What is the derivative (with respect to $x$)
of $P$?
\begin{sol}
$\ds\sum_{k=1}^n ka_kx^{k-1}$
\end{sol}
\end{ex}

%%%%%%%%%%
\begin{ex} 
Find a cubic polynomial whose graph has horizontal tangents at
$(-2 , 5)$ and $(2, 3)$.
\begin{sol}
$\ds x^3/16-3x/4+4$
\end{sol}
\end{ex}
 
%%%%%%%%%%
\begin{ex} 
Prove that $\ds{d\over dx}(cf(x))= cf'(x)$ using the
definition of the derivative.
\end{ex}

%%%%%%%%%%
\begin{ex} 
Suppose that $f$ and $g$ are differentiable at $x$. Show
that $f-g$ is differentiable at $x$ using the two linearity
properties from this section.
\end{ex}

%%%%%%%%%%
\begin{ex} 
Use the product rule to compute the derivative of $\ds f(x)=(2x-3)^2$.
 Sketch the function.  Find an equation of the tangent line to the curve at
 $x=2$.  Sketch the tangent line at $x=2$.
\begin{sol}
$f'=4(2x-3)$, $y=4x-7$
\end{sol}
\end{ex}

%%%%%%%%%%
\begin{ex} 
Suppose that $f$, $g$, and $h$ are differentiable functions.
Show that $(fgh)'(x) = f'(x) g(x)h(x) + f(x)g'(x) h(x) + f(x) g(x)
h'(x)$.
\end{ex}

%\exercise
%State and prove a rule to compute $(fghi)'(x)$, 
%similar to the rule in the previous problem.
%
%\remark{Product notation}
%Suppose $\ds f_1 , f_2 , \ldots f_n$ are functions.
%The product of all these functions can be written
%$$ \prod _{k=1 } ^n f_k.$$
%This is similar to the use of $\ds \sum$ to denote a 
%sum.
%For example,
%$$\prod _{k=1 } ^5 f_k =f_1 f_2 f_3 f_4 f_5$$
%and
%$$
%\prod _ {k=1 } ^n k = 1\cdot 2 \cdot \ldots \cdot n = n!.$$
%We sometimes use somewhat more complicated conditions; for example
%$$\prod _{k=1 , k\neq j } ^n f_k$$
%denotes the product of $\ds f_1$ through $\ds f_n$ except for $\ds f_j$.
%For example, 
%$$\prod _{k=1 , k\neq 4} ^5 x^k = x\cdot x^2 \cdot x^3 \cdot x^5 =
%x^{11}.$$
%\endremark
%
%\exercise
%  The \dfont{generalized product rule} 
%says that if $\ds f_1 , f_2 ,\ldots ,f_n$ are differentiable functions at
%  $x$ then
%$${d\over dx}\prod _{k=1 } ^n f_k(x) = 
%\sum _{j=1 } ^n \left(f'_j (x) \prod _{k=1 , k\neq j} ^n
%   f_k (x)\right).$$
%Verify that this is the same as your answer to the previous problem
%when $n=4$,
%and write out what this says when $n=5$.


%%%%%%%%%%%
%\begin{ex} 
%Find all points on the graph of
%$\ds f(x)=\sin^2(x)$ at which the tangent line is horizontal.
%\begin{sol} 
%$n\pi/2$, any integer $n$
%\end{sol}
%\end{ex}
%
%%%%%%%%%%%
%\begin{ex} 
%Find all points on the graph of $\ds f(x) = 2\sin(x) -
%\sin^2(x)$ at which the tangent line is horizontal.
%\begin{sol} 
%$\pi/2+n\pi$, any integer $n$
%\end{sol}
%\end{ex}
%
%%%%%%%%%%%
%\begin{ex} 
%Find an
 %equation for the tangent line to $\ds \sin^2(x)$ at 
%$x=\pi/3$.
%\begin{sol} 
%$\sqrt3x/2+3/4-\sqrt3\pi/6$
%\end{sol}
%\end{ex}
%
%%%%%%%%%%%
%\begin{ex} 
%Find an equation for the tangent line to $\ds \sec ^2 x$
%at $x=\pi/3$.
%\begin{sol} 
%$\ds 8\sqrt3x+4-8\sqrt3\pi/3$
%\end{sol}
%\end{ex}
%
%%%%%%%%%%%
%\begin{ex} 
%Find an equation for the tangent line to $\ds \cos ^2 x -
%\sin ^2 (4x)$ at $x=\pi/6$.
%\begin{sol} 
%$\ds 3\sqrt3x/2-\sqrt3\pi/4$
%\end{sol}
%\end{ex}
%
%%%%%%%%%%%
%\begin{ex} 
%Find the points on the curve $\ds y= x+ 2\cos x$ that have a
%horizontal tangent line.
%\begin{sol} 
%$\ds \pi/6+2n\pi$, $5\pi/6+2n\pi$, any integer $n$
%\end{sol}
%\end{ex}
%
%%%%%%%%%%%
%\begin{ex} 
%Let $C$ be a circle of radius $r$. Let $A$ be an arc on $C$
%subtending a central angle $\theta$. Let $B$ be the chord of
%$C$ whose endpoints are the endpoints of $A$. (Hence, $B$ also
%subtends $\theta$.) Let $s$ be the length of $A$
%and let $d$ be the length of $B$. Sketch a diagram of the situation
%and compute $\ds \lim_{\theta \to 0^+ } s/d$.
%\end{ex}

%%%%%%%%%%
\begin{ex} 
Compute the derivative of $\ds {x^3\over x^3-5x+10}$.
\begin{sol} 
$\ds {3x^2\over x^3-5x+10}-{x^3(3x^2-5)\over (x^3-5x+10)^2}$
\end{sol}
\end{ex}

%%%%%%%%%%
\begin{ex} 
Compute the derivative of $\ds {x^2+5x-3\over x^5-6x^3+3x^2-7x+1}$.
\begin{sol} 
$\ds {2x+5\over x^5-6x^3+3x^2-7x+1}-{(x^2+5x-3)(5x^4-18x^2+6x-7)\over(x^5-6x^3+3x^2-7x+1)^2}$
\end{sol}
\end{ex}

%%%%%%%%%%
\begin{ex} 
Compute the derivative of $\ds {x\over\sqrt{x-625}}$.
\begin{sol} 
$\ds \frac{x-1250}{2(x-625)^{3/2}}$
\end{sol}
\end{ex}

%%%%%%%%%%
\begin{ex} 
Compute the derivative of $\ds {\sqrt{x-5}\over x^{20}}$.
\begin{sol} 
$\ds \frac{200-39x}{2x^{21}\sqrt{x-5}}$
\end{sol}
\end{ex}

%%%%%%%%%%
\begin{ex} 
Find an equation for the tangent line to $\ds f(x) = (x^2 -
4)/(5-x)$ at $x= 3$.  
\begin{sol} 
$\ds y=17x/4-41/4$ 
\end{sol}
\end{ex}

%%%%%%%%%%
\begin{ex} 
Find an equation for the tangent line to 
$\ds f(x) = (x-2)/(x^3 + 4x - 1)$ at $x=1$.
\begin{sol} 
$y=11x/16-15/16$
\end{sol}
\end{ex}

%\exercise Let $P$ be a polynomial of degree $n$ and let $Q$ be a
%polynomial of degree $m$ (with $Q$ not the zero polynomial). 
%Using sigma notation we can write
%$$P=\sum _{k=0 } ^n a_k x^k,\qquad
%Q=\sum_{k=0}^m b_k x^k.
%$$
%Use sigma notation to write the derivative of the 
%{\dfont rational function\index{rational function}\/}
%$P/Q$.
%% \begin{sol} $\left(\sum_{k=0}^m b_k x^k\sum _{k=1}^n ka_k x^{k-1}-
%% \sum _{k=0 } ^n a_k x^k\sum_{k=1}^m kb_k x^k\right)/
%% \left(\sum_{k=0}^m b_k x^k\right)^2$
%% \end{sol}\end{ex}

%\exercise The curve $\ds y=1/(1+x^2)$ is an example of a class of
%curves each of which is called a {\dfont witch of
%Agnesi\index{witch of Agnesi}}. 
%Sketch the curve and find the tangent line to the curve at
%$x= 5$. (The word {\em witch\/} here is a mistranslation of the
%original Italian, as described at
%$$\hbox{\url{http://mathworld.wolfram.com/WitchofAgnesi.html} 
%\vb|http://mathworld.wolfram.com/WitchofAgnesi.html|\endurl}$$
%and 
%$$\eqalign{
%\hbox{\url{http://instructional1.calstatela.edu/sgray/Agnesi/WitchHistory/Historynamewitch.html} 
%\vb|http://|\endurl}%
%&\!\!\hbox{\url{http://instructional1.calstatela.edu/sgray/Agnesi/WitchHistory/Historynamewitch.html} 
%\vb|instructional1.calstatela.edu/sgray/Agnesi/|\endurl}\cr
%&\hbox{\url{http://instructional1.calstatela.edu/sgray/Agnesi/WitchHistory/Historynamewitch.html} 
%\vb|WitchHistory/Historynamewitch.html|\endurl.)}\cr
%}$$
%\begin{sol} $y=19/169-5x/338$
%\end{sol}\end{ex}
%%% \footnote{Due to a mistranslation of the Italian word
%%%   \emph{versiera} which actually refers to a rope that turns the
%%%   sail.}.  
 
%%%%%%%%%%
\begin{ex} 
If $f'(4) = 5$, $g'(4) = 12$, $(fg)(4)= f(4)g(4)=2$, and $g(4) = 6$,
compute $f(4)$ and $\ds{d\over dx}{f\over g}$ at 4.
\begin{sol} 
$13/18$
\end{sol}
\end{ex}

\end{enumialphparenastyle}