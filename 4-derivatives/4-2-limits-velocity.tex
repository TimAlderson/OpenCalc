\subsection*{Velocity}\label{sec:Velocity}

We started this section by saying, ``It is often useful to know
how sensitive the value of $y$ is to small changes in $x$.'' We have
seen one purely mathematical example of this, involving the function $f(x)=\sqrt{625-x^2}$. 
Here is a more applied example.

With careful measurement it might be possible to discover that the height of a
dropped ball $t$ seconds after it is released is $\ds h(t)=h_0-kt^2$. 
(Here $h_0$ is the initial height of the ball, when $t=0$,
and $k$ is some number determined by the experiment.)  A natural
question is then, ``How fast is the ball going at time $t$?'' We can
certainly get a pretty good idea with a little simple arithmetic. To
make the calculation more concrete, let's use units of meters and seconds and say that $\ds h_0=100$ meters and $k=4.9$.
Suppose we're interested in the speed at $t=2$. We know that when
$t=2$ the height is $100-4\cdot 4.9=80.4$ meters. A second later, at $t=3$,
the height is $100-9\cdot 4.9=55.9$ meters. The change in height during that second is
 $55.9-80.4=-24.5$ meters. The negative sign means the height has decreased, as we expect for a falling ball, and the number 24.5 is the average speed of the ball during the time interval, in meters per second.

We might guess that 24.5 meters per second is not a terrible estimate of the speed at
$t=2$, but certainly we can do better. At $t=2.5$ the height is
$\ds 100-4.9(2.5)^2=69.375$ meters. During the half second from $t=2$ to $t=2.5$, the change in height is
 $69.375-80.4=-11.025$ meters giving an average speed of 
$11.025/(1/2)=22.05$ meters per second. This should be a better estimate
of the speed at $t=2$. So it's clear now how to get better and better
approximations: compute average speeds over shorter and shorter time
intervals. Between $t=2$ and $t=2.01$, for example, the ball drops
0.19649 meters in one hundredth of a second, at an average speed of
19.649 meters per second.
 
We still might reasonably ask for the precise speed at $t=2$ (the {\em instantaneous} speed)
rather than just an approximation to it. For this, once again, we need a limit. Let's calculate
the average speed during the time interval from $t=2$ to $t=2+\Delta t$ without specifying a particular value for $\Delta t$.
The change in height during the time interval from $t=2$ to $t=2+\Delta t$ is
\begin{align*}
h(2+\Delta t)-h(2)
&=(100-4.9(2+\Delta t)^2)-80.4\\
&=100-4.9(4+4\Delta t+\Delta t^2)-80.4\\
&=100-19.6-19.6\Delta t-4.9\Delta t^2-80.4\\
&=-19.6\Delta t-4.9\Delta t^2\\
&=-\Delta t(19.6+4.9\Delta t)
\end{align*}
The average speed during this time interval is then
$$\frac{\Delta t(19.6+4.9\Delta t)}{\Delta t}=19.6+4.9\Delta t.$$

When $\Delta t$ is very small, this is very close to 19.6. Indeed,
$\lim_{\Delta x\to 0}(19.6+4.9\Delta t)=19.6$. So the exact speed at $t=2$ is 19.6 meters per second.

At this stage we need to make a distinction between \textit{speed} and
\textit{velocity}. Velocity is signed speed, that is, speed with a
direction indicated by a sign (positive or negative). Our algebra
above actually told us that the instantaneous velocity of the ball at $t=2$ is
$-19.6$ meters per second. The number 19.6 is the speed and the
negative sign indicates that the motion is directed downwards (the
direction of decreasing height).

In the language of the previous section, we might have started with
$\ds f(x)=100-4.9x^2$ and asked for the slope of the tangent line at
$x=2$. We would have answered that question by computing $$
\lim_{\Delta x\to 0}\frac{f(2+\Delta x) - f(2)}{\Delta x}
=\lim_{\Delta x\to 0}\frac{-19.6\Delta x-4.9\Delta x^2}{\Delta x}
=-19.6-4.9\Delta x=19.6
$$ 
The algebra is the same. Thus, the velocity of the ball is the value
of the derivative of a certain function, namely, of the function that
gives the position of the ball.

The upshot is that this problem, finding the velocity of the ball, is
\ifont{exactly} the same problem mathematically as finding the slope
of a curve. This may already be enough evidence to convince you that
whenever some quantity is changing (the height of a curve or the
height of a ball or the size of the economy or the distance of a space
probe from earth or the population of the world) the \textit{rate} at which the
quantity is changing can, in principle, be computed in exactly the
same way, by finding a derivative.

%%%%%%%%%%%%%%%%%%%%%%%%%%%%%%%%%%%%%%%%%%%%
\Opensolutionfile{solutions}[ex]
\section*{Exercises for Section \ref{sec:Slope}}

\begin{enumialphparenastyle}

%%%%%%%%%%
\begin{ex}
Draw the graph of the function $\ds y=f(x)=\sqrt{169-x^2}$ between $x=0$
and $x=13$.  Find the slope $\Delta y/\Delta x$ of the chord between the
points of the circle lying over (a) $x=12$ and $x=13$, (b) $x=12$ and
$x=12.1$,  (c) $x=12$ and $x=12.01$, (d) $x=12$ and $x=12.001$.  Now use
the geometry of tangent lines on a circle to find (e) the exact value of the
derivative $f'(12)$.  Your answers to (a)--(d) should be getting closer and
closer to your answer to (e).
\begin{sol}
$-5$, $-2.47106145$, $-2.4067927$, $-2.400676$, $-2.4$
\end{sol}
\end{ex}

%%%%%%%%%%
\begin{ex}
Use geometry to find the derivative $f'(x)$ of the function
$\ds f(x)=\sqrt{625-x^2}$ in the text for each of the following $x$: (a) 20,
(b) 24, (c) $-7$, (d) $-15$.  Draw a graph of the upper semicircle, and
draw the tangent line at each of these four points.
\begin{sol}
$-4/3$, $-24/7$, $7/24$, $3/4$
\end{sol}
\end{ex}

%%%%%%%%%%
\begin{ex}
Draw the graph of the function $y=f(x)=1/x$ between $x=1/2$ and $x=4$.
Find the slope of the chord between (a) $x=3$ and $x=3.1$, (b) $x=3$ and
$x=3.01$, (c) $x=3$ and $x=3.001$.  Now use algebra to find a simple
formula for the slope of the chord between $(3,f(3))$ and $(3+\Delta
x,f(3+\Delta x))$.  Determine what happens when $\Delta x$ approaches 0.
In your graph of $y=1/x$, draw the straight line through the point
$(3,1/3)$ whose slope is this limiting value of the difference quotient as
$\Delta x$ approaches 0.
\begin{sol}
$-0.107526881$, $-0.11074197$, $-0.1110741$, 
$\ds{-1\over3(3+\Delta x)}\rightarrow {-1\over9}$
\end{sol}
\end{ex}

%%%%%%%%%%
\begin{ex}
Find an algebraic expression for the difference quotient $\ds \bigl(f(1+\Delta
x)-f(1)\bigr)/\Delta x$ when $\ds f(x)=x^2-(1/x)$.  Simplify the expression as
much as possible.  Then determine what happens as $\Delta x$ approaches 0.
That value is $f'(1)$.
\begin{sol}
$\ds{3+3\Delta x+\Delta x^2\over1+\Delta x}\rightarrow3$ 
\end{sol}
\end{ex}

%%%%%%%%%%
\begin{ex}
Draw the graph of $\ds y=f(x)=x^3$ between $x=0$ and $x=1.5$.  Find the slope
of the chord between (a) $x=1$ and $x=1.1$, (b) $x=1$ and $x=1.001$, (c)
$x=1$ and $x=1.00001$.  Then use algebra to find a simple formula for the
slope of the chord between $1$ and $1+\Delta x$.  (Use the expansion
$\ds (A+B)^3=A^3+3A^2B+3AB^2+B^3$.)  Determine what happens as $\Delta x$
approaches 0, and in your graph of $\ds y=x^3$ draw the straight line through
the point $(1,1)$ whose slope is equal to the value you just found.
\begin{sol}
$3.31$, $3.003001$, $3.0000$,\hfill\break
 $3+3\Delta x+\Delta x^2\rightarrow3$
\end{sol}
\end{ex}

%%%%%%%%%%
\begin{ex}\label{ex:derivative of a line}
Find an algebraic expression for the difference quotient $(f(x+\Delta
x)-f(x))/\Delta x$ when $f(x)=mx+b$.  Simplify the expression as
much as possible.  Then determine what happens as $\Delta x$ approaches 0.
That value is $f'(x)$.
\begin{sol}
$m$
\end{sol}
\end{ex}


%%%%%%%%%%
\begin{ex}
Sketch the unit circle.  Discuss the behavior of the slope
of the tangent line at various angles around the circle.  Which
trigonometric function gives the slope of the tangent line at an angle
$\theta$?  Why? Hint: think in terms of ratios of sides of triangles.
\end{ex}

%%%%%%%%%%
\begin{ex}
Sketch the parabola $\ds y=x^2$.  For what values of $x$ on the parabola
is the slope of the tangent line positive?  Negative?  What do you notice
about the graph at the point(s) where the sign of the slope changes from
positive to negative and vice versa?
\end{ex}

%%%%%%%%%%
\begin{ex}
An object is traveling in a straight line so that its position (that
is, distance from some fixed point) is
given by this table:
\begin{table}[!ht]
\begin{tabular}{|c|c|c|c|c|}
\hline
time (seconds)& 0& 1& 2& 3\\
\hline
distance (meters)& 0& 10& 25& 60\\
\hline
\end{tabular}
\end{table}

Find the average speed of the object during the following time
intervals: $[0,1]$, $[0,2]$, $[0,3]$,
$[1,2]$, $[1,3]$, $[2,3]$. If you had to guess the speed at
$t=2$ just on the basis of these, what would you guess?
\begin{sol}
$10$, $25/2$, $20$, $15$, $25$, $35$.
\end{sol}
\end{ex}

%%%%%%%%%%
\begin{ex}
Let $\ds y=f(t)=t^2$, where $t$ is the time in seconds and $y$ is the distance
in meters that an object falls on a certain airless planet.  Draw a graph
of this function between $t=0$ and $t=3$.  Make a table of the average
speed of the falling object between (a) 2 sec and 3 sec, (b) 2 sec and
2.1 sec, (c) 2 sec and 2.01 sec, (d) 2 sec and 2.001 sec.  Then use algebra
to find a simple formula for the average speed between time $2$ and time
$2+
\Delta t$.  (If you substitute $\Delta t=1,\>0.1,\>0.01,\>0.001$ in this
formula you should again get the answers to parts (a)--(d).)  Next, in your
formula for average speed (which should be in simplified form) determine
what happens as $\Delta t$ approaches zero.  This is the instantaneous
speed.  Finally, in your graph of $\ds y=t^2$ draw the straight line
through the point $(2,4)$ whose slope is the instantaneous velocity you
just computed; it should of course be the tangent line.
\begin{sol}
$5$, $4.1$, $4.01$, $4.001$, $4+\Delta t\rightarrow 4$
\end{sol}
\end{ex}

%%%%%%%%%%
\begin{ex}
If an object is dropped from an 80-meter high window, its height $y$ above
the ground at time $t$ seconds is given by the formula $\ds y=f(t)=80-4.9t^2$.
(Here we are neglecting air resistance; the graph of this function was
shown in figure~\ref{fig:data plot}.)  Find the average velocity of
the falling object between (a) 1 sec and 1.1 sec, (b) 1 sec and 1.01 sec,
(c) 1 sec and 1.001 sec.  Now use algebra to find a simple formula for the
average velocity of the falling object between 1 sec and $1+\Delta t$ sec.
Determine what happens to this average velocity as $\Delta t$ approaches 0.
That is the instantaneous velocity at time $t=1$ second (it will be negative,
because the object is falling).
\begin{sol}
$-10.29$, $-9.849$, $-9.8049$, \hfill\break
$-9.8-4.9\Delta t\rightarrow -9.8$
\end{sol}
\end{ex}

\end{enumialphparenastyle}