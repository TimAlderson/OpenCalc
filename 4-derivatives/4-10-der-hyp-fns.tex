% pathAPEXCalculus_Source/text/06_Hyperbolic_Functions.tex
%c13ffb9  on Jun 15, 2015
%@APEXCalculus APEXCalculus Updated files from Version 2.0 to Version 3.0.
%1 contributor
%RawBlameHistory     421 lines (353 sloc)  20.7 KB
\section{Hyperbolic Functions}\label{sec:hyperbolic}

This section defines the derivatives of the hyperbolic functions and describes some of their properties. Recall their definition.

\begin{definition}{Hyperbolic Functions}{def:hyperbolic_functions} 
{\noindent%
\begin{minipage}{.5\textwidth}
\begin{enumerate}
\item		$\ds \cosh x = \frac{e^x+e^{-x}}2$\index{hyperbolic function!definition}
\item		$\ds \sinh x = \frac{e^x-e^{-x}}2$
\item		$\ds \tanh x = \frac{\sinh x}{\cosh x}$
\end{enumerate}
\end{minipage}
\begin{minipage}{.5\textwidth}
\begin{enumerate}\addtocounter{enumi}{3}
\item		$\ds \sech x = \frac{1}{\cosh x}$
\item		$\ds \csch x = \frac{1}{\sinh x}$
\item		$\ds \coth x = \frac{\cosh x}{\sinh x}$
\end{enumerate}
\end{minipage}
}\\
\end{definition}

\begin{example}{Exploring properties of hyperbolic functions}{ex_hf1}
\noindent Use Definition \ref{def:hyperbolic_functions} to rewrite the following expressions.
\begin{enumerate}
\item		$\frac{d}{dx}\big(\cosh x\big)$
\item		$\frac{d}{dx}\big(\sinh x\big)$
\item		$\frac{d}{dx}\big(\tanh x\big)$
\end{enumerate}
\end{example}

\begin{solution}
{
\begin{enumerate}
\item  \hfill$\begin{aligned}[t]
	\frac{d}{dx}\big(\cosh x\big) &= \frac{d}{dx}\left(\frac{e^x+e^{-x}}2\right) \\
					&= \frac{e^x-e^{-x}}2\\
					&= \sinh x.
	\end{aligned}\hfill$

So $\frac{d}{dx}\big(\cosh x\big) = \sinh x.$
	
\item  \hfill$\begin{aligned}[t]
	\frac{d}{dx}\big(\sinh x\big) &= \frac{d}{dx}\left(\frac{e^x-e^{-x}}2\right) \\
					&= \frac{e^x+e^{-x}}2\\
					&= \cosh x.
	\end{aligned}\hfill$

So $\frac{d}{dx}\big(\sinh x\big) = \cosh x.$
	
\item  \hfill$\begin{aligned}[t]
	\frac{d}{dx}\big(\tanh x\big) &= \frac{d}{dx}\left(\frac{\sinh x}{\cosh x}\right) \\
					&= \frac{\cosh x \cosh x - \sinh x \sinh x}{\cosh^2 x}\\
					&= \frac{1}{\cosh^2 x}\\
					&= \sech^2 x.
	\end{aligned}\hfill$

So $\frac{d}{dx}\big(\tanh x\big) = \sech^2 x.$	
\end{enumerate}
\vskip-\baselineskip
}
\end{solution}

The following Key Idea summarizes the derivatives relating to hyperbolic functions. Each can be verified by referring back to Definition \ref{def:hyperbolic_functions}.
\textbf{Derivatives}
\begin{enumerate}
\item $\frac{d}{dx}\big(\cosh x\big) = \sinh x$
\item $\frac{d}{dx}\big(\sinh x\big) = \cosh x$
\item $\frac{d}{dx}\big(\tanh x\big) = \sech^2 x$
\item $\frac{d}{dx}\big(\sech x\big) = -\sech x\tanh x$
\item $\frac{d}{dx}\big(\csch x\big) = -\csch x\coth x$
\item $\frac{d}{dx}\big(\coth x\big) = -\csch^2x$
\end{enumerate}


%We practice using Key Idea \ref{idea:hyperbolic_identities}.\\

%\example{ex_hf2}{Derivatives and integrals of hyperbolic functions}{
%Evaluate the following derivatives and integrals.

%\begin{minipage}[t]{.5\linewidth}
%\begin{enumerate}
%\item		$\ds\frac{d}{dx}\big(\cosh 2x\big)$
%\item		$\ds\int \sech^2(7t-3)\ dt$
%\end{enumerate}
%\end{minipage}
%\begin{minipage}[t]{.5\linewidth}
%\begin{enumerate}\addtocounter{enumi}{2}
%\item		$\ds \int_0^{\ln 2} \cosh x\ dx$
%\end{enumerate}
%\end{minipage}
%}
%{\begin{enumerate}
%\item		Using the Chain Rule directly, we have $\frac{d}{dx} \big(\cosh 2x\big) = 2\sinh 2x$.
%
%Just to demonstrate that it works, let's also use the Basic Identity found in Key Idea \ref{idea:hyperbolic_identities}: $\cosh 2x = \cosh^2x+\sinh^2x$.
%\begin{align*}\frac{d}{dx}\big(\cosh 2x\big) = \frac{d}{dx}\big(\cosh^2x+\sinh^2x\big) &= 2\cosh x\sinh x+ 2\sinh x\cosh x\\ &= 4\cosh x\sinh x.
%\end{align*}
%Using another Basic Identity, we can see that $4\cosh x\sinh x = 2\sinh 2x$. We get the same answer either way.

%\item	  We employ substitution, with $u = 7t-3$ and $du = 7dt$. Applying Key Ideas \ref{idea:linearsub}  and \ref{idea:hyperbolic_identities} we have:
%$$ \int \sech^2 (7t-3)\ dt = \frac17\tanh (7t-3) + C.$$
%
%\item		$$\int_0^{\ln 2} \cosh x\ dx = \sinh x\Big|_0^{\ln 2} = \sinh (\ln 2) - \sinh 0 = \sinh(\ln 2).$$
%
%We can simplify this last expression as $\sinh x$ is based on exponentials:
%$$\sinh(\ln 2) = \frac{e^{\ln 2}-e^{-\ln 2}}2 = \frac{2-1/2}{2} = \frac34.$$
%\end{enumerate}
%\vskip-1.5\baselineskip




