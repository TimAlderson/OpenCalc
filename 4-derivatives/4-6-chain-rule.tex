\section{The Chain Rule}\label{sec:ChainRule}
Let $h(x)=\sqrt{625-x^2}$.
The rules stated previously do not allow us to find $h'(x)$.
However, $h(x)$ is a composition of two functions.
Let $f(x)=\sqrt x$ and $g(x)=625-x^2$.
Then we see that 
$$h(x)=(f\circ g)(x).$$
From our rules we know that $f'(x)=\frac{1}{2}x^{-1/2}$ and $g'(x)=-2x$, thus it would be convenient to have a rule which allows us to differentiate $f\circ g$ in terms of $f'$ and $g'$.
This gives rise to the chain rule.

\begin{formulabox}[The Chain Rule]
If $g$ is differentiable at $x$ and $f$ is differentiable at $g(x)$, then the composite function $h=f\circ g$ [recall $f\circ g$ is defined as $f(g(x))$] is differentiable at $x$ and $h'(x)$ is given by:
$$h'(x)=f'(g(x))\cdot g'(x).$$
\end{formulabox}

The chain rule has a particularly simple expression if we use the
Leibniz notation for the derivative. The quantity $f'(g(x))$ is the
derivative of $f$ with $x$ replaced by $g$; this can be written 
$df/dg$. As usual, $g'(x)=dg/dx$. Then the chain rule becomes
$${df\over dx} = {df\over dg}{dg\over dx}.$$
This looks like trivial arithmetic, but it is not: $dg/dx$ is not a
fraction, that is, not literal division, but a single symbol that
means $g'(x)$. Nevertheless, it turns out that what looks like trivial
arithmetic, and is therefore easy to remember, is really true.

It will take a bit of practice to make the use of the chain rule come
naturally---it is more complicated than the earlier differentiation
rules we have seen.

\begin{example}{Chain Rule}{ChainRule}
Compute the derivative of $\ds \sqrt{625-x^2}$.
\end{example}

\begin{solution} 
 We already know that the
answer is $\ds -x/\sqrt{625-x^2}$, computed directly from the limit. In
the context of the chain rule, we have $\ds f(x)=\sqrt{x}$,
$\ds g(x)=625-x^2$. We know that $\ds f'(x)=(1/2)x^{-1/2}$, so $\ds f'(g(x))=
(1/2)(625-x^2)^{-1/2}$. Note that this is a two step computation:
first compute $f'(x)$, then replace $x$ by $g(x)$. Since $g'(x)=-2x$
we have
$$f'(g(x))g'(x)={1\over 2\sqrt{625-x^2}}(-2x)={-x\over
    \sqrt{625-x^2}}.
$$
\end{solution}

\begin{example}{Chain Rule}{ChainRule2}
Compute the derivative of $\ds 1/\sqrt{625-x^2}$.
\end{example}

\begin{solution} 
This is a quotient with
a constant numerator, so we could use the quotient rule, but it is
simpler to use the chain rule. The function is $\ds (625-x^2)^{-1/2}$, the
composition of $\ds f(x)=x^{-1/2}$ and $\ds g(x)=625-x^2$. We compute
$\ds f'(x)=(-1/2)x^{-3/2}$ using the power rule, and then
$$f'(g(x))g'(x)={-1\over 2(625-x^2)^{3/2}}(-2x)={x\over (625-x^2)^{3/2}}.$$
\end{solution}

In practice, of course, you will need to use more than one of the
rules we have developed to compute the derivative of a complicated
function.

\begin{example}{Derivative of Quotient}{DerivativeQuotient}
Compute the derivative of $$f(x)={x^2-1\over x\sqrt{x^2+1}}.$$
\vspace{-0.5cm}
\end{example}

\begin{solution} 
The ``last'' operation here is division, so to get started we need to
use the quotient rule first. This gives
\begin{eqnarray*}
f'(x)&=&{(x^2-1)'x\sqrt{x^2+1}-(x^2-1)(x\sqrt{x^2+1})'\over
x^2(x^2+1)}\cr
&=&{2x^2\sqrt{x^2+1}-(x^2-1)(x\sqrt{x^2+1})'\over
x^2(x^2+1)}.\cr
\end{eqnarray*}
Now we need to compute the derivative of $\ds x\sqrt{x^2+1}$. This is a
product, so we use the product rule:
$${d\over dx}x\sqrt{x^2+1}=x{d\over dx}\sqrt{x^2+1}+\sqrt{x^2+1}.$$
Finally, we use the chain rule:
$${d\over dx}\sqrt{x^2+1}={d\over dx}(x^2+1)^{1/2}=
{1\over 2}(x^2+1)^{-1/2}(2x)={x\over \sqrt{x^2+1}}.$$
And putting it all together:
\begin{eqnarray*}
f'(x)&=&{2x^2\sqrt{x^2+1}-(x^2-1)(x\sqrt{x^2+1})'\over
x^2(x^2+1)}.\cr
&=&{2x^2\sqrt{x^2+1}-(x^2-1)\left(x{\ds{x\over \sqrt{x^2+1}}}
+\sqrt{x^2+1}\right)\over
x^2(x^2+1)}.\cr
\end{eqnarray*}
This can be simplified of course, but we have done all the calculus,
so that only algebra is left.
\end{solution}

\begin{example}{Chain of Composition}{ChainComposition}
Compute the derivative of $\ds \sqrt{1+\sqrt{1+\sqrt{x}}}$. 
\end{example}

\begin{solution} 
Here we have a
more complicated chain of compositions, so we use the chain rule
twice.
At the outermost ``layer'' we have the function
$\ds g(x)=1+\sqrt{1+\sqrt{x}}$ plugged into $\ds f(x)=\sqrt{x}$, so applying
the chain rule once gives 
$${d\over dx}\sqrt{1+\sqrt{1+\sqrt{x}}}=
{1\over 2}\left(1+\sqrt{1+\sqrt{x}}\right)^{-1/2}{d\over dx}
\left(1+\sqrt{1+\sqrt{x}}\right).$$
Now we need the derivative of $\ds \sqrt{1+\sqrt{x}}$. Using the chain
rule again:
$${d\over dx}\sqrt{1+\sqrt{x}}={1\over
  2}\left(1+\sqrt{x}\right)^{-1/2}{1\over 2}x^{-1/2}.$$
So the original derivative is 
\begin{eqnarray*}
{d\over dx}\sqrt{1+\sqrt{1+\sqrt{x}}}&=&
{1\over 2}\left(1+\sqrt{1+\sqrt{x}}\right)^{-1/2}
{1\over
  2}\left(1+\sqrt{x}\right)^{-1/2}{1\over 2}x^{-1/2}.\cr
&=&{1\over 8 \sqrt{x}\sqrt{1+\sqrt{x}}\sqrt{1+\sqrt{1+\sqrt{x}}}}
\end{eqnarray*}
\end{solution}

Using the chain rule, the power rule, and the product rule, it is
possible to avoid using the quotient rule entirely.

\begin{example}{Derivative of Quotient without Quotient Rule}{DerivativeQuotientWithoutQuotientRule}
Compute the derivative of $\ds f(x)={x^3\over x^2+1}$.
\end{example}

\begin{solution} 
Write 
$\ds f(x)=x^3(x^2+1)^{-1}$, then
\begin{eqnarray*}
f'(x)&=&x^3{d\over dx}(x^2+1)^{-1}+3x^2(x^2+1)^{-1}\cr
\\
&=&x^3(-1)(x^2+1)^{-2}(2x)+3x^2(x^2+1)^{-1}\cr
\\
&=&-2x^4(x^2+1)^{-2}+3x^2(x^2+1)^{-1}\cr
\\
&=&{-2x^4\over (x^2+1)^{2}}+{3x^2\over x^2+1}\cr
\\
&=&{-2x^4\over (x^2+1)^{2}}+{3x^2(x^2+1)\over (x^2+1)^{2}}\cr
\\
&=&{-2x^4+3x^4+3x^2\over (x^2+1)^{2}}={x^4+3x^2\over (x^2+1)^{2}}
\end{eqnarray*}
Note that we already had the derivative on the second line; all the
rest is simplification. It is easier to get to this answer by using
the quotient rule, so there's a trade off: more work for fewer
memorized formulas.
\end{solution}


%%%%%%%%%%%%%%%%%%%%%%%%%%%%%%%%%%%%%%%%%%%%%%%%%
\Opensolutionfile{solutions}[ex]
\section*{Exercises for Section \ref{sec:ChainRule}}

\begin{enumialphparenastyle}

Find the derivatives of the functions. For extra practice, and to
check your answers, do some of these in more than one way if
possible. 

\begin{ex}  $\ds x^4-3x^3+(1/2)x^2+7x-\pi$
\begin{sol} 
$\ds 4x^3-9x^2+x+7$
\end{sol}
\end{ex}

\begin{ex}  $\ds x^3-2x^2+4\sqrt{x}$
\begin{sol} 
$\ds 3x^2-4x+2/\sqrt{x}$
\end{sol}
\end{ex}

\begin{ex}  $\ds (x^2+1)^3$
\begin{sol} 
$\ds 6(x^2+1)^2x$
\end{sol}
\end{ex}

\begin{ex}  $\ds x\sqrt{169-x^2}$
\begin{sol} 
$\ds \sqrt{169-x^2}-x^2/\sqrt{169-x^2}$
\end{sol}
\end{ex}

\begin{ex}  $\ds (x^2-4x+5)\sqrt{25-x^2}$
\begin{sol} 
$\ds  (2x-4)\sqrt{25-x^2}-$\hfill\break$(x^2-4x+5)x/\sqrt{25-x^2}$
\end{sol}
\end{ex}

\begin{ex}  $\ds \sqrt{r^2-x^2}$, $r$ is a constant
\begin{sol} 
$\ds -x/\sqrt{r^2-x^2}$
\end{sol}
\end{ex}

\begin{ex}  $\ds \sqrt{1+x^4}$
\begin{sol} 
$\ds 2x^3/\sqrt{1+x^4}$
\end{sol}
\end{ex}

\begin{ex}  $\ds {1\over\sqrt{5-\sqrt{x}}}$.
\begin{sol} 
$\ds{1\over 4\sqrt{x}(5-\sqrt{x})^{3/2}}$
\end{sol}
\end{ex}

\begin{ex}  $\ds (1+3x)^2$
\begin{sol} 
$\ds  6+18x$
\end{sol}
\end{ex}

\begin{ex}  $\ds{(x^2+x+1)\over(1-x)}$
\begin{sol} 
$\ds {2 x + 1\over1 - x }+{x^2  + x + 1\over(1 - x)^2}$
\end{sol}
\end{ex}

\begin{ex}  $\ds{\sqrt{25-x^2}\over x}$
\begin{sol} 
$\ds  -1/\sqrt{25-x^2}-\sqrt{25-x^2}/x^2$
\end{sol}
\end{ex}

\begin{ex}  $\ds\sqrt{{169\over x}-x}$
\begin{sol} 
$\ds{1\over2}\left({-169\over x^2}-1\right)\Big/\sqrt{{169\over x}-x}$
\end{sol}
\end{ex}

\begin{ex}  $\ds \sqrt{x^3-x^2-(1/x)}$
\begin{sol} 
$ \ds{3x^2-2x+1/x^2\over 2\sqrt{x^3-x^2-(1/x)}}$
\end{sol}
\end{ex}

\begin{ex}  $\ds 100/(100-x^2)^{3/2}$
\begin{sol} 
$ \ds{300 x \over(100-x^2)^{5/2}}$
\end{sol}
\end{ex}

\begin{ex}  $\ds {\root 3 \of{x+x^3}}$
\begin{sol} 
$ \ds{ 1 + 3 x^2\over3(x+x^3)^{2/3}}$
\end{sol}
\end{ex}

\begin{ex}  $\ds \sqrt{(x^2+1)^2+\sqrt{1+(x^2+1)^2}}$
\begin{sol} 
$\ds \left(4x(x^2+1)+{4x^3+4x\over2\sqrt{1+(x^2+1)^2}}\right)\Big/$\hfill\break$2\sqrt{(x^2+1)^2+\sqrt{1+(x^2+1)^2}}$
\end{sol}
\end{ex}

\begin{ex}  $\ds (x+8)^5$
\begin{sol} 
$\ds 5(x+8)^4$
\end{sol}
\end{ex}

\begin{ex}  $\ds (4-x)^3$
\begin{sol} 
$\ds -3(4-x)^2$
\end{sol}
\end{ex}

\begin{ex}  $\ds (x^2+5)^3$
\begin{sol} 
$\ds 6x(x^2+5)^2$
\end{sol}
\end{ex}

\begin{ex}  $\ds (6-2x^2)^3$
\begin{sol} 
$\ds -12x(6-2x^2)^2$
\end{sol}
\end{ex}

\begin{ex}  $\ds (1-4x^3)^{-2}$
\begin{sol} 
$\ds 24x^2(1-4x^3)^{-3}$
\end{sol}
\end{ex}

\begin{ex}  $\ds 5(x+1-1/x)$
\begin{sol} 
$\ds 5+5/x^2$
\end{sol}
\end{ex}

\begin{ex}  $\ds 4(2x^2-x+3)^{-2}$
\begin{sol} 
$\ds -8(4x-1)(2x^2-x+3)^{-3}$
\end{sol}
\end{ex}

\begin{ex}  $\ds {1\over 1+1/x}$
\begin{sol} 
$\ds 1/(x+1)^2$
\end{sol}
\end{ex}

\begin{ex}  $\ds {-3\over 4x^2-2x+1}$
\begin{sol} 
$\ds 3(8x-2)/(4x^2-2x+1)^2$
\end{sol}
\end{ex}

\begin{ex}  $\ds (x^2+1)(5-2x)/2$
\begin{sol} 
$\ds -3x^2+5x-1$
\end{sol}
\end{ex}

\begin{ex}  $\ds (3x^2+1)(2x-4)^3$
\begin{sol} 
$\ds 6x(2x-4)^3+6(3x^2+1)(2x-4)^2$
\end{sol}
\end{ex}

\begin{ex}  $\ds{x+1\over x-1}$
\begin{sol} 
$\ds -2/(x-1)^2$
\end{sol}
\end{ex}

\begin{ex}  $\ds{x^2-1\over x^2+1}$
\begin{sol} 
$\ds 4x/(x^2+1)^2$
\end{sol}
\end{ex}

\begin{ex}  $\ds{(x-1)(x-2)\over x-3}$
\begin{sol} 
$\ds (x^2-6x+7)/(x-3)^2$
\end{sol}
\end{ex}

\begin{ex}  $\ds{2x^{-1}-x^{-2}\over 3x^{-1}-4x^{-2}}$
\begin{sol} 
$\ds -5/(3x-4)^2$
\end{sol}
\end{ex}

\begin{ex}  $\ds 3(x^2+1)(2x^2-1)(2x+3)$
\begin{sol} 
$\ds 60x^4+72x^3+18x^2+18x-6$
\end{sol}
\end{ex}

\begin{ex}  $\ds{1\over (2x+1)(x-3)}$
\begin{sol} 
$\ds (5-4x)/((2x+1)^2(x-3)^2)$
\end{sol}
\end{ex}

\begin{ex}  $\ds ((2x+1)^{-1}+3)^{-1}$
\begin{sol} 
$\ds 1/(2(2+3x)^2)$
\end{sol}
\end{ex}

\begin{ex}  $\ds (2x+1)^3(x^2+1)^2$
\begin{sol} 
$\ds 56x^6+72x^5+110x^4+100x^3+60x^2+28x+6$
\end{sol}
\end{ex}

\begin{ex}   Find an equation for the tangent line to 
$\ds f(x) = (x-2)^{1/3}/(x^3 + 4x - 1)^2$ at $x=1$.
\begin{sol} 
$y=23x/96-29/96$
\end{sol}
\end{ex}

\begin{ex}  Find an equation for the tangent line to $\ds y=9x^{-2}$ at $(3,1)$.
\begin{sol} 
$y=3-2x/3$
\end{sol}
\end{ex}

\begin{ex}  Find an equation for the tangent line to $\ds (x^2-4x+5)\sqrt{25-x^2}$ 
at $(3,8)$.
\begin{sol} 
$y=13x/2-23/2$
\end{sol}
\end{ex}

\begin{ex}  Find an equation for the tangent line to $\ds \ds{(x^2+x+1)\over(1-x)}$ 
at $(2,-7)$.
\begin{sol} 
$y=2x-11$
\end{sol}
\end{ex}

\begin{ex}  Find an equation for the tangent line to 
$\ds \sqrt{(x^2+1)^2+\sqrt{1+(x^2+1)^2}}$
at $\ds (1,\sqrt{4+\sqrt{5}})$.
\begin{sol} 
$\ds y={20+2\sqrt5\over5\sqrt{4+\sqrt5}}\,x+{3\sqrt5\over5\sqrt{4+\sqrt5}}$
\end{sol}
\end{ex}

\begin{ex}
Let $y=f(x)$ and $x=g(t)$. If $g(1)=2$, $f(2)=3$, $g'(1)=4$ and $f'(2)=5$, find the derivative of $f\circ g$ at 1.
\begin{sol}
	$(f(g(1)))'=20$
\end{sol}
\end{ex}

\begin{ex}
Express the derivative of $g(x)=x^2f(x^2)$ in terms of $f$ and the derivative of $f$.
\begin{sol}
	$g'(x)=2x(f(x^2)+x^2f'(x^2))$
\end{sol}
\end{ex}

\end{enumialphparenastyle}