\subsection{Distance between Two Points and Midpoints}\label{sec:DistanceAndMidpoints}

Another important concept in Geometry is the notion of length.  If we are going to unite Algebra and Geometry using the Cartesian Plane, then we need to develop an algebraic understanding of what distance in the plane means.\\

Given two points $(x_1,y_1)$ and $(x_2,y_2)$, recall that their
horizontal distance from one another is $\Delta x=x_2-x_1$ and their
vertical distance from one another is $\Delta y=y_2-y_1$. Actually,
the word ``distance'' normally denotes ``positive distance''. $\Delta
x$ and $\Delta y$ are {\it signed\/} distances, but this is clear from
context. The (positive) distance from one point to the other
is the length of the hypotenuse of a right triangle with legs $|\Delta
x|$ and $|\Delta y|$, as shown in Figure~\ref{fig:distance between
points}.  The Pythagorean Theorem states that the distance between
the two points is the square root of the sum of the squares of the
horizontal and vertical sides:\\
$$| \, \Delta x \,|^{2} + | \, \Delta y \, |^{2} =  d^{2}$$ \\



\figure[!ht]
\centerline{\vbox{\beginpicture
\normalgraphs
%\ninepoint
\setcoordinatesystem units <0.5truein,0.5truein>
\setplotarea x from 0 to 3, y from 0 to 2
\putrule from 0 0 to 3 0
\putrule from 3 0 to 3 2
\plot 0 0 3 2 /
\put {$(x_1,y_1)$} [r] <-5pt,0pt> at 0 0
\put {$(x_2,y_2)$} [l] <5pt,0pt> at 3 2
\put {$\Delta x$} [t] <0pt,-5pt> at 1.5 0
\put {$\Delta y$} [l] <5pt,0pt> at 3 1
\endpicture}}
\caption{Distance between two points (here, $\Delta x$ and $\Delta y$ are positive). \label{fig:distance between points}}
\endfigure

\begin{formulabox}[Distance Formula]
The distance between points $(x_1,y_1)$ and $(x_2,y_2)$ is
$$\hbox{distance}=\sqrt{(\Delta x)^2+(\Delta y)^2}=\sqrt{(x_2-x_1)^2+ (y_2-y_1)^2}$$
\end{formulabox}

\bigskip

\begin{example}{Distance Between Two Points}{DistanceExample}
Find and simplify the distance between $P(-2,3)$ and $Q(1,-3)$.
\end{example}
\begin{solution}	
The distance, $d$, between points $P(-2,3)$ and $Q(1,-3)$ is \\
$$\begin{array}{rcl}
d & = & \sqrt{ (x_2-x_1)^2+ (y_2-y_1)^2}\\
 & = & \sqrt{(1-(-2))^2+(-3-(3))^2} \\
 & = & \sqrt{9+36} \\
 & = & 3\sqrt{5} \\
 \end{array}$$
 So the distance  is $3\sqrt{5}$.
\end{solution}

\begin{example}{Point on a Line}
	FFind all of the points with $x$-coordinate $1$ which are $4$ units from the point $(3,2)$. 
\end{example}
\begin{solution}
	We shall soon see that the points we wish to find are on the line $x=1$, but for now we'll just view them as points of the form $(1,y)$.  Visually,\\
	$$ \includegraphics[scale=0.4]{images/distance-ex}$$
	We require that the distance from $(3,2)$ to $(1,y)$ be $4$.  The Distance Formula yields
	
	\[ \begin{array}{rclr} 
	d &  = & \sqrt{\left(x_{\mbox{\tiny$2$}}-x_{\mbox{\tiny$1$}}\right)^2+\left(y_{\mbox{\tiny$2$}}-y_{\mbox{\tiny$1$}}\right)^2}  & \\
	4 &  = & \sqrt{(1-3)^2+(y-2)^2} & \\
	4  & = & \sqrt{4+(y-2)^2} & \\ 
	4^2 & = & \left(\sqrt{4+(y-2)^2}\right)^2 &  \mbox{squaring both sides} \\
	16 & = & 4+(y-2)^2 & \\
	12 & = & (y-2)^2 & \\
	(y-2)^2 & = & 12 &  \\
	y - 2 & = & \pm \sqrt{12} & \mbox{extracting the square root} \\
	y-2 & = & \pm 2 \sqrt{3} & \\
	y & = & 2 \pm 2 \sqrt{3}  & 
	\end{array} \]
	We obtain two points:  $(1, 2 + 2 \sqrt{3})$ and $(1, 2-2 \sqrt{3}).$  
	
\end{solution}		

As a special case of the distance formula, suppose we want to know the
distance of a point $(x,y)$ to the origin.  According to the distance
formula, this is $$\sqrt{(x-0)^2+(y-0)^2}=\sqrt{x^2+y^2}.$$
A point $(x,y)$ is at a distance $r$ from the origin if and only if
$\sqrt{x^2+y^2}=r$, or, if we square both sides: $x^2+y^2=r^2$.  As illustrated
in the next section, this is the equation of the circle of radius, $r$, centered at the origin.

Furthermore, given two points we can determine the \dfont{midpoint} of the line segment joining the two points.\\

\begin{formulabox}[Midpoint Formula]
The midpoint of the line segment joining two points $(x_1,y_1)$ and 
$(x_2,y_2)$ is the point with coordinates:
$$\hbox{midpoint}=\left(\frac{x_1+x_2}{2},\frac{y_1+y_2}{2}\right).$$
\end{formulabox}

\bigskip

\begin{example}{Midpoint of a Line Segment}{Midpoint}
Find the midpoint of the line segment joining the given points: $(1,0)$ and $(5,-2)$.
\end{example}

\begin{solution} 
Using the \ifont{midpoint formula} on $(x_1,y_1)=(1,0)$ and $(x_2,y_2)=(5,-2)$ we get:
$$\left(\frac{1+5}{2},\frac{0+(-2)}{2}\right)=(3,-1)$$
Thus, the midpoint of the line segment occurs at $(3,-1)$.
\end{solution}


We close with a more abstract application of the Midpoint Formula. 

\begin{example}{Distance between points}
IIf $a \neq b$, prove that the line $y = x$ equally divides the line segment with endpoints $(a,b)$ and $(b,a)$.
\end{example}	
\begin{solution}	
To prove the claim, we use the Midpoint formula to find the midpoint  
	
	\[ \begin{array}{rcl}
	
	M & = & \left( \dfrac{a+b}{2},  \dfrac{b+a}{2} \right) \\
	& = & \left( \dfrac{a+b}{2},  \dfrac{a+b}{2} \right)  \\ \end{array} \]
	
	Since the $x$ and $y$ coordinates of this point are the same, we find that the midpoint lies on the line $y=x$, as required. \
\end{solution}
	


