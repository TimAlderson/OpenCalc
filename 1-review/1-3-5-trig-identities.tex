\subsection{Trigonometric Identities}
There are numerous trigonometric identities, including those relating to shift/periodicity, Pythagoras type identities, double-angle formulas, half-angle formulas and addition formulas.
We list these below.

\begin{enumerate}
\item[1.] \dfont{Shifts and periodicity}
$$\begin{array}{|rcl|rcl|rcl|}
\hline
~&~&~&~&~&~&~&~&~\\
\ds{\sin (\theta + 2\pi)} & = & \ds{\sin \theta} &
\ds{\cos (\theta + 2\pi)} &= & \ds{\cos \theta} &
\ds{\tan (\theta + 2\pi)} &= & \ds{\tan \theta} \\
~&~&~&~&~&~&~&~&~\\
\hline
~&~&~&~&~&~&~&~&~\\
\ds{\sin (\theta + \pi)} &= & \ds{-\sin \theta} &
\ds{\cos (\theta + \pi)} &= & \ds{-\cos \theta} &
\ds{\tan (\theta + \pi)} &= & \ds{\tan \theta} \\
~&~&~&~&~&~&~&~&~\\
\hline
~&~&~&~&~&~&~&~&~\\
\ds{\sin (-\theta)} &= & \ds{-\sin \theta} & 
\ds{\cos (-\theta)} &= & \ds{\cos \theta} & 
\ds{\tan (-\theta)} &= & \ds{-\tan \theta}  \\
~&~&~&~&~&~&~&~&~\\
\hline
~&~&~&~&~&~&~&~&~\\
\ds{\sin \left(\theta +\frac{\pi}{2}\right) } &= &  \ds{\cos \theta} &
\ds{\cos \left(\theta -\frac{\pi}{2}\right)} &= &  \ds{\sin \theta} &
\ds{\tan \left(\frac{\pi}{2}-\theta\right)} &= & \ds{\cot \theta}  \\
~&~&~&~&~&~&~&~&~\\
\hline
\end{array}$$
\end{enumerate}

\begin{multicols}{2}
\begin{enumerate}
\item[2.] \dfont{Pythagoras type formulas}

$ \begin{array}{|rcl|} \hline	
&& \\
\ds{\sin ^2 \theta + \cos ^2 \theta} & = & \ds{1}\\
&& \\
\ds{\tan^2 \theta + 1} & = & \ds{\sec^2 \theta} \\
&& \\
\ds{1 + \cot ^2 \theta}  & = & \ds{\csc^2 \theta} \\
&& \\ \hline
\end{array} $

\vspace{5mm} 
\item[3.] \dfont{Addition and subtraction formulas}

$\begin{array}{|rcl|} \hline
&& \\	
\ds{\sin(\theta+\phi)} & = & \ds{\sin\theta\cos\phi + \cos\theta\sin\phi} \\
&& \\
\ds{\cos(\theta+\phi)} & = & \ds{\cos\theta\cos\phi - \sin\theta\sin\phi} \\
&& \\
\ds{\tan(\theta+\phi)} & = & \ds{\frac{\tan\theta + \tan\phi}{1 - \tan\theta\tan\phi}}\\
&& \\
\ds{\sin(\theta-\phi)} & = & \ds{\sin\theta\cos\phi - \cos\theta\sin\phi} \\
&& \\
\ds{\cos(\theta-\phi)} & = & \ds{\cos\theta\cos\phi + \sin\theta\sin\phi} \\
&& \\ \hline 
\end{array} $


\item[4.] \dfont{Double-angle formulas}

$\begin{array}{|rcl|} \hline
&& \\	
\ds{\sin(2\theta)} & =  & \ds{2\sin\theta\cos\theta} \\
&& \\
\ds{\cos(2\theta)} & = & \ds{\cos^2\theta - \sin^2\theta}\\
& =  & \ds{2\cos^2\theta - 1} \\
& = & \ds{1-2\sin^2\theta}\\
&& \\ \hline 
\end{array}$

\vspace{4mm} 
\item[5.] \dfont{Half-angle formulas}

$ \begin{array}{|rcl|} \hline 
&& \\
\ds{\cos^2\theta} & = & \ds{\frac{1+\cos(2\theta)}{2}} \\
&& \\
\ds{\sin^2\theta} & = & \ds{\frac{1-\cos(2\theta)}{2}} \\
&& \\ \hline 
\end{array} $

\end{enumerate}
\end{multicols}

\begin{example}{Double Angle}{DoubleAngle}
Find all values of $x$ with $0\leq x\leq \pi$ such that $\sin 2x=\sin x$.
\end{example}

\begin{solution} 
Using the double-angle formula $\sin 2x = 2\sin x \cos x$ we have:

\hspace{3cm} $$\begin{array}{rcl}
\ds{2\sin x\cos x} & = & \ds{\sin x} \\
\ds{2\sin x\cos x- \sin x} & = & \ds{0} \\
\ds{\sin x \, (2\cos x - 1)} & = & \ds{0} \\
\end{array}$$

Thus, either \hspace{1cm} $\,\, \sin x=0 \,\, $ or $\,\, \cos x = 1/2\,\, $. \\

For the first case when $\,\, \sin x = 0 \,\, $, we get $\, x=0 \, $ or $\, x=\pi \,$.
For the second case when $\,\,\cos x = 1/2 \,\,$, we get $\, x=\pi/3 \,$ (use the special triangles and CAST rule to get this).
Thus, we have three solutions: $x=0, \, \pi/3, \, \pi$.
\end{solution}