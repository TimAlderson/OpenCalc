%%%%%%%%%%%%%%%%%%%%%%%%%%%%%%%%%%%%%%%%%%%%%%%%%%%%%%%%%%%%%%%%%%%%%%%
\subsection{Sets and Number Systems}

A \dfont{set} can be thought of as any collection of \ifont{distinct} objects 
considered as a whole. Typically, sets are represented using 
\dfont{set-builder notation} and are surrounded by braces.
Recall that $(,)$ are called \dfont{parentheses} or \dfont{round brackets};
$[,]$ are called \dfont{square brackets}; and $\{,\}$ are called \dfont{braces} or \dfont{curly brackets}. \\

\begin{example}{Sets}{Sets}
The collection $\{a,b,1,2\}$ is a set. It consists of the collection of
four distinct objects, namely, $a$, $b$, $1$ and $2$.
\end{example}

Let $S$ be any set. We use the notation $\dfont{x\in S}$ to mean that $x$ 
is an element \ifont{inside} of the set $S$, and the notation $\dfont{x\not\in S}$ to mean that $x$ is \ifont{not} an element of the set $S$. \\

\begin{example}{Set Membership}{SetMembership}
If $S=\{a,b,c\}$, then $a\in S$ but $d\not\in S$.
\end{example}

The \dfont{intersection} between two sets $S$ and $T$ is denoted 
by $S\cap T$ and is the collection of all elements that belong to 
\ifont{both} $S$ and $T$. The \dfont{union} between two sets $S$ 
and $T$ is denoted by $S\cup T$ and is the collection of all elements 
that belong to \ifont{either} $S$ or $T$ (or both). \\

\begin{example}{Union and Intersection}{UnionIntersection}
Let $S=\{a,b,c\}$ and $T=\{b,d\}$.
Then $S\cap T=\{b\}$ and $S\cup T=\{a,b,c,d\}$. 
Note that we do \ifont{not} write the element $b$ twice in $S\cup T$ even 
though $b$ is in both $S$ and $T$.
\end{example}

Numbers can be classified into sets called \dfont{number systems}.

\begin{center}
  \begin{tabular}{| c || c | c |}
    \hline
    $\mathbb{N}$ & the \dfont{natural} numbers 
    		& $\{1, 2, 3,\ldots\}$ \\ \hline
    $\mathbb{Z}$ & the \dfont{integers} 
    		& $\{\dots,-3,-2, -1, 0, 1, 2, 3,\dots\}$ \\ \hline
    $\mathbb{Q}$ & the \dfont{rational} numbers 
    		& Ratios of integers: $\left\{\frac{p}{q}\,:\,p,q\in\mathbb{Z},q\not=0\right\}$ \\ \hline
    $\mathbb{R}$ & the \dfont{real} numbers 
    		& Can be written using a finite or infinite \ifont{decimal expansion} \\ \hline
    $\mathbb{C}$ & the \dfont{complex} numbers 
    		& These allow us to solve equations such as $x^2+1=0$ \\
    \hline
  \end{tabular}
\end{center}

In the table, the set of rational numbers is written using
set-builder notation. The colon, $:$, used in this manner means \ifont{such that}.
Often times, a vertical bar $|$ may also be used to mean \ifont{such that}.
The expression $\left\{\frac{p}{q}\,:\,p,q\in\mathbb{Z},q\not=0\right\}$ can be read out loud as
\ifont{the set of all fractions $p$ over $q$ such that $p$ and $q$ are both integers and $q$ is not equal to zero}. \\

\begin{example}{Rational Numbers}{RationalNumbers}
The numbers $-\frac{3}{4}$, $2.647$, $17$, $0.\bar{7}$ are all rational numbers. 
You can think of rational numbers as \ifont{fractions} of one integer over another. 
Note that 2.647 can be written as a fraction: 
\[ 2.647=2.647\times\frac{1000}{1000}=\frac{2647}{1000}. \]
Also note that in the expression $0.\bar{7}$, the bar over the $7$ indicates 
that the $7$ is repeated forever: 
\[ 0.77777777\ldots=\frac{7}{9}.\]
\vspace{-0.5cm}
\end{example} 

All rational numbers are real numbers with the property that
their decimal expansion either \ifont{terminates} after a finite number 
of digits or begins to \ifont{repeat} the same finite sequence of digits over and over.
Real numbers that are not rational are called \dfont{irrational}. \\

\begin{example}{Irrational Numbers}{IrrationalNumbers}
Some of the most common irrational numbers include:
\begin{itemize}
	\item $\sqrt 2$.\quad 
			Can you prove this is irrational? (The proof uses a technique called \ifont{contradiction}.)
	\item $\pi$.\quad 
			Recall that $\pi$ (\dfont{pi}) is defined as the ratio of the circumference of a circle to its diameter and can be approximated by $3.14159265$.
	\item $e$.\quad 
			Sometimes called Euler's number, $e$ can be approximated by $2.718281828459$. 
			We will review the definition of $e$ in a later chapter.
\end{itemize}
\end{example}

Let $S$ and $T$ be two sets. If every element of $S$ is also an element of $T$, then
we say $S$ is a \dfont{subset} of $T$ and write $S\subseteq T$. Furthermore, if $S$ is
a subset of $T$ but not equal to $T$, we often write $S\subset T$.
The five sets of numbers in the table give an increasing sequence of sets:
\[ \mathbb{N} \subset \mathbb{Z} \subset \mathbb{Q} \subset \mathbb{R} \subset \mathbb{C}. \]

That is, all natural numbers are also integers, all integers are also rational numbers, all rational numbers are also real numbers, and all real numbers are also complex numbers.



