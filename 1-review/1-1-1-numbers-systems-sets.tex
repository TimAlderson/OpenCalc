%%%%%%%%%%%%%%%%%%%%%%%%%%%%%%%%%%%%%%%%%%%%%%%%%%%%%%%%%%%%%%%%%%%%%%%
\subsection{Sets and Number Systems}

A \dfont{set} can be thought of as any collection of \ifont{distinct} objects, called {\bf{elements}} of the set.\\

In general, there are three ways to describe sets.  

	
\begin{formulabox}[Ways to Describe Sets]
	
	\begin{enumerate}
		
		\item \textbf{Verbally:} Use a sentence to define a set.\index{set ! verbal description}
		
		\item \textbf{Roster Notation:}  Begin with a left brace `$\{$', list each element of the set \textit{only once} and then end with a right brace `$\}$'.\index{set ! roster method}
		
		\item \textbf{Set-Builder Notation:} A combination of verbal and roster notation using a ``dummy variable'' such as $x$.\index{set ! set-builder notation}\index{set-builder notation}
	\end{enumerate}
\end{formulabox} 	


\begin{example}{Roster Notation for Sets}{Sets}
	The collection $\{a,b,1,2\}$ is a set. It consists of the collection of
	four distinct objects, namely, $a$, $b$, $1$ and $2$. The order of the elements doesn't matter, so the same set could be described as $\{2,a,1,b\}$. 
\end{example}

Typically, sets are represented using 
\dfont{set-builder notation} and are surrounded by braces.
Recall that $(,)$ are called \dfont{parentheses} or \dfont{round brackets};
$[,]$ are called \dfont{square brackets}; and $\{,\}$ are called \dfont{braces} or \dfont{curly brackets}. \\

\begin{example}{Set-Builder Notation for Sets}{SetBuilder}
The expression $\displaystyle{ \{x \, \ssep  x \geq 0 \} }$ can be read as the set of all elements $x$ such that $x$ is greater than or equal to zero.  The vertical bar $\ssep$, which can also be written as a colon $:$, is a separator that can be read as "such that", "for which", or "with the property that". 
\end{example}	

Let $S$ be any set. We use the notation $\dfont{x\in S}$ to mean that $x$ 
is an element \ifont{inside} of the set $S$, and the notation $\dfont{x\not\in S}$ to mean that $x$ is \ifont{not} an element of the set $S$. \\


\begin{example}{Set Membership}{SetMembership}
If $S=\{a,b,c\}$, then $a\in S$ but $d\not\in S$.
\end{example}

%The \dfont{intersection} between two sets $S$ and $T$ is denoted 
%by $S\cap T$ and is the collection of all elements that belong to 
%\ifont{both} $S$ and $T$. The \dfont{union} between two sets $S$ 
%and $T$ is denoted by $S\cup T$ and is the collection of all elements 
%that belong to \ifont{either} $S$ or $T$ (or both). \\
%
%\begin{example}{Union and Intersection}{UnionIntersection}
%Let $S=\{a,b,c\}$ and $T=\{b,d\}$.
%Then $S\cap T=\{b\}$ and $S\cup T=\{a,b,c,d\}$. 
%Note that we do \ifont{not} write the element $b$ twice in $S\cup T$ even 
%though $b$ is in both $S$ and $T$.
%\end{example}

Numbers can be classified into sets called \dfont{number systems}.


\begin{formulabox}[Sets of Numbers]

\begin{enumerate}
	
	\item The \textbf{Empty Set}:\index{set ! empty}\index{empty set} $\emptyset=\{ \}=\{x\,|\,\mbox{$x
		\neq x$}\}$.  This is the set with no elements.  Like the number `$0$,' it  plays a vital role in mathematics.
	
	\item The \textbf{Natural Numbers}:\index{natural number ! set of}\index{natural number ! definition of} $\mathbb N= \{ 1, 2, 3,  \ldots\}$ The periods of ellipsis here indicate that the natural numbers contain $1$, $2$, $3$, `and so forth'.
	
	\item The \textbf{Whole Numbers}:\index{whole number ! set of}\index{whole number ! definition of} $\mathbb W = \{ 0, 1, 2, \ldots \}$
	
	\item The \textbf{Integers}:\index{integer ! set of}\index{integer ! definition of} $\mathbb Z=\{ \ldots, -1, -2, -1, 0, 1, 2, 3, \ldots \}$
	
	\item The \textbf{Rational Numbers}:\index{rational number ! set of}\index{rational number ! definition of} $\mathbb Q=\left\{\frac{a}{b} \, | \, a \in \mathbb Z \, \mbox{and} \, b \in \mathbb Z \right\}$.  \underline{Ratio}nal numbers are the \underline{ratio}s of integers (provided the denominator is not zero!)  It turns out that another way to describe the rational numbers is: \[\mathbb Q=\{x\,|\,\mbox{$x$ possesses a repeating or terminating decimal representation.}\}\]
	
	\item The \textbf{Real Numbers}:\index{real number ! set of}\index{real number ! definition of} $\mathbb R = \{ x\,|\,\mbox{$x$ possesses a decimal representation.}\}$
	
	\item The \textbf{Irrational Numbers}:\index{irrational number ! set of}\index{irrational number ! definition of} $\mathbb P = \{x\,|\,\mbox{$x$ is a non-rational real number.}\}$  Said another way, an \underline{ir}rational number is a decimal which neither repeats nor terminates.
	
	\item The \textbf{Complex Numbers}:\index{complex number ! set of}\index{complex number ! definition of} $\mathbb C=\{a+bi\,|\,\mbox{$a$,$b \in \mathbb R$ and $i=\sqrt{-1}$}\}$  Despite their importance, the complex numbers play only a minor role in the text.
	
\end{enumerate}
\end{formulabox}

It is important to note that every natural number is a whole number, which, in turn, is an integer.   Each integer is a rational number (take $b =1$ in the above definition for $\mathbb Q$) and the rational numbers are all real numbers, since they possess decimal representations.   If we take $b=0$ in the above definition of $\mathbb C$, we see that every real number is a complex number.  In this sense, the sets $\mathbb N$, $\mathbb W$, $\mathbb Z$, $\mathbb Q$, $\mathbb R$, and $\mathbb C$ are nested.



%\begin{center}
%  \begin{tabular}{| c || c | c |}
%    \hline
%    $\mathbb{N}$ & the \dfont{natural} numbers 
%    		& $\{1, 2, 3,\ldots\}$ \\ \hline
%    $\mathbb{Z}$ & the \dfont{integers} 
%    		& $\{\dots,-3,-2, -1, 0, 1, 2, 3,\dots\}$ \\ \hline
%    $\mathbb{Q}$ & the \dfont{rational} numbers 
%    		& Ratios of integers: $\left\{\frac{p}{q}\, \ssep\,p,q\in\mathbb{Z},q\not=0\right\}$ \\ \hline
%    $\mathbb{R}$ & the \dfont{real} numbers 
%    		& Can be written using a finite or infinite \ifont{decimal expansion} \\ \hline
%    $\mathbb{C}$ & the \dfont{complex} numbers 
%    		& These allow us to solve equations such as $x^2+1=0$ \\
%    \hline
%  \end{tabular}
%\end{center}

%In the table, the set of rational numbers is written using
%set-builder notation. The vertical bar $\ssep$, which can also be written as a colon $:$, is a separator that can be read as "such that", "for which", or "with the property that". 
%The expression $\left\{\frac{p}{q}\, \ssep\,p,q\in\mathbb{Z},q\not=0\right\}$ can be read out loud as
%\ifont{the set of all fractions $p$ over $q$ such that $p$ and $q$ are both integers and $q$ is not equal to zero}. 
%\\

\begin{example}{Rational Numbers}{RationalNumbers}
The numbers $-\frac{3}{4}$, $2.647$, $17$, $0.\bar{7}$ are all rational numbers. 
You can think of rational numbers as \ifont{fractions} of one integer over another. 
Note that 2.647 can be written as a fraction: 
\[ 2.647=2.647\times\frac{1000}{1000}=\frac{2647}{1000} \]
Also note that in the expression $0.\bar{7}$, the bar over the $7$ indicates 
that the $7$ is repeated forever: 
\[ 0.77777777\ldots=\frac{7}{9}\]
\vspace{-0.5cm}
\end{example} 

All rational numbers are real numbers with the property that
their decimal expansion either \ifont{terminates} after a finite number 
of digits or begins to \ifont{repeat} the same finite sequence of digits over and over.
Real numbers that are not rational are called \dfont{irrational}. \\



\begin{example}{Irrational Numbers}{IrrationalNumbers}
Some of the most common irrational numbers include:
\begin{itemize}
	\item $\sqrt 2$.\quad 
			Can you prove this is irrational? (The proof uses a technique called \ifont{contradiction}.)
	\item $\pi$.\quad 
			Recall that $\pi$ (\dfont{pi}) is defined as the ratio of the circumference of a circle to its diameter and can be approximated by $3.14159265$.
	\item $e$.\quad 
			Sometimes called Euler's number, $e$ can be approximated by $2.718281828459$. 
			We will review the definition of $e$ in a later chapter.
\end{itemize}
\end{example}

Let $S$ and $T$ be two sets. If every element of $S$ is also an element of $T$, then
we say $S$ is a \dfont{subset} of $T$ and write $S\subseteq T$. Furthermore, if $S$ is
a subset of $T$ but not equal to $T$, we often write $S\subset T$.
The five sets of numbers in the table give an increasing sequence of sets:
\[ \mathbb{N} \subset \mathbb{Z} \subset \mathbb{Q} \subset \mathbb{R} \subset \mathbb{C}. \]

That is, all natural numbers are also integers, all integers are also rational numbers, all rational numbers are also real numbers, and all real numbers are also complex numbers.\\

For the most part, this textbook focuses on sets whose elements come from the real numbers $\mathbb R$.  Recall that we may visualize $\mathbb R$ as a line. Segments of this line are called \textbf{intervals}\index{interval ! definition of} of numbers. Below is a summary of the so-called \textbf{interval notation}\index{interval ! notation for} associated with given sets of numbers.  For intervals with finite endpoints, we list the left endpoint, then the right endpoint.  We use square brackets, `$[$' or `$]$', if the endpoint is included in the interval and use a filled-in or `closed' dot to indicate membership in the interval. Otherwise, we use parentheses, `$($' or `$)$' and an `open' circle to indicate that the endpoint is not part of the set.  If the interval does not have finite endpoints, we use the symbols $-\infty$ to indicate that the interval extends indefinitely to the left and $\infty$ to indicate that the interval extends indefinitely to the right.  Since infinity is a concept, and not a number, we always use parentheses when using these symbols in interval notation, and use an appropriate arrow to indicate that the interval extends indefinitely in one (or both) directions.\\



