\subsection{The Quadratic Formula and Completing the Square}
The technique of \dfont{completing the square} allows us to solve quadratic equations and also to determine the center of a circle/ellipse or the vertex of a parabola.

The main idea behind completing the square is to turn:
$$ ax^2 + bx + c$$
into
$$a(x - h)^2 + k.$$
One way to complete the square is to use the following formula:
$$ax^2+bx+c=a\left(x+\frac{b}{2a}\right)^2-\frac{b^2}{4a^2}+c.$$
But this formula is a bit complicated, so some students prefer following the steps outlined in the next example. \\

\begin{example}{Completing the Square}{CompletingSquare}
Solve $2x^2+12x-32=0$ by completing the square.
\end{example}

\begin{solution}
In this instance, we will \ifont{not} divide by $2$ first (usually you would) in order to demonstrate what you should do when the `$a$' value is not $1$.

\bigskip

\begin{tabular}{rl}
$2x^2+12x-32=0$ & Start with original equation.\\
\\
$2x^2+12x=32$ & Move the number over to the other side.\\
\\
$2(x^2+6x)=32$ & Factor out the $a$ from the $ax^2+bx$ expression.\\
\\
$6~~\to~~\frac{6}{2}=3~~\to~~3^2=\dfont{9}$ & Take the number in front of $x$, \\
	&  \dfont{divide by $2$}, \\
	&  then \dfont{square} it.\\
\\
$\ifont{2}(x^2+6x+\dfont{9})=32+\ifont{2}\cdot\dfont{9}$ & Add the result to both sides, \\
	&  taking $a=2$ into account.\\
\\
$2(x+3)^2=50$ & Factor the resulting perfect square trinomial.\\
\\
~ & \ifont{You have now completed the square!}\\
\\
$(x+3)^2=25~~\to~~x=2 \mbox{ or } x=-8$ & To solve for $x$, simply divide by $a=2$ \\
	& and take square roots.\\
\end{tabular}
\end{solution}

Suppose we want to solve for $x$ in the quadratic
equation $ax^2+bx+c=0$, where $a\neq 0$.
The solution(s) to this equation are given by the \dfont{quadratic formula}.\\

\begin{theorem}{The Quadratic Formula}{Quadratic Formula}
\label{QuadForm}	 
The solutions to $ax^2+bx+c=0$ (with $a\neq 0$) are $\ds{x=\frac{-b\pm\sqrt{b^2-4ac}}{2a}}$.
\end{theorem}

\begin{proof}
To prove the Quadratic Formula (Theorem \ref{QuadForm} ) we use the technique of \ifont{completing 
the square}. The general technique involves taking an expression of the 
form $x^2+rx$ and trying to find a number we can add so that we end up 
with a perfect square (that is, $(x+n)^2$). It turns out if you add $(r/2)^2$ 
then you can factor it as a perfect square.

For example, suppose we want to solve for $x$ in the equation $ax^2+bx+c=0$, where $a\neq 0$.
Then we can move $c$ to the other side and divide by $a$ (remember, $a\neq 0$ so we can divide by it) to get
$$x^2+\frac{b}{a}x=-\frac{c}{a}.$$
To write the left side as a perfect square we use what was mentioned previously.
We have $r=(b/a)$ in this case, so we must add $(r/2)^2=(b/2a)^2$ to both sides
$$x^2+\frac{b}{a}x+\left(\frac{b}{2a}\right)^2=-\frac{c}{a}+\left(\frac{b}{2a}\right)^2.$$
We know that the left side can be factored as a perfect square
$$\left(x+\frac{b}{2a}\right)^2=-\frac{c}{a}+\left(\frac{b}{2a}\right)^2.$$
The right side simplifies by using the exponent rules and finding a common denominator
$$\left(x+\frac{b}{2a}\right)^2=\frac{-4ac+b^2}{4a^2}.$$
Taking the square root we get
$$x+\frac{b}{2a}=\pm\sqrt{\frac{-4ac+b^2}{4a^2}},$$
which can be rearranged as
$$x=\frac{-b\pm\sqrt{b^2-4ac}}{2a}.$$
In essence, the quadratic formula is just completing the square.
\end{proof}