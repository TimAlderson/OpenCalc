\subsection{The Absolute Value}

\begin{definition}{Absolute Value}{Absolute Value}
	\label{def:AbsoluteValue}
The \dfont{absolute value} of a number $x$ is written as $|x|$ and represents 
the \ifont{distance} $x$ is from zero. Mathematically, we define it as follows:
$$|x|=\left\{\begin{array}{cl}
x, & \mbox{if $x\geq 0$,}\\
-x, & \mbox{if $x<0$.}\\
\end{array}\right.$$
\end{definition}

Thus, if $x$ is a negative real number, then $-x$ is a positive real number.
The absolute value does \ifont{not} just turn minuses into pluses.
That is, $|2x-1|\neq 2x+1$.
You should be familiar with the following properties.\\

\begin{formulabox}[Absolute Value Properties]
\begin{enumerate}
	\item $|x|\geq 0$.
	\item $|xy|=|x||y|$.
	\item $\displaystyle{\left|\frac{1}{x} \right|=\frac{1}{|x|} }$ \hspace{2mm} when $x\neq 0$.
	\item $|-x|=|x|$.
	\item $|x+y|\leq |x|+|y|$. This is called the \dfont{triangle inequality}.
	\item $\sqrt{x^2}=|x|$.
\end{enumerate}
\end{formulabox}

\begin{example}{$\sqrt{x^2}=|x|$}{sqrtabs}
Observe that $\sqrt{(-3)^2}$ gives an answer of $3$, not $-3$.
\end{example}

When solving inequalities with absolute values, the following are helpful. %(here $a>0$ is a positive number):
%\begin{itemize}
%	\item $|x|=a$ means $x=\pm a$.
%	\item $|x|\leq a$ means $x\geq -a$ \ifont{and} $x\leq a$  (that is, $-a\leq x\leq a$).
%	\item $|x|\geq a$ means $x\leq -a$ \ifont{or} $x\geq a$.
%\end{itemize}
%
%In the above, we do require $a>0$, otherwise we end up with false expressions.
%For example, $|x|=-1$ has no solutions because $|x|$ is always positive (or zero), 
%so it will never equal $-1$. On the other hand, $|x|=1$ has the solutions $x=\pm 1$.
 
\medskip
{\bf Case 1: $a>0$.}\vspace{-0.1cm}
\begin{itemize}\setlength{\itemsep}{0 in}
\item $|x|=a$ has solutions $x=\pm a$. 
\item $|x|\leq a$ means $x\geq -a$ {\bf and} $x\leq a$  (that is, $-a\leq x\leq a$). 
\item $|x|<a$ means $x< -a$ {\bf and} $x< a$  (that is, $-a< x< a$). 
\item $|x|\geq a$ means $x\leq -a$ {\bf or} $x\geq a$. 
\item $|x|> a$ means $x< -a$ {\bf or} $x> a$. 
\end{itemize}

{\bf Case 2: $a<0$.}\vspace{-0.1cm}
\begin{itemize}\setlength{\itemsep}{0 in}
\item $|x|=a$ has no solutions. 
\item Both $|x|\leq a$ and $|x|<a$ have no solutions. 
\item Both $|x|\geq a$ and $|x|>a$ have solution set $\{x|x\in\R\}$. 
\end{itemize}

{\bf Case 3: $a=0$.}\vspace{-0.1cm}
\begin{itemize}\setlength{\itemsep}{0 in}
\item $|x|=0$ has solution $x=0$. 
\item $|x|<0$ has no solutions. 
\item $|x|\leq 0$ has solution $x=0$. 
\item $|x|>0$ has solution set $\{x\in\R|x\neq 0\}$. 
\item $|x|\geq 0$ has solution set $\{x|x\in\R\}$.
\end{itemize}