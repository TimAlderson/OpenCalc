\section{Additional Exercises}\label{sec:MoreReviewExercises}

\Opensolutionfile{solutions}[ex]
%%%%%%%%%%%%%%%%%%%%%%%%%%%%%%%%%%%%%%%%%%%%

\begin{enumialphparenastyle}

These problems require a comprehensive knowledge of the skills reviewed in
this chapter. They are not in any particular order. A proficiency in these
skills will help you a long way as your learn the calculus material in the
following chapters.

%%%%%%%%%%
\begin{ex}
Rationalize the denominator for each of the
following expressions. That is, re-write the expression in such a way that
no square roots appear in the denominator. Also, simplify your answers if
possible.
\begin{enumerate}
	\item	$\dfrac{1}{\sqrt{2}}$
	\item	$\dfrac{3h}{\sqrt{x+h+1}-\sqrt{x+1}}$
\end{enumerate}
\begin{sol}
\begin{enumerate}
	\item	$\dfrac{\sqrt{2}}{2}$
	\item	$3(\sqrt{x+h+1}+\sqrt{x+1})$
\end{enumerate}
\end{sol}
\end{ex}

%%%%%%%%%%
\begin{ex}
Solve the following equations.
\begin{enumerate}
	\item	$2-5(x-3)=4-10x$
	\item	$2x^2-5x=3$
	\item	$x^2-x-3=0$
	\item	$x^2+x+3=0$
	\item	$\sqrt{x^2+9}=2x$
\end{enumerate}
\begin{sol}
\begin{enumerate}
	\item	$-13/5$
	\item	$-1/2,3$
	\item	$(1\pm\sqrt{13})/2$
	\item	No real solutions
	\item	$\sqrt{3}$
\end{enumerate}
\end{sol}
\end{ex}

%%%%%%%%%%
\begin{ex}
By means of counter-examples, show why it is wrong
to say that the following equations hold for all real numbers for which the
expressions are defined.
\begin{enumerate}
	\item	$(x-2)^2=x^2-2^2$
	\item	$\dfrac{1}{x+h}=\dfrac{1}{x}+\dfrac{1}{h}$
	\item	$\sqrt{x^2+y^2}=x+y$
\end{enumerate}
\begin{sol}
Counter-examples may vary.
\begin{enumerate}
	\item	$x=3$
	\item	$x=h=1$
	\item	$x=y=1$
\end{enumerate}
\end{sol}
\end{ex}

%%%%%%%%%%
\begin{ex}
Find an equation of the line passing through the
point $(-2,5)$ and parallel to the line $x+3y-2=0$.
\begin{sol}
	$x+3y-13=0$, or equivalents such as $y=-\frac{1}{3}x+\frac{13}{3}$
\end{sol}
\end{ex}

%%%%%%%%%%
\begin{ex}
Solve $\dfrac{x^2-1}{3x-1}\leq 1$.
\begin{sol}
	$(-\infty,0]\cup(\frac{1}{3},3]$
\end{sol}
\end{ex}

%%%%%%%%%%
\begin{ex}
Explain why the following expression never
represents a real number (for any real number $x$): $\sqrt{x-2}+\sqrt{1-x}$.
\begin{sol}
	It is impossible for both $x-2$ and $1-x$ to be non-negative
	for the same real number $x$.
\end{sol}
\end{ex}

%%%%%%%%%%
\begin{ex}
Simplify the expression $\dfrac{\left[3(x+h)^2+4\right]-\left[3x^2+4\right]}{h}$ as much as possible.
\begin{sol}
	$6x+3h$
\end{sol}
\end{ex}

%%%%%%%%%%
\begin{ex}
Simplify the expression $\dfrac{\frac{x+h}{2(x+h)-1}-\frac{x}{2x-1}}{h}$ as much as possible.
\begin{sol}
	$-1/\left[(2x+2h-1)(2x-1)\right]$
\end{sol}
\end{ex}

%%%%%%%%%%
\begin{ex}
Simplify the expression $-\sin x(\cos x+3\sin x)-\cos x(-\sin x+3\cos x)$.
\begin{sol}
	$-3$
\end{sol}
\end{ex}

%%%%%%%%%%
\begin{ex}
Solve the equation $\cos x=\frac{\sqrt{3}}{2}$ on
the interval $0\leq x\leq 2\pi$.
\begin{sol}
	$\pi /6,$ $5\pi /6$
\end{sol}
\end{ex}

%%%%%%%%%%
\begin{ex}
Find an angle $\theta$ such that $0\leq\theta\leq\pi$ and $\cos\theta=\cos\frac{38\pi}{5}$.
\begin{sol}
	$2\pi/5$
\end{sol}
\end{ex}

%%%%%%%%%%
\begin{ex}
What can you say about $\dfrac{\left\vert x\right\vert+\left\vert 4-x\right\vert}{x-2}$ when $x$ is a large
(positive) number?
\begin{sol}
	It is equal to 2 for all $x$ larger than 4.
\end{sol}
\end{ex}

%%%%%%%%%%
\begin{ex}
Find an equation of the circle with centre in $(-2,3)$ and passing through the point $(1,-1)$.
\begin{sol}
	$(x+2)^2+(y-3)^2=25$.
\end{sol}
\end{ex}

%%%%%%%%%%
\begin{ex}
Find the centre and radius of the circle described
by $x^2+y^2+6x-4y+12=3$.
\begin{sol}
	Centre is $(-3,2)$ and radius is 2.
\end{sol}
\end{ex}

%%%%%%%%%%
\begin{ex}
If $y=9x^2+6x+7$, find all possible values of $y$.
\begin{sol}
	$y$ could be any real number greater than or equal to 6.
\end{sol}
\end{ex}

%%%%%%%%%%
\begin{ex}
Simplify $\left(\dfrac{3x^2 y^3 z^{-1}}{18x^{-1}yz^3}\right)^2$.
\begin{sol}
	$x^6 y^4/(36z^8)$
\end{sol}
\end{ex}

%%%%%%%%%%
\begin{ex}
If $y=\dfrac{3x+2}{1-4x}$, then what is $x$ in
terms of $y$?
\begin{sol}
	$x=(y-2)/(3+4y)$
\end{sol}
\end{ex}

%%%%%%%%%%
\begin{ex}
Divide $x^2+3x-5$ by $x+2$ to obtain the
quotient and the remainder. Equivalently, find polynomial $Q(x)$
and constant $R$ such that
\[\frac{x^2+3x-5}{x+2}=Q(x)+\frac{R}{x+2}.\]
\begin{sol}
	$Q(x)=x+1$, $R=-7$
\end{sol}
\end{ex}

\end{enumialphparenastyle}