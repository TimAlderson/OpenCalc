\subsection{Solving Inequalities that Contain Absolute Values}
We start by solving an equality that contains an absolute value.
To do so, we recall that if $a\geq 0$ then the solution to 
$|x|=a$ is $x=\pm a$. In cases where we are not sure if the right 
side is positive or negative, we must perform a check at the end. \\

\begin{example}{Absolute Value Equality}{AbsoluteValueEquality}
Solve for $x$ in $|2x+3|=2-x$.
\end{example}

\begin{solution} 
This means that either:
\[\begin{array}{rclcrcl}
	2x+3 & = & 2-x & \hspace{5mm} \text{or} \hspace{5mm} & \hspace{3mm} 2x+3 & = & -(2-x)\\
	2x+3 & = & 2-x &  & 2x+3 & = & -2+x\\
	3x & = & -1 &  & x & = & -5\\
	x & = &-1/3 &  & x & = & -5\\
\end{array}\]
Since we do not know if the right side $``2-x"$ is positive or negative, we must perform a check of our answers and omit any that are incorrect.\\

If $x=-1/3$, then we have 

\hspace{4cm} $\begin{array}{rcl}
\displaystyle{\left| 2 (-1/3) +3 \right|} & = & \displaystyle{2-\left( -1/3 \right)} \\
\displaystyle{\left| -2 /3 \, +3\right|} & = & \displaystyle{2+ \, 1/3 } \\
	\displaystyle{ \left| 7 / 3 \right|} & = & \displaystyle{ 7/3 } \\
	\displaystyle{ 7/3 } & = &  \displaystyle{ 7/3 } \\
\end{array} $	
	

% \left( \frac{-1}{3} \right)+3\right|}

Since the left hand side of the equation equals the right hand side, $x=-1/3$ is a solution.\\

Now checking to see if $x=-5$ satisfies the expression. We have

\hspace{4cm} $\begin{array}{rcl}
\displaystyle{\left| 2 (-5) +3 \right|} & = & \displaystyle{2-\left( -5 \right)} \\
\displaystyle{\left| -10 \, +3\right|} & = & \displaystyle{2+ \, 5 } \\
	\displaystyle{ \left| -7 \right|} & = & \displaystyle{ 7 } \\
	\displaystyle{ 7 } & = &  \displaystyle{ 7 } \\
\end{array} $	

Since the left hand side equals the right hand side,  $x=-5$ is a solution.\\
Therefore, the solution to the absolute value equality is $x=-5, \, -1/3$. 
\end{solution}

We next look at absolute values and inequalities. \\

\begin{example}{Absolute Value Inequality}{AbsoluteValueInequality}
Solve $|x-5|<7$.
\end{example}

\begin{solution} 
This simply means $-7<x-5<7$.
Adding $5$ to each gives $-2<x<12$.
Therefore the solution is the interval $(-2,12)$.
\end{solution}

In some questions you must be careful when multiplying by a negative number as in the next problem. \\

\begin{example}{Absolute Value Inequality}{AbsoluteValueInequality2}
Solve $|2-z|<7$.
\end{example}

\begin{solution} 
This simply means $-7<2-z<7$.
Subtracting $2$ gives: $-9<-z<5$.
Now multiplying by $-1$ gives: $9>z>-5$. \ifont{Remember to reverse the inequality signs!}
We can rearrange this as $-5<z<9$.
Therefore the solution is the interval $(-5,9)$.
\end{solution}

\bigskip

\begin{example}{Absolute Value Inequality}{AbsoluteValueInequality3}
Solve $|2-z|\geq 7$.
\end{example}

\begin{solution} 
Recall that for $a>0$, $|x|\geq a$ means $x\leq -a$ or $x\geq a$.
Thus, either $2-z\leq -7$ \ifont{or} $2-z\geq 7$.
Either $9\leq z$ \ifont{or} $-5 \geq z$.
Either $z\geq 9$ \ifont{or} $z \leq -5$.
In interval notation, either $z$ is in $[9,\infty)$ \ifont{or} $z$ is in $(-\infty,-5]$.
All together, we get our solution to be: $(-\infty,-5]\cup [9,\infty)$.
\end{solution}

In the previous two examples the \ifont{only} difference is that one had $<$ in 
the question and the other had $\geq$. Combining the two solutions gives the 
\ifont{entire} real number line! \\

\begin{example}{Absolute Value Inequality}{AbsoluteValueInequality4}
Solve $0<|x-5|\leq 7$.
\end{example}

\begin{solution} 
We split this into two cases.

(1) For $0<|x-5|$ note that we always have that an absolute value is positive or zero (i.e., $0\leq |x-5|$ is always true).
So, for this part, we need to avoid $0=|x-5|$ from occurring. 
Thus, $x$ \ifont{cannot} be $5$, that is, $x\neq 5$.\\

(2) For $|x-5|\leq 7$, we have $-7\leq x-5\leq 7$.
Adding $5$ to each gives $-2\leq x\leq 12$.
Therefore the solution to $|x-5|\leq 7$ is the interval $[-2,12]$.

To combine (1) and (2) we need combine $x\neq 5$ with $x\in[-2,12]$.
Omitting $5$ from the interval $[-2,12]$ gives our solution to be: $[-2,5)\cup(5,12]$.
\end{solution}
