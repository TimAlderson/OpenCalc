The emphasis in this course is on problems---doing calculations and story
problems.  To master problem solving one needs a tremendous amount of
practice doing problems. The more problems you do the better you will
be at doing them, as patterns will start to emerge in both the
problems and in successful approaches to them. You will learn quickly and effectively
 if you devote some time to doing problems every day.

Typically the most difficult problems are story problems, since they
require some effort before you can begin calculating.
Here are some pointers for doing story problems:

\begin{enumerate}

\item
Carefully read each problem twice before writing anything.

\item 
Assign letters to quantities that are described only in words;
draw a diagram if appropriate.

\item
Decide which letters are constants and which are variables.  A letter
stands for a constant if its value remains the same throughout the problem.

\item
Using mathematical notation, write down what you know and then write down
what you want to find.

\item
Decide what category of problem it is (this might be obvious if the
problem comes at the end of a particular chapter, but will not necessarily
be so obvious if it comes on an exam covering several chapters).

\item
Double check each step as you go along; don't wait until the end to
check your work.

\item
Use common sense; if an answer is out of the range of practical
possibilities, then check your work to see where you went wrong.

\end{enumerate}