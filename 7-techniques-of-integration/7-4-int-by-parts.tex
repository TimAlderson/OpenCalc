\section{Integration by Parts}{}{}\label{sec:Parts}
%We have already seen that recognizing the product rule can be useful,
%when we noticed that
%$$\int \sec^3u+\sec u \tan^2u\,du=\sec u \tan u.$$
%As with substitution, we do not have to rely on insight or cleverness
%to discover such antiderivatives; there is a technique that will often
%help to uncover the product rule.
%
%Start with the product rule:
%$${d\over dx}f(x)g(x)=f'(x)g(x)+f(x)g'(x).$$
%We can rewrite this as
%$$f(x)g(x)=\int f'(x)g(x)\,dx +\int f(x)g'(x)\,dx,$$
%and then
%$$\int f(x)g'(x)\,dx=f(x)g(x)-\int f'(x)g(x)\,dx.$$
%This may not seem particularly useful at first glance, but it turns
%out that in many cases we have an integral of the form
%$$\int f(x)g'(x)\,dx$$
%but that 
%$$\int f'(x)g(x)\,dx$$
%is easier. This technique for turning one integral into another is
%called {\dfont integration by parts},
%and is usually written in more compact form. If we let $u=f(x)$ and
%$v=g(x)$ then $du=f'(x)\,dx$ and $dv=g'(x)\,dx$ and 
%$$\int u\,dv = uv-\int v\,du.$$
%To use this technique we need to identify likely candidates for
%$u=f(x)$ and $dv=g'(x)\,dx$.
%
%\begin{example}{Product of a Linear Function and Logarithm}{Product of a Linear Function and Logarithm}\label{Product of a Linear Function and Logarithm}
%Evaluate $\ds\int x\ln x\,dx$. 
%\end{example}
%
%\begin{solution}
%Let $u=\ln x$ so $du=1/x\,dx$. Then
%we must let $dv=x\,dx$ so $\ds v=x^2/2$ and
%$$
% \int x\ln x\,dx={x^2\ln x\over 2}-\int {x^2\over2}{1\over x}\,dx=
% {x^2\ln x\over 2}-\int {x\over2}\,dx={x^2\ln x\over 2}-{x^2\over4}+C.
%$$\vskip-10pt
%\end{solution}
%
%\begin{example}{Product of a Linear Function and Trigonometric Function}{Product of a Linear Function and Trigonometric Function}\label{Product of a Linear Function and Trigonometric Function}
%Evaluate $\ds\int x\sin x\,dx$. 
%\end{example}
%
%\begin{solution}
%Let $u=x$ so $du=dx$. Then
%we must let $dv=\sin x\,dx$ so $v=-\cos x$ and
%$$\int x\sin x\,dx=-x\cos x-\int -\cos x\,dx=
%-x\cos x+\int \cos x\,dx=-x\cos x+\sin x+C.$$\vskip-10pt
%\end{solution}

Here's a simple integral that we can't yet evaluate:
$$\int x\cos x \,dx.$$
It's a simple matter to take the derivative of the integrand using the Product Rule, but there is no Product Rule for integrals.  However, this section introduces \textit{Integration by Parts}, a method of integration that is based on the Product Rule for derivatives. It will enable us to evaluate this integral.

The Product Rule says that if $u$ and $v$ are functions of $x$, then  $(uv)' = u\primeskip'v + uv\primeskip'$.  For simplicity, we've written $u$ for $u(x)$ and $v$ for $v(x)$.  Suppose we integrate both sides with respect to $x$.  This gives
$$\int (uv)'\,dx = \int (u\primeskip'v+uv\primeskip')\,dx.$$
By the Fundamental Theorem of Calculus, the left side integrates to $uv$.  The right side can be broken up into two integrals, and we have
$$\int (uv)'\,dx = \int u\primeskip'v\,dx + \int uv\primeskip'\,dx.$$
rearranging we get
$$\int uv\primeskip'\,dx = \int (uv)'\,dx - \int u\primeskip'v\,dx.$$
which gives 
$$\int uv\primeskip'\,dx = uv - \int u\primeskip'v\,dx.$$
Using differential notation, we can write $du = u\primeskip'(x)dx$ and $dv=v\primeskip'(x)dx$ and the expression above can be written as follows:
$$\int u\,dv = uv - \int v\,du.$$
This is the Integration by Parts formula. For reference purposes, we state this in a theorem.


\begin{theorem}{Integration by Parts}{IBP}
{Let $u$ and $v$ be differentiable functions of $x$ on an interval $I$ containing $a$ and $b$. Then 
	$$\int u\ dv = uv - \int v\ du,$$ and \index{integration!by parts}
	$$\int_{x=a}^{x=b} u\ dv = uv\Big|_a^b - \int_{x=a}^{x=b}v\ du.$$
}
\end{theorem}


Let's try an example to understand our new technique.\\

\begin{example}{Integrating using Integration by Parts}{ex_ibp1}
	{
	Evaluate $\ds\int x\cos{x}\ dx$.}	
\end{example}	

\begin{solution}
{The key to Integration by Parts is to identify part of the integrand as ``$u$'' and part as ``$dv$.'' Regular practice will help one make good identifications, and later we will introduce some principles that help. For now, let  $u=x$ and $dv=\cos{x}\ dx$.
	
It is generally useful to make a small table of these values as done below. Right now we only know $u$ and $dv$ as shown on the left of Figure \ref{fig:ibp1}; on the right we fill in the rest of what we need. If $u = x$, then $du = dx$. Since $dv = \cos x\ dx$, $v$ is an antiderivative of $\cos x$. We choose $v = \sin x$.\\
	
	\noindent\begin{minipage}{\textwidth}
		\noindent\begin{minipage}[t]{.45\textwidth}
			%\centering
			\vskip-10pt
			\begin{align*}
			u&= x & v&=\text{?}\\
			du&= \text{?} & dv&=\cos x\ dx
			\end{align*}
		\end{minipage}\begin{minipage}[t]{.1\textwidth}\centering\vskip15pt$\Rightarrow$\end{minipage}
		\begin{minipage}[t]{.45\textwidth}
			\vskip-10pt
			\begin{align*}
			u&= x & v&=\sin x\\
			du&= dx & dv&=\cos x\ dx
			\end{align*}
		\end{minipage}
		\captionsetup{type=figure}%
		\caption{Setting up Integration by Parts.}\label{fig:ibp1}
	\end{minipage}\\
	\vskip\baselineskip
	
	%  On the right side of the formula we can see that we need $du$ and $v$.  We get $du$ by taking the derivative of $u$, and we get $du=(1)\,dx$, or simply $du=dx$.  We get $v$ by finding an antiderivative of $dv$.  Here we get $v=\sin x $.  
	Now substitute all of this into the Integration by Parts formula, giving
	$$\int x\cos x\,dx = x\sin x - \int \sin x \,dx.$$
	We can then integrate $\sin x$ to get $-\cos x + C$ and overall our answer is
	$$\int x\cos x\ dx = x\sin x + \cos x + C.$$
	Note how the antiderivative contains a product, $x\sin x$. This product is what makes Integration by Parts necessary.
}	
\end{solution}


The example above demonstrates how Integration by Parts works in general.  We try to identify $u$ and $dv$ in the integral we are given, and the key is that we usually want to choose $u$ and $dv$ so that $du$ is simpler than $u$ and $v$ is hopefully not too much more complicated than $dv$.  This will mean that the integral on the right side of the Integration by Parts formula, $\int v\,du$ will be simpler to integrate than the original integral $\int u\,dv$.

In the example above, we chose $u=x$ and $dv=\cos x\,dx$.  Then $du=dx$ was simpler than $u$ and $v=\sin x$ is no more complicated than $dv$.  Therefore, instead of integrating $x\cos x \,dx$, we could integrate $\sin x\,dx$, which we knew how to do.

A useful mnemonic for helping to determine $u$ is ``LIPET,'' where 
\begin{center}L = \textbf{L}ogarithmic, I = \textbf{I}nverse Trig., P = \textbf{P}olynomial (algebraic), E = \textbf{E}xponential, and T = \textbf{T}rigonometric.
\end{center}

If the integrand contains both a logarithmic and an polynomial term, in general letting $u$ be the logarithmic term works best, as indicated by L coming before P in LIPET.

Note: Some texts us ``LIATE,'' where A = \textbf{A}lgebraic. This method works just as well as LIPET.


%A useful mnemonic for helping to determine $u$ is ``LIATE,'' where 
%\begin{center}L = \textbf{L}ogarithmic, I = \textbf{I}nverse Trig., A = \textbf{A}lgebraic (polynomials), 
%	
%	T = \textbf{T}rigonometric, and E = \textbf{E}xponential.
%\end{center}
%
%If the integrand contains both a logarithmic and an algebraic term, in general letting $u$ be the logarithmic term works best, as indicated by L coming before A in LIATE.

We now consider another example.\\

\begin{example}{Integrating using Integration by Parts}{ex_ibp2}
	{
	Evaluate $\displaystyle \int x e^x\,dx$.}	
\end{example}

\begin{solution}
{Using the LIPET rule, we see that the integrand contains a  \textbf{P}olynomial term ($x$) and an \textbf{E}xponential term ($e^x$). Our mnemonic suggests letting $u$ be the polynomial term, so we choose $u=x$ and $dv=e^x\,dx$.  Then $du=dx$ and $v=e^x$ as indicated by the tables below.\\
	
	\noindent\begin{minipage}{\textwidth}
		\noindent\begin{minipage}[t]{.45\textwidth}
			%\centering
			\vskip-10pt
			\begin{align*}
			u&= x & v&=\text{?}\\
			du&= \text{?} & dv&=e^x\ dx
			\end{align*}
		\end{minipage}\begin{minipage}[t]{.1\textwidth}\centering\vskip15pt$\Rightarrow$\end{minipage}
		\begin{minipage}[t]{.45\textwidth}
			\vskip-10pt
			\begin{align*}
			u&= x & v&=e^x\\
			du&= dx & dv&=e^x\ dx
			\end{align*}
		\end{minipage}
		\captionsetup{type=figure}%
		\caption{Setting up Integration by Parts.}\label{fig:ibp2}
	\end{minipage}\\
	\vskip\baselineskip
	
	We see $du$ is simpler than $u$, while there is no change in going from $dv$ to $v$.  This is good.  The Integration by Parts formula gives
	$$\int x e^x\,dx = xe^x - \int e^x\,dx.$$
	The integral on the right is simple; our final answer is
	$$\int xe^x\ dx = xe^x - e^x + C.$$
	Note again how the antiderivatives contain a product term.
}	
\end{solution}	

\begin{example}{Product of a Linear Function and Logarithm}{Product of a Linear Function and Logarithm}\label{Product of a Linear Function and Logarithm}
Evaluate $\ds\int x\ln x\,dx$. 
\end{example}

\begin{solution}
Using the LIPET rule, we let $u=\ln x$ so $du=1/x\,dx$. Then
we must let $dv=x\,dx$ so $\ds v=x^2/2$ and
$$
 \int x\ln x\,dx={x^2\ln x\over 2}-\int {x^2\over2}{1\over x}\,dx=
 {x^2\ln x\over 2}-\int {x\over2}\,dx={x^2\ln x\over 2}-{x^2\over4}+C.
$$\vskip-10pt
\end{solution}



\begin{example}{Secant Cubed (again)}{Secant Cubed (again)}\label{Secant Cubed (again)}
Evaluate $\ds\int\sec^3 x\,dx$. 
\end{example}

\begin{solution}
Of course we already know the answer
to this, but we needed to be clever to discover it. Here we'll use the
new technique to discover the antiderivative.
Let $u=\sec x$ and $\ds dv=\sec^2 x\,dx$. Then $du=\sec x\tan x$ and
$v=\tan x$ and
\begin{eqnarray*}
  \int\sec^3 x\,dx&=&\sec x\tan x-\int \tan^2x\sec x\,dx\cr
  &=&\sec x\tan x-\int (\sec^2x-1)\sec x\,dx\cr
  &=&\sec x\tan x-\int \sec^3x\,dx +\int\sec x\,dx.\cr
\end{eqnarray*}
At first this looks useless---we're right back to
$\ds \int\sec^3x\,dx$. But looking more closely:
\begin{eqnarray*}
  \int\sec^3x\,dx&=&\sec x\tan x-\int \sec^3x\,dx +\int\sec x\,dx\cr
  \int\sec^3x\,dx+\int \sec^3x\,dx&=&\sec x\tan x +\int\sec x\,dx\cr
  2\int\sec^3x\,dx&=&\sec x\tan x +\int\sec x\,dx\cr
  \int\sec^3x\,dx&=&{\sec x\tan x\over2} +{1\over2}\int\sec x\,dx\cr
  &=&{\sec x\tan x\over2} +{\ln|\sec x+\tan x|\over2}+C.
\end{eqnarray*}\vskip-10pt
\end{solution}

\subsection*{Tabular Method}

\begin{example}{Product of a Polynomial and Trigonometric Function}{Product of a Polynomial and Trigonometric Function}\label{Product of a Polynomial and Trigonometric Function}
Evaluate $\ds\int x^2\sin x\,dx$. 
\end{example}

\begin{solution}
Let $u=x^2$, $dv=\sin x\,dx$; then $du=2x\,dx$ and $v=-\cos x$. 
Now $$\ds \int x^2\sin x\,dx=-x^2\cos x+\int 2x\cos x\,dx.$$ 
This is
better than the original integral, but we need to do integration by
parts again. Let $u=2x$, $dv=\cos x\,dx$; then
$du=2$ and $v=\sin x$, and
\begin{eqnarray*}
  \int x^2\sin x\,dx&=&-x^2\cos x+\int 2x\cos x\,dx\cr
  &=&-x^2\cos x+ 2x\sin x - \int 2\sin x\,dx\cr
  &=&-x^2\cos x+ 2x\sin x + 2\cos x + C.
\end{eqnarray*}\vskip-10pt
\end{solution}

Such repeated use of integration by parts is fairly common, but it can
be a bit tedious to accomplish, and it is easy to make
errors, especially sign errors involving the subtraction in the
formula. There is a nice tabular method to accomplish the calculation
that minimizes the chance for error and speeds up the whole
process. We illustrate with the previous example. Here is the
table:
$$\begin{array}{|c|c|c|}
\hline
\mbox{sign}& u& dv\\\hline
\mbox{+}& x^2& \sin x\\\hline
\mbox{-}& 2x& -\cos x\\\hline
\mbox{+}& 2& -\sin x\\\hline
\mbox{-}& 0& \cos x\\\hline
\end{array}$$
To form this table, we start with $u$ at the top of the second
column and repeatedly compute the derivative; starting with $dv$ at
the top of the third column, we repeatedly compute the
antiderivative. In the first column, we place a ``$-$'' in every
second row. To form the 
second table we combine the first and second columns by
ignoring the boundary; if you do this by hand, you may simply start
with two columns and add a ``$-$'' to every second row.

Alternatively, we can use the following table:
$$\begin{array}{|c|c|}
\hline
u& dv\\\hline
x^2& \sin x\\\hline
-2x& -\cos x\\\hline
2& -\sin x\\\hline
0& \cos x\\\hline
\end{array}$$
To compute with this second table we begin at the top. Multiply the
first entry in column $u$ by the second entry in column $dv$ to get
$\ds -x^2\cos x$, and add this to the integral of the product of the
second entry in column $u$ and second entry in column $dv$.  This
gives:
$$-x^2\cos x+\int 2x\cos x\,dx,$$
or exactly the result of the first application of integration by
parts.  Since this integral is not yet easy, we return to the table.
Now we multiply twice on the diagonal, $\ds (x^2)(-\cos x)$ and
$(-2x)(-\sin x)$ and then once straight across, $(2)(-\sin x)$, and
combine these as
$$-x^2\cos x+2x\sin x-\int 2\sin x\,dx,$$
giving the same result as the second application of integration by
parts. While this integral is easy, we may return yet once more to the
table. Now multiply three times on the diagonal to get $\ds
(x^2)(-\cos x)$, $(-2x)(-\sin x)$, and $(2)(\cos x)$, and once
straight across, $(0)(\cos x)$. We combine these as before to get
$$
  -x^2\cos x+2x\sin x +2\cos x+\int 0\,dx=
  -x^2\cos x+2x\sin x +2\cos x+C.
$$
Typically we would fill in the table one line at a time, until the
``straight across'' multiplication gives an easy integral. If we can
see that the $u$ column will eventually become zero, we can instead
fill in the whole table; computing the products as indicated will then
give the entire integral, including the ``$+C\,$'', as above.

\begin{example}{}{}
Do Example \ref{exa:ex_ibp2} again, using the tabular method.
\end{example}

\begin{solution}
Recognising that $ x $	has a derivative that vanishes, and $ e^x $ is easy to repeatedly integrate, we construct the table with $ u=x:$
$$\begin{array}{|c|c|}
\hline
u& dv\\\hline
x& e^x\\\hline
-1& e^x\\\hline
+0& e^x\\\hline
\end{array}$$
This gives
$$\int xe^x\ dx = xe^x - e^x + C.$$
which agrees with the result in Example \ref{exa:ex_ibp2}. 
\end{solution}


\subsection*{Some Classic Examples of IBP}

\begin{example}{Solving for the unknown integral}{ex_ibp4}
	{
	Evaluate $\displaystyle \int e^x\cos x \,dx$.}	
\end{example}	

\begin{solution}
{This is a classic problem.  Our mnemonic (LIPET) suggests letting $u$ be the exponential factor, sowe choose $u=e^x$ and hence $dv = \cos x\,dx$.  Then $du=e^x\,dx$ and $v=\sin x$ as shown below.\\	
	\noindent\begin{minipage}{\textwidth}
		\noindent\begin{minipage}[t]{.45\textwidth}
			%\centering
			\vskip-10pt
			\begin{align*}
			u&= e^x & v&=\text{?}\\
			du&= \text{?} & dv&=\cos x\ dx
			\end{align*}
		\end{minipage}\begin{minipage}[t]{.1\textwidth}\centering\vskip15pt$\Rightarrow$\end{minipage}
		\begin{minipage}[t]{.45\textwidth}
			\vskip-10pt
			\begin{align*}
			u&= e^x& v&=\sin x\\
			du&= e^x\ dx & dv&=\cos x\ dx
			\end{align*}
		\end{minipage}
		\captionsetup{type=figure}%
		\caption{Setting up Integration by Parts.}\label{fig:ibp4}
	\end{minipage}\\
	\vskip\baselineskip
	Notice that $du$ is no simpler than $u$, going against our general rule (but bear with us). The Integration by Parts formula yields
	$$\int e^x\cos x\ dx = e^x\sin x - \int e^x\sin x\,dx.$$
	The integral on the right is not much different than the one we started with, so it seems like we have gotten nowhere. Let's  keep working and apply Integration by Parts to the new integral, using $u=e^x$ and $dv = \sin x\,dx$. This leads us to the following:\\ %Then we get $du=e^x\,dx$ and $v=-\cos x$.  
	
	\noindent\begin{minipage}{\textwidth}
		\noindent\begin{minipage}[t]{.45\textwidth}
			%\centering
			\vskip-10pt
			\begin{align*}
			u&= e^x & v&=\text{?}\\
			du&= \text{?} & dv&=\sin x\ dx
			\end{align*}
		\end{minipage}\begin{minipage}[t]{.1\textwidth}\centering\vskip15pt$\Rightarrow$\end{minipage}
		\begin{minipage}[t]{.45\textwidth}
			\vskip-10pt
			\begin{align*}
			u&= e^x& v&=-\cos x\\
			du&= e^x\ dx & dv&=\sin x\ dx
			\end{align*}
		\end{minipage}
		\captionsetup{type=figure}%
		\caption{Setting up Integration by Parts (again).}\label{fig:ibp4a}
	\end{minipage}\\
	\vskip\baselineskip
	
	The Integration by Parts formula then gives:
	\begin{align*}
	\int e^x\cos x\,dx &= e^x\sin x - \left(-e^x\cos x - \int -e^x\cos x\,dx\right)\\
	&= e^x\sin x+ e^x\cos x - \int e^x\cos x\ dx.
	\end{align*}
	It seems we are back right where we started, as the right hand side contains $\int e^x\cos x\,dx$.  But this is actually a good thing.  
	
	Add $\ds\int e^x\cos x\ dx$ to both sides. This gives 
	\begin{align*}
	2\int e^x\cos x\ dx & = e^x\sin x + e^x\cos x \\
	\intertext{Now divide both sides by 2:}
	\int e^x\cos x\ dx & = \frac{1}{2}\big(e^x\sin x + e^x\cos x\big).
	\end{align*}
	
	Simplifying a little and adding the constant of integration, our answer is thus
	$$\int e^x\cos x\ dx = \frac12e^x\left(\sin x + \cos x\right)+C.$$
	\vskip-15pt
}	
\end{solution}	
								
								
\begin{example}{IBP: the antiderivative of $\ln x$}{ex_ibp5}
{Evaluate $\displaystyle \int \ln x\,dx$.}	
\end{example}									
								
\begin{solution}
{One may have noticed that we have rules for integrating the familiar trigonometric functions and $e^x$, but we have not yet given a rule for integrating $\ln x$.  That is because $\ln x$ can't easily be integrated with any of the rules we have learned up to this point.  But we can find its antiderivative by a clever application of Integration by Parts.  Set $u=\ln x$ and $dv=dx$.  This is a good, sneaky trick to learn as it can help in other situations. This determines $du=(1/x)\,dx$ and $v=x$ as shown below.\\
	
	\noindent\begin{minipage}{\textwidth}
		\noindent\begin{minipage}[t]{.45\textwidth}
			%\centering
			\vskip-10pt
			\begin{align*}
			u&= \ln x & v&=\text{?}\\
			du&= \text{?} & dv&=dx
			\end{align*}
		\end{minipage}\begin{minipage}[t]{.1\textwidth}\centering\vskip15pt$\Rightarrow$\end{minipage}
		\begin{minipage}[t]{.45\textwidth}
			\vskip-10pt
			\begin{align*}
			u&= \ln x& v&=x\\
			du&= 1/x\ dx & dv&=dx
			\end{align*}
		\end{minipage}
		\captionsetup{type=figure}%
		\caption{Setting up Integration by Parts.}\label{fig:ibp5}
	\end{minipage}\\
	\vskip\baselineskip
	Putting this all together in the Integration by Parts formula, things work out very nicely:
	$$\int \ln x\,dx = x\ln x - \int x\,\frac1x\,dx.$$
	The new integral simplifies to $\int 1\,dx$, which is about as simple as things get.  Its integral is $x+C$ and our answer is
	$$\int \ln x\ dx = x\ln{x} - x + C.$$
}\\
\end{solution}		


\begin{example}{Integrating using Int. by Parts: antiderivative of $\arctan x$}{ex_ibp6}
{Evaluate $\displaystyle \int \arctan x  \,dx$.}
\end{example}

\begin{solution}
{The same sneaky trick we used above works here.  Let $u=\arctan x$ and $dv=dx$.  Then $du=1/(1+x^2)\,dx$ and $v=x$.  The Integration by Parts formula gives
	$$\int \arctan x \,dx = x\arctan x - \int \frac x{1+x^2}\,dx.$$
	The integral on the right can be solved by substitution.  Taking $u=1+x^2$, we get $du=2x\,dx$.  The integral then becomes
	$$\int \arctan x \,dx = x\arctan x - \frac12\int \frac 1{u}\,du.$$
	The integral on the right evaluates to $\ln|u|+C$, which becomes $\ln(1+x^2)+C$.  Therefore, the answer is
	$$\int \arctan x\ dx = x\arctan x - \ln(1+x^2) + C.$$
}\\		
\end{solution}		




\subsection*{ Substitution Before Integration}

When taking derivatives, it was common to employ multiple rules (such as using both the Quotient and the Chain Rules). It should then come as no surprise that some integrals are best evaluated by combining integration techniques. In particular, here we illustrate making an ``unusual'' substitution first before using Integration by Parts.\\

\begin{example}{Integration by Parts after substitution}{ex_ibp8}
	{
	Evaluate $\ds \int \cos(\ln x)\ dx$.}	
\end{example}


\begin{solution}
{The integrand contains a composition of functions, leading us to think Substitution would be beneficial. Letting $u=\ln x$, we have $du = 1/x\ dx$. This seems problematic, as we do not have a $1/x$ in the integrand. But consider:
	$$du = \frac 1x\ dx \Rightarrow x\cdot du = dx.$$
	Since $u = \ln x$, we can use inverse functions and conclude that $x = e^u$. Therefore we have that
	\begin{align*}
	dx &= x\cdot du \\
	&= e^u\ du.
	\end{align*}
	We can thus replace $\ln x$ with $u$ and $dx$ with $e^u\ du$. Thus we rewrite our integral as 
	$$\int \cos(\ln x)\ dx = \int e^u\cos u \ du.$$
	We evaluated this integral in Example \ref{exa:ex_ibp4}. Using the result there, we have:
	\begin{align*}
	\int \cos(\ln x)\ dx &= \int e^u\cos u \ du \\
	&= \frac12e^u\big(\sin u + \cos u\big) + C \\
	&= \frac12e^{\ln x} \big(\sin(\ln x) + \cos (\ln x)\big)+C\\
	&= \frac12x \big(\sin(\ln x) + \cos (\ln x)\big)+C.
	\end{align*}
	\vskip-\baselineskip
}
\end{solution}		




\subsection*{Definite Integrals and Integration By Parts}


So far we have focused only on evaluating indefinite integrals. Of course, we can use Integration by Parts to evaluate definite integrals as well, as Theorem \ref{thm:IBP} states. We do so in the next example.\\

\begin{example}{Definite integration using Integration by Parts}{ex_ibp7}
	{
	Evaluate $\displaystyle \int_1^2 x^2 \ln x \,dx$.}	
\end{example}


\begin{solution}
{%Once again, our mnemonic suggests we let $u=\ln x$.  %(We could let $u = x^2$ and $dv = \ln x\ dx$, as we now know the antiderivatives of $\ln x$. However, letting $u = \ln x$ makes our next integral much simpler as it removes the logarithm from the integral entirely.)
	Our mnemonic suggests letting $u=\ln x$, hence $dv =x^2\,dx$. 
	%So we have $u=\ln x$ and $dv=x^2\,dx$.  
	We then get $du = (1/x)\,dx$ and $v=x^3/3$ as shown below.\\
	
	\noindent\begin{minipage}{\textwidth}
		\noindent\begin{minipage}[t]{.45\textwidth}
			%\centering
			\vskip-10pt
			\begin{align*}
			u&= \ln x & v&=\text{?}\\
			du&= \text{?} & dv&=x^2\ dx
			\end{align*}
		\end{minipage}\begin{minipage}[t]{.1\textwidth}\centering\vskip15pt$\Rightarrow$\end{minipage}
		\begin{minipage}[t]{.45\textwidth}
			\vskip-10pt
			\begin{align*}
			u&= \ln x& v&=x^3/3\\
			du&= 1/x\ dx & dv&=x^2\ dx
			\end{align*}
		\end{minipage}
		\captionsetup{type=figure}%
		\caption{Setting up Integration by Parts.}\label{fig:ibp7}
	\end{minipage}\\
	\vskip\baselineskip
	
	The Integration by Parts formula then gives
	\begin{align*}
	\int_1^2 x^2 \ln x\,dx &= \frac{x^3}3\ln x\bigg|_1^2 - \int_1^2 \frac{x^3}{3}\,\frac 1x\,dx \\
	&=  \frac{x^3}3\ln x\bigg|_1^2 - \int_1^2 \frac{x^2}{3}\,dx \\
	&=  \frac{x^3}3\ln x\bigg|_1^2 - \frac{x^3}{9}\bigg|_1^2\\
	&=  \left(\frac{x^3}3\ln x - \frac{x^3}{9}\right)\bigg|_1^2\\
	&=	\left(\frac83\ln 2 - \frac89\right)-\left(\frac13\ln 1 - \frac19\right) \\
	&= \frac83\ln 2 - \frac79 \\
	&\approx 1.07.
	\end{align*}
	\vskip-15pt
}	
\end{solution}		


In general, Integration by Parts is useful for integrating certain products of functions, like $\int x e^x\,dx$ or $\int x^3\sin x\,dx$.   It is also useful for integrals involving logarithms and inverse trigonometric functions.  

As stated before, integration is generally more difficult than derivation. We are developing tools for handling a large array of integrals, and experience will tell us when one tool is preferable/necessary over another. For instance, consider the three similar--looking integrals 
$$\int xe^x\,dx, \qquad  \int x e^{x^2}\,dx \qquad \text{and} \qquad \int xe^{x^3}\,dx.$$

While the first is calculated easily with Integration by Parts, the second is best approached with Substitution.  Taking things one step further, the third integral has no answer in terms of elementary functions, so none of the methods we learn in calculus will get us the exact answer.

Integration by Parts is a very useful method, second only to substitution. In the following sections of this chapter, we continue to learn other integration techniques.  
								
								
								

%%%%%%%%%%%%%%%%%%%%%%%%%%%%%%%%%%%%%%%%%%%%
\Opensolutionfile{solutions}[ex]
\section*{Exercises for \ref{sec:Parts}}

\begin{enumialphparenastyle}

Find the antiderivatives.

%%%%%%%%%%
\begin{ex}
 $\ds\int x\cos x\,dx$
\begin{sol}
 $\cos x+x\sin x+C$
\end{sol}
\end{ex}
%%%%%%%%%%

%%%%%%%%%%
\begin{ex}
 $\ds\int x^2\cos x\,dx$
\begin{sol}
 $\ds x^2\sin x-2 \sin x+2x\cos x +C$
\end{sol}
\end{ex}
%%%%%%%%%%

%%%%%%%%%%
\begin{ex}
 $\ds\int xe^x\,dx$
\begin{sol}
 $\ds (x-1)e^x +C$
\end{sol}
\end{ex}
%%%%%%%%%%

%%%%%%%%%%
\begin{ex}
 $\ds\int xe^{x^2}\,dx$
\begin{sol}
 $\ds (1/2)e^{x^2} +C$
\end{sol}
\end{ex}
%%%%%%%%%%

%%%%%%%%%%
\begin{ex}
 $\ds\int \sin^2 x\,dx$
\begin{sol}
 $(x/2)-\sin(2x)/4 +C=$\hfill\break$(x/2)-(\sin x\cos x)/2+C$
\end{sol}
\end{ex}
%%%%%%%%%%

%%%%%%%%%%
\begin{ex}
 $\ds\int \ln x\,dx$
\begin{sol}
 $x\ln x-x +C$
\end{sol}
\end{ex}
%%%%%%%%%%

%%%%%%%%%%
\begin{ex}
 $\ds\int x\arctan x\,dx$
\begin{sol}
 $\ds (x^2\arctan x +\arctan x -x)/2+C$
\end{sol}
\end{ex}
%%%%%%%%%%

%%%%%%%%%%
\begin{ex}
 $\ds\int x^3\sin x\,dx$
\begin{sol}
 $\ds -x^3\cos x+3x^2\sin x+6x\cos x-6\sin x+C$
\end{sol}
\end{ex}
%%%%%%%%%%

%%%%%%%%%%
\begin{ex}
 $\ds\int x^3\cos x\,dx$
\begin{sol}
 $\ds x^3\sin x+3x^2\cos x-6x\sin x-6\cos x+C$
\end{sol}
\end{ex}
%%%%%%%%%%

%%%%%%%%%%
\begin{ex}
 $\ds\int x\sin^2 x\,dx$
\begin{sol}
 $\ds x^2/4-(\cos^2 x)/4-(x\sin x\cos x)/2+C$
\end{sol}
\end{ex}
%%%%%%%%%%

%%%%%%%%%%
\begin{ex}
 $\ds\int x\sin x\cos x\,dx$
\begin{sol}
 $\ds x/4-(x\cos^2 x)/2+(\cos x\sin x)/4+C$
\end{sol}
\end{ex}
%%%%%%%%%%

%%%%%%%%%%
\begin{ex}
 $\ds\int \arctan(\sqrt x)\,dx$
\begin{sol}
 $x\arctan(\sqrt x)+\arctan(\sqrt x)-\sqrt{x}+C$
\end{sol}
\end{ex}
%%%%%%%%%%

%%%%%%%%%%
\begin{ex}
 $\ds\int \sin(\sqrt x)\,dx$
\begin{sol}
 $2\sin(\sqrt x)-2\sqrt x\cos(\sqrt x)+C$
\end{sol}
\end{ex}
%%%%%%%%%%

%%%%%%%%%%
\begin{ex}
 $\ds\int\sec^2 x\csc^2 x\,dx$
\begin{sol}
 $\sec x\csc x-2\cot x+C$
\end{sol}
\end{ex}
%%%%%%%%%%

\end{enumialphparenastyle}