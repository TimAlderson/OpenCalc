\section{Powers of Trigonometric Functions}{}{}\label{sec:Powers of trigonometric functions}
Functions involving trigonometric functions are useful as they are good at describing periodic behavior. % Learning techniques to integrate such functions helps solve problems involving periodic behavior and also helps build general problem solving skills.
 This section describes several techniques for finding antiderivatives of certain combinations of trigonometric functions.\\


Functions consisting of powers of the sine and cosine can be
integrated by using substitution and trigonometric identities. These
can sometimes be tedious, but the technique is straightforward. A similar 
technique is applicable to powers of secant and tangent (and also cosecant 
and cotangent, not discussed here).

%The trigonometric substitutions we will focus on in this section are summarized in the table below:
%\begin{center}
%	\begin{tabular}{|c|c|c|c|c|}
%	\hline
%	\thmfont{Substitution} & $u=\sin x$ & $u=\cos x$ & $u=\tan x$ & $u=\sec x$\\[0.5em]
%	\thmfont{Derivative} & $du=\cos x\, dx$ & $du=-\sin x\,dx$ & $du=\sec^2x\,dx$ & $du=\sec x\tan x\,dx$\\
%	\hline
%	\end{tabular}
%\end{center}

\subsection*{Integrals of the form $\ds \int \sin^m x\cos^n x\ dx$}



Using the technique of Substitution, we see the integral $\int \sin x\cos x\ dx$  could easily be evaluated  by letting $u=\sin x$ or by letting $u = \cos x$. This integral is easy since the power of both sine and cosine is $ 1 $.

We generalize this integral and consider integrals of the form $\int \sin^mx\cos^nx\ dx$, where $m,n$ are nonnegative integers. Our strategy for evaluating these integrals is to use the Pythagorean identity $\cos^2x+\sin^2x=1$ to convert high powers of one trigonometric function into the other, leaving a single sine or cosine term in the integrand. We summarize the general technique in the following


\begin{formulabox}[Integrals Involving Powers of Sine and Cosine]
	Consider $\ds \int \sin^mx\cos^nx\ dx$, where $m,n$ are nonnegative integers.\index{integration!of trig. powers}
		\begin{enumerate}
		\item		If $m$ is odd, then $m=2k+1$ for some integer $k$. Rewrite \small
				$$ \sin^mx = \sin^{2k+1}x = \sin^{2k}x\sin x = (\sin^2x)^k\sin x = (1-\cos^2x)^k\sin x.$$\normalsize
				Then \small
				$$\int \sin^mx\cos^nx\ dx = \int (1-\cos^2x)^k\sin x\cos^nx\ dx = -\int (1-u^2)^ku^n\ du,$$\normalsize
				where $u = \cos x$ and $du = -\sin x\ dx$. 
		\item		If $n$ is odd, then using substitutions similar to that outlined above we have
				\small
				$$ \int \sin^mx\cos^nx\ dx = \int u^m(1-u^2)^k\ du,$$ \normalsize
				where $u = \sin x$ and $du = \cos x\ dx$.
		\item		If both $m$ and $n$ are even, use the power--reducing identities
			\small$$  \cos^2x = \frac{1+\cos (2x)}{2} \quad \text{and}\quad \sin^2x = \frac{1-\cos(2x)}2$$\normalsize
		to reduce the degree of the integrand. Expand the result and try again.
		\end{enumerate}
%	NOTE: The integral will become more tedious as $m$ and $n$ get large, since multiple steps will be needed.
\end{formulabox}



Now a few examples to practice the approach.

\begin{example}{Integrating powers of sine and cosine}{}
Evaluate $\ds\int \sin^6 x\cos^5 x\,dx$.
\end{example}  

\begin{solution}
Since the power of cosine is odd, we use the substitution $\red{u=\sin x}$ and $du=\cos x\,dx$, that is, $\fbox{$dx$}=\fbox{$\ds\frac{du}{\cos x}$}$.
Then $\ds\int \sin^6 x\cos^5 x\,dx$ is equal to:
$${\def\arraystretch{2.2}
\begin{array}{>{\displaystyle}r>{\displaystyle}c>{\displaystyle}l>{\displaystyle}l}
 & = & \int u^6\cos^5x \,\fbox{$\ds\frac{du}{\cos x}$}	 & \mbox{Using the substitution} \\  
	&=&\int u^6\left(\cos^2x\right)^2 \,du  & \mbox{Canceling a $\cos x$ and rewriting $\cos^4x$}\\  
	&=&\int u^6(1-\sin^2x)^2\,du  & \mbox{Using trig identity $\cos^2x=1-\sin^2x$}\\  
	&=&\int u^6(1-u^2)^2\,du  & \mbox{Writing integral in terms of $u$'s}\\  
	&=&\int u^6-2u^8+u^{10}\,du  & \mbox{Expand and collect like terms}\\  
	&=&\frac{u^7}{7}-\frac{2u^9}{9}+\frac{u^{11}}{11}+C & \mbox{Integrating}\\  
	&=&\frac{\sin^7x}{7}-\frac{2\sin^9x}{9}+\frac{\sin^{11}x}{11}+C  & \mbox{Replacing $u$ back in terms of $x$}
\end{array}
}$$
\end{solution}


\begin{example}{Odd Power of Cosine}{}
Evaluate $\ds\int \cos^3 x\,dx$.
\end{example}  

\begin{solution}
Since the power of cosine is odd, we use the substitution $\red{u=\sin x}$ and $du=\cos x\,dx$.  
This may seem strange at first since we don't have $\sin x$ in the question, but it does work!  
$$\begin{array}{>{\displaystyle}r>{\displaystyle}c>{\displaystyle}l>{\displaystyle}l}
\int \cos^3 x\,\fbox{$dx$} & = & \int \cos^3x \,\fbox{$\ds\frac{du}{\cos x}$}	 & \mbox{Using the substitution} \\  
	&=&\int \cos^2x \,du  & \mbox{Canceling a $\cos x$}\\  
	&=&\int (1-\sin^2x)\,du  & \mbox{Using trig identity $\cos^2x=1-\sin^2x$}\\  
	&=&\int (1-u^2)\,du  & \mbox{Writing integral in terms of $u$'s}\\  
	&=&u-\frac{u^3}{3}+C & \mbox{Integrating}\\  
	&=&\sin x-\frac{\sin^3x}{3}+C  & \mbox{Replacing $u$ back in terms of $x$}
\end{array}$$
\end{solution}

\begin{example}{Integrating powers of sine and cosine}{Integrating powers of sine and cosine}\label{Odd Power of Sine}
Evaluate $\ds\int \sin^5 x\,dx$.
\end{example}

\begin{solution} 
Since the power of sine is odd, we factor out one $ \sin(x) $, and exploit the Pythagorean identity: $\sin^2x+\cos^2x=1$.
\begin{eqnarray*}
  \int \sin^5 x\,dx&=&\int \sin x \sin^4 x\,dx\\
	&=&
  \int \sin x (\sin^2 x)^2\,dx\\
	&=&
  \int \sin x (1-\cos^2 x)^2\,dx.
\end{eqnarray*}%\vskip-20pt
Now use $u=\cos x$, $du=-\sin x\,dx$:
\begin{eqnarray*}
  \int \sin x (1-\cos^2 x)^2\,dx&=&\int -(1-u^2)^2\,du\\
  &=&\int -(1-2u^2+u^4)\,du\\
  &=&\int 1+2u^2-u^4\,du\\	
  &=&-u+{\frac{2}{3}}u^3-{\frac{1}{5}}u^5+C\\
  &=&-\cos x+\frac{2}{3}\cos^3 x-\frac{1}{5}\cos^5x+C.
\end{eqnarray*}\vskip-20pt
\end{solution}

%Observe that by taking the substitution $u=\cos x$ in the last example, we ended up with an even power of sine from which we can use the Pythagorean identity $\sin^2x+\cos^2x=1$ to replace any remaining sine terms.
%We then ended up with a polynomial in $u$ in which we could expand and integrate quite easily.

%This technique works for products of powers of sine and cosine.
%We summarize it below.
%\begin{formulabox}[Products of Sine and Cosine]
%	When evaluating $\ds\int\sin^mx \cos^nx\,dx$:  
%	\begin{enumerate}
%	\item \ffont{The power of sine is odd ($m$ odd):}\\  
%		(a) Use $u=\cos x$ and $du=-\sin x\,dx$.\\  
%		(b) Replace $dx$ using (a), thus cancelling one power of $\sin x$ by the substitution of $du$, and be left with an even number of sine powers.\\  
%		(c) Use $\sin^2x=1-\cos^2x~(=1-u^2)$ to replace the leftover sines.  
%	\item \ffont{The power of cosine is odd ($n$ odd):}\\  
%		(a) Use $u=\sin x$ and $du=\cos x\,dx$.\\  
%		(b) Replace $dx$ using (a), thus cancelling one power of $\cos x$ by the substitution of $du$, and be left with an even number of cosine powers.\\  
%		(c) Use $\cos^2x=1-\sin^2x~(=1-u^2)$ to replace the leftover cosines.	  
%	\item \ffont{Both $m$ and $n$ are odd}:\\ Use either $1$ or $2$ (both will work).  
%	\item \ffont{Both $m$ and $n$ are even}:\\  
%	 Use $\cos^2x=\frac{1}{2}\left(1+\cos(2x)\right)$ and/or $\sin^2x=\frac{1}{2}\left(1-\cos(2x)\right)$ to reduce to a form that can be integrated.
%	\end{enumerate}
%	NOTE: The integral will become more tedious as $m$ and $n$ get large, since multiple steps will be needed.
%\end{formulabox}


\begin{example}{Integrating powers of sine and cosine}{Integrating powers of sine and cosine}\label{Odd Power of Sine}
Evaluate $\ds\int\sin^5x\cos^8x\ dx$.
\end{example}

\begin{solution} 
The power of the sine term is odd, so we rewrite $\sin^5x$ as $$\sin^5x = \sin^4x\sin x = (\sin^2x)^2\sin x = (1-\cos^2x)^2\sin x.$$

Our integral is now $\ds \int (1-\cos^2x)^2\cos^8x\sin x\ dx$. Let $u = \cos x$, hence $du = -\sin x\ dx$. Making the substitution and expanding the integrand gives
$$\int (1-\cos^2)^2\cos^8x\sin x\ dx = -\int (1-u^2)^2u^8\ du = -\int \big(1-2u^2+u^4\big)u^8\ du = -\int \big(u^8-2u^{10}+u^{12}\big)\ du.$$
This final integral is not difficult to evaluate, giving 
\begin{align*} -\int \big(u^8-2u^{10}+u^{12}\big)\ du &= -\frac19u^9 + \frac2{11}u^{11} - \frac1{13}u^{13} + C \\
										&=-\frac19\cos^9 x + \frac2{11}\cos^{11} x - \frac1{13}\cos^{13} x + C.
\end{align*}
\end{solution}






\begin{example}{Product of Even Powers of Sine and Cosine}{Product of Even Powers of Sine and Cosine}\label{Product of Even Powers of Sine and Cosine}
Evaluate $\ds\int\! \sin^2x\cos^2x\,dx$. 
\end{example}

\begin{solution} 
Use the formulas
$\ds \sin^2x =(1-\cos(2x))/2$ and $\ds \cos^2x =(1+\cos(2x))/2$ to get:
$$
  \int \sin^2x\cos^2x\,dx=\int {1-\cos(2x)\over2}\cdot
  {1+\cos(2x)\over2}\,dx.
$$
We then have
\begin{eqnarray*}
\int \sin^2x\cos^2x\,dx&=&\int {1-\cos(2x)\over2}\cdot{1+\cos(2x)\over2}\,dx\\
&=&{1\over4}\int 1-\cos^2 2x\,dx\\
&=&{1\over4}\left(x-\int\cos^2 2x\,dx\right)\\
&=&{1\over4}\left(x-{1\over2}\int 1+\cos4x \,dx\right)\\
&=&{1\over4}\left(x-{1\over2}\left(x+{\sin 4x\over 4}\right)\right)\\
&=&{1\over4}\left(x-{x\over2}-{\sin 4x\over 8}\right)+C
\end{eqnarray*}
\end{solution}


\begin{example}{Even Power of Sine}{Even Power of Sine}\label{Even Power of Sine}
Evaluate $\ds\int \sin^6 x\,dx$.
\end{example}

\begin{solution} 
Use $\ds \sin^2x =(1-\cos(2x))/2$ to
rewrite the function:
\begin{eqnarray*}
  \int \sin^6 x\,dx&=&\int (\sin^2 x)^3\,dx\\
	&=&\int {(1-\cos 2x)^3\over 8}\,dx\\
  &=&{1\over 8}\int 1-3\cos 2x+3\cos^2 2x-\cos^3 2x\,dx.
\end{eqnarray*}
Now we have four integrals to evaluate:
$$\int 1\,dx=x+C$$
and
$$\int -3\cos 2x\,dx = -{3\over 2}\sin 2x+C$$
The $\ds \cos^3 2x$ integral is like the previous example:
\begin{eqnarray*}
  \int -\cos^3 2x\,dx&=&\int -\cos 2x\cos^2 2x\,dx\\
  &=&\int -\cos 2x(1-\sin^2 2x)\,dx\\
  &=&\int -{1\over 2}(1-u^2)\,du\\
  &=&-{1\over 2}\left(u-{u^3\over 3}\right)+C\\
  &=&-{1\over 2}\left(\sin 2x-{\sin^3 2x\over 3}\right)+C.
\end{eqnarray*}
And finally we use another trigonometric identity,
$\ds \cos^2x=(1+\cos(2x))/2$:
$$
  \int 3\cos^2 2x\,dx=3\int {1+\cos 4x\over 2}\,dx=
  {3\over 2}\left(x+{\sin 4x\over 4}\right)+C.
$$
So at long last, gathering and combining the arbitrary constants we get
$$
  \int \sin^6 x\,dx = {x\over8} -{3\over 16}\sin 2x 
  -{1\over 16}\left(\sin 2x-{\sin^3 2x\over 3}\right)
  +{3\over 16}\left(x+{\sin 4x\over 4}\right)+C.
$$\vskip-10pt
\end{solution}


\begin{example}{Integrating powers of sine and cosine}{ex_trigint2}
Evaluate $\ds \int\sin^5x\cos^9x\ dx$.
\end{example}

\begin{solution} 
The powers of both the sine and cosine terms are odd, therefore we can apply our techniques to either power. We choose to work with the power of the cosine term since the previous example used the sine term's power.

We rewrite $\cos^9x$ as
\begin{align*} \cos^9 x &= \cos^8x\cos x \\
				&= (\cos^2x)^4\cos x \\
				&= (1-\sin^2x)^4\cos x \\
				&= (1-4\sin^2x+6\sin^4x-4\sin^6x+\sin^8x)\cos x.
\end{align*}

We rewrite the integral as 
$$\int\sin^5x\cos^9x\ dx = \int\sin^5x\big(1-4\sin^2x+6\sin^4x-4\sin^6x+\sin^8x\big)\cos x\ dx.$$

Now substitute and integrate, using $u = \sin x $ and $du = \cos x\ dx$.

\small\noindent
%\begin{gather}
%\int\sin^5x\big(1-4\sin^2x+6\sin^4x-4\sin^6x+\sin^8x\big)\cos x\ dx =\notag\\
%\int u^5(1-4u^2+6u^4-4u^6+u^8)\ du = \int\big(u^5-4u^7+6u^9-4u^{11}+u^{13}\big)\ du  \notag
%\end{gather}
$\ds \int\sin^5x\big(1-4\sin^2x+6\sin^4x-4\sin^6x+\sin^8x\big)\cos x\ dx =$% \\
\vskip-.8\baselineskip
\begin{align*} 
 \int u^5(1-4u^2+6u^4-4u^6+u^8)\ du &= \int\big(u^5-4u^7+6u^9-4u^{11}+u^{13}\big)\ du \\
				&= \frac16u^6-\frac12u^8+\frac35u^{10}-\frac13u^{12}+\frac{1}{14}u^{14}+C\\
				&= \frac16\sin^6 x-\frac12\sin^8 x+\frac35\sin^{10} x+\ldots\\
				&\phantom{=}-\frac13\sin^{12} x+\frac{1}{14}\sin^{14} x+C.
\end{align*}
\end{solution}


\noindent\textbf{Technology Note:} The work we are doing here can be a bit tedious, but the skills developed (problem solving, algebraic manipulation, etc.) are important. Nowadays problems of this sort are often solved using a computer algebra system. The powerful program \textit{Mathematica}\textsuperscript{\textregistered}, or \textit{Wolfram Alpha} integrates $\int \sin^5x\cos^9x\ dx$ as \small$$f(x)=-\frac{45 \cos (2 x)}{16384}-\frac{5 \cos (4 x)}{8192}+\frac{19 \cos (6
   x)}{49152}+\frac{\cos (8 x)}{4096}-\frac{\cos (10 x)}{81920}-\frac{\cos (12
   x)}{24576}-\frac{\cos (14 x)}{114688},$$\normalsize
which clearly has a different form than our answer in Example \ref{exa:ex_trigint2}, which is
$$g(x)=\frac16\sin^6 x-\frac12\sin^8 x+\frac35\sin^{10} x-\frac13\sin^{12} x+\frac{1}{14}\sin^{14} x.$$ Figure \ref{fig:trigint2} shows a graph of $f$ and $g$; they are clearly not equal, but they differ \emph{only by a constant}. That is $g(x) = f(x) + C$ for some constant $C$. So we have two different antiderivatives of the same function, meaning both answers are correct. \\%We leave it to the reader to recognize why both answers are correct.\\

\begin{figure}
\begin{center}
\begin{tikzpicture}
\begin{axis}[width=\marginparwidth+25pt,%
tick label style={font=\scriptsize},axis y line=middle,axis x line=middle,name=myplot,axis on top,%
			%x=.37\marginparwidth,
			%y=.37\marginparwidth,
%			xtick=\empty,% 
%			extra x ticks={.5,3},
%			extra x tick labels={$a$,$b$},
			ytick={-.002,.002,.004},
			yticklabels={$-0.002$,$0.002$,$0.004$},
			%minor y tick num=1,
%			extra y ticks={0.001},%
%			minor x tick num=4,
			ymin=-.003,ymax=0.005,%
			xmin=-.1,xmax=3.15,%
			scaled ticks=false
]

\addplot [{\colortwo},thick,smooth] coordinates {(0,-0.0027879) (0.15708,-0.0027856) (0.31416,-0.0026798) (0.47124,-0.0020322) (0.62832,-0.00060997) (0.7854,0.00089518)(0.94248,0.0017233) (1.0996,0.0019485) (1.2566,0.0019734) (1.4137,0.0019741) (1.5708,0.0019741) (1.7279,0.0019741) (1.885,0.0019734) (2.042,0.0019485) (2.1991,0.0017233) (2.3562,0.00089518) (2.5133,-0.00060997) (2.6704,-0.0020322) (2.8274,-0.0026798) (2.9845,-0.0027856) (3.1416,-0.0027879)};

\draw (axis cs:2.6,0.004) node {\scriptsize $g(x)$};

\addplot [{\colorone},thick,smooth] coordinates {(0,0) (0.15708,0) (0.31416,0.00010807) (0.47124,0.0007557) (0.62832,0.0021779) (0.7854,0.003683) (0.94248,0.0045111)
(1.0996,0.0047364) (1.2566,0.0047612) (1.4137,0.0047619)
(1.5708,0.0047619) (1.7279,0.0047619) (1.885,0.0047612) (2.042,0.0047364) (2.1991,0.0045111) (2.3562,0.003683) (2.5133,0.0021779) (2.6704,0.0007557) (2.8274,0.00010807) (2.9845,0) (3.1416,0)};

\draw (axis cs:2.4,-0.002) node {\scriptsize $f(x)$};
\end{axis}

\node [right] at (myplot.right of origin) {\scriptsize $x$};
\node [above] at (myplot.above origin) {\scriptsize $y$};
\end{tikzpicture}
\caption{A plot of $f(x)$ and $g(x)$ from Example \ref{exa:ex_trigint2} and the Technology Note. \label{fig:trigint2}}
\end{center}
\end{figure}


\begin{example}{Integrating powers of sine and cosine}{Integrating powers of sine and cosine}
Evaluate $\ds\int\cos^4x\sin^2x\ dx$.
\end{example}

\begin{solution} 
The powers of sine and cosine are both even, so we employ the power--reducing formulas and algebra as follows.
\begin{align*}
\int \cos^4x\sin^2x\ dx &= \int\left(\frac{1+\cos(2x)}{2}\right)^2\left(\frac{1-\cos(2x)}2\right)\ dx \\
				&= \int\frac{1+2\cos(2x)+\cos^2(2x)}4\cdot\frac{1-\cos(2x)}2\ dx\\
				&=	\int \frac18\big(1+\cos(2x)-\cos^2(2x)-\cos^3(2x)\big)\ dx
\end{align*}
The $\cos(2x)$ term is easy to integrate, especially with Key Idea \ref{idea:linearsub}. The $\cos^2(2x)$ term is another trigonometric integral with an even power, requiring the power--reducing formula again. The $\cos^3(2x)$ term is a cosine function with an odd power, requiring a substitution as done before. We integrate each in turn below.

$$\int\cos(2x)\ dx = \frac12\sin(2x)+C.$$

$$\int\cos^2(2x)\ dx = \int \frac{1+\cos(4x)}2\ dx = \frac12\big(x+\frac14\sin(4x)\big)+C.$$

Finally, we rewrite $\cos^3(2x)$ as $$\cos^3(2x) = \cos^2(2x)\cos(2x) = \big(1-\sin^2(2x)\big)\cos(2x).$$
Letting $u=\sin(2x)$, we have $du = 2\cos(2x)\ dx$, hence
\begin{align*}
\int \cos^3(2x)\ dx &= \int\big(1-\sin^2(2x)\big)\cos(2x)\ dx\\
							&= \int \frac12(1-u^2)\ du\\
							&= \frac12\Big(u-\frac13u^3\Big)+C\\
							&=	\frac12\Big(\sin(2x)-\frac13\sin^3(2x)\Big)+C
\end{align*}

Putting all the pieces together, we have
\begin{align*}
\int \cos^4x\sin^2x\ dx &=\int \frac18\big(1+\cos(2x)-\cos^2(2x)-\cos^3(2x)\big)\ dx \\
					&= \frac18\Big[x+\frac12\sin(2x)-\frac12\big(x+\frac14\sin(4x)\big)-\frac12\Big(\sin(2x)-\frac13\sin^3(2x)\Big)\Big]+C \\
					&=\frac18\Big[\frac12x-\frac18\sin(4x)+\frac16\sin^3(2x)\Big]+C
\end{align*}
\end{solution}




\subsection*{Integrals of the form {\small{$\ds \int\sin(mx)\sin(nx)\ dx,$ $\ds\int \cos(mx)\cos(nx)\ dx$}}, and {\small{$\ds\int \sin(mx)\cos(nx)\ dx$.}}}


Functions that contain products of sines and cosines of differing periods are important in many applications including the analysis of sound waves. Integrals of the form 
$$\int\sin(mx)\sin(nx)\ dx,\quad \int \cos(mx)\cos(nx)\ dx \quad \text{and}\quad\int \sin(mx)\cos(nx)\ dx$$
are best approached by first applying the Product to Sum Formulas found in the back cover of this text, namely
\begin{align*}
\sin(mx)\sin(nx) &= \frac12\Big[\cos\big((m-n)x\big)-\cos\big((m+n)x\big)\Big] \\
\cos(mx)\cos(nx) &= \frac12\Big[\cos\big((m-n)x\big)+\cos\big((m+n)x\big)\Big] \\
\sin(mx)\cos(nx) &=	\frac12\Big[\sin\big((m-n)x\big)+\sin\big((m+n)x\big)\Big]
\end{align*}

\begin{example}{Integrating products of $\sin(mx)$ and $\cos(nx)$}{ex_trigint4}
Evaluate $\ds\int\sin(5x)\cos(2x)\ dx$.
\end{example}

\begin{solution}
The application of the formula and subsequent integration are straightforward:
\begin{align*}
\int\sin(5x)\cos(2x)\ dx &= \int \frac12\Big[\sin(3x)+\sin(7x)\Big]\ dx \\
												&= -\frac16\cos(3x) - \frac1{14}\cos(7x) + C
\end{align*}
\end{solution}


\section*{Integrals of the form $\ds\int\tan^mx\sec^nx\ dx$.}

When evaluating integrals of the form $\int \sin^mx\cos^nx\ dx$, the Pythagorean identity allowed us to convert even powers of sine into even powers of cosine, and vise--versa. If, for instance, the power of sine was odd, we pulled out one $\sin x$ and converted the remaining even power of $\sin x$ into a function using powers of $\cos x$, leading to an easy substitution.

The same basic strategy applies to integrals of the form $\int \tan^mx\sec^n x\ dx$, albeit a bit more nuanced. The following three facts will prove useful:
\begin{itemize}
\item $\frac{d}{dx}(\tan x) = \sec^2x$, 
\item $\frac{d}{dx}(\sec x) = \sec x\tan x$ , and 
\item	$1+\tan^2x = \sec^2x$ (the Pythagorean Theorem).
\end{itemize}

If the integrand can be manipulated to separate a $\sec^2x$ term with the remaining secant power even, or if a $\sec x\tan x$ term can be separated with the remaining $\tan x$ power even, the Pythagorean Theorem can be employed, leading to a simple substitution. This strategy is outlined in the following.





%Next, we turn our attention to products of secant and tangent.
%Some we already know how to do.
%$$\int \sec^2x\,dx = \tan x+C\qquad\qquad\int\sec x\tan x\,dx=\sec x+C$$


\begin{formulabox}[Integrals Involving Powers of Tangent and Secant]
Consider $\ds\int\tan^mx\sec^nx\ dx$, where $m,n$ are nonnegative integers.\index{integration!of trig. powers}
\begin{enumerate}
\item		If $n$ is even, then $n=2k$ for some integer $k$. Rewrite $\sec^nx$ as 
$$\sec^nx = \sec^{2k}x = \sec^{2k-2}x\sec^2x = (1+\tan^2x)^{k-1}\sec^2x.$$
Then
$$\int\tan^mx\sec^nx\ dx=\int\tan^mx(1+\tan^2x)^{k-1}\sec^2x\ dx = \int u^m(1+u^2)^{k-1}\ du,$$
where $u = \tan x$ and $du = \sec^2x\ dx$.

\item		If $m$ is odd, then $m=2k+1$ for some integer $k$. Rewrite $\tan^mx\sec^nx$ as
$$\tan^mx\sec^nx = \tan^{2k+1}x\sec^nx = \tan^{2k}x\sec^{n-1}x\sec x\tan x = (\sec^2x-1)^k\sec^{n-1}x\sec x\tan x.$$
Then
$$\int\tan^mx\sec^nx\ dx=\int(\sec^2x-1)^k\sec^{n-1}x\sec x\tan x\ dx = \int(u^2-1)^ku^{n-1}\ du,$$
where $u = \sec x$ and $du = \sec x\tan x\ dx$.

\item If $n$ is odd and $m$ is even, then $m=2k$ for some integer $k$. Convert $\tan^mx $ to $(\sec^2x-1)^k$. Expand the new integrand and use Integration By Parts, with $dv = \sec^2x\ dx$.

\item		If $m$ is even and $n=0$, rewrite $\tan^mx$ as
$$\tan^mx = \tan^{m-2}x\tan^2x = \tan^{m-2}x(\sec^2x-1) = \tan^{m-2}\sec^2x-\tan^{m-2}x.$$
So
$$\int\tan^mx\ dx = \underbrace{\int\tan^{m-2}\sec^2x\ dx}_{\text{\small apply rule \#1}}\quad - \underbrace{\int\tan^{m-2}x\ dx}_{\text{\small apply rule \#4 again}}.$$

\end{enumerate}
\end{formulabox}

The techniques described in items 1 and 2 are relatively straightforward, but the techniques in items 3 and 4 can be rather tedious. A few examples will help with these methods.\\


We can integrate $\tan x$ quite easily using substitution.\\

\begin{example}{Integrating Tangent}{int_tan}
Evaluate $\ds\int\tan x\,dx$.
\end{example}  

\begin{solution}
Note that $\ds\tan x = \frac{\sin x}{\cos x}$ and let $u=\cos x$, so that $du=-\sin x\,dx$. %, i.e., $\fbox{$dx$}=\fbox{$\ds\frac{du}{-\sin x}$}$:
$$\begin{array}{>{\displaystyle}r>{\displaystyle}c>{\displaystyle}l>{\displaystyle}l}
\int \tan x\,dx & = & \int \frac{\sin x}{\cos x}\,\fbox{$dx$} & \mbox{Rewriting $\tan x$} \\  
	&=&\int \frac{\sin x}{u}\,\fbox{$\ds\frac{du}{-\sin x}$}  & \mbox{Using the substitution}\\  
	&=&-\int \frac{1}{u}\,du  & \mbox{Cancelling and pulling the $-1$ out}\\  
	&=&-\ln|u|+C & \mbox{Using formula $\ds\int\frac{1}{u}\,dx=\ln|u|+C$}\\  
	& = & -\ln|\cos x|+C & \mbox{Replacing $u$ back in terms of $x$}\\  
	& = & \ln|\sec x|+C & \mbox{Using log properties and $\sec x=1/\cos x$}\\
\end{array}$$
\end{solution}

%Let's take a moment to realize this result! A common mistake is to believe that $\int\tan x\,dx$ is $\sec^2(x)+C$ -- this is \emph{not} true.

Higher poers of $ \tan(x) $ require the methods outlined above.

\begin{example}{Integrating Tangent Squared}{}
Evaluate $\ds\int\tan^2 x\,dx$.
\end{example} 

\begin{solution}
The power on tangent is even, so we are in the fourth case outlined in the method. Ao we exploit the fact that   $\ds\tan^2 x = \sec^2x-1$.
$$\begin{array}{>{\displaystyle}r>{\displaystyle}c>{\displaystyle}l>{\displaystyle}l}
\int \tan^2 x\,dx & = & \int \sec^2x-1\,dx & \mbox{Rewriting $\tan x$} \\
	&=&\tan x-x+C & \mbox{Since $\ds\int\sec^2x\,dx=\tan x+C$}\\
\end{array}$$
\end{solution}

In problems with tangent and secant, two integrals come up frequently:
$$\ds \int\sec^3x\,dx\qquad\mbox{and}\qquad\int\sec x\,dx.$$
Both have relatively nice expressions but they are a bit tricky to discover. 

First we do $\ds\int\sec x\,dx$, which we
will need to compute $\ds \int\sec^3x\,dx$.

\begin{example}{Integral of Secant}{Integral of Secant}
Evaluate $\ds\int\sec x\,dx$.
\end{example}

\begin{solution}
\begin{eqnarray*}
  \int\sec x\,dx&=&\int\sec x\,{\sec x +\tan x\over \sec x +\tan x}\,dx\cr
  &=&\int{\sec^2 x +\sec x\tan x\over \sec x +\tan x}\,dx.\cr
\end{eqnarray*}
Now let $u=\sec x +\tan x$, $\ds du=\sec x \tan x + \sec^2x\,dx$, exactly
the numerator of the function we are integrating. Thus
\begin{eqnarray*}
  \int\sec x\,dx&=&\int{\sec^2 x +\sec x\tan x\over \sec x +\tan x}\,dx\cr
	&=&\int{1\over u}\,du=\ln |u|+C\cr
  &=&\ln|\sec x +\tan x|+C.
\end{eqnarray*}
\end{solution}

Now we compute the integral $\ds \int\sec^3 x\,dx$.

\begin{example}{Integral of Secant Cubed}{Integral of Secant Cubed}
Evaluate $\ds\int\sec^3 x\,dx$.
\end{example}

\begin{solution}
\begin{eqnarray*}
  \sec^3x&=&{\sec^3x\over2}+{\sec^3x\over2}={\sec^3x\over2}+{(\tan^2x+1)\sec
    x\over 2}\cr
		\\
  &=&{\sec^3x\over2}+{\sec x \tan^2 x\over2}+{\sec x\over 2}\\
	\\
	&=&
  {\sec^3x+\sec x \tan^2x\over 2}+{\sec x\over 2}.
\end{eqnarray*}

We already know how to integrate $\sec x$, so we just need the first
quotient. This is ``simply'' a matter of recognizing the product rule
in action:
$$\int \sec^3x+\sec x \tan^2x\,dx=\sec x \tan x.$$
So putting these together we get 
$$
  \int\sec^3x\,dx={\sec x \tan x\over2}+{\ln|\sec x +\tan x|\over2}+C,
$$
Note: Once we learn a technique called Integration by Parts, we will see another way to solve this integral.
\end{solution}

%For products of secant and tangent it is best to use the following guidelines.
%
%\begin{formulabox}[Products of Secant and Tangent]
%	When evaluating $\ds\int\sec^mx \tan^nx\,dx$:  
%	\begin{enumerate}
%	\item \ffont{The power of secant is even ($m$ even):}\\  
%		(a) Use $u=\tan x$ and $du=\sec^2 x\,dx$.\\  
%		(b) Cancel $\sec^2 x$ by the substitution of $dx$, and be left with an even number of secants.\\  
%		(c) Use $\sec^2x=1+\tan^2x~(=1+u^2)$ to replace the leftover secants.  
%	\item \ffont{The power of tangent is odd ($n$ odd):}\\  
%		(a) Use $u=\sec x$ and $du=\sec x\tan x\,dx$.\\  
%		(b) Cancel one $\sec x$ and one $\tan x$ by the substitution of $dx$.\\  
%		\quad~~The number of remaining tangents is even.\\  
%		(c) Use $\tan^2x=\sec^2x-1~(=u^2-1)$ to replace the leftover tangents.  
%	\item \ffont{$m$ is even or $n$ is odd}:\\ Use either $1$ or $2$ (both will work).  
%	\item \ffont{The power of secant is odd and the power of tangent is even}:\\  
%	 No guidelines. Remember that $\ds\int\sec x\,dx$ and $\ds\int\sec^3 x\,dx$ can usually be looked up.
%	\end{enumerate}
%\end{formulabox}

%\begin{example}{Even Power of Secant}{}
%Evaluate $\ds\int\sec^6x\tan^6x\,dx$.
%\end{example}  
%
%\begin{solution}
%Since the power of secant is even, we ue $u=\tan x$, so that $du=\sec^2x\,dx$.
%$${\def\arraystretch{2.2}
%\begin{array}{>{\displaystyle}r>{\displaystyle}c>{\displaystyle}l>{\displaystyle}l}
%\int \sec^6x\tan^6 x\,dx &=&\int \sec^6x\,(u^6)\,\fbox{$\ds\frac{du}{\sec^2x}$}  & \mbox{Using the substitution}\\  
%	&=&\int \sec^4x(u^6)\,du  & \mbox{Cancelling a $\sec^2x$}\\  
%	&=&\int (\sec^2x)^2(u^6)\,du  & \mbox{Rewriting $\sec^4x$}\\  
%	&=&\int (1+\tan^2x)^2(u^6)\,du  & \mbox{Using $\sec^2x=1+\tan^2x$}\\  
%	&=&\int (1+u^2)^2(u^6)\,du  & \mbox{Using the substitution}\\  
%	\end{array}
%}$$
%To integrate this product the easiest method is expand it into a polynomial and integrate term-by-term.
%$${\def\arraystretch{2.2}
%\begin{array}{>{\displaystyle}r>{\displaystyle}c>{\displaystyle}l>{\displaystyle}l}
%\int \sec^6x\tan^6 x\,dx	&=&\int ( u^6+2u^8+u^{10})\,du  & \mbox{Expanding}\\  
%	&=&\frac{u^7}{7}+\frac{2u^9}{9}+\frac{u^{11}}{11}+C & \mbox{Integrating}\\  
%	& = & \frac{\tan^7x}{7}+\frac{2\tan^9x}{9}+\frac{\tan^{11}x}{11}+C & \mbox{Rewriting in terms of $x$}\\
%\end{array}
%}$$
%\end{solution}


\begin{example}{Even Power of Secant}{intevensec}
Evaluate $\ds\int \tan^2x\sec^6x\ dx$.
\end{example}  

\begin{solution}
Since the power of secant is even, we use rule \#1 and pull out a $\sec^2x$ in the integrand. We convert the remaining powers of secant into powers of tangent.
\begin{align*}
\int \tan^2x\sec^6x\ dx &= \int\tan^2x\sec^4x\sec^2x\ dx \\
		&= \int \tan^2x\big(1+\tan^2x\big)^2\sec^2x\ dx \\
\intertext{Now substitute, with $u=\tan x$, with $du = \sec^2x\ dx$.}
		&=\int u^2\big(1+u^2\big)^2\ du\\
\intertext{We leave the integration and subsequent substitution to the reader. The final answer is}
		&=\frac13\tan^3x+\frac25\tan^5x+\frac17\tan^7x+C.
\end{align*}
\end{solution}



%\begin{example}{Odd Power of Tangent}{}
%Evaluate $\ds\int\sec^5x\tan x\,dx$.
%\end{example} 
%
%\begin{solution}
%Since the power of tangent is odd, we use $u=\sec x$, so that $du=  \sec x\tan x\,dx$.
%Then we have:
%$$\begin{array}{>{\displaystyle}r>{\displaystyle}c>{\displaystyle}l>{\displaystyle}l}
%\int \sec^5x\tan x\,dx &=&\int \sec^5x\tan x\,\fbox{$\ds\frac{du}{\sec x\tan x}$}  & \mbox{Substituting $dx$ first}\\  
%	&=&\int \sec^4x\,du  & \mbox{Cancelling}\\  
%	&=&\int u^4\,du  & \mbox{Using the substitution}\\  
%	&=&\frac{u^5}{5}+C & \mbox{Integrating}\\  
%	& = & \frac{\sec^5x}{5}+C & \mbox{Rewriting in terms of $x$}\\
%\end{array}$$
%\end{solution}


\begin{example}{Odd Power of Tangent}{intoddtan}
Evaluate $\ds\int\sec^5x\tan x\,dx$.
\end{example} 

\begin{solution}
Since the power of tangent is odd, we use rule \# 2, and factor out $\sec x\tan x$ and substitute $ u=\sec(x)$.
Then we have:
$$\begin{array}{>{\displaystyle}r>{\displaystyle}c>{\displaystyle}l>{\displaystyle}l}
\int \sec^5x\tan x\,dx &=&\int \sec^4x\sec x\tan x\; dx  & \mbox{Substituting $dx$ first}\\  
	%&=&\int \sec^4x\,du  & \mbox{Cancelling}\\  
	&=&\int u^4\,du  & \mbox{Using the substitution}\\  
	&=&\frac{u^5}{5}+C & \mbox{Integrating}\\  
	& = & \frac{\sec^5x}{5}+C & \mbox{Rewriting in terms of $x$}\\
\end{array}$$
\end{solution}



\begin{example}{Odd Power of Secant and Even Power of Tangent}{}
Evaluate $\ds\int\sec x\tan^2x\,dx$.
\end{example}

\begin{solution}
The guidelines don't help us in this scenario. However, since $\ds\tan^2x=\sec^2x-1$, we have
$${\def\arraystretch{2.2}
\begin{array}{>{\displaystyle}r>{\displaystyle}c>{\displaystyle}l>{\displaystyle}l}
\int \sec x\tan^2x\,dx & = & \int \sec x(\sec^2x-1)  \,dx &  \\  
	& = & \int (\sec^3x-\sec x)\,dx  &  \\  
	& = & \frac{1}{2}\left(\sec x\tan x+\ln|\sec x+\tan x|\right) - \ln|\sec x+\tan x|+C &  \\  
	& = & \frac{1}{2}\sec x\tan x+\frac{1}{2}\ln|\sec x+\tan x| - \ln|\sec x+\tan x|+C &  \\  
	& = & \frac{1}{2}\sec x\tan x-\frac{1}{2}\ln|\sec x+\tan x|+C & \\ 
\end{array}
}$$
\end{solution}


\begin{example}{Integrating powers of tangent and secant}{ex_trigint7}
{
Evaluate $\ds\int\tan^6x\ dx$.
}
\end{example}

\begin{solution}
{We employ rule \#4. 
\begin{align*}
\int \tan^6x\ dx &= \int \tan^4x\tan^2x\ dx \\
			&= \int\tan^4x\big(\sec^2x-1\big)\ dx\\
			&= \int\tan^4x\sec^2x\ dx - \int\tan^4x\ dx \\
\intertext{Integrate the first integral with substitution, $u=\tan x$; integrate the second by employing rule \#4 again.}
			&=	\frac15\tan^5x-\int\tan^2x\tan^2x\ dx\\
			&=	\frac15\tan^5x-\int\tan^2x\big(\sec^2x-1\big)\ dx \\
			&= \frac15\tan^5x -\int\tan^2x\sec^2x\ dx + \int\tan^2x\ dx\\
\intertext{Again, use substitution for the first integral and rule \#4 for the second.}
			&= \frac15\tan^5x-\frac13\tan^3x+\int\big(\sec^2x-1\big)\ dx \\
			&=	 \frac15\tan^5x-\frac13\tan^3x+\tan x - x+C.
\end{align*}
\vskip-\baselineskip
}
\end{solution}


Some of these examples were admittedly long, with repeated applications of the same rule. Try to not be overwhelmed by the length of the problem, but rather admire how robust this solution method is. A trigonometric function of a high power can be systematically reduced to trigonometric functions of lower powers until all antiderivatives can be computed. 

The next section introduces an integration technique known as Trigonometric Substitution, a clever combination of Substitution and the Pythagorean Theorem.

%%%%%%%%%%%%%%%%%%%%%%%%%%%%%%%%%%%%%%%%%%%%
\Opensolutionfile{solutions}[ex]
\section*{Exercises for \ref{sec:Powers of trigonometric functions}}

\begin{enumialphparenastyle}

Find the antiderivatives.

%%%%%%%%%%
\begin{ex}
 $\ds\int \sin^2 x\,dx$
\begin{sol}
 $x/2-\sin(2x)/4+C$
\end{sol}
\end{ex}
%%%%%%%%%%

%%%%%%%%%%
\begin{ex}
 $\ds\int \sin^3 x\,dx$
\begin{sol}
 $\ds -\cos x+(\cos^3x)/3+C$
\end{sol}
\end{ex}
%%%%%%%%%%

%%%%%%%%%%
\begin{ex}
 $\ds\int \sin^4 x\,dx$
\begin{sol}
 $3x/8-(\sin 2x)/4+(\sin 4x)/32+C$
\end{sol}
\end{ex}
%%%%%%%%%%

%%%%%%%%%%
\begin{ex}
 $\ds\int \cos^2 x\sin^3 x\,dx$
\begin{sol}
 $\ds (\cos^5 x)/5-(\cos^3x)/3+C$
\end{sol}
\end{ex}
%%%%%%%%%%

%%%%%%%%%%
\begin{ex}
 $\ds\int \cos^3 x\,dx$
\begin{sol}
 $\ds \sin x-(\sin^3x)/3+C$
\end{sol}
\end{ex}
%%%%%%%%%%

%%%%%%%%%%
\begin{ex}
 $\ds\int \cos^3 x \sin^2 x\,dx$
\begin{sol}
 $\ds (\sin^3x)/3-(\sin^5x)/5+C$
\end{sol}
\end{ex}
%%%%%%%%%%

%%%%%%%%%%
\begin{ex}
 $\ds\int \sin x (\cos x)^{3/2}\,dx$
\begin{sol}
 $\ds -2(\cos x)^{5/2}/5+C$
\end{sol}
\end{ex}
%%%%%%%%%%

%%%%%%%%%%
\begin{ex}
 $\ds\int \sec^2 x\csc^2 x\,dx$
\begin{sol}
 $\tan x-\cot x+C$
\end{sol}
\end{ex}
%%%%%%%%%%

%%%%%%%%%%
\begin{ex}
 $\ds\int \tan^3x \sec x\,dx$
\begin{sol}
 $\ds (\sec^3x)/3-\sec x+C$
\end{sol}
\end{ex}
%%%%%%%%%%

%%%%%%%%%%
\begin{ex}
 $\ds\int \left(\frac{1}{\csc x}+\frac{1}{\sec x}\right)\,dx$
\begin{sol}
 $\ds -\cos x+\sin x+C$
\end{sol}
\end{ex}
%%%%%%%%%%

%%%%%%%%%%
\begin{ex}
 $\ds\int\frac{\cos^2x+\cos x+1}{\cos^3x}\,dx$
\begin{sol}
 $\ds \frac{3}{2}\ln|\sec x+\tan x|+\tan x+\frac{1}{2}\sec x\tan x+C$
\end{sol}
\end{ex}
%%%%%%%%%%

%%%%%%%%%%
\begin{ex}
 $\ds\int x\sec^2(x^2)\tan^4(x^2)\,dx$
\begin{sol}
 $\ds \frac{\tan^5(x^2)}{10}+C$
\end{sol}
\end{ex}
%%%%%%%%%%

\end{enumialphparenastyle}