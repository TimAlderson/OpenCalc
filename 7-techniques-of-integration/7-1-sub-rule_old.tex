\section{Substitution Rule}\label{sec:SubRule}
Needless to say, most integration problems we will encounter will
not be so simply. That is to say we will require more than the basic
integration rules we have seen. Here's a slightly more complicated example: Find
$$\int 2x\cos(x^2)\,dx.$$
This is not a ``simple'' derivative, but a little thought reveals that
it must have come from an application of the chain rule. Multiplied
on the ``outside'' is $2x$, which is the derivative of the ``inside''
function $\ds x^2$. Checking:
$${d\over dx}\sin(x^2)  = \cos(x^2){d\over dx}x^2 = 2x\cos(x^2),$$
so 
$$\int 2x\cos(x^2)\,dx=\sin(x^2)+C .$$

To summarize: If we suspect that a given function is the derivative of
another via the chain rule, we let $u$ denote a likely candidate for
the inner function, then translate the given function so that it is
written entirely in terms of $u$, with no $x$ remaining in the
expression. If we can integrate this new function of $u$, then the
antiderivative of the original function is obtained by replacing $u$
by the equivalent expression in $x$.

\begin{theorem}{Substitution Rule for Indefinite Integrals}{SubstitutionRule}
If $u=g(x)$ is a differentiable function whose range is an interval $I$ and $f$ is continuous on $I$, then
$$\int f(g(x))g'(x)\,dx=\int f(u)\,du.$$
\end{theorem}

Even in simple cases you may prefer to use this mechanical procedure,
since it often helps to avoid silly mistakes. For example, consider
again this simple problem:
$$\int 2x\cos(x^2)\,dx.$$
Let $\ds u=x^2$, then $du/dx = 2x$ or $du = 2x\,dx$. Since we have exactly 
$2x\,dx$ in the original integral, we can replace it by $du$:
$$\int 2x\cos(x^2)\,dx=\int \cos u\,du=\sin u +C = \sin(x^2)+C.$$
This is not the only way to do the algebra, and typically there are
many paths to the correct answer. Another possibility, for example,
is: Since $du/dx = 2x$, $dx=du/2x$, and then the integral becomes
$$\int 2x\cos(x^2)\,dx=\int 2x\cos u\,{du\over 2x}=\int \cos u\,du.$$
The important thing to remember is that you must eliminate all
instances of the original variable $x$.

\begin{example}{Substitution Rule}{SubstitutionRuleex}
Evaluate $\ds\int(ax+b)^n\,dx$, assuming $a,b$ are
constants, $a\not=0$, and $n$ is a positive integer.
\end{example}

\begin{solution} 
We let $u=ax+b$ so $du=a\,dx$ or $dx=du/a$. Then
$$
  \int(ax+b)^n\,dx=\int {1\over a} u^n\,du={1\over a(n+1)}u^{n+1}+C=
  {1\over a(n+1)}(ax+b)^{n+1}+C.
$$
\end{solution}

\begin{example}{Substitution Rule}{SubstitutionRulex}
Evaluate $\ds\int \sin(ax+b)\,dx$, assuming that $a$ and $b$ are
constants and $a\not=0$.
\end{example}

\begin{solution} 
Again we let $u=ax+b$ so $du=a\,dx$ or $dx=du/a$. Then
$$
  \int\sin(ax+b)\,dx=\int {1\over a} \sin u\,du={1\over a}(-\cos u)+C=
-{1\over a}\cos(ax+b)+C.
$$
\end{solution}

\begin{formulabox}[Strategy for Substitution Rule]
A general strategy to follow is:
\begin{enumerate}\setlength{\itemsep}{0 in}
\item Choose a possible $u=u(x)$. \dfont{Tip:} Choose a substitution $u$ so that its derivate also appears in the integral (up to a constant).
\item Calculate $du=u'(x)~dx$.
\item Either replace $u'(x)~dx$ by $du$, or replace $dx$ by $\ds{\frac{du}{u'(x)}}$, and cancel.
\item Write the rest of the integrand in terms of $u$. If this is not possible, the substitution will not work: You must go back to step 1.
\item Find the indefinite integral. (Again, if this is not possible, try a different substitution, or a different method).
\item Rewrite the result in terms of $x$.
\end{enumerate}
\end{formulabox}

\begin{example}{Substitution}{Substitution}
Evaluate the following integral: $\ds\int \frac{2x}{\sqrt{1-4x^2}}\,dx.$
\end{example}

\begin{solution} 
We try the substitution:
$$u=1-4x^2.$$
Then,
$$du=-8x~dx$$
In the numerator we have $2x~dx$, so rewriting the differential gives:
$$-\frac{1}{4}du=2x~dx.$$
Then the integral is:
\begin{eqnarray*}
\int \frac{2x}{\sqrt{1-4x^2}}\,dx&=&\int \left(1-4x^2\right)^{-1/2}(2x~dx)\\
\\
&=&\int u^{-1/2}\left(-\frac{1}{4}du\right)\\
\\
&=&\left(\frac{-1}{4}\right)\frac{u^{1/2}}{1/2}+C\\
\\
&=&-\frac{\sqrt{1-4x^2}}{2}+C
\end{eqnarray*}
\end{solution}



\begin{example}{Substitution}{Substitution}
Evaluate the following integral: $\ds\int \sech^2(7t-3)\ dt$
\end{example}

\begin{solution} 
 We employ substitution, with $u = 7t-3$ and $du = 7dt$. We have:
$$ \int \sech^2 (7t-3)\ dt=  \frac17 \int \sech^2 (u)\ du= \frac17\tanh (u) + C = \frac17\tanh (7t-3) + C.$$
\end{solution}





\begin{example}{Substitution}{Substitution2}
Evaluate the following integral: $\ds\int \cos x(\sin x)^5\,dx.$
\end{example}

\begin{solution} 
In this question we will let $u=\sin x$.
Then,
$$du=\cos x~dx.$$
Thus, the integral becomes:
\begin{eqnarray*}
\int \cos x(\sin x)^5\,dx&=&\int u^5\,du\\
\\
&=&\frac{u^6}{6}+C\\
\\
&=&\frac{(\sin x)^6}{6}+C
\end{eqnarray*}
\end{solution}

\begin{example}{Substitution}{Substitution3}
Evaluate the following integral:
$\ds\int \frac{\cos(\sqrt x)}{\sqrt x}\,dx.$
%\vspace{-0.5cm}
\end{example}

\begin{solution} 
We use the substitution:
$$u=x^{1/2}.$$
Then,
$$du=\frac{1}{2}x^{-1/2}dx.$$
Rewriting the differential we get:
$$2~du=\frac{1}{\sqrt x}~dx.$$
The integral becomes:
\begin{eqnarray*}
\int \frac{\cos(\sqrt x)}{\sqrt x}\,dx&=&2\int \cos u\,du\\
\\
&=&2\sin u+C\\
\\
&=&2\sin(\sqrt x)+C
\end{eqnarray*}
\end{solution}

\begin{example}{Substitution}{Substitution4}
Evaluate the following integral:
$\ds\int 2x^3\sqrt{x^2+1}\,dx.$
%\vspace{-0.5cm}
\end{example}

\begin{solution} 
This problem is a little bit different than the previous ones.
It makes sense to let:
$$u=x^2+1,$$
then
$$du=2x~dx.$$
Making this substitution gives:
\begin{eqnarray*}
\int 2x^3\sqrt{x^2+1}\,dx&=&\int x^2\sqrt{x^2+1}(2x)\,dx\\
\\
&=&\int x^2u^{1/2}\,du\\
\end{eqnarray*}
This is a problem because our integrals can't have a mixture of two variables in them.
Usually this means we chose our $u$ incorrectly.
However, in this case we can eliminate the remaining $x$'s from our integral by using:
$$u=x^2+1\quad\to\quad x^2=u-1.$$
We get:
\begin{eqnarray*}
\int x^2u^{1/2}\,du&=&\int (u-1)u^{1/2}\,du\\
\\
&=&\int u^{3/2}-u^{1/2}\,du\\
\\
&=&\frac{2}{5}u^{5/2}-\frac{2}{3}u^{3/2}+C\\
\\
&=&\frac{2}{5}(x^2+1)^{5/2}-\frac{2}{3}(x^2+1)^{3/2}+C
\end{eqnarray*}
\end{solution}

The next example shows how to use the Substitution Rule when dealing with definite integrals.

\begin{example}{Substitution Rule}{SubstitutionRuledefex}
Evaluate $\ds\int_2^4 x\sin(x^2)\,dx$. 
\end{example}

\begin{solution} 
First we compute the
antiderivative, then evaluate the integral.
Let $\ds u=x^2$, so $du=2x\,dx$ or $x\,dx=du/2$. Then
$$
  \int x\sin(x^2)\,dx=\int {1\over 2} \sin u\,du={1\over 2}(-\cos u)+C=
  -{1\over 2}\cos(x^2)+C.
$$
Now
$$
  \int_2^4 x\sin(x^2)\,dx=\left.-{1\over 2}\cos(x^2)\right|_2^4
  =-{1\over 2}\cos(16)+{1\over 2}\cos(4).
$$
A somewhat neater alternative to this method is to change the original
limits to match the variable $u$. Since $\ds u=x^2$, when $x=2$, $u=4$,
and when $x=4$, $u=16$. So we can do this:
$$
  \int_2^4 x\sin(x^2)\,dx=
  \int_4^{16} {1\over 2} \sin u\,du=\left.-{1\over 2}(\cos u)\right|_4^{16}
  =-{1\over 2}\cos(16)+{1\over 2}\cos(4).
$$
An incorrect, and dangerous, alternative is something like this:
$$
  \int_2^4 x\sin(x^2)\,dx=\int_2^4 {1\over 2} \sin u\,du=
  \left.-{1\over 2}\cos (u)\right|_2^4=
  \left.-{1\over 2}\cos(x^2)\right|_2^4=-{1\over 2}\cos(16)+{1\over
  2}\cos(4).
$$
This is incorrect because $\ds\int_2^4 {1\over 2} \sin u\,du$
means that $u$ takes on values between 2 and 4, which is wrong. It
is dangerous, because it is very easy to get to 
the point $\ds\left.-{1\over 2}\cos (u)\right|_2^4$ and forget to substitute
$\ds x^2$ back in for $u$, thus getting the incorrect answer
$\ds -{1\over 2}\cos(4)+{1\over 2}\cos(2)$. An acceptable alternative is something like:
$$ 
  \int_2^4 x\sin(x^2)\,dx=\int_{x=2}^{x=4} {1\over 2} \sin u\,du=
  \left.-{1\over 2}\cos (u)\right|_{x=2}^{x=4}=
  \left.-{1\over 2}\cos(x^2)\right|_2^4=-{\cos(16)\over 2}+{\cos(4)\over2}.
$$
\end{solution}

To summarize, we have the following.

\begin{theorem}{Substitution Rule for Definite Integrals}{SubstitutionRuledef}
If $g'$ is continuous on $[a,b]$ and $f$ is continuous on the range of $u=g(x)$, then
$$\int_a^b f(g(x))g'(x)\,dx=\int_{g(a)}^{g(b)}f(u)\,du.$$
\end{theorem}

\begin{example}{Substitution Rule}{SubstitutionRuledef2}
Evaluate $\ds\int_{1/4}^{1/2}{\cos(\pi t)\over\sin^2(\pi t)}\,dt$. 
\end{example}

\begin{solution} 
Let $u=\sin(\pi t)$ so $du=\pi\cos(\pi t)\,dt$ or $du/\pi=\cos(\pi
t)\,dt$.
We change the limits to $\ds \sin(\pi/4)=\sqrt2/2$ and 
$\sin(\pi/2)=1$.
Then
$$
  \int_{1/4}^{1/2}{\cos(\pi t)\over\sin^2(\pi t)}\,dt=
  \int_{\sqrt2/2}^{1}{1\over \pi}{1\over u^2}\,du=
  \int_{\sqrt2/2}^{1} {1\over \pi}u^{-2}\,du=
  \left.{1\over \pi}{u^{-1}\over -1}\right|_{\sqrt2/2}^{1}=
  -{1\over\pi}+{\sqrt2\over\pi}.
$$
\end{solution}


%%%%%%%%%%%%%%%%%%%%%%%%%%%%%%%%%%%%%%%%%%%%%%%%%
\Opensolutionfile{solutions}[ex]
\section*{Exercises for Section \ref{sec:SubRule}}

\begin{enumialphparenastyle}

Find the following indefinite and definite integrals.

%%%%%%%%%%
\begin{ex}
 $\ds\int (1-t)^9\,dt$
\begin{sol}
 $\ds -(1-t)^{10}/10+C$
\end{sol}
\end{ex}

%%%%%%%%%%
\begin{ex}
 $\ds\int (x^2+1)^2\,dx$
\begin{sol}
 $\ds x^5/5+2x^3/3+x+C$
\end{sol}
\end{ex}

%%%%%%%%%%
\begin{ex}
 $\ds\int x(x^2+1)^{100}\,dx$
\begin{sol}
 $\ds (x^2+1)^{101}/202+C$
\end{sol}
\end{ex}

%%%%%%%%%%
\begin{ex}
 $\ds\int {1\over\root 3 \of {1-5t}}\,dt$ 
\begin{sol}
 $\ds -3(1-5t)^{2/3}/10+C$
\end{sol}
\end{ex}

%%%%%%%%%%
\begin{ex}
 $\ds\int \sin^3x\cos x\,dx$
\begin{sol}
 $\ds (\sin^4x)/4+C$
\end{sol}
\end{ex}

%%%%%%%%%%
\begin{ex}
 $\ds\int x\sqrt{100-x^2}\,dx$
\begin{sol}
 $\ds -(100-x^2)^{3/2}/3+C$
\end{sol}
\end{ex}

%%%%%%%%%%
\begin{ex}
 $\ds\int {x^2\over\sqrt{1-x^3}}\,dx$
\begin{sol}
 $\ds \ds -2\sqrt{1-x^3}/3+C$
\end{sol}
\end{ex}

%%%%%%%%%%
\begin{ex}
 $\ds\int \cos(\pi t)\cos\bigl(\sin(\pi t)\bigr)\,dt$
\begin{sol}
 $\ds \sin(\sin\pi t)/\pi+C$
\end{sol}
\end{ex}

%%%%%%%%%%
\begin{ex}
 $\ds\int {\sin x\over\cos^3 x}\,dx$
\begin{sol}
 $\ds \ds 1/(2\cos^2 x)=(1/2)\sec^2x+C$
\end{sol}
\end{ex}

%%%%%%%%%%
\begin{ex}
 $\ds\int\tan x\,dx$
\begin{sol}
 $-\ln|\cos x|+C$
\end{sol}
\end{ex}

%%%%%%%%%%
\begin{ex}
  $\ds\int_0^\pi\sin^5(3x)\cos(3x)\,dx$
\begin{sol}
 $0$
\end{sol}
\end{ex}

%%%%%%%%%%
\begin{ex}
 $\ds\int\sec^2x\tan x\,dx$
\begin{sol}
 $\ds \tan^2(x)/2+C$
\end{sol}
\end{ex}

%%%%%%%%%%
\begin{ex}
 $\ds\int_0^{\sqrt{\pi}/2} x\sec^2(x^2)\tan(x^2)\,dx$
\begin{sol}
 $1/4$
\end{sol}
\end{ex}

%%%%%%%%%%
\begin{ex}
 $\ds\int {\sin(\tan x)\over\cos^2x}\,dx$
\begin{sol}
 $-\cos(\tan x)+C$
\end{sol}
\end{ex}

%%%%%%%%%%
\begin{ex}
 $\ds\int_3^4 {1\over(3x-7)^2}\,dx$
\begin{sol}
 $1/10$
\end{sol}
\end{ex}

%%%%%%%%%%
\begin{ex}
 $\ds\int_0^{\pi/6}(\cos^2x - \sin^2x)\,dx$
\begin{sol}
 $\ds \sqrt3/4$
\end{sol}
\end{ex}

%%%%%%%%%%
\begin{ex}
 $\ds\int {6x\over(x^2 - 7)^{1/9}}\,dx$
\begin{sol}
 $\ds (27/8)(x^2-7)^{8/9}$
\end{sol}
\end{ex}

%%%%%%%%%%
\begin{ex}
 $\ds\int_{-1}^1 (2x^3-1)(x^4-2x)^6\,dx$
\begin{sol}
 $\ds -(3^7+1)/14$
\end{sol}
\end{ex}

%%%%%%%%%%
\begin{ex}
 $\ds\int_{-1}^1 \sin^7 x\,dx$
\begin{sol}
 $0$
\end{sol}
\end{ex}

%%%%%%%%%%
\begin{ex}
 $\ds\int f(x) f'(x)\,dx$ 
\begin{sol}
 $\ds f(x)^2/2$
\end{sol}
\end{ex}

\end{enumialphparenastyle}