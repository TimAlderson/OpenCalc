\section{Introduction to Functions}\label{sec:Functions}
A \dfont{function} $y=f(x)$ is a rule for determining
$y$ when we're given a value of $x$.  For example, the rule
$y=f(x)=2x+1$ is a function.  Any line $y=mx+b$ is called a \dfont{linear} 
function.  The graph of a function looks like a curve 
above (or below) the $x$-axis, where for any value
of $x$ the rule $y=f(x)$ tells us how far to go above (or below) the
$x$-axis to reach the curve.

\begin{definition}{FUNCTION}
	AA {\bf{function}} is a relation between a set of inputs and a set of permissible outputs with the property that each input is related to exactly one output. That is, a function $f$ assigns to each element $x$ in a set $D$ exactly one element, called $f(x)$ in a set $E$.
\end{definition}	

A function $f$ is defined by its set of inputs, called the {\bf{domain}} of $f$. The inputs are also referred to as the {\bf{independent variable}} of the function. The set of permissible outputs is called the {\bf{range}} of $f$ (or image of the function). Since outputs depend on the inputs, the output is referred to as the {\bf{dependent variable}}. For a function written in the form, $\displaystyle{f(x)}$, $x$ is also referred to as the {\bf{argument}} of the function. Thus, for $g(4+h)$ the input $4+h$ is the argument for the function $g$.  \\


\begin{figure}[H]
	$$\includegraphics[width=4.5in]{images/Domain1}$$
	\caption{Arrow diagrams showing (a) function and (b) non-function. \label{fig:arrowDiagrams}} 
\end{figure}	

The diagram in Figure~\xrefn{fig:arrowDiagrams} (a) represents a function with domain $\{1, 2, 3\}$ and set of ordered pairs $\{(1,D), \,\,(2,C), \,\,(3,C)\}$. The range of $f$ is $\{C, \,\,D\}$. However, this second diagram (Figure~\xrefn{fig:arrowDiagrams} (b)) does not represent a function. One reason is that $x=2$ corresponds to two values of $y$, $y=B$ and $y=C$.  \\

A useful way to visualize a function is by sketching its graph. If $f$ is a function with domain $D$, then its graph is the set of ordered pairs $\ds{(x,f(x))}$. Since the y-coordinate of ant point $(x,y)$ on the graph is $y=f(x)$, we can read the value of $f(x)$ from the graph as being the height of the graph corresponding to $x$. The graph of $f$ also allows us to see visualize the domain and range of the function. \\




Functions can be defined in various ways: by an algebraic formula or several
algebraic formulas, by a graph, or by an experimentally determined 
table of values. In the latter case, the table gives a bunch of 
points in the plane, which we might then
interpolate with a smooth curve, if that makes sense.

\vspace{3mm} 
Given a value of $x$, a function must give
at most one value of $y$.  Thus, vertical lines are not functions.  
%For example, the line $x=1$ has infinitely many values of $y$ if $x=1$. It
%is also true that 
%if $x$ is any number (not 1) there is no $y$ which corresponds to $x$,
%but that is not a problem---only multiple $y$ values is a problem.

One test to identify whether or not a curve in the $(x,y)$ coordinate 
system is a function is the following.

\begin{theorem}{The Vertical Line Test}{Vertical Line Test}
	\label{VertLineTest}
A curve in the $(x,y)$ coordinate system represents a function if and only if no vertical line intersects the curve more than once.
\end{theorem}

\begin{figure}[H]
	$$\includegraphics[width=5in]{images/VerticalLines1}$$
	
	\caption{(a) Vertical line intersects curve once for all $x$, therefore the curve represents a function. (b) Vertical Line intersects curve more than once for $x$, therefore this curve does not represent a function. }
\end{figure}

\begin{example}{Identifying Functions}{IDFunctions}
	Determine which equations represent $y$ as a function of $x$.  \\
	
\hspace{4mm} (a) \hspace{2mm} $x^3 + y^2 = 1$ \hspace{1.5cm} (b) \hspace{2mm} $x^2 + y^3 = 1$
	\hspace{1.5cm} (c) \hspace{2mm} $x^2y = 1 - 3y$
\end{example}
\begin{solution}	
For each of these equations, we solve for $y$ and determine whether each choice of $x$ will determine only one corresponding value of $y$.
	
(a) 		
 \[ \begin{array}{rclr} 
		x^3 + y^2 & = & 1 & \\
		y^2 & = & 1 - x^3 & \\
		\sqrt{y^2} & = & \sqrt{1 - x^3} & \mbox{extract square roots} \\
		y & = & \pm \sqrt{1 - x^3} & \\ 
		\end{array} \]
		
If we substitute $x=0$ into our equation for $y$, we get  $y = \pm \sqrt{1 - 0^3} = \pm 1$, so that $(0,1)$ and $(0,-1)$ are on the graph of this equation. Hence, this equation does not represent $y$ as a function of $x$. As predicted, the graph of $x^3+y^2=1$, shown in Figure~\xrefn{fig:graphFT}(a), clearly fails the Vertical Line Test (Theorem \ref{VertLineTest}), so the equation does not represent $y$ as a function of $x$.\\   

\vspace{3mm}
		
(b)
  \[ \begin{array}{rclr} 
		x^2 + y^3 & = & 1 & \\
		y^3 & = & 1 - x^2 & \\
		\sqrt[3]{y^3} & = & \sqrt[3]{1 - x^2} & \\
		y & = & \sqrt[3]{1 - x^2} & \\ 
		\end{array} \]
		
For every choice of $x$, the equation $y =  \sqrt[3]{1 - x^2}$ returns only \textbf{one} value of $y$.  Hence, this equation describes $y$ as a function of $x$. As predicted, the graph of $x^2+y^3=1$, shown in Figure~\xrefn{fig:graphFT}(b), clearly passes the Vertical Line Test (Theorem \ref{VertLineTest}), so the equation does represent $y$ as a function of $x$.\\   
		
(c) 
  \[ \begin{array}{rclr} 
		x^2y & = & 1 - 3y & \\
		x^2y + 3y & = & 1 & \\
		y \left(x^2 + 3\right) & = & 1 & \mbox{factor} \\
		y & = & \dfrac{1}{x^2 + 3} & \\ 
		\end{array} \]
     For each choice of $x$, there is only one value for $y$, so this equation describes $y$ as a function of~$x$. 		
     As predicted, the graph of $x^2y=1-3y$, shown in Figure~\xrefn{fig:graphFT}(c), clearly passes the Vertical Line Test (Theorem \ref{VertLineTest}), so the equation does represent $y$ as a function of $x$.\\   

\begin{figure}[H]	
$$\includegraphics[scale=0.38]{images/TestFunction-ex1}$$
\caption{\label{fig:graphFT}} 
\end{figure}
\end{solution}


\subsection{Function Notation and Domain}

 In this section, we focus more on the \textbf{process} \index{function ! as a process} by which the $x$ is matched with the $y$.  If we think of the domain of a function as a set of \textbf{inputs} and the range as a set of \textbf{outputs}, we can think of a function $f$ as a process by which each input $x$ is matched with only one output $y$.  Since the output is completely determined by the input $x$ and the process $f$, we symbolize the output with \index{function ! notation} \textbf{function notation}: `$f(x)$', read `$f$ \textbf{of} $x$.' In other words, $f(x)$ is the output which results by applying the process $f$ to the input $x$.  In this case, the parentheses here do not indicate multiplication, as they do elsewhere in Algebra.  This can cause confusion if the context is not clear, so you must read carefully. \\
 
 The value of $y$ is completely dependent on the choice of $x$.  For this reason,  $x$ is often called the \index{variable ! independent} \index{independent variable} \index{function ! independent variable of} \textbf{independent variable}, or \index{function ! argument} \index{argument ! of a function} \textbf{argument} of $f$, whereas $y$ is often called the \index{variable ! dependent} \index{dependent variable} \index{function ! dependent variable of} \textbf{dependent variable}. 
 
 \medskip
 
 As we shall see, the process of a function $f$ is usually described using an algebraic formula. For example, suppose a function $f$ takes a real number and performs the following two steps, in sequence
 
 \begin{enumerate}
 	
 	\item  multiply by 3
 	
 	\item  add 4
 	
 \end{enumerate}
 
 If we choose $5$ as our input,  in step 1 we multiply by $3$ to get $(5)(3) = 15$.  In step 2, we add 4 to our result from step 1 which yields $15 + 4 = 19$.  Using function notation, we would write  $f(5) = 19$ to indicate that the result of applying the process $f$ to the input $5$ gives the output $19$.  In general, if we use $x$ for the input, applying step 1 produces $3x$.  Following with step 2 produces $3x+4$ as our final output.  Hence for an input $x$, we get the output $f(x) = 3x + 4$.  Notice that to check our formula for the case $x=5$, we replace the occurrence of $x$ in the formula for $f(x)$ with $5$ to get $f(5) = 3(5) + 4 = 15 + 4 = 19$, as required.


\begin{minipage}{0.4\textwidth}
In addition to lines, another familiar example of a  
function is the parabola $f(x)=x^2$.  We can draw the graph of this
function by taking various values of $x$ (say, at regular intervals) and
plotting the points $(x,f(x))=(x,x^2)$.  Then connect the points with a
smooth curve.  (See Figure~\xrefn{fig:graphx2}.)\\
\end{minipage} 
\begin{minipage}{0.3\textwidth}
	$$\begin{array}{r|r}
		x & y=x^{2}\\ \hline
	\vdots & \vdots \\	
	-3 & (-3)^{2} = 9 \\
	-2 & (-2)^{2} = 4 \\
	-1 & (-1)^{2} = 1 \\
	0 & (0)^{2} = 0 \\
	1 & (1)^{2} = 1 \\
	2 & (2)^{2} = 4 \\
	3 & (3)^{2} = 9 \\
	\vdots & \vdots \\
\end{array} $$	
\end{minipage}
\begin{minipage}{0.3\textwidth}
	\begin{figure}[H]
	$$\includegraphics[scale=0.25]{images/Plotx2}$$
	\caption{ \label{fig:graphx2}}
	\end{figure} 
\end{minipage}
	

The two examples $f(x)=2x+1$ and $f(x)=x^2$ are both functions which
can be evaluated at {\it any} value of $x$ from negative infinity to
positive infinity.  For many functions, however, it only makes sense to
take $x$ in some interval or outside of some ``forbidden'' region.  The
interval of $x$-values at which we're allowed to evaluate the function is
called the \dfont{domain} of the function.

\begin{minipage}{0.6\textwidth}
	\begin{example}{Graph of a function}{GraphFunction}
		Given the graph of $f$ shown in Figure~\xrefn{fig:graphF}\\
		(a)  Find the values of $f(-2)\, , \,\, f(3) \, , \,$ and $\, f(5)$. \\
		(b) What are the domain and range of $f$?\\
	\end{example}
	\vspace{2.5cm}
\end{minipage}
\begin{minipage}{0.4\textwidth}
	\begin{figure}[H]
		$$\includegraphics[width=2.2in]{images/graphFunction1}$$
		\caption{ \label{fig:graphF}} 
	\end{figure}
\end{minipage}

\vspace{-2.5cm}
\begin{minipage}{0.6\textwidth}
	\begin{solution}
		(a) We see from the Figure  that the point $(-2, \,-3)$ lies on the graph of $f$, so the value of $f$ at $x=-2$ is $f(-2)=-3$. From the graph, when $x=3$, $f(3)=-1$ and when $x=5$, $f(5)=4$. \\
		(b) The graph of $f(x)$ is defined for $\displaystyle{-6 \leq x \leq 6}$, so the domain of $f$ is the closed interval $[-6, \, 6]$. The range of $f$ is the set of all $y$ -values obtained from evaluating the function on its domain. Therefore, the range of $f$ is the  closed interval from $[-3, \, 6]$. 
	\end{solution}
\end{minipage}	

\begin{example}{Evaluating a function}
	IIf $\,\,\displaystyle{f(x)=x^{2}-3x+5}$, \hspace{2mm} find  $\displaystyle{f(-1)} \,$ , \hspace{2mm}  $\displaystyle{f(0)} \, $ , \hspace{2mm}  $\displaystyle{f(11)}$
\end{example}
\begin{solution}
	We first evaluate $f(-1)$ by replacing $x$ with the argument $-1$ in the expression \\ $\displaystyle{f(x)=x^{2}-3x+5}$. This yields \hspace{2mm} $\displaystyle{f(-1)=(-1)^{2}-3(-1)+5= 9}$. \\
	
	Similarily,   \hspace{2mm} $\displaystyle{f(0)=(0)^{2}-3(0)+5= 5}$ \hspace{2mm} and  \hspace{2mm}  $\displaystyle{f(11)=(11)^{2}-3(11)+5= 9}$ \\
\end{solution}

\begin{example}{Evaluating a function}{EvalFn}
	Find and simplify the following.
	
	\begin{enumerate}
		
		\item $f(-1)$, $f(0)$, $f(2)$
		
		\item  $f(2x)$, $2 f(x)$
		
		\item $f(x+2)$, $f(x)+2$, $f(x) + f(2)$
		
	\end{enumerate}
\end{example}
\begin{solution}
	
	 \begin{enumerate} \item  To find $f(-1)$, we replace every occurrence of $x$ in the expression $f(x)$ with $-1$
			
			\[ \begin{array}{rclr}  
			f(-1) & = & -(-1)^2 + 3(-1) + 4 & \\
			& = & -(1) + (-3) + 4 & \\ 
			& = & 0 & \\ 
			\end{array} \]
			
			
			Similarly, $f(0) = -(0)^2 + 3(0) + 4 = 4$, \\
			
			and \hspace{9mm} $f(2) = -(2)^2 + 3(2) + 4 = -4+6+4 = 6$.
			
			\item To find $f(2x)$, we replace every occurrence of $x$ with the quantity $2x$
			
			\[ \begin{array}{rclr}  
			f(2x) & = & -(2x)^2 + 3(2x) + 4 & \\
			& = & -(4x^2) + (6x) + 4 & \\
			& = & -4x^2+6x+4 & \\ 
			\end{array} \]
			
			The expression $2f(x)$ means we multiply the expression $f(x)$ by $2$
			
			\[ \begin{array}{rclr}  
			2f(x) & = & 2\left(-x^2 + 3x + 4\right) & \\
			& = & -2x^2 + 6x + 8 \\ 
			\end{array} \]
			
			
			\item  To find $f(x+2)$, we replace every occurrence of $x$ with the quantity $x+2$
			
			\[ \begin{array}{rclr}  
			f(x+2) & = & -(x+2)^2 + 3(x+2) + 4 & \\
			& = & -\left(x^2 + 4x + 4\right) + (3x+6) + 4 & \\
			& = & -x^2-4x-4+3x+6+4 &  \\
			& = & -x^2-x+6 & 
			\end{array} \]
			
			To find $f(x)+2$, we add $2$ to the expression for $f(x)$
			
			\[ \begin{array}{rclr}  
			f(x) + 2 & = & \left(-x^2 + 3x + 4\right) + 2  & \\
			& = & -x^2 + 3x + 6 \\ 
			\end{array} \]
			
			From our work above, we see $f(2) = 6$ so that
			
			\[ \begin{array}{rclr}  
			f(x) + f(2) & = & \left(-x^2 + 3x + 4\right) + 6  & \\
			& = & -x^2 + 3x + 10 \\ 
			\end{array} \]
			
		\end{enumerate}
		

A few notes about the previous example are in order.  First note the difference between the answers for $f(2x)$ and $2f(x)$.  For $f(2x)$, we are multiplying the \textit{input} by $2$;  for $2 f(x)$, we are multiplying the \textit{output} by $2$.  As we see, we get entirely different results.  Along these lines, note that $f(x+2)$, $f(x) + 2$ and $f(x) + f(2)$ are three \textit{different} expressions as well.  Even though function notation uses parentheses, as does multiplication, there is \textit{no} general `distributive property' of function notation. Finally, note the practice of using parentheses when substituting one algebraic expression into another;  we highly recommend this practice as it will reduce careless errors. 
\end{solution}


\begin{example}{Evaluating a function}
	IIf $\,\,\displaystyle{f(x)=\sqrt{x^{2}+1}-\frac{3}{5-x}}$, \hspace{2mm} evaluate $f(-x)\,$  \hspace{2mm} and  \hspace{2mm}  $f(a+h)$.
\end{example}
\begin{solution}
	
	$$\begin{array}{rclrcl }
	f(-x) & = & \displaystyle{\sqrt{(-x)^{2}+1}-\frac{3}{5-(-x)}}  & \hspace{1.5cm} f(a+h) & = & \displaystyle{\sqrt{(a+h)^{2}+1}-\frac{3}{5-(a+h)}} \\
	&&&&& \\
	& = & \displaystyle{\sqrt{x^{2}+1}-\frac{3}{5+x}}         && = & \displaystyle{\sqrt{a^{2}+2ah+h^{2}+1}-\frac{3}{5-a-h}} \\
	\end{array}$$
\end{solution}                    







%\begin{figure}[h]
%%\centerline{
%\hbox to \hsize{\vbox{\beginpicture
%\normalgraphs
%%\ninepoint
%\setcoordinatesystem units <0.7truecm,0.7truecm>
%\setplotarea x from -3 to 3, y from -3 to 3
%\axis left shiftedto x=0 /
%\axis bottom shiftedto y=0 /
%\setquadratic
%\plot -2.5 3 0 0 2.5 3 /
%\put {$f(x)=x^2$} [t] <0pt,-5pt> at 0 -3
%\endpicture }\hfill
%\vbox{\beginpicture
%\normalgraphs
%%\ninepoint
%\setcoordinatesystem units <0.7truecm,0.7truecm>
%\setplotarea x from -3 to 3, y from -3 to 3
%\axis left shiftedto x=0 /
%\axis bottom shiftedto y=0 /
%\setquadratic
%\plot 0 0 1 1 3 1.732 /
%\put {$f(x)=\sqrt{x}$} [t] <0pt,-5pt> at 0 -3
%\endpicture }\hfill
%\vbox{\beginpicture 
%\normalgraphs
%%\ninepoint
%\setcoordinatesystem units <0.7truecm,0.7truecm>
%\setplotarea x from -3 to 3, y from -3 to 3
%\axis left shiftedto x=0 /
%\axis bottom shiftedto y=0 /
%\setquadratic
%\plot -3.000 -0.333 -2.867 -0.349 -2.733 -0.366 -2.600 -0.385 -2.467 -0.405 
%-2.333 -0.429 -2.200 -0.455 -2.067 -0.484 -1.933 -0.517 -1.800 -0.556 
%-1.667 -0.600 -1.533 -0.652 -1.400 -0.714 -1.267 -0.789 -1.133 -0.882 
%-1.000 -1.000 -0.867 -1.154 -0.733 -1.364 -0.600 -1.667 -0.467 -2.143 
%-0.333 -3.000 /
%\plot 0.333 3.000 0.467 2.143 0.600 1.667 0.733 1.364 0.867 1.154 
%1.000 1.000 1.133 0.882 1.267 0.789 1.400 0.714 1.533 0.652 
%1.667 0.600 1.800 0.556 1.933 0.517 2.067 0.484 2.200 0.455 
%2.333 0.429 2.467 0.405 2.600 0.385 2.733 0.366 2.867 0.349 
%3.000 0.333 /
%\put {$f(x)=1/x$} [t] <0pt,-5pt> at 0 -3
%\endpicture}}
%%}
%\caption{Some graphs. \label{fig:some graphs}}
%
%\end{figure} 

\begin{example}{Domain of an a quadratic function}{FindDomain}
	Find the domain and range of $\ds{f(x)=x^{2}}$.
\end{example}

\begin{minipage}{0.6\textwidth}
	\begin{solution} 	
		The domain of $\ds{f(x)=x^{2}}$ is the set of all real numbers, $\ds{\R }$, where the graph of $f$ consists of all pairs of real numbers $\ds{(x, \,x^{2})}$. The graph of $f$ represents a parabola. The range of $f$ consists of all values of $f(x)=x^{2}$, or more precisely, the set of non-negative real numbers. So the range of $f$ is $\ds{\{y \in \mathbb{R} \, \ssep \, y \geq 0 \} = [0,\infty)}$.   \\
	\end{solution}
\end{minipage}
\begin{minipage}{0.4\textwidth}
	$$\includegraphics[width=1.5in]{images/x2}$$
\end{minipage}

\begin{example}{Domain of the Square-Root Function}{DomainofSquare-RootFunction}
	Find the domain of $\ds{f(x)=\sqrt{x}}$. 
\end{example}

\begin{solution}
The square-root function $(x)=\sqrt{x}$ is the rule
which says, given an $x$-value, take the non-negative number whose
square is $x$.  This rule only makes sense if $x\ge 0$. % is positive or zero.
We say that the domain of this function is $x\ge 0$, or more formally
$\left\{x\in\mathbb{R}\, \ssep \,x\ge 0\right\}$.  Alternately, we
can use interval notation, and write that the domain is $[0,\infty)$.
The fact that the domain of $y=\sqrt{x}$ is $[0,\infty)$ means that in the
graph of this function (see Table~\ref{tab:elemfns})
we have points $(x,y)$ only above $x$-values on the right side of the
$x$-axis.
\end{solution}

\begin{example}{Domain of the Identity Function}{DomainofSquare-RootFunction}
	Find the domain of $\ds{f(x)=\frac{1}{x} }$. 
\end{example}

\begin{solution}
Another example of a function whose domain is not the entire $x$-axis
is: $f(x)=1/x$, the reciprocal function. (See Table~\ref{tab:elemfns}.)  We cannot substitute $x=0$
in this formula.  The function makes sense, however, for any nonzero
$x$, so we take the domain to be: $\{x\in\mathbb{R}\, \ssep \,x\ne 0\}$.  The graph
of this function does not have any point $(x,y)$ with $x=0$.  As $x$
gets close to 0 from either side, the graph goes off toward $\pm$ infinity.
We call the vertical line $x=0$ an \dfont{asymptote}.
\end{solution}


To summarize, two reasons why certain $x$-values are excluded from the
domain of a function are the following.

\begin{formulabox}[Restrictions for the Domain of a Function]

1.  Division by zero is not allowed. That is, if $\ds{\,\,f(x)=\frac{P(x)}{Q(x)}\,\,}$ then $\ds{\,\, Q(x) \neq 0}$. \\
2.  Real square roots of a negative number do not exist. That is, if $\ds{\,\, f(x) = \sqrt{D(x)} \,\,}$ \\
then $\ds{\,\, D(x) \geq 0}$.  \\
\\
We will encounter some other ways in which functions might be undefined later.
\end{formulabox}

\begin{example}{Finding domain of a rational function with a square root}{domain}
	Find the domain of 
	$$f(x)={\frac{1}{\sqrt{4x-x^2}}}$$
	\vspace{-0.5cm}
\end{example}

\begin{solution} 
	To answer this question, we must rule out the $x\,$-values that make
	$4x-x^2$ negative (because we cannot take the square root of a
	negative number)
	and also the $x\,$-values that make $4x-x^2$ zero (because if $4x-x^2=0$, then
	when we take the square root we get 0, and we cannot divide by 0).
	In other words, the domain consists of all $x$ for which $4x-x^2$ is
	strictly positive.  That is, \\
	$$\begin{array}{rcl}
	4x-x^{2} & > & 0 \\
	x (4-x) & > & 0 \hspace{5mm} \longrightarrow \,\,0 < x < 4 \\
	\end{array}$$
	
	In interval notation, the domain is the interval $(0,4)$.
\end{solution}



\subsection{Modeling with Functions}\label{sec:ModelingFn}

The importance of Mathematics to our society lies in its value to approximate, or \textbf{model}\index{mathematical model}\index{model ! mathematical} real-world phenomenon.  Whether it be used to predict the high temperature on a given day, determine the hours of daylight on a given day, or predict population trends of various and sundry real and mythical beasts, Mathematics is second only to literacy in the importance humanity's development.

\medskip

It is important to keep in mind that anytime Mathematics is used to approximate reality, there are always limitations to the model.  For example, suppose grapes are on sale at the local market for $\$1.50$ per pound. Then one pound of grapes costs $\$1.50$, two pounds of grapes cost $\$3.00$, and so forth.  Suppose we want to develop a formula which relates the cost of buying grapes to the amount of grapes being purchased.  Since these two quantities vary from situation to situation, we assign them variables.  Let $c$ denote the cost of the grapes and let $g$ denote the amount of grapes purchased. To find the cost $c$ of the grapes, we multiply the amount of grapes $g$ by the price $\$1.50$ dollars per pound to get \[c = 1.5 g\]  In order for the units to be correct in the formula, $g$ must be measured in \textit{pounds} of grapes in which case the computed value of $c$ is measured in \textit{dollars}.  Since we're interested in finding the cost $c$ given an amount $g$, we think of $g$ as the independent variable and $c$ as the dependent variable.  Using the language of function notation, we write \[c(g) = 1.5 g\] where $g$ is the amount of grapes purchased (in pounds) and $c(g)$ is the cost (in dollars).  For example, $c(5)$ represents the cost, in dollars, to purchase $5$ pounds of grapes. In this case, $c(5) = 1.5(5) = 7.5$, so it would cost $\$ 7.50$. If, on the other hand, we wanted to find the \textit{amount} of grapes we can purchase for $\$5$, we would need to set $c(g) = 5$ and solve for $g$.  In this case, $c(g)=1.5g$, so solving  $c(g) = 5$ is equivalent to solving $1.5g = 5$  Doing so gives $g = \frac{5}{1.5} = 3.\overline{3}$. This means we can purchase exactly $3.\overline{3}$ pounds of grapes for $\$5$.  Of course, you would be hard-pressed to buy exactly $3.\overline{3}$ pounds of grapes, and this leads us to our next topic of discussion, the \index{domain ! applied}\index{applied domain of a function}\textbf{applied domain} of a function.

\medskip

Even though, mathematically, $c(g) = 1.5g$ has no domain restrictions (there are no denominators and no even-indexed radicals), there are certain values of $g$ that don't make any physical sense.  For example, $g = -1$ corresponds to `purchasing' $-1$ pounds of grapes. Also, unless the `local market' mentioned is the State of California (or some other exporter of grapes), it also doesn't make much sense for $g = 500,000,000$, either. So the reality of the situation limits what $g$ can be, and these limits determine the applied domain of $g$.  Typically, an applied domain is stated explicitly.  In this case, it would be common to see something like $c(g) = 1.5g$, $0 \leq g \leq 100$, meaning the number of pounds of grapes purchased is limited from $0$ up to $100$. The upper bound here, $100$ may represent the inventory of the market, or some other limit as set by local policy or law.  Even with this restriction, our model has its limitations.  As we saw above, it is virtually impossible to buy exactly  $3.\overline{3}$ pounds of grapes so that our cost is exactly $\$5$.  In this case, being sensible shoppers, we would most likely `round down' and purchase $3$ pounds of grapes or however close the market scale can read to $3.\overline{3}$ without being over.  It is time for a more sophisticated example.

\begin{example}{Application}{Application}
	The height $h$ in feet of a model rocket above the ground $t$ seconds after lift-off is given by \[ h(t) = \left\{ \begin{array}{rcl} -5t^2 + 100t, & \mbox{if} & 0 \leq t \leq 20 \\ 0, & \mbox{if} & t > 20 \\ \end{array} \right.\]
	\begin{enumerate}
		
		\item Find and interpret $h(10)$ and $h(60)$.
		
		\item Solve $h(t) = 375$ and interpret your answers.
	\end{enumerate}
\end{example}
\begin{solution}
	\begin{enumerate} 
\item We first note that the independent variable here is $t$, chosen because it represents time.  Secondly, the function is broken up into two rules:  one formula for values of $t$ between $0$ and $20$ inclusive, and another for values of $t$ greater than 20. Since $t=10$ satisfies the inequality $0 \leq t \leq 20$,  we use the first formula listed,  $h(t) = -5t^2 + 100t$, to find $h(10)$.  We get $h(10) = -5(10)^2 + 100(10) = 500$.  Since $t$ represents the number of seconds since lift-off and $h(t)$ is the height above the ground in feet, the equation $h(10) = 500$ means that $10$ seconds after lift-off, the model rocket is $500$ feet above the ground. To find $h(60)$, we note that $t=60$ satisfies $t > 20$, so we use the rule $h(t) = 0$.  This function returns a value of $0$ regardless of what value is substituted in for $t$, so $h(60) = 0$.  This means that $60$ seconds after lift-off, the rocket is $0$ feet above the ground;  in other words, a minute after lift-off, the rocket has already returned to Earth.

\item Since the function $h$ is defined in pieces, we need to solve $h(t) = 375$ in pieces.  For $0 \leq t \leq 20$, $h(t) =  -5t^2 + 100t$, so for these values of $t$, we solve $-5t^2 + 100t = 375$.  Rearranging terms, we get $5t^2 - 100t + 375 = 0$, and factoring gives $5(t-5)(t-15) = 0$. Our answers are  $t=5$ and $t=15$, and since both of these values of $t$ lie between $0$ and $20$, we keep both solutions.  For $t>20$, $h(t) = 0$, and in this case, there are no solutions to $0=375$.  In terms of the model rocket,  solving $h(t) = 375$ corresponds to finding when, if ever, the rocket reaches $375$ feet above the ground. Our two answers, $t=5$ and $t=15$ correspond to the rocket reaching this altitude \textit{twice} -- once $5$ seconds after launch, and again $15$ seconds after launch.
\end{enumerate}
\end{solution}	

The type of function in the previous example is called a \textbf{piecewise-defined} function, or `piecewise' function for short.  Many real-world phenomena (e.g. postal rates, income tax formulas are modeled by such functions.  

Here is another example where the domain of a function is restricted. For example, if $y$ is the area of a
square of side $x$, then we can write $f(x)=x^2$.  In a purely
mathematical context the domain of the function $y=x^2$ is all of
$\mathbb{R}$. However, in the story-problem context of finding areas of squares,
we restrict the domain to positive values of $x$, because a square
with negative or zero side makes no sense.

In pure mathematics, we usually take the domain to be all
values of $x$ at which the formulas can be evaluated. However, in
a physical application problem there might be further restrictions on the domain
because only certain values of $x$ are of interest or make practical
physical sense. For example, the volume $V$ of a sphere is given by the formula $V(r)=\frac{4}{3}\pi r^3$ where $r$ is the radius of the sphere. The domain of $V$ is all $r \geq 0$, since it makes no sense to have a negative radius.\\ 


%Also, letters different from $f$ may be used.  For example, if $y$ is
%the velocity of something at time $t$, we may write $y=v(t)$ with
%the letter $v$ (instead of $f$) standing for the velocity function (and
%$t$ playing the role of $x$).
%
%The letter playing the role of $x$ is called the \dfont{independent
%variable}, and the letter playing the role of $y$ is called the
%\dfont{dependent variable} (because its
%value ``depends on'' the value of the independent
%variable).  In story problems, when one has to translate from English 
%into mathematics, a crucial step is to
%determine what letters stand for variables.  If only words and no
%letters are given, then we have to decide which letters to use.  Some
%letters are traditional.  For example, almost always, $t$ stands for
%time.
%
\begin{example}{Open Box}{openbox} 
An open-top box is made from an $a\times b$ rectangular piece of
cardboard by cutting out a square of side $x$ from each of the four
corners, and then folding the sides up and sealing them with duct
tape.  Find a formula for the volume $V$ of the box as a function of
$x$, and find the domain of this function.
\end{example}

\begin{solution} 
	
\begin{wrapfigure}{r}{0.4\textwidth}
	\begin{center}
		\includegraphics[width=0.38\textwidth]{images/openbox}
	\end{center}
	\caption{}
\end{wrapfigure}	
	
The box we get will have height $x$ and rectangular base of
dimensions $a-2x$ by $b-2x$.  Thus, 
$$V(x)=x(a-2x)(b-2x)$$
Here $a$ and $b$ are constants, and $V$ is the variable that depends
on $x$, i.e., $V$ is playing the role of $y$.  

This formula makes mathematical sense for any $x$, but in the physical
problem the domain is much less.  In the first place, $x$ must be
positive.  In the second place, it must be less than half the length
of either of the sides of the cardboard.  Thus, the domain is
$$\left\{x\in\mathbb{R}\,\ssep\,0<x<{\frac{1}{2}}(\hbox{minimum~of~$a$~and~$b$})\right\}.$$
In interval notation we write: the domain is the interval
$(0,\min(a,b)/2)$. You might think about whether we could allow 0 or 
(the minimum~of~$a$~and~$b$) to be in the domain. They make a certain
physical sense, though we normally would not call the result a box. If we
were to allow these values, what would be the corresponding volumes?
Does that volume make sense?



\end{solution}



\begin{example}{Circles and function}{circle}
	Can a circle of radius $r$ centered at the origin be expressed as a function? 
	
\end{example}

\begin{solution} 
	The equation of a circle of radius $r$ centered at the origin is usually given in the form
	$x^2+y^2=r^2$.  To write the equation in the form $y=f(x)$ we solve
	for $y$, obtaining $y=\pm\sqrt{r^2-x^2}$.  But {\it this is not a
		function}, because when we substitute a value in $(-r,r)$ for $x$
	there are two corresponding values of $y$.  To get a function, we must
	choose one of the two signs in front of the square root.  If we choose
	the positive sign, for example, we get the upper semicircle
	$f(x)=\sqrt{r^2- x^2}$ (see Figure~\ref{fig:upper semicircle}).  The
	domain of this function is the interval $[-r,r]$, i.e., $x$ must be
	between $-r$ and $r$ (including the endpoints). 
	If $x$ is outside of
	that interval, then $r^2-x^2$ is negative, and we cannot take the
	square root.  In terms of the graph, this just means that there are no
	points on the curve whose $x$-coordinate is greater than $r$ or less
	than $-r$.
	
	\figure[!ht]
	\centerline{\vbox{\beginpicture
			\normalgraphs
			%\ninepoint
			\setcoordinatesystem units <0.5truein,0.5truein>
			\setplotarea x from -3.1 to 3.1, y from 0 to 3.1
			\axis bottom ticks withvalues {$-r$} {$r$} / at -3 3 / /
			\axis left shiftedto x=0 /
			\circulararc 180 degrees from 3 0 center at 0 0
			\endpicture}}
	\caption{Upper semicircle $f(x)=\sqrt{r^2- x^2}$. \label{fig:upper semicircle}}
	\endfigure
	
\end{solution}


A function does not always have to be given by a single formula as the next example demonstrates.

\begin{example}{Piecewise Velocity}{PiecewiseVelocityExample}
Suppose that $v(t)$ is the velocity function for a car
which starts out from rest (zero velocity) at time $t=0$; then
increases its speed steadily to 20 m/sec, taking 10 seconds to do
this; then travels at constant speed 20 m/sec for 15 seconds; and
finally applies the brakes to decrease speed steadily to 0, taking 5
seconds to do this. Express the velocity of the car as a function of time $t$.  
\end{example}

\begin{solution}
	The formula for $v(t)$ is different in each of
	the three time intervals.  \\
	
\begin{minipage}{0.3\textwidth}
	For $0 \leq t \leq 10$: \\
	
	\vspace{-4mm}
\hspace{1cm} 	$\begin{array}{c|c} 
		t & v \\ \hline
		0 & 0 \\
		10 & 20 \\
\end{array}$

represented by $v(t)=2t$ 
\end{minipage}
\begin{minipage}{0.3\textwidth}
For $10 \leq t \leq 25$: \\

\vspace{-4mm}
\hspace{1cm} $\begin{array}{c|c} 
t & v \\ \hline
10 & 20 \\
25 & 20 \\
\end{array}$

represented by $v(t)=20$ 
\end{minipage}
\begin{minipage}{0.35\textwidth}
	For $25 \leq t \leq 30$: \\
	
	\vspace{-4mm} 
\hspace{1cm}	$\begin{array}{c|c} 
	t & v \\ \hline
	25 & 20 \\
	30 & 0 \\
	\end{array}$
	
represented by $v(t)=-4t+120$
	\end{minipage}	
	
	
\vspace{2mm} 	
 The velocity of the car  can be expressed as the following piecewise function 

\vspace{-0.5cm}
\begin{minipage}{0.5\textwidth}	
\hspace{1cm} $\displaystyle{ v(t) = \left\{ \begin{array}{cl}
	2t, & 0 \leq t \leq 10 \\
	20, & 10 \leq t \leq 25 \\
	-4t+120, & 25 \leq t \leq 30 \\
	\end{array} \right.}$	\\
	
The graph of this function is shown in Figure~\ref{fig:piecewise velocity}.

\vspace{5mm}
\end{minipage}
\begin{minipage}{0.5\textwidth}	
	
\begin{figure}[H]
	$$\includegraphics[scale=0.3]{images/Velocity}$$
	\caption{ \label{fig:piecewise velocity} } 
\end{figure} 		

%\figure[!h]
%\centerline{\vbox{\beginpicture
%\normalgraphs
%%\ninepoint
%\setcoordinatesystem units <4truemm,2truemm>
%\setplotarea x from 0 to 31, y from 0 to 22
%\axis bottom ticks withvalues {$10$} {$25$} {$30$} / at 10 25 30 / /
%\axis left ticks numbered from 0 to 20 by 10 /
%\plot 0 0 10 20 25 20 30 0 /
%\put {$t$} [l] <3pt,0pt> at 31 0
%\put {$v$} [b] <0pt,3pt> at 0 22
%\endpicture}}
%\caption{A velocity function. \label{fig:piecewise velocity}}
%\endfigure

\end{minipage} 

\end{solution}
	
\pagebreak
%%%%%%%%%%%%%%%%%%%%%%%%%%%%%%%%%%%%%%%%%%%%
\Opensolutionfile{solutions}[ex]
\section*{Exercises for \ref{sec:Functions}}

\begin{enumialphparenastyle}

\begin{multicols}{2}
%%%%%%%%%
\begin{ex}
Determine whether each curve represents a function. \\

\vspace{-1.3cm}
$$\includegraphics[scale=0.25]{images/ExFN3}$$	

\begin{sol}
	Using the Vertical Line Test,
	(A) function, \hspace{3mm} (B) not a function, \hspace{3mm} (C) not a function, \hspace{3mm} (D) function, \hspace{3mm} (E) function, \hspace{3mm} (F) not a function. 	
\end{sol}
\end{ex}

%%%%%%%%%%
\begin{ex}
Determine whether or not the equation represents $y$ as a function of $x$.\\

\begin{multicols}{2}
	\begin{enumerate}
		\item $y = x^{3} - x$ 
		\item $y = \sqrt{x - 2}$
		\item $x^{3}y = -4$ 
		\item $x^{2} - y^{2} = 1$
		\item $y = \dfrac{x}{x^{2} - 9}$
		\item $x = -6$
		\item  $x = y^2 + 4$
		\item $y = x^2 + 4$
		\item $x^2 + y^2 = 4$
		\item $y = \sqrt{4-x^2}$
		\item $x^2 - y^2 = 4$
		\item $x^3 + y^3 = 4$
		\item $2x + 3y = 4$
		\item $2xy = 4$
		\item $x^2 = y^2$ 
		\end{enumerate} 
\end{multicols}
\begin{sol}
\begin{multicols}{2}
	\begin{enumerate}
		\item $y = x^{3} - x$  function
		\item $y = \sqrt{x - 2}$ function
		\item $x^{3}y = -4$  function
		\item $x^{2} - y^{2} = 1$ not a function
		\item $y = \dfrac{x}{x^{2} - 9}$ function
		\item $x = -6$ not a function
		\item  $x = y^2 + 4$ not a function
		\item $y = x^2 + 4$ function
		\item $x^2 + y^2 = 4$ not a function
		\item $y = \sqrt{4-x^2}$ function
		\item $x^2 - y^2 = 4$not a function
		\item $x^3 + y^3 = 4$ function
		\item $2x + 3y = 4$ function
		\item $2xy = 4$ function 
		\item $x^2 = y^2$ not a function 
	\end{enumerate} 
\end{multicols} 	
\end{sol}
\end{ex}

	
	
	
%%%%%%%%%%%%%%%
\begin{ex}
If $\,\, f(x) = \sqrt{x-4}-3x \,\,$, find \\
$f(-4)\, , \,\, f(4) \, , \, \text{and} \,\, f(8)$. 

\begin{sol}
	$f(-4)$ is undefined, \hspace{3mm} $f(4)=-12$, \hspace{3mm} $f(8)=-22$
\end{sol} 	
\end{ex}

\begin{ex}
If $\displaystyle{f(x)=2x^{2}-3x+2}$, find the following. \\

\begin{tabular}{lll}
(a) $f(2)$ & \hspace{2mm} (b) $f(-2)$ & \hspace{2mm} (c) $f(a)$  \\
&& \\ 
(d)  $f(-a)$ & \hspace{2mm} (e) $f(a+1)$ & \hspace{2mm} (f) $f(2a)$ \\
&& \\ 
(g) $2f(a)$ & \hspace{2mm} (h) $\displaystyle{\left[ f(a) \right]^{2}}$ & \hspace{2mm} (i) $f(a+h)$ \\
\end{tabular}

\begin{sol}
\begin{tabular}{ll}
(a) $f(2)=6$ & (b) $f(-2)=16$ \\
(c) $f(a)=2a^{2}-3a+2$  & (d)  $f(-a)=2a^{2}+3a+2$ \\
(e) $f(a+1)=2a^{2}+a+1$ & (f) $f(2a)=8a^{2}-6x+2$ \\
(g) $2f(a)=4x^{2}-6x+4$ & (h) $\displaystyle{\left[ f(a) \right]^{2}=4x^{4}-12x^{3}+17x^{2}-12x+4}$ \\
(i) $f(a+h)=2a^{2}+4ah+2h^{2}-3a-3h+2$ & \\
\end{tabular} 
\end{sol}
\end{ex}

%%%%%%%%%%%%%%%%%%%%%%%%%%%%%%%%%%%%
\begin{ex}
Use the given function $f$ to find and simplify the following:\\
 \begin{multicols}{3}
	\begin{itemize}
		\item  $f(2)$
		\item  $f(-2)$
		\item  $f(2a)$
		\item  $2 f(a)$
		\item $f(a+2)$
		\item  $\displaystyle{f \left( \frac{2}{a} \right)}$
		\item $\displaystyle{\frac{f(a)}{2}}$
		\item  $f(a + h)$
		\item $\displaystyle{f(a) + f(2)}$
	\end{itemize}
\end{multicols}

\begin{multicols}{2}
	(a) \hspace{2mm}  $f(x) = 2x-5$ \\
	\\
    (b) \hspace{2mm} $f(x) = 5-2x$	\\
    \\
    (c) \hspace{2mm} $f(x) = 2x^2 - 1$\\
    \\
    (d) \hspace{2mm} $f(x) = \dfrac{2}{x}$ \\
    \\ 
	(e) \hspace{2mm} $\displaystyle{f(x) = 3x^2+3x-2}$\\
	\\
	(f) \hspace{2mm}  $f(x) = \sqrt{2x+1}$\\
	\\
	(g) \hspace{2mm} $f(x) = -7$\\
	\\
	(h) \hspace{2mm}  $f(x) = \dfrac{x}{2}$\\
\end{multicols}
\begin{sol}
(a) \hspace{2mm} For $f(x) = 2x-5$ \\
$f(2) = -1$\\
$f(-2) = -9$\\
$f(2a) = 4a-5$	\\
$2 f(a) = 4a-10$\\
$f(a+2) = 2a-1$\\
$f(a) + f(2) = 2a-6$\\	
$f \left( \frac{2}{a} \right) = \frac{4}{a} - 5$ \\
$\hphantom{f \left( \frac{2}{a} \right)} = \frac{4-5a}{a}$\\
$\frac{f(a)}{2} =\frac{2a-5}{2}$\\
$f(a + h) = 2a + 2h - 5$\\
		
\vspace{3mm}	

(b) \space{2mm} For $f(x) = 5-2x$\\
$f(2) = 1$\\
$f(-2) = 9$\\
$f(2a) = 5-4a$	\\
$2 f(a) = 10-4a$\\
$f(a+2) = 1-2a$\\
$f(a) + f(2) = 6-2a$\\
$f \left( \frac{2}{a} \right) = 5 - \frac{4}{a}$ \\
$\hphantom{f \left( \frac{2}{a} \right)} = \frac{5a-4}{a}$\\
$\frac{f(a)}{2} = \frac{5-2a}{2}$\\
$f(a + h) = 5-2a-2h$\\

\vspace{3mm} 

(c) \hspace{2mm} For $f(x) = 2x^2-1$\\
$f(2) = 7$\\
$f(-2) = 7$\\
$2 f(a) = 4a^2-2$\\
$f(a+2) = 2a^2+8a+7$\\
$f(a) + f(2) = 2a^2+6$\\
$f \left( \frac{2}{a} \right) = \frac{8}{a^2} - 1$ \\
$\hphantom{f \left( \frac{2}{a} \right)} = \frac{8-a^2}{a^2}$\\
$\frac{f(a)}{2} =  \frac{2a^2-1}{2}$\\
$f(a + h) = 2a^2+4ah+2h^2-1$\\
		
\vspace{2mm}

(d) \hspace{2mm} For $f(x) = 3x^2+3x-2$\\
$f(2) = 16$\\
$f(-2) = 4$\\
$f(2a) = 12a^2+6a-2$\\
$2 f(a) = 6a^2+6a-4$\\
$f(a+2) = 3a^2+15a+16$\\
\small $f(a) + f(2) = 3a^2+3a+14$ \normalsize\\
$f \left( \frac{2}{a} \right) = \frac{12}{a^2} + \frac{6}{a} - 2$ \\
$\hphantom{f \left( \frac{2}{a} \right)} = \frac{12+6a-2a^2}{a^2}$\\
$\frac{f(a)}{2} =  \frac{3a^2+3a-2}{2}$\\
$f(a + h) = 3a^2 + 6ah + 3h^2+3a+3h-2$\\
		
\vspace{3mm}	

(e) \hspace{2mm} For $f(x) = \sqrt{2x+1}$\\
$f(2) = \sqrt{5}$\\
$f(-2)$ is not real \\
$f(2a) = \sqrt{4a+1}$\\
$2 f(a) = 2\sqrt{2a+1}$\\
$f(a+2) = \sqrt{2a+5}$\\
\small $f(a) + f(2) =\sqrt{2a+1} + \sqrt{5}$ \normalsize\\
$f \left( \frac{2}{a} \right) = \sqrt{\frac{4}{a} + 1}$ \\
$\hphantom{f \left( \frac{2}{a} \right)} = \sqrt{\frac{a+4}{a}}$\\
$\frac{f(a)}{2} = \frac{\sqrt{2a+1}}{2}$\\
$f(a + h) = \sqrt{2a+2h+1}$\\

\vspace{3mm} 
	
(f) \hspace{2mm}  For $f(x) = -7$\\
$f(2) = -7$\\
$f(-2) = -7$\\
$f(2a) = -7$\\
$2 f(a) = -14$\\
$f(a+2) = -7$\\
$f(a) + f(2) = -14$\\
$f \left( \frac{2}{a} \right) = -7$ \\
$\frac{f(a)}{2} = \frac{-7}{2}$\\
$f(a + h) = -7$\\

\vspace{3mm}

For $f(x) = \frac{x}{2}$\\
$f(2) = 1$\\
$f(-2) = -1$\\
$f(2a) = a$\\
$2 f(a) = a$\\
$f(a+2) = \frac{a+2}{2}$\\
$f(a) + f(2) = \frac{a}{2}+ 1$ \\
$\hphantom{f(a) + f(2)} = \frac{a+2}{2}$\\
$f \left( \frac{2}{a} \right) = \frac{1}{a}$\\
$\frac{f(a)}{2} =  \frac{a}{4}$\\
$f(a + h) = \frac{a+h}{2}$\\
		
\vspace{3mm}	

(g) For $f(x) = \frac{2}{x}$\\
$f(2) = 1$\\
$f(-2) = -1$\\
$f(2a) = \frac{1}{a}$\\
$2 f(a) = \frac{4}{a}$\\
$f(a+2) = \frac{2}{a+2}$\\
$f(a) + f(2) = \frac{2}{a}+1$ \\
$\hphantom{f(a)+f(2)}=\frac{a+2}{2}$\\
$f \left( \frac{2}{a} \right) = a$\\
$\frac{f(a)}{2} =  \frac{1}{a}$\\
$f(a + h) = \frac{2}{a+h}$\\

\end{sol}
\end{ex}





%%%%%%%%%%%%%%%%%%%%%%
\begin{ex}
An on-line comic book retailer charges shipping costs according to the following formula \[{\displaystyle S(n) = \left\{ \begin{array}{rcl}  1.5 n + 2.,5 &  & 1 \leq n \leq 14  \\
	0,  & & n \geq 15
	\end{array} \right. }\]

where $n$ is the number of  comic books purchased and $S(n)$ is the shipping cost in dollars.

\begin{enumerate}
	
	\item  What is the cost to ship 10 comic books?  %  Ans:  $S(10) = 17.5$, $\$ 17.50$.
	
	\item  What is the significance of the formula $S(n) = 0$ for $n \geq 15$?   % Ans:  There is free shipping on orders of $15$ or more comic books. 
\end{enumerate} 		
		
\begin{sol}
(a) \hspace{2mm} $S(10) = 17.5$, so it costs $\$ 17.50$ to ship 10 comic books. \\
(b) \hspace{2mm} There is free shipping on orders of $15$ or more comic books. \\
			
\end{sol}	
\end{ex}
	
	
%%%%%%%%%%%%%%%%%%%%%%%
\begin{ex}
The cost $C$ (in dollars) to talk $m$ minutes a month on a mobile phone plan is modeled by   \[{\displaystyle C(m) = \left\{ \begin{array}{rcl} 25, & & 0 \leq m \leq 1000 \\
	25+0.1(m-1000), &  & m > 1000
	\end{array} \right. }\]

\begin{enumerate}
	
	\item  How much does it cost to talk $750$ minutes per month with this plan?  % Ans:  $C(750) = 25$, $\$ 25$.
	
	\item  How much does it cost to talk $20$ hours a month with this plan?  % Ans:  $C(1200) = 45$, $\$ 45$. 
	
	\item  Explain the terms of the plan verbally.  % Ans:  It costs $\$25$ for up to $1000$ minutes and $10$ cents per minute for each minute over $1000$ minutes.
	
\end{enumerate}	
\begin{sol}
	
(a) \hspace{2mm} $C(750) = 25$, so it costs $\$ 25$ to talk 750 minutes per month with this plan. \\

(b) \hspace{2mm} Since $20 \, \text{hours} = 1200 \, \text{minutes}$, we substitute $m = 1200$ and get  $C(1200) = 45$.  It costs $\$ 45$ to talk 20 hours per month with this plan. \\ 

(c) \hspace{2mm}  It costs $\$25$ for up to $1000$ minutes and $10$ cents per minute for each minute over $1000$ minutes.

	
\end{sol}		
\end{ex}
		



%%%%%%%%%%%%%%%%%%%%%%%%%%%%%
\begin{ex}
Find the domain of each of the following functions:

\begin{tabular}{ll}
	(a)	$\ds f(x)=x^2+1$ & (b)	$\ds f(x)=\sqrt{2x-3}$ \\
	& \\ [0em]
	(c) $\ds g(x)=\frac{1}{x+1}$ & (d) $\ds y=\frac{1}{x^2-1}$ \\
	& \\ [0em]
	(e) $\ds f(x)=\sqrt{\frac{-1}{x}}$ & (f) $\ds g(t)={\root 3 \of t}$ \\
	& \\ [0em]
	(g) $\ds f(x)=\frac{x+4}{x^{2}-9}$ & (h) $\ds y=\sqrt{1-x^2}$ \\
	& \\ [0em]
	(i) $\ds y=\sqrt{1-\frac{1}{x} }$ & (j) $\ds h(x)=\frac{1}{\sqrt{1-(3x)^2}}$ \\
	& \\ [0em]
	(k) $\ds f(s)=\sqrt{s}+\frac{1}{s-1}$ & (l) $\ds f(x)=\frac{1}{\sqrt{x}-1}$ \\
	& \\ [0em]
	(m) $\ds y=\frac{\sqrt{2x-1}}{x^{2}-x}$ & (n)  $\ds y=\frac{\sin x}{\sqrt{2x-1}}$
\end{tabular}

\begin{sol}
\begin{enumerate}
	\item	$\ds \{x\mid x\in \R\}$, i.e., all $x$
	\item	$\ds \{x\mid x\ge 3/2\}$
	\item	$\ds \{x\mid x\not=-1\}$
	\item	$\ds \{x\mid x\not=1 \hbox{ and } x\not=-1\}$
	\item	$\ds \{x\mid x<0\}$
	\item	$\ds \{t\mid t\in \R\}$, i.e., all $t$
	\item	$\ds \{x\mid x\not \pm 3\}$
	\item	$\ds \{x\mid -1\le x\le 1\}$
	\item	$\ds \{x\mid x\ge 1\}$
	\item	$\ds \{x\mid -1/3< x< 1/3\}$
	\item	$\ds \{s\mid s\ge0  \hbox{ and } s\not=1\}$
	\item	$\ds \{x\mid x\ge0  \hbox{ and } x\not=1\}$
	\item  $\ds \{x\mid x\not=0,\, 1\}$
	\item  $\ds \{x\mid x> 1/2\}$
\end{enumerate}
\end{sol}
\end{ex}

%%%%%%%%%%%%%%%%%%%%%%
\begin{ex}
Find the (implied) domain of the function.


%\begin{multicols}{2}
(a) \hspace{2mm}  $f(x) = x^{4} - 13x^{3}  - 19$ \\
\\
(b) \hspace{2mm}  $f(x) = x^2 + 4$\\
	\\	
(c) \hspace{2mm}	$f(x) = \dfrac{x-2}{x+1}$\\
\\
(d) \hspace{2mm} $f(x) = \dfrac{3x}{x^2+x-2}$\\
\\
(e) \hspace{2mm}  $f(x) = \dfrac{2x}{x^2+3}$\\
\\
(f) \hspace{2mm} $f(x) = \dfrac{2x}{x^2-3}$\\
\\
(g) \hspace{2mm}  $f(x) = \dfrac{x+4}{x^2 - 36}$\\
\\
(h) \hspace{2mm} $f(x) = \dfrac{x-2}{x-2}$  \\
\\
(i) \hspace{2mm}  $f(x) = \sqrt{3-x}$\\
\\
(j) \hspace{2mm} $f(x) = \sqrt{2x+5}$  \\
\\
(k) \hspace{2mm} $f(x) = 9x\sqrt{x+3}$\\
\\
(l) \hspace{2mm} $f(x) = \dfrac{\sqrt{7-x}}{x^2+1}$  \\
\\
(m) \hspace{2mm} $f(x) = \sqrt{6x-2}$\\
\\
(n) \hspace{2mm} $f(x) = \dfrac{6}{\sqrt{6x-2}}$\\
\\
(o) \hspace{2mm}  $f(x) = \sqrt[3]{6x-2}$\\
\\
(p) \hspace{2mm} $f(x) = \dfrac{6}{4 - \sqrt{6x-2}}$\\
\\
(q) \hspace{2mm}  $f(x) = \dfrac{\sqrt{6x-2}}{x^2-36}$\\
\\
(r) \hspace{2mm} $f(x) = \dfrac{\sqrt[3]{6x-2}}{x^2+36}$\\
\\
(s) \hspace{2mm}  $s(t) = \dfrac{t}{t - 8}$\\
\\
(t) \hspace{2mm} $Q(r) = \dfrac{\sqrt{r}}{r - 8}$\\
\\
(u) \hspace{2mm}  $b(\theta) = \dfrac{\theta}{\sqrt{\theta - 8}}$\\
\\
(v) \hspace{2mm} $A(x) = \sqrt{x - 7} + \sqrt{9 - x}$\\
\\
(w) \hspace{2mm} $\alpha(y) = \sqrt[3]{\dfrac{y}{y - 8}}$\\
\\
(x) \hspace{2mm} $g(v) = \dfrac{1}{4 - \dfrac{1}{v^{2}}}$\\
\\
(y) \hspace{2mm}  $T(t) = \dfrac{\sqrt{t} - 8}{5-t}$ \\
\\
(z) \hspace{2mm} $u(w) = \dfrac{w - 8}{5 - \sqrt{w}}$ \\
%\end{multicols} 	
\begin{sol}
\begin{multicols}{2}
	\begin{enumerate}
		\item $(-\infty, \infty)$
		\item  $(-\infty, \infty)$
		\item $(-\infty, -1) \cup (-1, \infty)$
		\item  $(-\infty,-2) \cup (-2,1) \cup (1, \infty)$
		\item $(-\infty, \infty)$
		\item  $(-\infty, -\sqrt{3}) \cup (-\sqrt{3}, \sqrt{3}) \cup (\sqrt{3}, \infty)$
		\item  $(-\infty, -6) \cup (-6,6) \cup (6, \infty)$
		\item $(-\infty, 2) \cup (2, \infty)$
		\item  $(-\infty, 3]$
		\item $\left[-\frac{5}{2}, \infty \right)$  
		\item  $[-3, \infty)$
		\item $(-\infty, 7]$  
		\item    $\left[ \frac{1}{3}, \infty \right)$
		\item   $\left( \frac{1}{3}, \infty \right)$	
		\item   $(-\infty, \infty)$	
		\item   $\left[ \frac{1}{3}, 3 \right) \cup (3, \infty)$
		\item  $\left[ \frac{1}{3}, 6 \right) \cup (6, \infty)$
		\item   $(-\infty, \infty)$
		\item $(-\infty, 8) \cup (8, \infty)$
		\item $[0, 8) \cup (8, \infty)$
		\item $(8, \infty)$
		\item $[7, 9]$
		\item $(-\infty, 8) \cup (8, \infty)$
		\item $\left( -\infty, -\frac{1}{2} \right) \cup \left( -\frac{1}{2}, 0 \right) \cup \left(0, \frac{1}{2}$ 
		\item $[0, 5) \cup (5,\infty)$
		\item $[0, 25) \cup (25, \infty)$
\end{enumerate}
\end{multicols}

\end{sol}	
\end{ex}
	
	






%%%%%%%%%%
\begin{ex}
A farmer wants to build a fence along a river.  He has
500 feet of fencing and wants to enclose a rectangular pen on three
sides (with the river providing the fourth side).  If $x$ is the
length of the side perpendicular to the river, determine the area of
the pen as a function of $x$.  What is the domain of this function?
\begin{sol}
$A=x(500-2x)$, $\ds \{x\mid 0\le x\le 250\}$
\end{sol}
\end{ex}

%%%%%%%%%
\begin{ex}
A can in the shape of a cylinder is to be made with a total
of 100 square centimeters of material in the side, top, and bottom;
the manufacturer wants the can to hold the maximum possible
volume. Write the volume as a function of the radius $r$ of the can;
find the domain of the function.
\begin{sol}
$\ds V=r(50-\pi r^2)$, $\ds \{r\mid 0< r\le \sqrt{50/\pi}\}$
\end{sol}
\end{ex}

%%%%%%%%%
\begin{ex}
A can in the shape of a cylinder is to be made to hold a
volume of one liter (1000 cubic centimeters). The manufacturer wants
to use the least possible material for the can. Write the surface area
of the can (total of the top, bottom, and side) as a function of the
radius $r$ of the can; find the domain of the function.
\begin{sol}
$\ds A=2\pi r^2+2000/r$, $\ds \{r\mid 0<r<\infty\}$
\end{sol}
\end{ex}

\begin{ex}
Suppose that it cost $\displaystyle{5 \cents \,}$ per minute to park at the airport with the rate dropping to $\displaystyle{3 \cents \,}$ after $9$ $\textsc{P.M.}$ Find and graph the cost function $c(t)$ for values of $t$ satisfying $0 \leq t \leq 120$. Assume that $t$ is the number of minutes after $8 \, \textsc{P.M.}$. 	
\end{ex}

\begin{ex}
The Canadian Federal tax rates for 2017 are shown in the following Table. \\

(a) Write a piecewise definition for the tax due $T(x)$ on an income of $x$ dollars. 	\\
(b) Find the tax due on a taxable income of $\$100,000$. \\

\vspace{-1cm}
$$\includegraphics[scale=0.45]{images/piecwiseTaxBracket}$$
\end{ex}


\end{multicols}

\end{enumialphparenastyle}

\pagebreak
