\subsection{Slant Asymptotes}\label{subsec:SlantAsymptotes}
Some functions may have slant (or {\it oblique}) asymptotes, which are neither vertical nor horizontal. If
\[\ds\lim_{x\to\infty}\left[f(x)-(mx+b)\right]=0\]
then the straight line $y=mx+b$ is a \dfont{slant asymptote} to $f(x)$. Visually, the vertical distance between $f(x)$ and $y=mx+b$ is decreasing towards 0 and the curves do not intersect or cross at any point as $x$ approaches infinity. Similarly when $x\to-\infty$.

\begin{example}{Slant Asymptote in a Rational Function}{SlantAsymptRationalFunction}
Find the slant asymptotes of $\ds f(x)=\frac{-3x^2+4}{x-1}$.
\end{example}
\begin{solution}
Note that this function has no horizontal asymptotes since $f(x)\to-\infty$ as $x\to\infty$ and $f(x)\to\infty$ as $x\to-\infty$.

In rational functions, slant asymptotes occur when the degree in the numerator is one greater than in the denominator. We use long division to rearrange the function:
\[\ds\frac{-3x^2+4}{x-1}=-3x-3+\frac{1}{x-1}.\]
The part we're interested in is the resulting polynomial $-3x-3$. This is the line $y=mx+b$ we were seeking, where $m=-3$ and $b=-3$. Notice that
\[\ds\lim_{x\to\infty}\frac{-3x^2+4}{x-1}-(-3x-3)=\lim_{x\to\infty}\frac{1}{x-1}=0\]
and
\[\ds\lim_{x\to-\infty}\frac{-3x^2+4}{x-1}-(-3x-3)=\lim_{x\to-\infty}\frac{1}{x-1}=0.\]
Thus, $y=-3x-3$ is a slant asymptote of $f(x)$.
\end{solution}

Although rational functions are the most common type of function we encounter with slant asymptotes, there are other types of functions we can consider that present an interesting challenge.

\begin{example}{Slant Asymptote}{SlantOtherFunction}
Show that $y=2x+4$ is a slant asymptote of $f(x)=2x-3^x+4$.
\end{example}
\begin{solution}
This is because  
\[\lim_{x\to -\infty}[f(x)-(2x+4)]=\lim_{x\to -\infty}(-3^x)=0.\]
\end{solution}

We note that $\lim_{x\to \infty}[f(x)-(2x+4)]=\lim_{x\to \infty}(-3^x)=-\infty$.
So, the vertical distance between $%
y=f(x)$ and the line $y=2x+4$ decreases toward 0 only when $%
x\to -\infty $ and not when $x\to \infty$. The graph of $f$
approaches the slant asymptote $y=2x+4$ only at the far left and not at the
far right. One might ask if $y=f(x)$ approaches a slant
asymptote when $x\to \infty$. The answer turns out to be no, but we
will need to know something about the relative growth rates of the
exponential functions and linear functions in order to prove this.
Specifically, one can prove that when the base is greater than 1 the
exponential functions grows faster than any power function as $x\to
\infty$. This can be phrased like this: For any $a>1$ and any $n>0$,%
\begin{equation*}
\lim_{x\to \infty}\frac{a^x}{x^n}=\infty \text{ and }%
\lim_{x\to \infty}\frac{x^n}{a^x}=0.
\end{equation*}%
These facts are most easily proved with the aim of something called the L'H\^{o}pital's Rule.