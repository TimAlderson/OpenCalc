\subsection{Vertical Asymptotes}\label{subsec:VerticalAsymptotes}
The line $x=a$ is called a \dfont{vertical asymptote} of $f(x)$ if \ifont{at least one} of the following is true:
$$\lim_{x\to a}f(x)=\infty\qquad\lim_{x\to a^-}f(x)=\infty\qquad\lim_{x\to a^+}f(x)=\infty$$
$$\lim_{x\to a}f(x)=-\infty\qquad\lim_{x\to a^-}f(x)=-\infty\qquad\lim_{x\to a^+}f(x)=-\infty$$ 

\begin{example}{Vertical Asymptotes}{VerticalAsymptotes}
Find the vertical asymptotes of
$\ds f(x)=\frac{2x}{x-4}$.
%\vspace{-0.5cm}
\end{example}

\begin{solution} 
In the definition of vertical asymptotes we need a certain limit to be $\pm\infty$.
Candidates would be to consider values not in the domain of $f(x)$, such as $a=4$.
As $x$ approaches $4$ but is larger than $4$ then $x-4$ is a small positive number and $2x$ is close to $8$, so the quotient $2x/(x-4)$ is a large positive number.
Thus we see that
$$\lim_{x\to 4^+}\frac{2x}{x-4}=\infty.$$
Thus, at least one of the conditions in the definition above is satisfied. Therefore $x=4$ is a vertical asymptote.
\end{solution}