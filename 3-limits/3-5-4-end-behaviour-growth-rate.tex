\subsection{End Behaviour and Comparative Growth Rates}

Let us now look at the last two subsections and go deeper. In the last two
subsections we looked at horizontal and slant asymptotes. Both are special
cases of the end behaviour of functions, and both concern situations where
the graph of a function approaches a straight line as $x\to \infty$
or $-\infty$. But not all functions have this kind of end behviour. For
example, $f(x)=x^2$ and $f(x)=x^3$ do not
approach a straight line as $x\to \infty$ or $-\infty$. The best
we can say with the notion of limit developed at this stage are that%
\begin{eqnarray*}
	\lim_{x\to \infty}x^2 &=&\infty\text{, }\lim_{x\to -\infty}x^2=\infty, \\
	\lim_{x\to \infty}x^3 &=&\infty\text{, }\lim_{x\to -\infty}x^3=-\infty.
\end{eqnarray*}%
Similarly, we can describe the end behaviour of transcendental functions
such as $f(x)=e^x$ using limits, and in this case, the graph
approaches a line as $x\to -\infty$ but not as $x\to \infty$.
\begin{equation*}
\lim_{x\to -\infty}e^x=0\text{, }\lim_{x\to \infty}e^x=\infty.
\end{equation*}

People have found it useful to make a finer distinction between these end
behaviours all thus far captured by the symbols $\infty$ and $-\infty$.
Specifically, we will see that the above functions have different growth
rates at infinity. Some increases to infinty faster than others.
Specifically,

\begin{definition}{Comparative Growth Rates}{CompareGrowthRates}
Suppose that $f$ and $g$ are two functions such that $\lim\limits_{x%
	\to \infty }f\left( x\right) =\infty $ and $\lim\limits_{x%
	\to \infty }g\left( x\right) =\infty .$ We say that $f\left(
x\right) $ grows faster than $g\left( x\right) $ as $x\to \infty $
if the following holds:%
\begin{equation*}
\lim_{x\to \infty }\frac{f\left( x\right) }{g\left( x\right) }%
=\infty ,
\end{equation*}%
or equivalently,%
\begin{equation*}
\lim_{x\to \infty }\frac{g\left( x\right) }{f\left( x\right) }=0.
\end{equation*}
\end{definition}

Here are a few obvious examples:

\begin{example}{}{ComparingMonomials}
Show that if $m>n$ are two positive integers, then $f(x)=x^m$ grows faster
than $g(x)=x^n$ as $x\to \infty$.
\end{example}
\begin{solution}
Since $m>n,$ $m-n$ is a positive integer. Therefore,%
\begin{equation*}
\lim_{x\to \infty}\frac{f(x)}{g(x)}=\lim_{x\to \infty}\frac{x^m}{x^n}=\lim_{x\to \infty}x^{m-n}=\infty.
\end{equation*}
\end{solution}

\begin{example}{}{CompareMonicPoly}
Show that if $m>n$ are two positive integers, then any monic polynomial
$P_{m}(x)$ of degree $m$ grows faster than any monic polynomial
$P_{n}(x)$ of degree $n$ as $x\to \infty$. [Recall that
a polynomial is monic if its leading coefficient is 1.]
\end{example}
\begin{solution}
By assumption, $P_{m}(x)=x^m+$ terms of degrees less than 
$m=x^{m}+a_{m-1}x^{m-1}+\ldots$, and $P_{n}(x)=x^n+$ terms of
degrees less than $n=x^{n}+b_{n-1}x^{n-1}+\ldots$. Dividing the numerator and
denominator by $x^n$, we get%
\begin{equation*}
\lim_{x\to \infty}\frac{f(x)}{g(x)}=\lim_{x\to \infty}\frac{x^{m-n}+a_{m-1}x^{m-n-1}+\ldots}{1+\frac{%
		b_{n-1}}{x}+\ldots}=\lim_{x\to \infty}x^{m-n}(\frac{1+\frac{%
		a_{m-1}}{x}+\ldots}{1+\frac{b_{n-1}}{x}+\ldots}) =\infty,
\end{equation*}%
since the limit of the bracketed fraction is 1 and the limit of $x^{m-n}$ is 
$\infty$, as we showed in Example~\ref{exa:ComparingMonomials}.
\end{solution}

\begin{example}{}{HighestDegreeTermGrowth}
Show that a polynomial grows exactly as fast as its highest degree term
as $x\to \infty$ or $-\infty$. That is, if $P(x)$ is
any polynomial and $Q(x)$ is its highest degree term, then both limits
\begin{equation*}
\lim_{x\to \infty}\frac{P(x)}{Q(x)}\text{ and }\lim_{x\to -\infty}\frac{P(x)}{Q(x)}
\end{equation*}%
are finite and nonzero.
\end{example}
\begin{solution}
Suppose that $P(x)=a_{n}x^{n}+a_{n-1}x^{n-1}+\ldots+a_{1}x+a_{0},$ where $a_{n}\neq 0.$ Then the
highest degree term is $Q\left( x\right) =a_{n}x^{n}.$ So,%
\begin{equation*}
\lim_{x\to \infty }\frac{P\left( x\right) }{Q\left( x\right) }%
=\lim_{x\to \infty }\left( a_{n}+\frac{a_{n-1}}{x}+...+\frac{a_{1}}{%
	x^{n-1}}+\frac{a_{0}}{x^{n}}\right) =a_{n}\neq 0.
\end{equation*}
\end{solution}

Let's state a theorem we mentioned when we discussed the last example in the
last subsection:

\begin{theorem}{}{}
Let $n$ be any positive integer and let $a>1$. Then $f(x)=a^x$
grows faster than $g(x)=x^n$ as $x\to \infty$:
\begin{equation*}
\lim_{x\to \infty}\frac{a^x}{x^n}=\infty\text{, }\lim_{x\to \infty}\frac{x^n}{a^x}=0.
\end{equation*}%
In particular,%
\begin{equation*}
\lim_{x\to \infty}\frac{e^x}{x^n}=\infty\text{, }\lim_{x\to \infty}\frac{x^n}{e^x}=0.
\end{equation*}
\end{theorem}

The easiest way to prove this is to use the L'H\^{o}pital's Rule, which we will
introduce in a later chapter. For now, one can plot and compare the graphs
of an exponential function and a power function. Here is a comparison
between $f(x)=x^2$ and $g(x)=2^x$:

\begin{center}
\begin{tikzpicture}[]
\begin{axis}[
	axis lines=middle,
	ymin=0,
	ymax=130,
	xmin=0,
	xmax=7,
	xlabel={$x$},
	]
	\addplot [mark=none,samples=200,red] {x^2};
	\addplot [mark=none,samples=200,blue] {2^x};
\end{axis}
\end{tikzpicture}
\end{center}

Notice also that as $x\to -\infty$, $x^n$ grows in size but
$e^x$ does not. More specifically, $x^n\to \infty$ or $-\infty$
according as $n$ is even or odd, while $e^x\to 0$. So, it is
meaningless to compare their ``growth''
rates, although we can still calculate the limit
\begin{equation*}
\lim_{x\to -\infty}\frac{e^x}{x^n}=0.
\end{equation*}

Let's see an application of our theorem.

\begin{example}{}{}
Find the horizontal asymptote(s) of $f(x)=\dfrac{x^3+2e^x}{e^x-4x^2}$.
\end{example}
\begin{solution}
To find horizontal asymptotes, we calculate the limits of $f(x)$
as $x\to \infty$ and $x\to -\infty$. For $x\to \infty$,
we divide the numerator and the denominator by $e^x$,
and then we take limit to get%
\begin{equation*}
\lim_{x\to \infty}\dfrac{x^3+2e^x}{e^x-4x^2}=%
\lim_{x\to \infty}\dfrac{\frac{x^3}{e^x}+2}{1-4\frac{x^2}{e^x}}%
=\frac{0+2}{1-4(0)}=2.
\end{equation*}%
For $x\to -\infty$, we divide the numerator and the denominator by $x^2$ to get 
\begin{equation*}
\lim_{x\to -\infty}\dfrac{x^3+2e^x}{e^x-4x^2}%
=\lim_{x\to -\infty}\dfrac{x+2\frac{e^x}{x^2}}{\frac{e^x}{x^2}-4}.
\end{equation*}%
The denominator now approaches $0-4=-4$. The numerator has limit $-\infty$.
So, the quotient has limit $\infty$: 
\begin{equation*}
\lim_{x\to -\infty}\dfrac{x+2\frac{e^x}{x^2}}{\frac{e^x}{x^2}-4}=\infty.
\end{equation*}%
So, $y=2$ is a horizontal asymptote. The function $y=f(x)$
approaches the line $y=2$ as $x\to \infty$. And this is the only
horizontal asymptote, since the function $y=f(x)$ does not
approach any horizontal line as $x\to -\infty$.
\end{solution}

Since the growth rate of a polynomial is the same as that of its leading
term, the following is obvious:

\begin{example}{}{}
If $P(x)$ is any polynomial, then%
\begin{equation*}
\lim_{x\to \infty}\frac{P(x)}{e^x}=0.
\end{equation*}
\end{example}

Also, if $r$ is any real number, then we can place it between two
consecutive integers $n$ and $n+1.$ For example, $\sqrt{3}$ is between 1 and
2, $e$ is between 2 and 3, and $\pi$ is between 3 and 4. Then the following
is totally within our expectation:

\begin{example}{}{}
Prove that if $a>1$ is any basis and $r>0$ is any exponent,
then $f(x)=a^x$ grows faster than $g(x)=x^r$
as $x\to \infty$.
\end{example}
\begin{solution}
Let $r$ be between consecutive integers $n$ and $n+1$. Then for
all $x>1$, $x^{n}\leq x^{r}\leq x^{n+1}$. Dividing by $a^{x}$, we get%
\begin{equation*}
\frac{x^n}{a^x}\leq \frac{x^r}{a^x}\leq \frac{x^{n+1}}{a^x}.
\end{equation*}%
Since 
\begin{equation*}
\lim_{x\to \infty}\frac{x^n}{a^x}=0.
\end{equation*}
\end{solution}

What about exponential functions with different bases? We recall from the
graphs of the exponential functions that for any base $a>1$,%
\begin{equation*}
\lim_{x\to \infty}a^x=\infty.
\end{equation*}

So, the exponential functions with bases greater than 1 all grow to infinity
as $x\to \infty$. How do their growth rates compare?

\begin{theorem}{}{}
If $1<a<b$, then $f(x)=b^x$ grows faster than
$g(x)=a^x$ as $x\to \infty$.
\end{theorem}
\begin{proof}
Proof. Since $a<b$, we have $\frac{b}{a}>1$. So,%
\begin{equation*}
\lim_{x\to \infty}\frac{b^x}{a^x}=\lim_{x\to \infty}\left(\frac{b}{a}\right)^x=\infty.
\end{equation*}
\end{proof}

Another function that grows to infinity as $x\to \infty$ is
$g(x)=\ln x$. Recall that the natural logarithmic function is
the inverse of the exponential function $y=e^x$. Since $e^x$ grows very
fast as $x$ increases, we should expect $\ln x$ to grow very slowly as $x$
increases. The same applies to logarithmic functions with any basis $a>1$.
This is the content of the next theorem.

\begin{theorem}{}{}
Let $r$ be any positive real number and $a>1$. Then 

\begin{enumerate}[(a)]
\item	$f(x)=x^r$ grows faster than $g(x)=\ln x$ as $x\to \infty$.
\item	$f(x)=x^r$ grows faster than $g(x)=\log_{a}x$ as $x\to \infty$.
\end{enumerate}
\end{theorem}
\begin{proof}
\begin{enumerate}
\item	We use a change of variable. Letting $t=\ln x$, then $x=e^t$.
So, $x\to \infty$ if and only if $t\to \infty$, and 
\begin{equation*}
\lim_{x\to \infty}\frac{\ln x}{x^r}=\lim_{t\to \infty}\frac{t}{(e^t)^r}%
=\lim_{t\to \infty}\frac{t}{(e^r)^t}.
\end{equation*}%
Now, since $r>0$, $a=e^r>1$. So, $a^t$ grows as $t$ increases, and it
grows faster than $t$ as $t\to \infty$. Therefore,%
\begin{equation*}
\lim_{x\to \infty}\frac{\ln x}{x^{r}}%
=\lim_{t\to \infty}\frac{t}{(e^r)^t}=\lim_{t\to \infty}\frac{t}{a^t}=0.
\end{equation*}
\item	The change of base identity $\log_{a}x=\dfrac{\ln x}{\ln a}$ implies
that $\log_{a}x$ is simply a constant multple of $\ln x$. The result now
follows from (a).
\end{enumerate}
\end{proof}