\section{Infinite Limits and Limits at Infinity}\label{sec:InfLimits}
We occasionally want to know what happens to some quantity
when a variable gets very large or ``goes to infinity''.

\begin{example}{Limit at Infinity}{LimitInfinity}
What happens to the function $\ds \cos(1/x)$ as $x$ goes to infinity? It
seems clear that as $x$ gets larger and larger, $1/x$ gets closer and
closer to zero, so $\cos(1/x)$ should be getting closer and closer to 
$\cos(0)=1$.
\end{example}

As with ordinary limits, this concept of ``limit at infinity'' can be
made precise. Roughly, we want $\ds \lim_{x\to \infty}f(x)=L$ to mean that
we can make $f(x)$ as close as we want to $L$ by making $x$ large
enough.

\begin{definition}{Limit at Infinity (Formal Definition)}{LimitAtInfinity}
If $f$ is a function, we say that $\ds \lim_{x\to
  \infty}f(x)=L$ if for every $\epsilon>0$ there is an $N > 0$ so that
  whenever $x>N$, $|f(x)-L|<\epsilon$.
We may similarly define $\ds \lim_{x\to-\infty}f(x)=L$.
\end{definition}

We include this definition for completeness, but we will not explore it in
detail. Suffice it to say that such limits behave in much the same way
that ordinary limits do; in particular there is a direct analog of 
Theorem~\ref{thm:PropertiesLimits}.

\begin{example}{Limit at Infinity}{LimitInfinity2}
Compute $\ds\lim_{x\to \infty}{2x^2-3x+7\over x^2+47x+1}$.
\end{example}

\begin{solution} 
As $x$ goes to infinity both the numerator and denominator go to
infinity. We divide the numerator and
denominator by $\ds x^2$:
$$\lim_{x\to \infty}{2x^2-3x+7\over x^2+47x+1}=
\lim_{x\to \infty}{2-\ds{3\over x}+\ds{7\over x^2}\over
1+\ds{47\over x}+\ds{1\over x^2}}.$$
Now as $x$ approaches infinity, all the quotients with some power of
$x$ in the denominator approach zero, leaving 2 in the numerator and 1
in the denominator, so the limit again is 2.
\end{solution}

In the previous example, we \ifont{divided by the highest power of $x$ that occurs in the denominator} in order to evaluate the limit.
We illustrate another technique similar to this.

\begin{example}{Limit at Infinity}{LimitInfinity3}
Compute the following limit:
$$\lim_{x\to\infty}\frac{2x^2+3}{5x^2+x}.$$
\vspace{-0.5cm}
\end{example}

\begin{solution} 
As $x$ becomes large, both the numerator and denominator become large, so it isn't clear what happens to their ratio. The highest power of $x$ in the denominator is $x^2$, therefore we will divide every term in both the numerator and denominator by $x^2$ as follows:
$$\lim_{x\to\infty}\frac{2x^2+3}{5x^2+x}=\lim_{x\to\infty}\frac{2+3/x^2}{5+1/x}.$$
Most of the limit rules from last lecture also apply to infinite limits, so we can write this as:
$$=\frac{\ds{\lim_{x\to\infty}2+3\lim_{x\to\infty}\frac{1}{x^2}}}{\ds{\lim_{x\to\infty}5+\lim_{x\to\infty}\frac{1}{x}}}=\frac{2+3(0)}{5+0}=\frac{2}{5}.$$
Note that we used the theorem above to get that $\ds{\lim_{x\to\infty}\frac{1}{x}=0}$ and $\ds{\lim_{x\to\infty}\frac{1}{x^2}=0}$.

A shortcut technique is to analyze only the \ifont{leading terms} of the numerator and denominator. A leading term is a term that has the highest power of $x$. If there are multiple terms with the same exponent, you must include all of them.\\
\ifont{Top:} The leading term is $2x^2$.\\
\ifont{Bottom:} The leading term is $5x^2$.\\
Now only looking at leading terms and ignoring the other terms we get:
$$\lim_{x\to\infty}\frac{2x^2+3}{5x^2+x}=\lim_{x\to\infty}\frac{2x^2}{5x^2}=\frac{2}{5}.$$
\end{solution}

We next look at limits whose value is infinity (or minus infinity).

\begin{definition}{Infinite Limit (Useable Definition)}{Infinite Limit}
In general, we will write
$$\lim_{x\to a}f(x)=\infty$$
if we can make the value of $f(x)$ arbitrarily large by taking $x$ to be sufficiently close to $a$ (on either side of $a$) but not equal to $a$.
Similarly, we will write
$$\lim_{x\to a}f(x)=-\infty$$
if we can make the value of $f(x)$ arbitrarily large and \blue{negative} by taking $x$ to be sufficiently close to $a$ (on either side of $a$) but not equal to $a$.
\end{definition}

This definition can be modified for one-sided limits as well as limits with $x\to a$ replaced by $x\to\infty$ or $x\to-\infty$.

\begin{example}{Limit at Infinity}{LimitInfinity4}
Compute the following limit: $\ds\lim_{x\to\infty}(x^3-x).$
\end{example}

\begin{solution} 
One might be tempted to write:
$$\lim_{x\to\infty}x^3-\lim_{x\to\infty}x=\infty-\infty,$$
however, we do not know what $\infty-\infty$ is, as $\infty$ is not a real number and so cannot be treated like one.
We instead write:
$$\lim_{x\to\infty}(x^3-x)=\lim_{x\to\infty}x(x^2-1).$$
As $x$ becomes arbitrarily large, then both $x$ and $x^2-1$ become arbitrarily large, and hence their product $x(x^2-1)$ will also become arbitrarily large. Thus we see that
$$\lim_{x\to\infty}(x^3-x)=\infty.$$
\end{solution}

\begin{example}{Limit at Infinity and Basic Functions}{LimitInfinity5}
We can easily evaluate the following limits by observation:
\[ \begin{array}{ll}
1.~\ds\lim_{x\to \infty} \frac{6}{\sqrt{x^3}}=0 & \qquad 2.~\ds\lim_{x\to -\infty} x-x^2=-\infty \\
\\
3.~\ds\lim_{x\to \infty}x^3+x=\infty            & \qquad 4.~\ds\lim_{x\to \infty} \cos(x)=\mbox{DNE} \\
\\
5.~\ds\lim_{x\to \infty} e^x=\infty             & \qquad 6.~\ds\lim_{x\to -\infty} e^x=0 \\
\\
7.~\ds\lim_{x\to 0^+} \ln x=-\infty             & \qquad 8.~\ds\lim_{x\to 0} \cos(1/x)=\mbox{DNE} \\
\end{array} \]
\end{example}

Often, the shorthand notation $\ds\frac{1}{0^+}=+\infty$ and $\ds\frac{1}{0^-}=-\infty$ is used to represent the following two limits respectively:
$$\lim_{x\to 0^+}\frac{1}{x}=+\infty\qquad\mbox{and}\qquad\lim_{x\to 0^-}\frac{1}{x}=-\infty.$$
Using the above convention we can compute the following limits.

\begin{example}{Limit at Infinity and Basic Functions}{LimitInfinity6}
Compute $\ds{\lim_{x\to 0^+} e^{1/x}}$, $\ds{\lim_{x\to 0^-} e^{1/x}}$ and $\ds{\lim_{x\to 0} e^{1/x}}$.
\end{example}

\begin{solution} 
We have:
$$\lim_{x\to 0^+} e^{\frac{1}{x}}=e^{\frac{1}{0^+}}=e^{+\infty}=\infty.$$
$$\lim_{x\to 0^-} e^{\frac{1}{x}}=e^{\frac{1}{0^-}}=e^{-\infty}=0.$$
Thus, as left-hand limit $\neq$ right-hand limit, 
$$\lim_{x\to 0} e^{\frac{1}{x}}=\mbox{DNE}.$$
\end{solution}


%%%%%%%%%%%%%%%%%%%%%%%%%%%%%%%%%%%%%%%%%
% Subsections to include
\subsection{Vertical Asymptotes}\label{subsec:VerticalAsymptotes}
The line $x=a$ is called a \dfont{vertical asymptote} of $f(x)$ if \ifont{at least one} of the following is true:
$$\lim_{x\to a}f(x)=\infty\qquad\lim_{x\to a^-}f(x)=\infty\qquad\lim_{x\to a^+}f(x)=\infty$$
$$\lim_{x\to a}f(x)=-\infty\qquad\lim_{x\to a^-}f(x)=-\infty\qquad\lim_{x\to a^+}f(x)=-\infty$$ 

\begin{example}{Vertical Asymptotes}{VerticalAsymptotes}
Find the vertical asymptotes of
$\ds f(x)=\frac{2x}{x-4}$.
%\vspace{-0.5cm}
\end{example}

\begin{solution} 
In the definition of vertical asymptotes we need a certain limit to be $\pm\infty$.
Candidates would be to consider values not in the domain of $f(x)$, such as $a=4$.
As $x$ approaches $4$ but is larger than $4$ then $x-4$ is a small positive number and $2x$ is close to $8$, so the quotient $2x/(x-4)$ is a large positive number.
Thus we see that
$$\lim_{x\to 4^+}\frac{2x}{x-4}=\infty.$$
Thus, at least one of the conditions in the definition above is satisfied. Therefore $x=4$ is a vertical asymptote.
\end{solution}
\subsection{Horizontal Asymptotes}\label{subsec:HorizontalAsymptotes}
The line $y=L$ is a \dfont{horizontal asymptote} of $f(x)$ if either
$$\lim_{x\to\infty}f(x)=L\qquad\mbox{or}\qquad\lim_{x\to-\infty}f(x)=L.$$

\begin{example}{Horizontal Asymptotes}{HorizontalAsymptotes}
Find the horizontal asymptotes of $\ds f(x)=\frac{|x|}{x}$.
\end{example}

\begin{solution} 
We must compute two infinite limits.
First,
$$\lim_{x\to\infty}\frac{|x|}{x}.$$
Notice that for $x$ arbitrarily large that $x>0$, so that $|x|=x$.
In particular, for $x$ in the interval $(0,\infty)$ we have
$$\lim_{x\to\infty}\frac{|x|}{x}=\lim_{x\to\infty}\frac{x}{x}=1.$$
Second, we must compute
$$\lim_{x\to-\infty}\frac{|x|}{x}.$$
Notice that for $x$ arbitrarily large negative that $x<0$, so that $|x|=-x$.
In particular, for $x$ in the interval $(-\infty,0)$ we have
$$\lim_{x\to -\infty}\frac{|x|}{x}=\lim_{x\to -\infty}\frac{-x}{x}=-1.$$
Therefore there are two horizontal asymptotes, namely, $y=1$ and $y=-1$.
\end{solution}
\subsection{Slant Asymptotes}\label{subsec:SlantAsymptotes}
Some functions may have slant (or {\it oblique}) asymptotes, which are neither vertical nor horizontal. If
\[\ds\lim_{x\to\infty}\left[f(x)-(mx+b)\right]=0\]
then the straight line $y=mx+b$ is a \dfont{slant asymptote} to $f(x)$. Visually, the vertical distance between $f(x)$ and $y=mx+b$ is decreasing towards 0 and the curves do not intersect or cross at any point as $x$ approaches infinity. Similarly when $x\to-\infty$.

\begin{example}{Slant Asymptote in a Rational Function}{SlantAsymptRationalFunction}
Find the slant asymptotes of $\ds f(x)=\frac{-3x^2+4}{x-1}$.
\end{example}
\begin{solution}
Note that this function has no horizontal asymptotes since $f(x)\to-\infty$ as $x\to\infty$ and $f(x)\to\infty$ as $x\to-\infty$.

In rational functions, slant asymptotes occur when the degree in the numerator is one greater than in the denominator. We use long division to rearrange the function:
\[\ds\frac{-3x^2+4}{x-1}=-3x-3+\frac{1}{x-1}.\]
The part we're interested in is the resulting polynomial $-3x-3$. This is the line $y=mx+b$ we were seeking, where $m=-3$ and $b=-3$. Notice that
\[\ds\lim_{x\to\infty}\frac{-3x^2+4}{x-1}-(-3x-3)=\lim_{x\to\infty}\frac{1}{x-1}=0\]
and
\[\ds\lim_{x\to-\infty}\frac{-3x^2+4}{x-1}-(-3x-3)=\lim_{x\to-\infty}\frac{1}{x-1}=0.\]
Thus, $y=-3x-3$ is a slant asymptote of $f(x)$.
\end{solution}

Although rational functions are the most common type of function we encounter with slant asymptotes, there are other types of functions we can consider that present an interesting challenge.

\begin{example}{Slant Asymptote}{SlantOtherFunction}
Show that $y=2x+4$ is a slant asymptote of $f(x)=2x-3^x+4$.
\end{example}
\begin{solution}
This is because  
\[\lim_{x\to -\infty}[f(x)-(2x+4)]=\lim_{x\to -\infty}(-3^x)=0.\]
\end{solution}

We note that $\lim_{x\to \infty}[f(x)-(2x+4)]=\lim_{x\to \infty}(-3^x)=-\infty$.
So, the vertical distance between $%
y=f(x)$ and the line $y=2x+4$ decreases toward 0 only when $%
x\to -\infty $ and not when $x\to \infty$. The graph of $f$
approaches the slant asymptote $y=2x+4$ only at the far left and not at the
far right. One might ask if $y=f(x)$ approaches a slant
asymptote when $x\to \infty$. The answer turns out to be no, but we
will need to know something about the relative growth rates of the
exponential functions and linear functions in order to prove this.
Specifically, one can prove that when the base is greater than 1 the
exponential functions grows faster than any power function as $x\to
\infty$. This can be phrased like this: For any $a>1$ and any $n>0$,%
\begin{equation*}
\lim_{x\to \infty}\frac{a^x}{x^n}=\infty \text{ and }%
\lim_{x\to \infty}\frac{x^n}{a^x}=0.
\end{equation*}%
These facts are most easily proved with the aim of something called the L'H\^{o}pital's Rule.
\subsection{End Behaviour and Comparative Growth Rates}

Let us now look at the last two subsections and go deeper. In the last two
subsections we looked at horizontal and slant asymptotes. Both are special
cases of the end behaviour of functions, and both concern situations where
the graph of a function approaches a straight line as $x\to \infty$
or $-\infty$. But not all functions have this kind of end behviour. For
example, $f(x)=x^2$ and $f(x)=x^3$ do not
approach a straight line as $x\to \infty$ or $-\infty$. The best
we can say with the notion of limit developed at this stage are that%
\begin{eqnarray*}
	\lim_{x\to \infty}x^2 &=&\infty\text{, }\lim_{x\to -\infty}x^2=\infty, \\
	\lim_{x\to \infty}x^3 &=&\infty\text{, }\lim_{x\to -\infty}x^3=-\infty.
\end{eqnarray*}%
Similarly, we can describe the end behaviour of transcendental functions
such as $f(x)=e^x$ using limits, and in this case, the graph
approaches a line as $x\to -\infty$ but not as $x\to \infty$.
\begin{equation*}
\lim_{x\to -\infty}e^x=0\text{, }\lim_{x\to \infty}e^x=\infty.
\end{equation*}

People have found it useful to make a finer distinction between these end
behaviours all thus far captured by the symbols $\infty$ and $-\infty$.
Specifically, we will see that the above functions have different growth
rates at infinity. Some increases to infinty faster than others.
Specifically,

\begin{definition}{Comparative Growth Rates}{CompareGrowthRates}
Suppose that $f$ and $g$ are two functions such that $\lim\limits_{x%
	\to \infty }f\left( x\right) =\infty $ and $\lim\limits_{x%
	\to \infty }g\left( x\right) =\infty .$ We say that $f\left(
x\right) $ grows faster than $g\left( x\right) $ as $x\to \infty $
if the following holds:%
\begin{equation*}
\lim_{x\to \infty }\frac{f\left( x\right) }{g\left( x\right) }%
=\infty ,
\end{equation*}%
or equivalently,%
\begin{equation*}
\lim_{x\to \infty }\frac{g\left( x\right) }{f\left( x\right) }=0.
\end{equation*}
\end{definition}

Here are a few obvious examples:

\begin{example}{}{ComparingMonomials}
Show that if $m>n$ are two positive integers, then $f(x)=x^m$ grows faster
than $g(x)=x^n$ as $x\to \infty$.
\end{example}
\begin{solution}
Since $m>n,$ $m-n$ is a positive integer. Therefore,%
\begin{equation*}
\lim_{x\to \infty}\frac{f(x)}{g(x)}=\lim_{x\to \infty}\frac{x^m}{x^n}=\lim_{x\to \infty}x^{m-n}=\infty.
\end{equation*}
\end{solution}

\begin{example}{}{CompareMonicPoly}
Show that if $m>n$ are two positive integers, then any monic polynomial
$P_{m}(x)$ of degree $m$ grows faster than any monic polynomial
$P_{n}(x)$ of degree $n$ as $x\to \infty$. [Recall that
a polynomial is monic if its leading coefficient is 1.]
\end{example}
\begin{solution}
By assumption, $P_{m}(x)=x^m+$ terms of degrees less than 
$m=x^{m}+a_{m-1}x^{m-1}+\ldots$, and $P_{n}(x)=x^n+$ terms of
degrees less than $n=x^{n}+b_{n-1}x^{n-1}+\ldots$. Dividing the numerator and
denominator by $x^n$, we get%
\begin{equation*}
\lim_{x\to \infty}\frac{f(x)}{g(x)}=\lim_{x\to \infty}\frac{x^{m-n}+a_{m-1}x^{m-n-1}+\ldots}{1+\frac{%
		b_{n-1}}{x}+\ldots}=\lim_{x\to \infty}x^{m-n}(\frac{1+\frac{%
		a_{m-1}}{x}+\ldots}{1+\frac{b_{n-1}}{x}+\ldots}) =\infty,
\end{equation*}%
since the limit of the bracketed fraction is 1 and the limit of $x^{m-n}$ is 
$\infty$, as we showed in Example~\ref{exa:ComparingMonomials}.
\end{solution}

\begin{example}{}{HighestDegreeTermGrowth}
Show that a polynomial grows exactly as fast as its highest degree term
as $x\to \infty$ or $-\infty$. That is, if $P(x)$ is
any polynomial and $Q(x)$ is its highest degree term, then both limits
\begin{equation*}
\lim_{x\to \infty}\frac{P(x)}{Q(x)}\text{ and }\lim_{x\to -\infty}\frac{P(x)}{Q(x)}
\end{equation*}%
are finite and nonzero.
\end{example}
\begin{solution}
Suppose that $P(x)=a_{n}x^{n}+a_{n-1}x^{n-1}+\ldots+a_{1}x+a_{0},$ where $a_{n}\neq 0.$ Then the
highest degree term is $Q\left( x\right) =a_{n}x^{n}.$ So,%
\begin{equation*}
\lim_{x\to \infty }\frac{P\left( x\right) }{Q\left( x\right) }%
=\lim_{x\to \infty }\left( a_{n}+\frac{a_{n-1}}{x}+...+\frac{a_{1}}{%
	x^{n-1}}+\frac{a_{0}}{x^{n}}\right) =a_{n}\neq 0.
\end{equation*}
\end{solution}

Let's state a theorem we mentioned when we discussed the last example in the
last subsection:

\begin{theorem}{}{}
Let $n$ be any positive integer and let $a>1$. Then $f(x)=a^x$
grows faster than $g(x)=x^n$ as $x\to \infty$:
\begin{equation*}
\lim_{x\to \infty}\frac{a^x}{x^n}=\infty\text{, }\lim_{x\to \infty}\frac{x^n}{a^x}=0.
\end{equation*}%
In particular,%
\begin{equation*}
\lim_{x\to \infty}\frac{e^x}{x^n}=\infty\text{, }\lim_{x\to \infty}\frac{x^n}{e^x}=0.
\end{equation*}
\end{theorem}

The easiest way to prove this is to use the L'H\^{o}pital's Rule, which we will
introduce in a later chapter. For now, one can plot and compare the graphs
of an exponential function and a power function. Here is a comparison
between $f(x)=x^2$ and $g(x)=2^x$:

\begin{center}
\begin{tikzpicture}[]
\begin{axis}[
	axis lines=middle,
	ymin=0,
	ymax=130,
	xmin=0,
	xmax=7,
	xlabel={$x$},
	]
	\addplot [mark=none,samples=200,red] {x^2};
	\addplot [mark=none,samples=200,blue] {2^x};
\end{axis}
\end{tikzpicture}
\end{center}

Notice also that as $x\to -\infty$, $x^n$ grows in size but
$e^x$ does not. More specifically, $x^n\to \infty$ or $-\infty$
according as $n$ is even or odd, while $e^x\to 0$. So, it is
meaningless to compare their ``growth''
rates, although we can still calculate the limit
\begin{equation*}
\lim_{x\to -\infty}\frac{e^x}{x^n}=0.
\end{equation*}

Let's see an application of our theorem.

\begin{example}{}{}
Find the horizontal asymptote(s) of $f(x)=\dfrac{x^3+2e^x}{e^x-4x^2}$.
\end{example}
\begin{solution}
To find horizontal asymptotes, we calculate the limits of $f(x)$
as $x\to \infty$ and $x\to -\infty$. For $x\to \infty$,
we divide the numerator and the denominator by $e^x$,
and then we take limit to get%
\begin{equation*}
\lim_{x\to \infty}\dfrac{x^3+2e^x}{e^x-4x^2}=%
\lim_{x\to \infty}\dfrac{\frac{x^3}{e^x}+2}{1-4\frac{x^2}{e^x}}%
=\frac{0+2}{1-4(0)}=2.
\end{equation*}%
For $x\to -\infty$, we divide the numerator and the denominator by $x^2$ to get 
\begin{equation*}
\lim_{x\to -\infty}\dfrac{x^3+2e^x}{e^x-4x^2}%
=\lim_{x\to -\infty}\dfrac{x+2\frac{e^x}{x^2}}{\frac{e^x}{x^2}-4}.
\end{equation*}%
The denominator now approaches $0-4=-4$. The numerator has limit $-\infty$.
So, the quotient has limit $\infty$: 
\begin{equation*}
\lim_{x\to -\infty}\dfrac{x+2\frac{e^x}{x^2}}{\frac{e^x}{x^2}-4}=\infty.
\end{equation*}%
So, $y=2$ is a horizontal asymptote. The function $y=f(x)$
approaches the line $y=2$ as $x\to \infty$. And this is the only
horizontal asymptote, since the function $y=f(x)$ does not
approach any horizontal line as $x\to -\infty$.
\end{solution}

Since the growth rate of a polynomial is the same as that of its leading
term, the following is obvious:

\begin{example}{}{}
If $P(x)$ is any polynomial, then%
\begin{equation*}
\lim_{x\to \infty}\frac{P(x)}{e^x}=0.
\end{equation*}
\end{example}

Also, if $r$ is any real number, then we can place it between two
consecutive integers $n$ and $n+1.$ For example, $\sqrt{3}$ is between 1 and
2, $e$ is between 2 and 3, and $\pi$ is between 3 and 4. Then the following
is totally within our expectation:

\begin{example}{}{}
Prove that if $a>1$ is any basis and $r>0$ is any exponent,
then $f(x)=a^x$ grows faster than $g(x)=x^r$
as $x\to \infty$.
\end{example}
\begin{solution}
Let $r$ be between consecutive integers $n$ and $n+1$. Then for
all $x>1$, $x^{n}\leq x^{r}\leq x^{n+1}$. Dividing by $a^{x}$, we get%
\begin{equation*}
\frac{x^n}{a^x}\leq \frac{x^r}{a^x}\leq \frac{x^{n+1}}{a^x}.
\end{equation*}%
Since 
\begin{equation*}
\lim_{x\to \infty}\frac{x^n}{a^x}=0.
\end{equation*}
\end{solution}

What about exponential functions with different bases? We recall from the
graphs of the exponential functions that for any base $a>1$,%
\begin{equation*}
\lim_{x\to \infty}a^x=\infty.
\end{equation*}

So, the exponential functions with bases greater than 1 all grow to infinity
as $x\to \infty$. How do their growth rates compare?

\begin{theorem}{}{}
If $1<a<b$, then $f(x)=b^x$ grows faster than
$g(x)=a^x$ as $x\to \infty$.
\end{theorem}
\begin{proof}
Proof. Since $a<b$, we have $\frac{b}{a}>1$. So,%
\begin{equation*}
\lim_{x\to \infty}\frac{b^x}{a^x}=\lim_{x\to \infty}\left(\frac{b}{a}\right)^x=\infty.
\end{equation*}
\end{proof}

Another function that grows to infinity as $x\to \infty$ is
$g(x)=\ln x$. Recall that the natural logarithmic function is
the inverse of the exponential function $y=e^x$. Since $e^x$ grows very
fast as $x$ increases, we should expect $\ln x$ to grow very slowly as $x$
increases. The same applies to logarithmic functions with any basis $a>1$.
This is the content of the next theorem.

\begin{theorem}{}{}
Let $r$ be any positive real number and $a>1$. Then 

\begin{enumerate}[(a)]
\item	$f(x)=x^r$ grows faster than $g(x)=\ln x$ as $x\to \infty$.
\item	$f(x)=x^r$ grows faster than $g(x)=\log_{a}x$ as $x\to \infty$.
\end{enumerate}
\end{theorem}
\begin{proof}
\begin{enumerate}
\item	We use a change of variable. Letting $t=\ln x$, then $x=e^t$.
So, $x\to \infty$ if and only if $t\to \infty$, and 
\begin{equation*}
\lim_{x\to \infty}\frac{\ln x}{x^r}=\lim_{t\to \infty}\frac{t}{(e^t)^r}%
=\lim_{t\to \infty}\frac{t}{(e^r)^t}.
\end{equation*}%
Now, since $r>0$, $a=e^r>1$. So, $a^t$ grows as $t$ increases, and it
grows faster than $t$ as $t\to \infty$. Therefore,%
\begin{equation*}
\lim_{x\to \infty}\frac{\ln x}{x^{r}}%
=\lim_{t\to \infty}\frac{t}{(e^r)^t}=\lim_{t\to \infty}\frac{t}{a^t}=0.
\end{equation*}
\item	The change of base identity $\log_{a}x=\dfrac{\ln x}{\ln a}$ implies
that $\log_{a}x$ is simply a constant multple of $\ln x$. The result now
follows from (a).
\end{enumerate}
\end{proof}



%%%%%%%%%%%%%%%%%%%%%%%%%%%%%%%%%%%%%%%%%
\Opensolutionfile{solutions}[ex]
\section*{Exercises for \ref{sec:InfLimits}}

\begin{enumialphparenastyle}

%%%%%%%%%%
\begin{ex}
Compute the following limits.
\begin{multicols}{3}
\begin{enumerate}
	\item	$\ds\lim_{x\to \infty} \sqrt{x^2+x}-\sqrt{x^2-x}$
	\item	$\ds\lim_{x\to\infty} {e^x + e^{-x}\over e^x -e^{-x}}$
	\item	$\ds\lim_{t\to1^+}{(1/t)-1\over t^2-2t+1}$
	\item	$\ds\lim_{t\to\infty}{t+5-2/t-1/t^3\over 3t+12-1/t^2}$
	\item	$\ds\lim_{y\to\infty}{\sqrt{y+1}+\sqrt{y-1}\over y}$
	\item	$\ds\lim_{x\to 0^+}{3+x^{-1/2}+x^{-1}\over 2+4x^{-1/2}}$
	\item	$\ds\lim_{x\to\infty}{x+x^{1/2}+x^{1/3}\over x^{2/3}+x^{1/4}}$
	\item	$\ds\lim_{t\to\infty}{1-\sqrt{t\over t+1}\over 2-\sqrt{4t+1\over t+2}}$
	\item	$\ds\lim_{t\to\infty}{1-{t\over t-1}\over 1-\sqrt{t\over t-1}}$
	\item	$\ds\lim_{x\to-\infty}{x+x^{-1}\over 1+\sqrt{1-x}}$
	\item	$\ds\lim_{x\to1^+}{\sqrt{x}\over x-1}$
	\item	$\ds\lim_{x\to\infty}{x^{-1}+x^{-1/2}\over x+x^{-1/2}}$
	\item	$\ds\lim_{x\to\infty}{x+x^{-2}\over 2x+x^{-2}}$
	\item	$\ds\lim_{x\to\infty}{5+x^{-1}\over 1+2x^{-1}}$
	\item	$\ds\lim_{x\to\infty}{4x\over\sqrt{2x^2+1}}$
	\item	$\ds\lim_{x\to\infty}{(x+5)\left({1\over 2x}+{1\over x+2}\right)}$
	\item	$\ds\lim_{x\to0^+}{(x+5)\left({1\over 2x}+{1\over x+2}\right)}$
	\item	$\ds\lim_{x\to2}{x^3-6x-2\over x^3-4x}$
\end{enumerate}
\end{multicols}
\begin{sol}
\begin{multicols}{3}
\begin{enumerate}
	\item	$1$
	\item	$1$
	\item	$-\infty$
	\item	$1/3$
	\item	$0$
	\item	$\infty$
	\item	$\infty$
	\item	$2/7$
	\item	$2$
	\item	$-\infty$
	\item	$\infty$
	\item	$0$
	\item	$1/2$
	\item	$5$
	\item	$\ds 2\sqrt2$
	\item	$3/2$
	\item	$\infty$
	\item	does not exist
\end{enumerate}
\end{multicols}
\end{sol}
\end{ex}


%%%%%%%%%%
\begin{ex} 
The function $\ds f(x) = {x\over\sqrt{x^2+1}}$ has two horizontal asymptotes.  Find them and give a rough sketch of $f$ with its horizontal asymptotes. 
\begin{sol}
$y=1$ and $y=-1$
\end{sol}
\end{ex}

%%%%%%%%%%
\begin{ex}
Find the vertical asymptotes of $\ds f(x)=\frac{\ln x}{x-2}$.
\begin{sol}
	$x=0$ and $x=2$.
\end{sol}
\end{ex}

%%%%%%%%%%
\begin{ex}
Suppose that a falling object reaches velocity $v(t)=50(1-e^{-t/5})$ at time $t$, where distance is measured in $m$ and time $s$. What is the object's terminal velocity, i.e. the value of $v(t)$ as $t$ goes to infinity?
\end{ex}

%%%%%%%%%%
\begin{ex}
Find the slant asymptote of $f(x)=\dfrac{x^2+x+6}{x-3}$.
\begin{sol}
	$y=x+4$
\end{sol}
\end{ex}


%%%%%%%%%%
\begin{ex}
	Compute the following limits.
	\begin{enumerate}
		\item	$\ds\lim_{x\to -\infty}(2x^3-x)$
		\item	$\ds\lim_{x\to \infty}\tan^{-1}(e^x)$
		\item	$\ds\lim_{x\to -\infty}\tan^{-1}(e^x)$
		\item	$\ds\lim_{x\to \infty}\dfrac{e^x+x^4}{x^3+5\ln x}$
		\item	$\ds\lim_{x\to \infty}\dfrac{2^x+5(3^x)}{3(2^x)-3^x}$
		\item	$\ds\lim_{x\to -\infty}\dfrac{2^x+5(3^x)}{3(2^x)-3^x}$
		\item	$\ds\lim_{x\to 0^{+}}\sqrt{x}\ln x$ [Hint: Let $t=1/x$]
	\end{enumerate}
	\begin{sol}
		\begin{enumerate}
			\item	$-\infty$
			\item	$\pi/2$
			\item	0
			\item	$\infty$
			\item	$-5$
			\item	1/3
			\item	0
		\end{enumerate}
	\end{sol}
\end{ex}


\end{enumialphparenastyle}