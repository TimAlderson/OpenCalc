\section{Computing Limits: Algebraically}\label{sec:ComputingLimitsAlg}
\subsection*{Properties of limits}
With reference to Theorem \ref{thm:LimitProduct}, we can derive 
a handful of theorems to give us the tools to compute many limits
without explicitly working with the precise definition of a limit.

\begin{theorem}{Limit Properties}{PropertiesLimits}
 Suppose that $ L, M $ and $ k $ are real numbers where  $\ds \lim_{x\to a}f(x)=L$ and $\ds \lim_{x\to a}g(x)=M$. Then
\begin{itemize}
\item[Constant Multiple Rule] $\ds\lim_{x\to a} kf(x) =\ds k\lim_{x\to a}f(x)=kL$
\item $\ds\lim_{x\to a} (f(x)+g(x)) = \ds\lim_{x\to a}f(x)+\lim_{x\to a}g(x)=L+M$
\item $\ds\lim_{x\to a} (f(x)-g(x)) = \ds\lim_{x\to a}f(x)-\lim_{x\to a}g(x)=L-M$
\item $\ds\lim_{x\to a} (f(x)g(x)) = \ds\lim_{x\to a}f(x)\cdot\lim_{x\to a}g(x)=LM$
\item $\ds\lim_{x\to a} \frac{f(x)}{g(x)} = \ds\frac{\ds\lim_{x\to a}f(x)}{\ds\lim_{x\to a}g(x)}=\frac{L}{M},\hbox{ if $M$ is not 0}$
\end{itemize}
\end{theorem}

Roughly speaking, these rules say that to compute the limit of an
algebraic expression, it is enough to compute the limits of the
``innermost bits'' and then combine these limits. This often means
that it is possible to simply plug in a value for the variable, since
$\ds \lim_{x\to a} x =a$.

\begin{example}{Limit Properties}{LimitProperties}
Compute $\ds\lim_{x\to 1}{x^2-3x+5\over x-2}$.
\end{example}

\begin{solution} 
 If we apply the theorem in all its gory detail, we get
\begin{eqnarray*}
\lim_{x\to 1}{x^2-3x+5\over x-2}&=&
{\ds\lim_{x\to 1}(x^2-3x+5)\over \ds\lim_{x\to1}(x-2)}\cr
\\
&=&{(\ds\lim_{x\to 1}x^2)-(\ds\lim_{x\to1}3x)+(\ds\lim_{x\to1}5)\over 
  (\ds\lim_{x\to1}x)-(\ds\lim_{x\to1}2)}\cr
\\
&=&{(\ds\lim_{x\to 1}x)^2-3(\ds\lim_{x\to1}x)+5\over (\ds\lim_{x\to1}x)-2}\cr
\\
&=&{1^2-3\cdot1+5\over 1-2}\cr
\\
&=&{1-3+5\over -1} = -3
\end{eqnarray*}
\end{solution}
 
It is worth commenting on the trivial limit $\ds \lim_{x\to1}5$. From one
point of view this might seem meaningless, as the number 5 can't
``approach'' any value, since it is simply a fixed number. However, 5 can,
and should, be interpreted here as the function that has value 5
everywhere, $f(x)=5$, with graph a horizontal line. From this point of
view it makes sense to ask what happens to the values of the function (height of the graph)
as $x$ approaches 1.

We're primarily interested in limits
that aren't so easy, namely, limits in which a denominator approaches
zero. There are a handful of algebraic tricks that work on many of
these limits.

\begin{example}{Zero Denominator}{zerodenominator}
Compute $\ds\lim_{x\to1}{x^2+2x-3\over x-1}$.
\end{example}

\begin{solution} 
 We can't simply plug in $x=1$ because that makes the denominator
 zero.  However:
\begin{eqnarray*}
\lim_{x\to1}{x^2+2x-3\over x-1}&=&\lim_{x\to1}{(x-1)(x+3)\over x-1}\cr
\\
&=&\lim_{x\to1}(x+3)=4\
\end{eqnarray*}
\end{solution}

The technique used to solve the previous example can be referred to as \ifont{factor and cancel}.
Its validity comes from the fact that we are allowed to cancel $x-1$ from the numerator and denominator.
Remember in Calculus that we have to make sure we don't cancel zeros, so we require $x-1\neq 0$ in order
to cancel it. But looking back at the definition of a limit using $x\to 1$, the key point for this example
is that we are taking values of $x$ close to $1$ but \ifont{not} equal to $1$. This is
exactly what we wanted ($x\neq 1$) in order to cancel this common factor.

While Theorem~\ref{thm:PropertiesLimits} is very helpful, we
need a bit more to work easily with limits. Since the theorem applies
when some limits are already known, we need to know the behavior of
some functions that cannot themselves be constructed from the simple
arithmetic operations of the theorem, such as $\ds\sqrt{x}$. Also,
there is one other extraordinarily useful way to put functions
together: composition. If $f(x)$ and
$g(x)$ are functions, we can form two functions by composition:
$f(g(x))$ and $g(f(x))$. For example, if $\ds f(x)=\sqrt{x}$ and $\ds
\ds g(x)=x^2+5$, then $\ds f(g(x))=\sqrt{x^2+5}$ and $\ds
g(f(x))=(\sqrt{x})^2+5=x+5$.  Here is a companion to
Theorem~\ref{thm:PropertiesLimits} for composition:

\begin{theorem}{Limit of Composition}{LimitComposition}
Suppose that $\ds \lim_{x\to a}g(x)=L$ and $\ds \lim_{x\to L}f(x)=f(L)$. Then
$$\lim_{x\to a} f(g(x)) = f(L).$$
\end{theorem}

Note the special form of the condition on $f$: it is not enough to
know that $\ds\lim_{x\to L}f(x) = M$, though it is a bit tricky to see
why. We have included an example in the exercise section to illustrate this tricky
point for those who are interested. Many of the most familiar functions do have this property, and
this theorem can therefore be applied. For example:

\begin{theorem}{Continuity of Roots}{ContinuityRoots}
 Suppose that $n$ is a positive integer. Then
$$\lim_{x\to a}\root n\of{x} = \root n\of{a},$$
provided that $a$ is positive if $n$ is even.
\end{theorem}

This theorem is not too difficult to prove from the definition of limit.

Another of the most common algebraic tricks is called \ifont{rationalization}. 
Rationalizing makes use of the difference of squares formula $(a-b)(a+b)=a^2-b^2$.
Here is an example.

\begin{example}{Rationalizing}{Rationalizing}
Compute $\ds\lim_{x\to-1} {\sqrt{x+5}-2\over x+1}$.
\end{example}

\begin{solution} 
\begin{eqnarray*}
\lim_{x\to-1} {\sqrt{x+5}-2\over x+1}&=&
\lim_{x\to-1} {\sqrt{x+5}-2\over x+1}\cdot{\sqrt{x+5}+2\over \sqrt{x+5}+2}\cr
\\
&=&\lim_{x\to-1} {x+5-4\over (x+1)(\sqrt{x+5}+2)}\cr
\\
&=&\lim_{x\to-1} {x+1\over (x+1)(\sqrt{x+5}+2)}\cr
\\
&=&\lim_{x\to-1} {1\over \sqrt{x+5}+2}={1\over4}
\end{eqnarray*}
At the very last step we have used Theorems~\ref{thm:LimitComposition} and \ref{thm:ContinuityRoots}.
\end{solution}

\begin{example}{Left and Right Limit}{leftright}
Evaluate $\ds\lim_{x\to 0}{x\over|x|}$.
\end{example}

\begin{solution} 
The function $f(x)=x/|x|$ is undefined at 0; when $x>0$, $|x|=x$ and
so $f(x)=1$; when $x<0$, $|x|=-x$ and $f(x)=-1$. Thus
$$\ds \lim_{x\to 0^-}{x\over|x|}=\lim_{x\to 0^-}-1=-1$$
while 
$$\ds \lim_{x\to 0^+}{x\over|x|}=\lim_{x\to 0^+}1=1.$$
The limit of $f(x)$ must be equal to both the left and right limits; since they are
different, the limit $\ds \lim_{x\to 0}{x\over|x|}$ does not exist.
\end{solution}


%%%%%%%%%%%%%%%%%%%%%%%%%%%%%%%%%%%%%%%%%%%%
\Opensolutionfile{solutions}[ex]
\section*{Exercises for \ref{sec:ComputingLimitsAlg}}

\begin{enumialphparenastyle}

%%%%%%%%%%
\begin{ex}
Compute the limits. If a limit does not exist, explain why.
\begin{multicols}{2}
\begin{enumerate}
	\item	$\ds \lim_{x\to 3}{x^2+x-12\over x-3}$
	\item	$\ds \lim_{x\to 1}{x^2+x-12\over x-3}$
	\item	$\ds \lim_{x\to -4}{x^2+x-12\over x-3}$
	\item	$\ds \lim_{x\to 2} {x^2+x-12\over x-2}$
	\item	$\ds \lim_{x\to 1} {\sqrt{x+8}-3\over x-1}$
	\item	$\ds \lim_{x\to 0^+} \sqrt{{1\over x}+2} - \sqrt{1\over x}$
	\item	$\ds\lim _{x\to 2} 3$
	\item	$\ds\lim _{x\to 4 } 3x^3 - 5x $
	\item	$\ds \lim _{x\to 0 } {4x - 5x^2\over x-1}$
	\item	$\ds\lim _{x\to 1 } {x^2 -1 \over x-1 }$
	\item	$\ds\lim _{x\to 0^ + } {\sqrt{2-x^2 }\over x}$
	\item	$\ds\lim _{x\to 0^ + } {\sqrt{2-x^2}\over x+1}$
	\item	$\ds\lim _{x\to a } {x^3 -a^3\over x-a}$
	\item	$\ds\lim _{x\to 2 } (x^2 +4)^3$
\end{enumerate}
\end{multicols}
\begin{sol}
\begin{multicols}{2}
\begin{enumerate}
	\item	7
	\item	5
	\item	0
	\item	undefined
	\item	$1/6$
	\item	0
	\item	3
	\item	172
	\item	0
	\item	2
	\item	does not exist
	\item	$\ds \sqrt2$
	\item	$\ds 3a^2$
	\item	512
\end{enumerate}
\end{multicols}
\end{sol}
\end{ex}

%%%%%%%%%%
\begin{ex}
Let $f(x)=\left\{ 
\begin{array}{cc}
1 & \text{if }x\neq 0 \\ 
0 & \text{if }x=0%
\end{array}%
\right.$ and $g(x)=0$. What are the values of $
L=\lim_{x\to 0}g(x)$ and $M=\lim_{x\to L}f(x)$? Is it true that $\lim_{x\to	0}f(g(x))=M$? What are some noteworthy differences
between this example and Theorem~\ref{thm:LimitComposition}?
\begin{sol}
	$L=0$ and $M=1.$ No.
\end{sol}
\end{ex}

\end{enumialphparenastyle}

\clearpage