\subsection{Absolute Extrema}\label{subsec:AbsoluteExtremaSubsection}

Absolute extrema are also commonly referred to as
\dfont{global extrema}. Unlike local extrema, which are
only ``extreme" relative to points ``close to'' them, an
absolute (or global) extrema is ``extreme" relative to \emph{all}
other points in the interval under consideration.

\begin{definition}{Absolute Maxima and Minima}{AbsMaxMinDef}
	A real-valued function $f$ has an \deffont{absolute maximum}
	on an interval $I$ at $x_0$ if $f(x_0)$ is the largest value
	of $f$ on $I$; in other words, $f(x_0)\geq f(x)$ for all $x$
	in the domain of $f$ that are in $I$.
	
	\medskip
	A real-valued function $f$ has an \deffont{absolute minimum}
	on an interval $I$ at $x_0$ if $f(x_0)$ is the smallest value
	of $f$ on $I$; in other words, $f(x_0)\leq f(x)$ for all $x$
	in the domain of $f$ that are in $I$.
\end{definition}

\begin{example}{Absolute Extrema}{AbsoluteExtremaExamples}
	Consider the function $f(x)=x^2$ on the interval $(-\infty,\infty)$.
	This parabola has an absolute minimum at $x=0$.
	However, it does not have an absolute maximum.
	
	\medskip
	Consider the function $f(x)=|x|$ on the interval $[-1,2]$.
	This graph looks like a check mark. It has an absolute minimum
	at $x=0$ and an absolute maximum at $x=2$.
	
	\medskip
	Consider the function $f(x)=\cos x$ on the interval $[0,\pi]$.
	It has an absolute minimum at $x=\pi$ and an absolute maximum at $x=0$.
	
	\medskip
	Consider the function $f(x)=e^x$ on any interval $[a,b]$,
	where $a<b$. Since this exponential function is increasing,
	it has an absolute minimum at $x=a$ and an absolute maximum at $x=b$.
\end{example}

Like Fermat's Theorem, the following theorem has an intuitive appeal.
However, unlike Fermat's Theorem, the proof relies on a more advanced
concept called \textbf{compactness}, which will only be covered in a course
typically entitled Analysis. So, we will be content with understanding the
statement of the theorem.

\begin{theorem}{Extreme-Value Theorem}{ExtremeValueTheorem}
	If a function $f$ is continuous on a closed interval $[a,b]$,
	then $f$ has both an absolute maximum and an absolute minimum on $[a,b]$.
\end{theorem}

Although this theorem tells us that an absolute extremum exists, it
does not tell us what it is or how to find it.

Note that if an absolute extremum is inside the interval (i.e. not
an endpoint), then it must also be a local extremum. This immediately
tells us that to find the absolute extrema of a function on an interval,
we need only examine the local extrema inside the interval, and the
endpoints of the interval.

We can devise a method for finding absolute extrema for a function $f$ on a closed interval $[a,b]$:

\begin{enumerate}
	\item Verify the function is continuous on $[a,b]$.
	\item Find the derivative and determine all critical values of $f$ that are in $[a,b]$.
	\item Evaluate the function at the critical values found in Step 2 and the end points of the interval.
	\item Identify the absolute extrema.
\end{enumerate}

Why must a function be continuous on a closed interval in order to
use this theorem? Consider the following example.

\begin{example}{Absolute Extrema of a $1/x$}{AbsExtReciprocal}
	Find any absolute extrema for $f(x)=1/x$ on the interval $[-1,1]$.
\end{example}
\begin{solution}
	The function $f$ is not continuous at $x=0$. Since $0\in [-1,1]$,
	$f$ is not continous on the closed interval:
	\begin{align*}
		\lim_{x\to 0^+}f(x)&=+\infty	\\
		\lim_{x\to 0^-}f(x)&=-\infty\, ,
	\end{align*}
	so we are \emph{unable} to apply the Extreme-Value Theorem. Therefore,
	$f(x)=1/x$ does not have an absolute maximum or an absolute minimum on $[-1,1]$.
\end{solution}

However, if we consider the same function on an interval where it is
continuous, the theorem will apply. This is illustrated in the following example.

\begin{example}{Absolute Extrema of a $1/x$}{AbsExtReciprocal}
	Find any absolute extrema for $f(x)=1/x$ on the interval $[1,2]$.
\end{example}
\begin{solution}
	The function $f$ is continous on the interval, so we can apply the
	Extreme-Value Theorem. We begin with taking the derivative to be
	$f'(x)=-1/x^2$ which has a critical value at $x=0$, but since this
	critical value is not in $[1,2]$ we ignore it. The only points where
	an extrema can occur are the endpoints of the interval. To find the
	maximum or minimum we can simply evaluate the function: $f(1)=1$ and
	$f(2)=1/2$, so the absolute maximum is at $x=1$ and the absolute minimum is at $x=2$.
\end{solution}

Why must an interval be closed in order to use the above theorem? Recall
the difference between open and
closed intervals. Consider a function $f$ on the open interval $(0,1)$.
If we choose successive values of $x$ moving closer and closer to $1$,
what happens? Since 1 is not included in the interval we will not attain
exactly the value of 1. Suppose we reach a value of 0.9999 --- is it
possible to get closer to 1? Yes: There are infinitely many real numbers
between 0.9999 and 1. In fact, any conceivable real number close to 1
will have infinitely many real numbers between itself and 1. Now, suppose
$f$ is decreasing on $(0,1)$: As we approach 1, $f$ will continue to
decrease, even if the difference between successive values of $f$ is
slight. Similarly if $f$ is increasing on $(0,1)$.

Consider a few more examples:

\begin{example}{Determining Absolute Extrema}{AbsExtOne}
	Determine the absolute extrema of $f(x)=x^3-x^2+1$ on the interval $[-1,2]$.
\end{example}
\begin{solution}
	First, notice $f$ is continuous on the closed interval $[-1,2]$, so
	we're able to use Theorem~\ref{thm:ExtremeValueTheorem} to determine
	the absolute extrema. The derivative is $f'(x)=3x^2-2x$, and the critical
	values are $x=0,2/3$ which are both in the interval $[-1,2]$. In order
	to find the absolute extrema, we must consider all critical values that
	lie within the interval (that is, in $(-1,2)$) \emph{and} the endpoints of the interval.
	\begin{align*}
		f(-1)&=(-1)^3-(-1)^2+1=-1	\\
		f(0)&=(0)^3-(0)^2+1=1	\\
		f(2/3)&=(2/3)^3-(2/3)^2+1=23/27	\\
		f(2)&=(2)^3-(2)^2+1=5
	\end{align*}
	
	The absolute maximum is at (2,5) and the absolute minimum is at (-1,-1).
\end{solution}

\begin{example}{Determining Absolute Extrema}{AbsExtOne}
	Determine the absolute extrema of $f(x)=-9/x-x+10$ on the interval $[2,6]$.
\end{example}
\begin{solution}
	First, notice $f$ is continuous on the closed interval $[2,6]$,
	so we're able to use Theorem~\ref{thm:ExtremeValueTheorem} to 
	determine the absolute extrema. The function is not continuous
	at $x=0$, but we can ignore this fact since 0 is not in $[2,6]$.
	The derivative is $f'(x)=9/x^2-1$, and the critical values are
	$x=\pm 3$, but only $x=+3$ is in the interval. In order to find
	the absolute extrema, we must consider all critical values that
	lie within the interval \emph{and} the endpoints of the interval.
	\begin{align*}
		f(2)&=-9/(2)-(2)+10=7/2=3.5	\\
		f(3)&=-9/(3)-(3)+10=4	\\
		f(6)&=-9/(6)-(6)+10=5/2=2.5
	\end{align*}
	
	The absolute maximum is at (3,4) and the absolute minimum is at (6,2.5).
\end{solution}

When we are trying to find the absolute extrema of a function on an
open interval, we cannot use the Extreme Value Theorem. However, if
the function is continuous on the interval, many of the same ideas apply.
In particular, if an absolute extremum exists, it must also be a local extremum.
In addition to checking values at the local extrema, we must check the behaviour
of the function as it approaches the ends of the interval.

Some examples to illustrate this method.

\begin{example}{Extrema of Secant}{SecantExtrema}
	Find the extrema of $\sec(x)$ on $(-\pi/2,\pi/2)$.
\end{example}
\begin{solution}
	Notice $\sec(x)$ is continuous on $(-\pi/2,\pi/2)$ and has one local minimum at 0. Also
	\[\lim_{x\to(-\pi/2)^+}\sec(x)=\lim_{x\to(\pi/2)^-}\sec(x)=+\infty\, ,\]
	so $\sec(x)$ has no absolute maximum, but the point $(0,1)$ is the absolute minimum.
\end{solution}

A similar approach can be used for infinite intervals.

\begin{example}{Extrema of $\frac{x^2}{x^2+1}$}{ExtofRationalFraction}
	Find the extrema of $\ds\frac{x^2}{x^2+1}$ on $(-\infty,\infty)$.
\end{example}
\begin{solution}
	Since $x^2+1\neq 0$ for all $x$ in $(-\infty,\infty)$ the function is continuous on this interval. This function has only one critical value at $x=0$, which is the local minimum and also the absolute minimum. Now, $\lim_{x\to\pm\infty}\frac{x^2}{x^2+1}=1$, so the function does not have an absolute maximum: It continues to increase towards 1, but does not attain this exact value. 
\end{solution}


\Opensolutionfile{solutions}[ex]
%%%%%%%%%%%%%%%%%%%%%%%%%%%%%%%%%%%%%%%%%%%%
\section*{Exercises for \ref{subsec:AbsoluteExtremaSubsection}}

\begin{enumialphparenastyle}

\begin{ex}
	Find the absolute extrema for $f(x)=-\frac{x+4}{x-4}$ on $[0,3]$.
	\begin{sol}
		Absolute maximum $(3,7)$; Absolute minimum $(0,1)$. 
	\end{sol}
\end{ex}

\begin{ex}
	Find the absolute extrema for $f(x)=-\frac{x+4}{x-4}$ on $[0,3]$.
	\begin{sol}
		Absolute maximum $(3,7)$; Absolute minimum $(0,1)$. 
	\end{sol}
\end{ex}

\begin{ex}
	Find the absolute extrema for $f(x)=\csc(x)$ on $[0,\pi]$.
	\begin{sol}
		Absolute minimum $(\pi/2,1)$; No absolute maximum.
	\end{sol}
\end{ex}

\begin{ex}
	Find the absolute extrema for $f(x)=\ln(x)/x^2$ on $[1,4]$.
	\begin{sol}
		Absolute minimum $(1,0)$; Absolute maximum $(e^{1/2},\frac{1}{2e})$.
	\end{sol}
\end{ex}

\begin{ex}
	Find the absolute extrema for $f(x)=x\sqrt{1-x^2}$ on $[-1,1]$.
	\begin{sol}
		Absolute minimum $(1,0)$; Absolute maximum $(e^{1/2},\frac{1}{2e})$.
	\end{sol}
\end{ex}

\begin{ex}
	Find the absolute extrema for $f(x)=xe^{-x^2/32}$ on $[0,2]$.
	\begin{sol}
		Absolute minimum $(0,0)$; Absolute maximum $(2,2e^{1/8})$.
	\end{sol}
\end{ex}

\begin{ex}
	Find the absolute extrema for $f(x)=x-\tan^{-1}(2x)$ on $[0,2]$.
	\begin{sol}
		Absolute minimum $(1/2,\frac{2-\pi}{4})$; Absolute maximum $(2,2-\tan^{-1}(4))$.
	\end{sol}
\end{ex}

\begin{ex}
	Find the absolute extrema for $f(x)=\frac{x}{x^2+1}$.
	\begin{sol}
		Absolute maximum $(1,1/2)$; Absolute minimum $(-1,-1/2)$. 
	\end{sol}
\end{ex}

For the following exercises, sketch a potential graph of a continuous function on the closed interval $[0,4]$ with the given properties.

\begin{multicols}{2}
	\begin{ex}
		Absolute minimum at 0, absolute maximum at 2, local minimum at 3.
	\end{ex}
	
	\begin{ex}
		Absolute maximum at 1, absolute minimum at 2, local maximum at 3.
	\end{ex}
	
	\begin{ex}
		Absolute minimum at 4, absolute maximum at 1, local minimum at 2, local maxima at 1 and 3.
	\end{ex}
\end{multicols}

\end{enumialphparenastyle}