In this chapter we explore how to use derivative and
differentiation to solve a variety of problems, some mathematical and some
practical. We explore some applications which motivated and were formalized
in the definition of the derivative, and look at a few clever uses of the tangent line
(which has immediate geometric ties to the definition of the derivative).

% % % %
% The following paragraphs are from Joseph Ling
% % % %
%
%We will begin with \textbf{problems concerning rates of change}
%(which first motivated and was formalized in the definition of the
%derivative) \textbf{and a few clever, practical uses of the tangent line}
%(which had immediate geometric tie to the definition of the derivative).
%Specifically, in Section~\ref{sec:RelatedRates}, we learn to determine how fast a quantity is
%changing with time when we know enough about how and how fast a related
%quantity is changing with time. In Section~\ref{sec:Approx}, we use tangent lines to
%locally approximate functions, thus transforming questions that concern the
%functions into approximate questions that concern their tangent lines. This
%strategy is employed to estimate function values, changes in function
%values, and the roots of functions. The applications in these two sections
%are direct results of the definition of the derivative and the well
%established methods of differentiation.
%
%In Section~\ref{sec:MVT} we explore a deeper understanding of the logical consequences
%of (continuity and) differentiability. Specifically, we discuss the Mean
%Value Theorem, which is the foundational theorem behind all the applications
%in subsequent sections of this chapter. The Mean Value Theorem tells us that
%there is an intimate connection between the net change in function value
%over an interval and certain value of the derivative on that interval,
%provided the function is ``sufficiently nice,'' the meaning of which will be made precise in terms
%of continuity and differentiability. A logical consequence of this is that
%information about the function and information about its derivative are
%intimately related. By studying suitable features of the derivative, we can
%answer questions that concern the function itself. This idea plays out in
%Section~\ref{sec:MVT}, where we see that the derivative of a function uniquely
%determines the function up to a constant on an interval; then in Section~\ref{sec:LH},
%where we see how the limit of some quotient $\frac{f\left( x\right) }{g\left( x\right) }$
%can be found by computing the limit of the quotient of
%derivatives $\frac{f^{\prime }\left( x\right) }{g^{\prime }\left( x\right) }$
%instead, provided the latter yields a definite answer$;$ and then again in
%Section~\ref{sec:CurveSketching}, where we see that the shape of the graph of $f\left( x\right) $
%can be determined by studying the roots and the signs of the derivatives $%
%f^{\prime }\left( x\right) $ and $f^{\prime \prime }\left( x\right) ;$ and
%finally in Section~\ref{sec:Optimization}, where we apply the mathematical techniques developed
%in Section~\ref{sec:CurveSketching} to solve practical, real-life (word)
%problems.