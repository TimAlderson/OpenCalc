\section{L'H\^opital's Rule}\label{sec:LH}\label{sec:lhopitals_rule}

This section is concerned with a technique of evaluating certain limits that will be useful for determining asymptotes, and also in the following chapters, where integration is discussed.

Our treatment of limits exposed us to ``$ 0/0 $'', an indeterminate form. If $\ds \lim_{x\to c}f(x)=0$ and $\ds \lim_{x\to c} g(x) =0$, we do not conclude that $\ds \lim_{x\to c} f(x)/g(x)$ is $0/0$; rather, we use $0/0$ as notation to describe the fact that both the numerator and denominator approach $ 0 $. The expression $ 0/0 $ has no numeric value; other work must be done to evaluate the limit.


Other indeterminate forms exist; they are: %Limits may seeming evaluate to
 $\infty/\infty$, $0\cdot\infty$, $\infty-\infty$, $0^0$, $1^\infty$ and $\infty^0$. %, expressions which have no inherent value. 
 Just as ``$ 0/0 $'' does not mean ``divide $ 0 $ by $ 0 $,'' the expression ``$\infty/\infty$'' does not mean ``divide infinity by infinity.'' Instead, it means ``a quantity is growing without bound and is being divided by another quantity that is growing without bound.'' We cannot determine from such a statement what value, if any, results in the limit. Likewise, ``$0\cdot \infty$'' does not mean ``multiply zero by infinity.'' Instead, it means ``one quantity is shrinking to zero, and is being multiplied by a quantity that is growing without bound.'' We cannot determine from such a description what the result of such a limit will be.

This section introduces l'H\^opital's Rule, a method of resolving limits that produce the indeterminate forms $ 0/0 $ and $\infty/\infty$. We'll also show how algebraic manipulation can be used to convert other indeterminate expressions into one of these two forms so that our new rule can be applied.

\begin{theorem}{L'H\^opital's Rule, Part 1}{LHR}
{Let $\ds \lim_{x\to c}f(x) = 0$ and $\ds \lim_{x\to c}g(x)=0$, where $f$ and $g$ are differentiable functions on an open interval $I$ containing $c$, and $g\primeskip'(x)\neq 0$ on $I$ except possibly at $c$. Then \index{limit!L'H\^opital's Rule}\index{L'H\^opital's Rule}
$$ \lim_{x\to c} \frac{f(x)}{g(x)} = \lim_{x\to c} \frac{\fp(x)}{g\primeskip'(x)}.$$
}
\end{theorem}

We demonstrate the use of l'H\^opital's Rule in the following examples; we will often use ``LHR'' as an abbreviation of ``l'H\^opital's Rule.''\\


\begin{example}{Using l'H\^opital's Rule}{ex_lhr1}
{
Evaluate the following limits, using l'H\^opital's Rule as needed.

\noindent%
\begin{minipage}[t]{.5\textwidth}
\begin{enumerate}
\item		$\ds \lim_{x\to0}\frac{\sin x}x$
\item		$\ds \lim_{x\to 1}\frac{\sqrt{x+3}-2}{1-x}$
\end{enumerate}
\end{minipage}
\begin{minipage}[t]{.5\textwidth}
\begin{enumerate}\addtocounter{enumi}{2}
\item		$\ds \lim_{x\to0}\frac{x^2}{1-\cos x}$
\item		$\ds \lim_{x\to 2}\frac{x^2+x-6}{x^2-3x+2}$
\end{enumerate}
\end{minipage}
}

\end{example}


\begin{solution}
{\begin{enumerate}
\item		We could solve this using the Squeeze Theorem, but l'H\^opital's Rule is much simpler to apply:
$$\lim_{x\to0}\frac{\sin x}x \;\;\;\;\left(\to \frac00\right)\stackrel{\ \text{ by LHR \rule[-5pt]{0pt}{3pt}} \ }{=} \lim_{x\to0} \frac{\cos x}{1}=1.$$

\item	\hfill $\ds \lim_{x\to 1}\frac{\sqrt{x+3}-2}{1-x} \;\;\;\;\left(\to \frac00\right)	 \stackrel{\ \text{ by LHR \rule[-5pt]{0pt}{3pt}} \ }{=} \lim_{x \to 1} \frac{\frac12(x+3)^{-1/2}}{-1} =-\frac 14.$\hfill\null 

\item		\hfill $\ds \lim_{x\to 0}\frac{x^2}{1-\cos x} \;\;\;\;\left(\to \frac00\right) \stackrel{\ \text{ by LHR \rule[-5pt]{0pt}{3pt}} \ }{=}  \lim_{x\to 0} \frac{2x}{\sin x}.$ \hfill\null

This latter limit also evaluates to the $ 0/0 $ indeterminate form. To evaluate it, we apply l'H\^opital's Rule again.

\hfill $\ds  \lim_{x\to 0} \frac{2x}{\sin x}  \stackrel{\ \text{ by LHR \rule[-5pt]{0pt}{3pt}} \ }{=} \frac{2}{\cos x} = 2 .$ \hfill\null

Thus $\ds \lim_{x\to0}\frac{x^2}{1-\cos x}=2.$

\item		We already know how to evaluate this limit; first factor the numerator and denominator. We then have: 
$$\lim_{x\to 2}\frac{x^2+x-6}{x^2-3x+2} = \lim_{x\to 2}\frac{(x-2)(x+3)}{(x-2)(x-1)} = \lim_{x\to 2}\frac{x+3}{x-1} = 5.$$
We now show how to solve this using l'H\^opital's Rule.

$$\lim_{x\to 2}\frac{x^2+x-6}{x^2-3x+2} \;\;\;\;\left(\to \frac00\right)\stackrel{\ \text{ by LHR \rule[-5pt]{0pt}{3pt}} \ }{=}  \lim_{x\to 2}\frac{2x+1}{2x-3} = 5.$$
\end{enumerate}
\vskip-\baselineskip
}
\end{solution}


%
Note that at each step where l'H\^opital's Rule was applied, it was \emph{needed}: the initial limit returned the indeterminate form of ``$0/0$.'' If the initial limit returns, for example, $ 1/2 $, then l'H\^opital's Rule does not apply.

The following theorem extends our initial version of l'H\^opital's Rule in two ways. It allows the technique to be applied to the indeterminate form $\infty/\infty$ and to limits where $x$ approaches $\pm\infty$.

\begin{theorem}{L'H\^opital's Rule, Part 2}{thm:LHR2}
{\begin{enumerate}
\item		Let $\ds\lim_{x\to a}f(x) = \pm\infty$ and $\ds\lim_{x\to a}g(x)=\pm \infty$, where $f$ and $g$ are differentiable on an open interval $I$ containing $a$. Then \index{limit!L'H\^opital's Rule}\index{L'H\^opital's Rule}
$$\lim_{x\to a} \frac{f(x)}{g(x)} = \lim_{x\to a}\frac{\fp(x)}{g\primeskip'(x)}.$$

\item		Let $f$ and $g$ be differentiable functions on the open interval $(a,\infty)$ for some value $a$, where $g\primeskip'(x)\neq 0$ on $(a,\infty)$ and $\ds\lim_{x\to\infty} f(x)/g(x)$ returns either 0/0 or $\infty/\infty$. Then
$$\lim_{x\to \infty} \frac{f(x)}{g(x)} = \lim_{x\to \infty}\frac{\fp(x)}{g\primeskip'(x)}.$$
A similar statement can be made for limits where $x$ approaches $-\infty$.
\end{enumerate}
}
\end{theorem}


\begin{example}{Using l'H\^opital's Rule with limits involving $\infty$}{ex_LHR2}
{
Evaluate the following limits.\\

$\ds 1.\ \lim_{x\to\infty} \frac{3x^2-100x+2}{4x^2+5x-1000} \qquad\qquad 2. \ \lim_{x\to \infty}\frac{e^x}{x^3}.$
}

\end{example}


\begin{solution}
{\begin{enumerate}
\item		We can evaluate this limit already using other techniques%Theorem \ref{thm:lim_rational_fn_at_infty}
; the answer is $ 3/4 $. We apply l'H\^opital's Rule to demonstrate its applicability.
$$\lim_{x\to\infty} \frac{3x^2-100x+2}{4x^2+5x-1000} \;\;\;\;\left(\to \frac\infty\infty \right) \stackrel{\ \text{ by LHR \rule[-5pt]{0pt}{3pt}} \ }{=} \lim_{x\to\infty} \frac{6x-100}{8x+5} \;\;\;\;\left(\to \frac\infty\infty \right) \stackrel{\ \text{ by LHR \rule[-5pt]{0pt}{3pt}} \ }{=} \lim_{x\to\infty} \frac68 = \frac34.$$

\item		$\ds \lim_{x\to \infty}\frac{e^x}{x^3} \;\;\;\;\left(\to \frac\infty\infty \right) \stackrel{\ \text{ by LHR \rule[-5pt]{0pt}{3pt}} \ }{=} \lim_{x\to\infty} \frac{e^x}{3x^2} \;\;\;\;\left(\to \frac\infty\infty \right) \stackrel{\ \text{ by LHR \rule[-5pt]{0pt}{3pt}} \ }{=} \lim_{x\to\infty} \frac{e^x}{6x} \;\;\;\;\left(\to \frac\infty\infty \right) \stackrel{\ \text{ by LHR \rule[-5pt]{0pt}{3pt}} \ }{=} \lim_{x\to\infty} \frac{e^x}{6} = \infty.$

Recall that this means that the limit does not exist; as $x$ approaches $\infty$, the expression $e^x/x^3$ grows without bound. We can infer from this that $e^x$ grows ``faster'' than $x^3$; as $x$ gets large, $e^x$ is far larger than $x^3$. (This has important implications in computing when considering efficiency of algorithms.)
\end{enumerate}
}
\end{solution}



\subsection*{Indeterminate Forms $0\cdot\infty$ and $\infty-\infty$}


L'H\^opital's Rule can only be applied to ratios of functions. When faced with an indeterminate form such as $0\cdot\infty$ or $\infty-\infty$, we can sometimes apply algebra to rewrite the limit so that l'H\^opital's Rule can be applied. We demonstrate the general idea in the next example.
\index{limit!indeterminate form}\index{indeterminate form}\\



\begin{example}{L'H\^opital's Rule}{LHRule3}
Compute the following limits:\\
\noindent
\begin{minipage}[t]{.5\textwidth}
\begin{enumerate}
\item		$\ds \lim_{x\to0^+} x\cdot e^{1/x}$
\item		$\ds \lim_{x\to0^-} x\cdot e^{1/x}$
\item       $\ds\lim_{x\to 0^+} x\ln x$
\end{enumerate}
\end{minipage}
\begin{minipage}[t]{.5\textwidth}
\begin{enumerate}\addtocounter{enumi}{3}
\item		$\ds \lim_{x\to\infty} \ln(x+1)-\ln x$
\item		$\ds \lim_{x\to\infty} x^2-e^x$
\end{enumerate}
\end{minipage}
\end{example}


\begin{solution}
{\begin{enumerate}
\item		As $x\rightarrow 0^+$, $x\rightarrow 0$ and $e^{1/x}\rightarrow \infty$. Thus we have the indeterminate form $0\cdot\infty$. We rewrite the expression $x\cdot e^{1/x}$ as $\ds\frac{e^{1/x}}{1/x}$; now, as $x\rightarrow 0^+$, we get the indeterminate form $\infty/\infty$ to which l'H\^opital's Rule can be applied. 
$$ \lim_{x\to0^+} x\cdot e^{1/x} = \lim_{x\to 0^+} \frac{e^{1/x}}{1/x} \stackrel{\ \text{ by LHR \rule[-5pt]{0pt}{3pt}} \ }{=} \lim_{x\to 0^+}\frac{(-1/x^2)e^{1/x}}{-1/x^2} =\lim_{x\to 0^+}e^{1/x} =\infty.$$

Interpretation: $e^{1/x}$ grows ``faster'' than $x$ shrinks to zero, meaning their product grows without bound.

\item		As $x\rightarrow 0^-$, $x\rightarrow 0$ and $e^{1/x}\rightarrow e^{-\infty}\rightarrow 0$. The the limit evaluates to $0\cdot 0$ which is not an indeterminate form. We conclude then that $$\lim_{x\to 0^-}x\cdot e^{1/x} = 0.$$

\item As $x\to 0^+$, $\ln x\to -\infty$, so the product is indeterminate of the form $\pm 0\cdot\infty $.
 So we can  apply L'H\^opital's Rule after re-writing it in the form $\frac{\infty }{\infty }$:
$$x\ln x = \frac{\ln x}{1/x}=\frac{\ln x}{x^{-1}}.$$
Now as $x$ approaches zero, both the numerator and denominator
approach infinity (one $-\infty$ and one $+\infty$, but only the size
is important). Using  L'H\^opital's Rule:
$$\lim_{x\to 0^+} {\ln x\over x^{-1}}\stackrel{\ \text{ by LHR \rule[-5pt]{0pt}{3pt}} \ }{=}
\lim_{x\to 0^+} {1/x\over -x^{-2}} =\lim_{x\to 0^+} -x = 0.$$
Interpretation: $x$ approaches zero much faster than the $\ln x$ approaches $-\infty$.
\item		This limit initially evaluates to the indeterminate form $\infty-\infty$. By applying a logarithmic rule, we can rewrite the limit as 
$$ \lim_{x\to\infty} \ln(x+1)-\ln x = \lim_{x\to \infty} \ln \left(\frac{x+1}x\right).$$

As $x\rightarrow \infty$, the argument of the $\ln$ term approaches $\infty/\infty$, to which we can apply l'H\^opital's Rule.
$$\lim_{x\to\infty} \frac{x+1}x \stackrel{\ \text{ by LHR \rule[-5pt]{0pt}{3pt}} \ }{=} \frac11=1.$$

Since $x\rightarrow \infty$ implies $\ds\frac{x+1}x\rightarrow 1$, it follows that 
$$x\rightarrow \infty \quad \text{ implies }\quad \ln\left(\frac{x+1}x\right)\rightarrow \ln 1=0.$$

Thus $$ \lim_{x\to\infty} \ln(x+1)-\ln x = \lim_{x\to \infty} \ln \left(\frac{x+1}x\right)=0.$$
Interpretation: since this limit evaluates to 0, it means that for large $x$, there is essentially no difference between $\ln (x+1)$ and $\ln x$; their difference is essentially 0.

\item		The limit $\ds \lim_{x\to\infty} x^2-e^x$ initially returns the indeterminate form $\infty-\infty$. We can rewrite the expression by factoring out $x^2$; $\ds x^2-e^x = x^2\left(1-\frac{e^x}{x^2}\right).$ We need to evaluate how $e^x/x^2$ behaves as $x\rightarrow \infty$:
$$\lim_{x\to\infty}\frac{e^x}{x^2} \stackrel{\ \text{ by LHR \rule[-5pt]{0pt}{3pt}} \ }{=} \lim_{x\to\infty} \frac{e^x}{2x} \stackrel{\ \text{ by LHR \rule[-5pt]{0pt}{3pt}} \ }{=} \lim_{x\to\infty} \frac{e^x}{2} = \infty.$$

Thus $\lim_{x\to\infty}x^2(1-e^x/x^2)$ evaluates to $\infty\cdot(-\infty)$, which is not an indeterminate form; rather, $\infty\cdot(-\infty)$ evaluates to $-\infty$. We conclude that 
$\ds \lim_{x\to\infty} x^2-e^x = -\infty.$

Interpretation: as $x$ gets large, the difference between $x^2$ and $e^x$ grows very large.
\end{enumerate}
}
\end{solution}




\subsection*{Indeterminate Forms\ \ $0^0$, $1^\infty$ and $\infty^0$}


When faced with an indeterminate form that involves a power, it often helps to employ the natural logarithmic function. The following Key Idea expresses the concept, which is followed by an example that demonstrates its use.


\begin{formulabox}[Limits Involving Indeterminate Powers] %Forms $0^0$, $1^\infty$ and $\infty^0$
{\label{idea:LHRpower}
If $\ds \lim_{x\to c} \ln\big(f(x)\big) = L$, then 
$\ds \lim_{x\to c} f(x) = \lim_{x\to c} e^{\ln(f(x))} = e\,^L.$ 
\index{limit!indeterminate form}\index{indeterminate form}
}
\end{formulabox}



%


\begin{example}{Using l'H\^opital's Rule with indeterminate forms involving exponents }{ex_LHR4}
{Evaluate the following limits.
\begin{enumerate}
\item $\ds \lim_{x\to\infty} \left(1+\frac1x\right)^x  $ 
\item  $ \ds \lim_{x\to0^+} x^x$
\item $\ds{\lim_{x\to 1^+}x^{1/(x-1)}}.$
\end{enumerate}
}
\end{example}


\begin{solution}
{\begin{enumerate}
\item		%This equivalent to a special limit given in Theorem \ref{thm:lim_continuous}; 
This limit has important applications within mathematics and finance. Note that the exponent approaches $\infty$ while the base approaches $ 1 $, leading to the indeterminate form $1^\infty$. Let $f(x) = (1+1/x)^x$; the problem asks to evaluate $\ds\lim_{x\to\infty}f(x)$. Let's first evaluate $\ds \lim_{x\to\infty}\ln\big(f(x)\big)$.
\begin{align*}
\lim_{x\to\infty}\ln\big(f(x)\big) & = \lim_{x\to\infty} \ln \left(1+\frac1x\right)^x \\
			&= \lim_{x\to\infty} x\ln\left(1+\frac1x\right)\\
			&=  \lim_{x\to\infty} \frac{\ln\left(1+\frac1x\right)}{1/x}\\
			\intertext{This produces the indeterminate form 0/0, so we apply l'H\^opital's Rule.}
			&\stackrel{\ \text{ by LHR \rule[-5pt]{0pt}{3pt}} \ }{=}	\lim_{x\to\infty} \frac{\frac{1}{1+1/x}\cdot(-1/x^2)}{(-1/x^2)} \\
			&= \lim_{x\to\infty}\frac{1}{1+1/x}\\
			&= 1.
\end{align*}
Thus $\ds\lim_{x\to\infty} \ln \big(f(x)\big) = 1.$ We return to the original limit and apply Key Idea \ref{idea:LHRpower}.

$$\lim_{x\to\infty}\left(1+\frac1x\right)^x = \lim_{x\to\infty} f(x) =  \lim_{x\to\infty}e^{\ln (f(x))} = e^1 = e.$$


\item		This limit leads to the indeterminate form $0^0$. Let $f(x) = x^x$ and consider first $\ds\lim_{x\to0^+} \ln\big(f(x)\big)$. 
\begin{align*}
\lim_{x\to0^+} \ln\big(f(x)\big) &= \lim_{x\to0^+} \ln\left(x^x\right) \\
			&= \lim_{x\to0^+} x\ln x \\
			&= \lim_{x\to0^+} \frac{\ln x}{1/x}.\\
			\intertext{This produces the indeterminate form $-\infty/\infty$ so we apply l'H\^opital's Rule.}
			&\stackrel{\ \text{ by LHR \rule[-5pt]{0pt}{3pt}} \ }{=}	\lim_{x\to0^+} \frac{1/x}{-1/x^2} \\
			&= \lim_{x\to0^+} -x \\
			&= 0.
\end{align*}
Thus $\ds\lim_{x\to0^+} \ln\big(f(x)\big) =0$. We return to the original limit and apply Key Idea \ref{idea:LHRpower}.
$$\lim_{x\to0^+} x^x = \lim_{x\to0^+} f(x) = \lim_{x\to0^+} e^{\ln(f(x))} = e^0 = 1.$$
This result is supported by the graph of $f(x)=x^x$ given in Figure \ref{fig:LHR4}.

\mfigure{.8}{A graph of $f(x)=x^x$ supporting the fact that as $x\to 0^+$, $f(x)\to 1$.}{fig:LHR4}{\begin{tikzpicture}
\begin{axis}[width=.6\textwidth,%
tick label style={font=\scriptsize},axis y line=middle,axis x line=middle,name=myplot,axis on top,%
			%x=.37\marginparwidth,
			%y=.37\marginparwidth,
%			xtick=\empty,% 
%			extra x ticks={.5,3},
%			extra x tick labels={$a$,$b$},
			ytick={1,2,3,4},
%			yticklabels={$-0.002$,$0.002$,$0.004$},
			%minor y tick num=1,
%			extra y ticks={0.001},%
%			minor x tick num=4,
			ymin=-.4,ymax=4.5,%
			xmin=-.1,xmax=2.2,%
			scaled ticks=false
]

\addplot [{\colorone},thick,smooth] coordinates {(0.01,0.955)(0.02,0.9247)(0.03,0.9001)(0.04,0.8792)(0.05,0.8609)(0.06,0.8447)(0.07,0.8302)(0.08,0.817)(0.09,0.8052)(0.1,0.7943)(0.2,0.7248)(0.3,0.6968)(0.4,0.6931)(0.5,0.7071)(0.6,0.736)(0.7,0.7791)(0.8,0.8365)(0.9,0.9095)(1.,1.)(1.1,1.111)(1.2,1.245)(1.3,1.406)(1.4,1.602)(1.5,1.837)(1.6,2.121)(1.7,2.465)(1.8,2.881)(1.9,3.386)(2.,4.)
};

\draw (axis cs:1,1) node [below right] {\scriptsize $f(x)=x^x$};



%\draw (axis cs:2.4,-0.002) node {\scriptsize $f(x)$};
\end{axis}

\node [right] at (myplot.right of origin) {\scriptsize $x$};
\node [above] at (myplot.above origin) {\scriptsize $y$};
\end{tikzpicture}
\item 
This limit is of the type $\mathquotes{1^\infty}$.
To deal with this type of limit we will again use logarithms.
Let
$$L=\lim_{x\to 1^+}x^{1/(x-1)}.$$
Now, take the natural log of both sides:
$$\ln L=\lim_{x\to 1^+}\ln\left(x^{1/(x-1)}\right).$$
Using log properties we have:
$$\ln L=\lim_{x\to 1^+}\frac{\ln x}{x-1}.$$
The right side limit is now of the type $0/0$, therefore, we can apply L'H\^opital's Rule:
$$\ln L=\lim_{x\to 1^+}\frac{\ln x}{x-1}\stackrel{\ \text{ by LHR \rule[-5pt]{0pt}{3pt}} \ }{=}\lim_{x\to 1^+}\frac{1/x}{1}=1$$
Thus, $\ln L=1$ and hence, our original limit (denoted by $L$) is: $L=e^1=e$. That is,
$$L=\lim_{x\to 1^+}x^{1/(x-1)}=e.$$
In this case, even though our limit had a type of $\mathquotes{1^\infty}$, it actually had a value of $e$.
}
\end{enumerate}
}
\end{solution}





























%Our brief revisit of limits will be rewarded in the next section where we consider \textit{improper integration.} So far, we have only considered definite integrals where the bounds are finite numbers, such as $\ds \int_0^1 f(x)\ dx$. Improper integration considers integrals where one, or both, of the bounds are ``infinity.'' Such integrals have many uses and applications, in addition to generating ideas that are enlightening.


%The following application of derivatives allows us to compute certain limits.
%
%\begin{definition}{Limits of the Indeterminate Forms $\frac{0}{0}$ and $\frac{\infty}{\infty}$}{IndeterminateLimits}
%A limit of a quotient $\lim\limits_{x\rightarrow a}\frac{f\left( x\right) }{%
%	g\left( x\right) }$ is said to be an \textbf{indeterminate form of the type }%
%$\frac{0}{0}$ if both $f\left( x\right) \rightarrow 0$ and $g\left( x\right)
%\rightarrow 0$ as $x\rightarrow a.$ Likewise, it is said to be an \textbf{%
%	indeterminate form of the type} $\frac{\infty }{\infty }$ if both $f\left(
%x\right) \rightarrow \pm \infty $ and $g\left( x\right) \rightarrow \pm
%\infty $ as $x\rightarrow a$ (Here, the two $\pm $ signs are independent of
%each other).
%\end{definition}
%
%\begin{theorem}{L'H\^opital's Rule}{LHRule}
%For a limit $\lim\limits_{x\rightarrow a}\frac{f\left( x\right) }{g\left(
%	x\right) }$ of the indeterminate form $\frac{0}{0}$ or $\frac{\infty }{%
%	\infty },$ $\lim\limits_{x\rightarrow a}\frac{f\left( x\right) }{g\left(
%	x\right) }=\lim\limits_{x\rightarrow a}\frac{f^{\prime }\left( x\right) }{%
%	g^{\prime }\left( x\right) }$ if $\lim\limits_{x\rightarrow a}\frac{%
%	f^{\prime }\left( x\right) }{g^{\prime }\left( x\right) }$ exists or equals $%
%\infty $ or $-\infty$.
%\end{theorem}
%
%This theorem is somewhat difficult to prove, in part because it
%incorporates so many different possibilities, so we will not prove it
%here.
%
%We should also note that there may be instances where we would need to apply L'H\^opital's Rule multiple times, but we must confirm that $\lim_{x\to a}\frac{f'(x)}{g'(x)}$ is still indeterminate before we attempt to apply L'H\^opital's Rule again. Finally, we want to mention that L'H\^opital's rule is also valid for one-sided limits and limits at infinity.
%
%\begin{example}{L'H\^opital's Rule}{lhrule0}
%Compute $\ds\lim_{x\to \pi}\frac{x^2-\pi^2}{\sin x}$.
%\end{example}
%
%\begin{solution} 
%We use L'H\^opital's Rule: Since the numerator and denominator
%both approach zero,
%$$\lim_{x\to \pi}\frac{x^2-\pi^2}{\sin x}=
%\lim_{x\to \pi}\frac{2x}{\cos x},$$
%provided the latter exists. But in fact this is an easy limit, since
%the denominator now approaches $-1$, so 
%$$\lim_{x\to \pi}\frac{x^2-\pi^2}{\sin x}=\frac{2\pi}{-1} = -2\pi.$$
%\end{solution}
%
%\begin{example}{L'H\^opital's Rule}{LHRule1}
%Compute $\ds\lim_{x\to \infty}{2x^2-3x+7\over
%x^2+47x+1}$.
%\end{example}
%
%\begin{solution} 
%As $x$ goes to infinity, both the numerator and denominator go to
%infinity, so we may apply L'H\^opital's Rule:
%$$\lim_{x\to \infty}\frac{2x^2-3x+7}{x^2+47x+1}=
%\lim_{x\to \infty}\frac{4x-3}{2x+47}.$$
%In the second quotient, it is still the case that the numerator and
%denominator both go to infinity, so we are allowed to use
%L'H\^opital's Rule again:
%$$\lim_{x\to \infty}\frac{4x-3}{2x+47}=\lim_{x\to \infty}\frac{4}{2}=2.$$
%So the original limit is 2 as well.
%\end{solution}
%
%\begin{example} {L'H\^opital's Rule}{LHRule2}
%Compute $\ds\lim_{x\to 0}\frac{\sec x - 1}{\sin x}$.
%\end{example}
%
%\begin{solution} 
%Both the numerator and denominator approach zero, so applying 
%L'H\^opital's Rule:
%$$\lim_{x\to 0}\frac{\sec x - 1}{\sin x}=
%\lim_{x\to 0}\frac{\sec x\tan x}{\cos x}=\frac{1\cdot 0}{1}=0.$$
%\end{solution}
%
%L'H\^{o}pital's rule concerns limits of a quotient that are indeterminate forms. But not all
%functions are given in the form of a quotient. But all the same, nothing
%prevents us from re-writing a given function in the form of a quotient.
%Indeed, some functions whose given form involve either a product $f\left(
%x\right) g\left( x\right) $ or a power $f\left( x\right) ^{g\left( x\right)
%} $ carry indeterminacies such as $0\cdot \pm \infty $ and $1^{\pm \infty }.$
%Something small times something numerically large (positive or negative)
%could be anything. It depends on how small and how large each piece turns
%out to be. A number close to 1 raised to a numerically large (positive or
%negative) power could be anything. It depends on how close to 1 the base is,
%whether the base is larger than or smaller than 1, and how large the
%exponent is (and its sign). We can use suitable algebraic manipulations to
%relate them to indeterminate quotients. We will illustrate with two
%examples, first a product and then a power.
%
%\begin{example}{L'H\^opital's Rule}{LHRule3}
%Compute $\ds\lim_{x\to 0^+} x\ln x$.
%\end{example}
%
%\begin{solution} 
%This doesn't appear to be suitable for L'H\^opital's Rule, but it also
%is not ``obvious''. As $x$ approaches zero, $\ln x$ goes to $-\infty$,
%so the product looks like:
%$$(\hbox{something very small})\cdot(\hbox{something very large and negative}).$$
%This could be anything: it depends on {\it how small} and
%{\it how large} each piece of the function turns out to be. 
%As defined earlier, this is a type of $\pm\mathquotes{0\cdot\infty}$, which is
%indeterminate. So we can in fact apply L'H\^opital's Rule after re-writing
%it in the form $\frac{\infty }{\infty }$:
%$$x\ln x = \frac{\ln x}{1/x}=\frac{\ln x}{x^{-1}}.$$
%Now as $x$ approaches zero, both the numerator and denominator
%approach infinity (one $-\infty$ and one $+\infty$, but only the size
%is important). Using  L'H\^opital's Rule:
%$$\lim_{x\to 0^+} {\ln x\over x^{-1}}=
%\lim_{x\to 0^+} {1/x\over -x^{-2}} =\lim_{x\to 0^+} {1\over x}(-x^2)=
%\lim_{x\to 0^+} -x = 0.$$
%One way to interpret this is that since $\ds\lim_{x\to
%  0^+}x\ln x = 0$, the $x$ approaches zero much faster than the $\ln x$
%approaches $-\infty$.
%\end{solution}
%
%Finally, we illustrate how a limit of the type $\mathquotes{1^\infty}$ can be indeterminate.
%
%\begin{example} {L'H\^opital's Rule}{LHRule4}
%Evaluate $\ds{\lim_{x\to 1^+}x^{1/(x-1)}}.$
%\end{example}
%
%\begin{solution} 
%Plugging in $x=1$ (from the right) gives a limit of the type $\mathquotes{1^\infty}$.
%To deal with this type of limit we will use logarithms.
%Let
%$$L=\lim_{x\to 1^+}x^{1/(x-1)}.$$
%Now, take the natural log of both sides:
%$$\ln L=\lim_{x\to 1^+}\ln\left(x^{1/(x-1)}\right).$$
%Using log properties we have:
%$$\ln L=\lim_{x\to 1^+}\frac{\ln x}{x-1}.$$
%The right side limit is now of the type $0/0$, therefore, we can apply L'H\^opital's Rule:
%$$\ln L=\lim_{x\to 1^+}\frac{\ln x}{x-1}=\lim_{x\to 1^+}\frac{1/x}{1}=1$$
%Thus, $\ln L=1$ and hence, our original limit (denoted by $L$) is: $L=e^1=e$. That is,
%$$L=\lim_{x\to 1^+}x^{1/(x-1)}=e.$$
%In this case, even though our limit had a type of $\mathquotes{1^\infty}$, it actually had a value of $e$.
%\end{solution}

%%%%%%%%%%%%%%%%%%%%%%%%%%%%%%%%%%%%%%%%%
\Opensolutionfile{solutions}[ex]
\section*{Exercises for \ref{sec:LH}}

Compute the following limits.

\begin{multicols}{2}[]

\begin{enumialphparenastyle}



%%%%%%%%%%
\begin{ex} 
$\ds\lim_{x\to 0} {\cos x -1\over \sin x}$
\begin{sol}
 $0$
\end{sol}
\end{ex}

%%%%%%%%%%
\begin{ex} 
$\ds\lim_{x\to \infty} {e^x\over x^3}$
\begin{sol}
 $\infty$
\end{sol}
\end{ex}

%%%%%%%%%%
\begin{ex} 
$\ds\lim_{x\to \infty} {\ln x\over x}$
\begin{sol}
 $0$
\end{sol}
\end{ex}

%%%%%%%%%%
\begin{ex} 
$\ds\lim_{x\to \infty} {\ln x\over \sqrt{x}}$
\begin{sol}
 $0$
\end{sol}
\end{ex}

%%%%%%%%%%
\begin{ex} 
$\ds\lim_{x\to0}{\sqrt{9+x}-3\over x}$
\begin{sol}
 $1/6$
\end{sol}
\end{ex}

%%%%%%%%%%
\begin{ex} 
$\ds\lim_{x\to2}{2-\sqrt{x+2}\over 4-x^2}$
\begin{sol}
 $1/16$
\end{sol}
\end{ex}

%%%%%%%%%%
\begin{ex} 
 $\ds\lim_{x\to1}{\sqrt{x}-1\over \sqrt[3]{x}-1}$
\begin{sol}
 $3/2$
\end{sol}
\end{ex}

%%%%%%%%%%
\begin{ex} 
 $\ds\lim_{x\to0}{(1-x)^{1/4}-1\over x}$
\begin{sol}
 $-1/4$
\end{sol}
\end{ex}

%%%%%%%%%%
\begin{ex} 
 $\ds\lim_{t\to 0}{\left(t+{1\over t}\right)((4-t)^{3/2}-8)}$
\begin{sol}
 $-3$
\end{sol}
\end{ex}

%%%%%%%%%%
\begin{ex} 
 $\ds\lim_{t\to 0^+}\left({1\over t}+{1\over\sqrt{t}}\right)
(\sqrt{t+1}-1)$
\begin{sol}
 $1/2$
\end{sol}
\end{ex}

%%%%%%%%%%
\begin{ex} 
 $\ds\lim_{x\to 0}{x^2\over\sqrt{2x+1}-1}$
\begin{sol}
 $0$
\end{sol}
\end{ex}

%%%%%%%%%%
\begin{ex} 
 $\ds\lim_{u\to 1}{(u-1)^3\over (1/u)-u^2+3/u-3}$
\begin{sol}
 $-1$
\end{sol}
\end{ex}

%%%%%%%%%%
\begin{ex} 
 $\ds\lim_{x\to 0}{2+(1/x)\over 3-(2/x)}$
\begin{sol}
 $-1/2$
\end{sol}
\end{ex}

%%%%%%%%%%
\begin{ex} 
 $\ds\lim_{x\to 0^+}{1+5/\sqrt{x}\over 2+1/\sqrt{x}}$
\begin{sol}
 $5$
\end{sol}
\end{ex}

%%%%%%%%%%
\begin{ex} 
 $\ds\lim_{x\to\pi/2}{\cos x\over (\pi/2)-x}$
\begin{sol}
 $1$
\end{sol}
\end{ex}

%%%%%%%%%%
\begin{ex} 
 $\ds\lim_{x\to0}{e^x-1\over x}$
\begin{sol}
 $1$
\end{sol}
\end{ex}

%%%%%%%%%%
\begin{ex} 
 $\ds\lim_{x\to0}{x^2\over e^x-x-1}$
\begin{sol}
 $2$
\end{sol}
\end{ex}

%%%%%%%%%%
\begin{ex} 
 $\ds\lim_{x\to1}{\ln x\over x-1}$
\begin{sol}
 $1$
\end{sol}
\end{ex}

%%%%%%%%%%
\begin{ex} 
 $\ds\lim_{x\to0}{\ln(x^2+1)\over x}$
\begin{sol}
 $0$
\end{sol}
\end{ex}

%%%%%%%%%%
\begin{ex} 
 $\ds\lim_{x\to1}{x\ln x\over x^2-1}$
\begin{sol}
 $1/2$
\end{sol}
\end{ex}

%%%%%%%%%%
\begin{ex} 
 $\ds\lim_{x\to0}{\sin(2x)\over\ln(x+1)}$
\begin{sol}
 $2$
\end{sol}
\end{ex}

%%%%%%%%%%
\begin{ex} 
 $\ds\lim_{x\to1}{x^{1/4}-1\over x}$
\begin{sol}
 $0$
\end{sol}
\end{ex}

%%%%%%%%%%
\begin{ex} 
 $\ds\lim_{x\to1}{\sqrt{x}-1\over x-1}$
\begin{sol}
 $1/2$
\end{sol}
\end{ex}

%%%%%%%%%%
\begin{ex} 
 $\ds\lim_{x\to0}{3x^2+x+2\over x-4}$
\begin{sol}
 $-1/2$
\end{sol}
\end{ex}

%%%%%%%%%%
\begin{ex} 
 $\ds\lim_{x\to0}{\sqrt{x+1}-1\over \sqrt{x+4}-2}$
\begin{sol}
 $2$
\end{sol}
\end{ex}

%%%%%%%%%%
\begin{ex} 
 $\ds\lim_{x\to0}{\sqrt{x+1}-1\over \sqrt{x+2}-2}$
\begin{sol}
 $0$
\end{sol}
\end{ex}

%%%%%%%%%%
\begin{ex} 
 $\ds\lim_{x\to0^+}{\sqrt{x+1}+1\over\sqrt{x+1}-1}$
\begin{sol}
 $\infty$
\end{sol}
\end{ex}

%%%%%%%%%%
\begin{ex} 
 $\ds\lim_{x\to0}{\sqrt{x^2+1}-1\over\sqrt{x+1}-1}$
\begin{sol}
 $0$
\end{sol}
\end{ex}

%%%%%%%%%%
\begin{ex} 
 $\ds\lim_{x\to1}{(x+5)\left({1\over 2x}+{1\over x+2}\right)}$
\begin{sol}
 $5$
\end{sol}
\end{ex}

%%%%%%%%%%
\begin{ex} 
 $\ds\lim_{x\to2}{x^3-6x-2\over x^3+4}$
\begin{sol}
 $-1/2$
\end{sol}
\end{ex}


%%%%%%%%%%
\begin{ex} 
 {$\ds \lim_{x\to \pi} \frac{\sin x}{x-\pi}$}

\begin{sol}
  {$-1$}
\end{sol}

\end{ex}


%%%%%%%%%%
\begin{ex} 
{$\ds \lim_{x\to \pi/4} \frac{\sin x-\cos x}{\cos (2x)}$}
 
\begin{sol}
 {$-\sqrt{2}/2$}
\end{sol}

\end{ex}

%%%%%%%%%%
\begin{ex} 
 {$\ds \lim_{x\to 0} \frac{\sin (5x)}{x}$}

\begin{sol}
  {$5$}
\end{sol}

\end{ex}

%%%%%%%%%%
\begin{ex} 
{$\ds \lim_{x\to 0^+} \frac{e^x-1}{x^2}$}
 
\begin{sol}
 {$\infty$}
\end{sol}

\end{ex}

%%%%%%%%%%
\begin{ex} 
 {$\ds \lim_{x\to 0^+} \frac{e^x-x-1}{x^2}$}

\begin{sol}
  {$1/2$}
\end{sol}

\end{ex}

%%%%%%%%%%
\begin{ex} 
 {$\ds \lim_{x\to 0^+} \frac{x-\sin x}{x^3-x^2}$}

\begin{sol}
  {$0$}
\end{sol}

\end{ex}

%%%%%%%%%%
\begin{ex} 
 {$\ds \lim_{x\to \infty} \frac{\Big(\ln x\Big)^2}{x}$}

\begin{sol}
  {$0$}
\end{sol}

\end{ex}

%%%%%%%%%%
\begin{ex} 
 {$\ds \lim_{x\to \infty} x^3-x^2$}
 
\begin{sol}
 {$\infty$}
\end{sol}

\end{ex}

%%%%%%%%%%
\begin{ex} 
{$\ds \lim_{x\to \infty} \sqrt{x}-\ln x$}

\begin{sol}
 {$\infty$} 
\end{sol}

\end{ex}

%%%%%%%%%%
\begin{ex} 
{$\ds \lim_{x\to -\infty} xe^x$}

\begin{sol}
 {$0$} 
\end{sol}

\end{ex}

%%%%%%%%%%
\begin{ex} 
 {$\ds \lim_{x\to 0^+} (\sin x)^{x}$ \\ Hint: use the Squeeze Theorem.}

\begin{sol}
 {$1$} 
\end{sol}

\end{ex}

%%%%%%%%%%
\begin{ex} 
{$\ds \lim_{x\to 1^+} (1-x)^{1-x}$}
 
\begin{sol}
 {$1$}
\end{sol}

\end{ex}


%%%%%%%%%%
\begin{ex} 
{$\ds \lim_{x\to \infty} (x)^{1/x}$}

\begin{sol}
 {$1$} 
\end{sol}

\end{ex}

%%%%%%%%%%
\begin{ex} 
{$\ds \lim_{x\to 1^+} (\ln x)^{1-x}$}
 
\begin{sol}
 {$1$}
\end{sol}

\end{ex}

%%%%%%%%%%
\begin{ex} 
 {$\ds \lim_{x\to \infty} (1+x)^{1/x}$}
 
\begin{sol}
{$1$} 
\end{sol}

\end{ex}

%%%%%%%%%%
\begin{ex} 
 {$\ds \lim_{x\to \infty} (1+x^2)^{1/x}$}

\begin{sol}
 {$1$} 
\end{sol}

\end{ex}

%%%%%%%%%%
\begin{ex} 
 {$\ds \lim_{x\to \pi/2} \tan x\cos x$}

\begin{sol}
  {$1$}
\end{sol}

\end{ex}

%%%%%%%%%%
\begin{ex} 
{$\ds \lim_{x\to \pi/2} \tan x\sin (2x)$}
 
\begin{sol}
 {$2$}
\end{sol}

\end{ex}

%%%%%%%%%%
\begin{ex} 
{$\ds \lim_{x\to 1^+} \frac{1}{\ln x} - \frac{1}{x-1}$}

\begin{sol}
{$1/2$}  
\end{sol}

\end{ex}

%%%%%%%%%%
\begin{ex} 
 {$\ds \lim_{x\to 3^+} \frac{5}{x^2-9} - \frac{x}{x-3}$}
 
\begin{sol}
{$-\infty$} 
\end{sol}

\end{ex}

%%%%%%%%%%
\begin{ex} 
{$\ds \lim_{x\to \infty} x\tan (1/x)$}

\begin{sol}
{$1$}  
\end{sol}

\end{ex}

%%%%%%%%%%
\begin{ex} 
{$\ds \lim_{x\to \infty} \frac{(\ln x)^3}{x}$}
 
\begin{sol}
{$0$} 
\end{sol}

\end{ex}

%%%%%%%%%%
\begin{ex} 
 {$\ds \lim_{x\to 1} \frac{x^2+x-2}{\ln x}$}

\begin{sol}
  {$3$}
\end{sol}

\end{ex}

%%%%%%%%%%
\begin{ex} 
{$\ds \lim_{x\to 0^+} xe^{1/x}$}

\begin{sol}
 {$\infty$} 
\end{sol}

\end{ex}



\begin{ex}
Discuss what happens if we try to use L'H\^{o}pital's rule to find the limit $\lim\limits_{x\rightarrow \infty}\dfrac{x+\sin x}{x+1}$.
\end{ex}

\end{enumialphparenastyle}

\end{multicols}
\clearpage
