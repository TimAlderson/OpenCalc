\subsection{Intervals of Increase/Decrease, and the First Derivative Test}\label{sec:FirstDer}
The method of Section~\ref{subsec:LocalExtremaSubsection} for deciding whether there is a
local maximum or minimum at a critical value is not always
convenient. We can instead use information about the derivative
$f'(x)$ to decide; since we have already had to compute the derivative
to find the critical values, there is often relatively little extra
work involved in this method.

How can the derivative tell us whether there is a maximum, minimum, or
neither at a point? Suppose that $f$ is differentiable at and around $x=a$, and suppose further that $a$ is a critical point of $f$. Then we have several possibilities:
\begin{enumerate}
\item	There is a local maximum at $x=a$. This happens if $f'(x)>0$ as we approach $x=a$ from the left (i.e. when $x$ is in the vicinity of $a$, and $x<a$) and $f'(x)<0$ as we move to the right of $x=a$ (i.e. when $x$ is in the vicinity of $a$, and $x>a$).
\item	There is a local minimum at $x=a$. This happens if $f'(x)<0$ as we approach $x=a$ from the left (i.e. when $x$ is in the vicinity of $a$, and $x<a$) and $f'(x)>0$ as we move to the right of $x=a$ (i.e. when $x$ is in the vicinity of $a$, and $x>a$).
\item	There is neither a local maximum or local minimum at $x=a$. If $f'(x)$ does not change from negative to positive, or from positive to negative, as we move from the left of $x=a$ to the right of $x=a$ (that is, $f'(x)$ is positive on both sides of $x=a$, or negative on both sides of $x=a$) then there is neither a maximum nor minimum when $x=a$.
\end{enumerate}
See the first graph in Figure~\ref{fig:max and min points}
and the graph in Figure~\ref{fig:non extremum}
for examples.

\begin{example}{Local Maximum and Minimum}{localmaxandmin}
Find all local maximum and minimum points for $f(x)=\sin x+\cos
x$ using the first derivative test.  
\end{example}

\begin{solution} 
The derivative is $f'(x)=\cos
x-\sin x$ and from Example~\ref{max and min} the critical
values we need to consider are $\pi/4$ and $5\pi/4$.

We analyze the graphs of $\sin x$ and $\cos x$.
Just to the left of $\pi/4$ the cosine is larger than the
  sine, so $f'(x)$ is positive; just to the right the cosine is
  smaller than the sine, so $f'(x)$ is negative. This means there is a
  local maximum at $\pi/4$. Just to the left of $5\pi/4$ the cosine is
  smaller than the sine, and to the right the cosine is larger than
  the sine. This means that the derivative $f'(x)$ is negative to the
  left and positive to the right, so $f$ has a local minimum at
  $5\pi/4$.
\end{solution}

The above observations have obvious intuitive appeal as
you examine the graphs in Figures~\ref{fig:max and min points} and \ref{fig:non extremum}. We can extend these ideas
further and then formulate and prove a theorem: If the graph of $f$ is
increasing before (i.e., to the left of) $x=a$ and decreasing after (i.e.,
to the right of) $x=a,$ then there is a local maximum at $x=a.$ If the graph
of $f$ is decreasing before $x=a$ and increasing after $x=a,$ then there is
a local minimum at $x=a.$ If the graph of $f$ is consistently increasing on
either side of $x=a$ or consistently decreasing on either side of $x=a,$
then there is neither a local maximum nor a local minimum at $x=a.$ We can
prove the following theorem using the Mean Value Theorem.

\begin{theorem}{Intervals of Increase and Decrease}{IntervalsIncDecTheorem}
If $f^{\prime }\left( x\right) >0$ for every $x$ in an interval, then $f$ is
increasing on that interval.

If $f^{\prime }\left( x\right) <0$ for every $x$ in an interval, then $f$ is
decreasing on that interval.
\end{theorem}
\begin{proof}
We will prove the increasing case. The proof of the decreasing case
is similar. Suppose that $f^{\prime }\left( x\right) >0$ on an interval $I.$
Then $f$ is differentiable, and hence also, continuous on $I.$ If $x_{1}$
and $x_{2}$ are any two numbers in $I$ and $x_{1}<x_{2},$ then $f$ is
continuous on $\left[ x_{1},x_{2}\right] $ and differentiable on $\left(
x_{1},x_{2}\right) .$ By the Mean Value Theorem, there is some $c$ in $%
\left( x_{1},x_{2}\right) $ such that 
\begin{equation*}
f^{\prime }\left( c\right) =\frac{f\left( x_{2}\right) -f\left( x_{1}\right) 
}{x_{2}-x_{1}}.
\end{equation*}%
But $c$ must be in $I,$ and thus, since $f^{\prime }\left( x\right) >0$ for
every $x$ in $I,$ $f^{\prime }\left( c\right) >0$. Also, since $x_{1}<x_{2},$
we have $x_{2}-x_{1}>0.$ Therefore, both the left hand side and the
denominator of the right hand side are positive. It follows that the
numerator of the right hand must be positive. That is, $f\left( x_{2}\right)
-f\left( x_{1}\right) >0,$ or in other words, $f\left( x_{1}\right) <f\left(
x_{2}\right) .$ This shows that between $x_{1}$ and $x_{2}$ in $I,$ the
larger one, $x_{2},$ necessarily has the larger function value, $f\left(
x_{2}\right) ,$ and the smaller one, $x_{1},$ necessarily have the smaller
function value, $f\left( x_{1}\right) .$ This means that $f$ is increasing
on $I$.
\end{proof}

\begin{example}{Local Minimum and Maximum}{LocalMaxMinExample}
Consider the function $f\left( x\right) =x^{4}-2x^{2}.$ Find
where $f$ is increasing and where $f$ is decreasing. Use this information to
find the local maximum and minimum points of $f$.
\end{example}
\begin{solution}
We compute $f^{\prime }\left( x\right) $ and analyze its sign. 
\begin{equation*}
f^{\prime }\left( x\right) =4x^{3}-4x=4x\left( x^{2}-1\right) =4x\left(
x-1\right) \left( x+1\right) .
\end{equation*}%
The solution of the inequality $f^{\prime }\left( x\right) >0$ is $\left(
-1,0\right) \cup \left( 1,\infty \right) $. So, $f$ is increasing on the
interval $\left( -1,0\right) $ and on the interval $\left( 1,\infty \right)
. $ The solution of the inequality $f^{\prime }\left( x\right) <0$ is $%
\left( -\infty ,-1\right) \cup \left( 0,1\right) .$ So, $f$ is decreasing on
the interval $\left( -\infty ,-1\right) $ and on the interval $\left(
0,1\right) .$ Therefore, at the critical points $-1,$ $0$ and $1,$
respectively, $f$ has a local minimum, a local maximum and a local minimum.
\end{solution}

%\figure[!ht]
%\vbox{\beginpicture
%\normalgraphs
%%\ninepoint
%\setcoordinatesystem units <1.5truecm,1.5truecm>
%\setplotarea x from 0 to 6.28, y from -1 to 1
%\axis left shiftedto x=0 /
%\axis bottom shiftedto y=0 ticks withvalues {${\pi\over4}$}
%      {$5\pi\over4$} / at 0.7853981635 3.926990818 / /
%\setquadratic
%\plot
%0.000 0.000 
%0.063 0.063 0.126 0.125 0.188 0.187 0.251 0.249 
%0.314 0.309 0.377 0.368 0.440 0.426 0.503 0.482 0.565 0.536 
%0.628 0.588 0.691 0.637 0.754 0.685 0.817 0.729 0.880 0.771 
%0.942 0.809 1.005 0.844 1.068 0.876 1.131 0.905 1.194 0.930 
%1.257 0.951 1.319 0.969 1.382 0.982 1.445 0.992 1.508 0.998 
%1.571 1.000 1.634 0.998 1.696 0.992 1.759 0.982 1.822 0.969 
%1.885 0.951 1.948 0.930 2.011 0.905 2.073 0.876 2.136 0.844 
%2.199 0.809 2.262 0.771 2.325 0.729 2.388 0.685 2.450 0.637 
%2.513 0.588 2.576 0.536 2.639 0.482 2.702 0.426 2.765 0.368 
%2.827 0.309 2.890 0.249 2.953 0.187 3.016 0.125 3.079 0.063 
%3.142 0.000 3.204 -0.063 3.267 -0.125 3.330 -0.187 3.393 -0.249 
%3.456 -0.309 3.519 -0.368 3.581 -0.426 3.644 -0.482 3.707 -0.536 
%3.770 -0.588 3.833 -0.637 3.896 -0.685 3.958 -0.729 4.021 -0.771 
%4.084 -0.809 4.147 -0.844 4.210 -0.876 4.273 -0.905 4.335 -0.930 
%4.398 -0.951 4.461 -0.969 4.524 -0.982 4.587 -0.992 4.650 -0.998 
%4.712 -1.000 4.775 -0.998 4.838 -0.992 4.901 -0.982 4.964 -0.969 
%5.027 -0.951 5.089 -0.930 5.152 -0.905 5.215 -0.876 5.278 -0.844 
%5.341 -0.809 5.404 -0.771 5.466 -0.729 5.529 -0.685 5.592 -0.637 
%5.655 -0.588 5.718 -0.536 5.781 -0.482 5.843 -0.426 5.906 -0.368 
%5.969 -0.309 6.032 -0.249 6.095 -0.187 6.158 -0.125 6.220 -0.063 
%6.283 0.000 /
%\plot
%0.000 1.000 0.063 0.998 0.126 0.992 0.188 0.982 0.251 0.969 
%0.314 0.951 0.377 0.930 0.440 0.905 0.503 0.876 0.565 0.844 
%0.628 0.809 0.691 0.771 0.754 0.729 0.817 0.685 0.880 0.637 
%0.942 0.588 1.005 0.536 1.068 0.482 1.131 0.426 1.194 0.368 
%1.257 0.309 1.319 0.249 1.382 0.187 1.445 0.125 1.508 0.063 
%1.571 0.000 1.634 -0.063 1.696 -0.125 1.759 -0.187 1.822 -0.249 
%1.885 -0.309 1.948 -0.368 2.011 -0.426 2.073 -0.482 2.136 -0.536 
%2.199 -0.588 2.262 -0.637 2.325 -0.685 2.388 -0.729 2.450 -0.771 
%2.513 -0.809 2.576 -0.844 2.639 -0.876 2.702 -0.905 2.765 -0.930 
%2.827 -0.951 2.890 -0.969 2.953 -0.982 3.016 -0.992 3.079 -0.998 
%3.142 -1.000 3.204 -0.998 3.267 -0.992 3.330 -0.982 3.393 -0.969 
%3.456 -0.951 3.519 -0.930 3.581 -0.905 3.644 -0.876 3.707 -0.844 
%3.770 -0.809 3.833 -0.771 3.896 -0.729 3.958 -0.685 4.021 -0.637 
%4.084 -0.588 4.147 -0.536 4.210 -0.482 4.273 -0.426 4.335 -0.368 
%4.398 -0.309 4.461 -0.249 4.524 -0.187 4.587 -0.125 4.650 -0.063 
%4.712 0.000 4.775 0.063 4.838 0.125 4.901 0.187 4.964 0.249 
%5.027 0.309 5.089 0.368 5.152 0.426 5.215 0.482 5.278 0.536 
%5.341 0.588 5.404 0.637 5.466 0.685 5.529 0.729 5.592 0.771 
%5.655 0.809 5.718 0.844 5.781 0.876 5.843 0.905 5.906 0.930 
%5.969 0.951 6.032 0.969 6.095 0.982 6.158 0.992 6.220 0.998 
%6.283 1.000 /
%\endpicture}
%\label{fig:sin and cos}
%\caption{The sine and cosine.}
%\endfigure


%%%%%%%%%%%%%%%%%%%%%%%%%%%%%%%%%%%%%%%%%%%%
\Opensolutionfile{solutions}[ex]
\section*{Exercises for \ref{sec:FirstDer}}

\begin{enumialphparenastyle}

Find all critical points and identify them as local maximum points, local minimum points, or neither.

%%%%%%%%%%
\begin{ex}
 $\ds y=x^2-x$ 
\begin{sol}
 min at $x=1/2$
\end{sol}
\end{ex}

%%%%%%%%%%
\begin{ex}
 $\ds y=2+3x-x^3$ 
\begin{sol}
 min at $x=-1$, max at $x=1$
\end{sol}
\end{ex}

%%%%%%%%%%
\begin{ex}
 $\ds y=x^3-9x^2+24x$
\begin{sol}
 max at $x=2$, min at $x=4$
\end{sol}
\end{ex}

%%%%%%%%%%
\begin{ex}
 $\ds y=x^4-2x^2+3$ 
\begin{sol}
 min at $x=\pm 1$, max at $x=0$.
\end{sol}
\end{ex}

%%%%%%%%%%
\begin{ex}
 $\ds y=3x^4-4x^3$
\begin{sol}
 min at $x=1$
\end{sol}
\end{ex}

%%%%%%%%%%
\begin{ex}
 $\ds y=(x^2-1)/x$
\begin{sol}
 none
\end{sol}
\end{ex}

%%%%%%%%%%
\begin{ex}
 $\ds y=3x^2-(1/x^2)$ 
\begin{sol}
 none
\end{sol}
\end{ex}

%%%%%%%%%%
\begin{ex}
 $y=\cos(2x)-x$ 
\begin{sol}
 min at $x=7\pi/12+k\pi$, max at $x=-\pi/12+k\pi$, for integer $k$.
\end{sol}
\end{ex}

%%%%%%%%%%
\begin{ex}
$\ds f(x) = (5-x)/(x+2)$
\begin{sol}
 none
\end{sol}
\end{ex}

%%%%%%%%%%
\begin{ex}
 $\ds f(x) = |x^2 - 121|$
\begin{sol}
 max at $x=0$, min at $x=\pm 11$
\end{sol}
\end{ex}

%%%%%%%%%%
\begin{ex}
 $\ds f(x) = x^3/(x+1)$
\begin{sol}
 min at $x=-3/2$, neither at $x=0$
\end{sol}
\end{ex}

%%%%%%%%%%%
%\begin{ex}
% $\ds f(x)= \cases{x^2 \sin(1/x)  & $x\neq 0$ \cr
% 0  & $x=0$\cr}$

%%%%%%%%%%
\begin{ex}
 $\ds f(x) = \sin ^2 x$
\begin{sol}
 min at $n\pi$, max at $\pi/2+n\pi$
\end{sol}
\end{ex}

%%%%%%%%%%
\begin{ex}
 Find the maxima and minima of $f(x)=\sec x$.
\begin{sol}
 min at $2n\pi$, max at $(2n+1)\pi$
\end{sol}
\end{ex}

%%%%%%%%%%
\begin{ex}
  Let $\ds f(\theta) = \cos^2(\theta) -
 2\sin(\theta)$.  Find the intervals where $f$ is increasing and the
 intervals where $f$ is decreasing in $[0,2\pi]$.  Use this
 information to classify the critical points of $f$ as either local
 maximums, local minimums, or neither.
\begin{sol}
 min at $\pi/2+2n\pi$, max at $3\pi/2+2n\pi$
\end{sol}
\end{ex}

%%%%%%%%%%
\begin{ex}
 Let $r>0$. Find the local
maxima and minima of the function $\ds f(x)
=\sqrt{r^2 -x^2 }$ on its domain $[-r,r]$.
\end{ex}

%%%%%%%%%%
\begin{ex}
 Let $\ds f(x) =a x^2 + bx + c$ with $a\neq 0$. Show that $f$
has exactly one critical point using the first derivative test. Give
conditions on $a$ and $b$ which guarantee that the critical point will
be a maximum. It is possible to see this without using calculus at
all; explain.
\end{ex}

\end{enumialphparenastyle}