\subsection{The Second Derivative Test}\label{sec:SecondDer}
The basis of the first derivative test is that if the derivative
changes from positive to negative at a point at which the derivative
is zero then there is a local maximum at the point, and similarly for
a local minimum. If $f'$ changes from positive to negative it is
decreasing; this means that the derivative of $f'$, $f''$, might be negative,
and if in fact $f''$ is negative then $f'$ is definitely
decreasing. From this we determine that there is a local maximum at the point in question. Note
that $f'$ might change from positive to negative while $f''$ is
zero, in which case $f''$ gives us no information about the critical
value. Similarly, if $f'$ changes from negative to positive there is a
local minimum at the point, and $f'$ is increasing. If $f''>0$ at the
point, this tells us that $f'$ is increasing, and so there is a local
minimum.

\begin{example}{Second Derivative}{secondderivative}
Consider again $f(x)=\sin x + \cos x$,  with $f'(x)=\cos x-\sin x$ and
$ f''(x)=-\sin x -\cos x$. Use the second derivative test to determine which critical points are local maximum or minima.
\end{example}
\begin{solution}
Since $\ds f''(\pi/4)=-\sqrt{2}/2-\sqrt2/2=-\sqrt2<0$,
we know there is a local maximum at $\pi/4$. Since
$\ds f''(5\pi/4)=-(-\sqrt{2}/2)-(-\sqrt2/2)=\sqrt2>0$, there is a local
minimum at $5\pi/4$.
\end{solution}

When it works, the second derivative test is often the easiest way to
identify local maximum and minimum points. Sometimes the test fails,
and sometimes the second derivative is quite difficult to evaluate; in
such cases we must fall back on one of the previous tests.

\begin{example}{Second Derivative}{secondderivativetwo}
Let $\ds f(x)=x^4$ and $\ds g(x)=-x^4$. Classify the critical points of $f(x)$ and $g(x)$ as either maximum or minimum.
\end{example}
\begin{solution}
The derivatives for $f(x)$ are $f'(x)=4x^3$ and $\ds f''(x)=12x^2$.
Zero is the only critical value, but $f''(0)=0$, so
the second derivative test tells us nothing. However, $f(x)$ is
positive everywhere except at zero, so clearly $f(x)$ has a local
minimum at zero.

On the other hand, for $g(x)=-x^4$, $g'(x)=-4x^3$ and $g''(x)=-12x^2$. So $g(x)$ also has zero
as its only only critical value, and the second derivative is again zero, but $-x^4$ has a local maximum at zero.
\end{solution}


%%%%%%%%%%%%%%%%%%%%%%%%%%%%%%%%%%%%%%%%%%%%
\Opensolutionfile{solutions}[ex]
\section*{Exercises for \ref{sec:SecondDer}}

\begin{enumialphparenastyle}

Find all local maximum and minimum points by the second derivative test. 


%%%%%%%%%%
\begin{ex}
 $\ds y=x^2-x$ 
\begin{sol}
 min at $x=1/2$
\end{sol}
\end{ex}

%%%%%%%%%%
\begin{ex}
 $\ds y=2+3x-x^3$ 
\begin{sol}
 min at $x=-1$, max at $x=1$
\end{sol}
\end{ex}

%%%%%%%%%%
\begin{ex}
 $\ds y=x^3-9x^2+24x$
\begin{sol}
 max at $x=2$, min at $x=4$
\end{sol}
\end{ex}

%%%%%%%%%%
\begin{ex}
 $\ds y=x^4-2x^2+3$ 
\begin{sol}
 min at $x=\pm 1$, max at $x=0$.
\end{sol}
\end{ex}

%%%%%%%%%%
\begin{ex}
 $\ds y=3x^4-4x^3$
\begin{sol}
 min at $x=1$
\end{sol}
\end{ex}

%%%%%%%%%%
\begin{ex}
 $\ds y=(x^2-1)/x$
\begin{sol}
 none
\end{sol}
\end{ex}

%%%%%%%%%%
\begin{ex}
 $\ds y=3x^2-(1/x^2)$ 
\begin{sol}
 none
\end{sol}
\end{ex}

%%%%%%%%%%
\begin{ex}
 $y=\cos(2x)-x$ 
\begin{sol}
 min at $x=7\pi/12+n\pi$, max at $x=-\pi/12+n\pi$, for integer $n$.
\end{sol}
\end{ex}

%%%%%%%%%%
\begin{ex}
 $\ds y = 4x+\sqrt{1-x}$
\begin{sol}
 max at $x=63/64$
\end{sol}
\end{ex}

%%%%%%%%%%
\begin{ex}
 $\ds y = (x+1)/\sqrt{5x^2 + 35}$
\begin{sol}
 max at $x=7$
\end{sol}
\end{ex}

%%%%%%%%%%
\begin{ex}
 $\ds y= x^5 - x$
\begin{sol}
 max at $\ds -5^{-1/4}$, min at $\ds 5^{-1/4}$
\end{sol}
\end{ex}

%%%%%%%%%%
\begin{ex}
 $\ds y = 6x +\sin 3x$
\begin{sol}
 none
\end{sol}
\end{ex}

%%%%%%%%%%
\begin{ex}
 $\ds y = x+ 1/x$
\begin{sol}
 max at $-1$, min at $1$
\end{sol}
\end{ex}

%%%%%%%%%%
\begin{ex}
 $\ds y = x^2+ 1/x$
\begin{sol}
 min at $\ds 2^{-1/3}$
\end{sol}
\end{ex}

%%%%%%%%%%
\begin{ex}
 $\ds y = (x+5)^{1/4}$
\begin{sol}
 none
\end{sol}
\end{ex}

%%%%%%%%%%
\begin{ex}
 $\ds y = \tan^2 x$
\begin{sol}
 min at $n\pi$
\end{sol}
\end{ex}

%%%%%%%%%%
\begin{ex}
 $\ds y =\cos^2 x - \sin^2 x$
\begin{sol}
 max at $n\pi$, min at $\pi/2+n\pi$
\end{sol}
\end{ex}

%%%%%%%%%%
\begin{ex}
 $\ds y = \sin^3 x$
\begin{sol}
 max at $\pi/2+2n\pi$, min at $3\pi/2+2n\pi$
\end{sol}
\end{ex}

\end{enumialphparenastyle}