\subsection{Taylor Polynomials}\label{sec:Taylor}
We can go beyond first order derivatives to create polynomials approximating a function as closely as we wish, these are called $Taylor$ $Polynomials$.

While our linear approximation $L(x)=f'(a)(x-a)+f(a)$ at a point $a$ was a polynomial of degree 1 such that both $L(a)=f(a)$ and $L'(a)=f'(a)$, we can now form a polynomial
\[ T_n(x)=a_0+a_1(x-a)+a_2(x-a)^2+a_3(x-a)^3+\dots +a_n(x-a)^n \]
which has the same first $n$ derivatives at $x=a$ as the function $f$.

By successively computing the derivatives of $T_n$, we obtain:
\[ \begin{array}{l}
a_0 = f(a)=\frac{f(a)}{0!}\\
a_1 = \frac{f'(a)}{1!}\\
a_2 = \frac{f''(a)}{2!}\\
\cdots \\
a_k = \frac{f^{(k)}(a)}{k!}\\
\cdots a_n =\frac{f^{(n)}(a)}{n!}
\end{array} \]
where $f^{(k)}(x)$ is the $k^{th}$ derivative of $f(x)$, and
$n!=n(n-1)(n-2)\ldots (2)(1) $, referred to as \ifont{factorial} notation.

Here is an example.

\begin{example}{Approximate e using Taylor Polynomials}{approximate e using taylor polynomials}
Approximate $e^x$ using Taylor polynomials at $a=0$, and use this to approximate $e$.
\end{example}

\begin{solution}
In this case we use the function $f(x)=e^x$ at $a=0$, and therefore
\[ T_n(x)=a_0+a_1x+a_xx^2+a_3x^3+\ldots +a_nx^n \]

Since all derivatives $f^{(k)}(x)=e^x$, we get:
\[ \begin{array}{l}
a_0=f(0)=1 \\
a_1=\frac{f'(0)}{1!}=1 \\
a_2=\frac{f''(0)}{2!}=\frac{1}{2!} \\
a_3=\frac{f'''(0)}{3!}=\frac{1}{3!} \\
\cdots \\
a_k=\frac{f^{(k)}(0)}{k!}=\frac{1}{k!} \\
\cdots \\
a_n=\frac{f^{(n)}(0)}{n!}=\frac{1}{n!} 
\end{array} \]
Thus
\[ \begin{array}{l}
T_1(x)=1+x=L(x) \\
T_2(x)=1+x+\frac{x^2}{2!} \\
T_3(x)=1+x+\frac{x^2}{2!}+\frac{x^3}{3!}
\end{array} \]
and in general
\[ T_n(x)=1+x+\frac{x^2}{2!}+\frac{x^3}{3!}+\cdots +\frac{x^n}{n!}. \]

Finally we can approximate $e=f(1)$ by simply calculating $T_n(1)$. A few values are:
\[ \begin{array}{l}
T_1(1)=1+1=2 \\
T_2(1)=1+1+\frac{1^2}{2!}=2.5 \\
T_4(1)=1+1+\frac{1^2}{2!}+\frac{1^3}{3!}=2.\overline{6} \\
T_8(1)=2.71825396825 \\
T_{20}(1)=2.71828182845
\end{array} \]

We can continue this way for larger values of $n$, but $T_{20}(1)$ is already a pretty good approximation of $e$, and we took only 20 terms!
\end{solution}
%
%\subsubsection{Taylor's Theorem}\label{subsubsec:TaylorsTheoremsubsubsec}
%We have now seen how a polynomial is used to approximate a function $f$ near $a$. But how close is our approximation to the actual function? That is to say, can we measure the error in our approximation?
%
%\begin{theorem}{Taylor's Theorem}{TaylorTheorem}
%Suppose $f$ is defined and has $n+1$ continuous derivatives on an open interval $I$ containing $a$. Then for each $x$ in the interval,
%\[\ds f(x)=\left[\sum_{k=0}^n\frac{f^{(k)}(a)}{k!}(x-a)^k\right]+R_{n+1}(x)\]
%where the error term is 
%\[R_{n+1}(x)=\frac{f^{(n+1)}(z)}{(n+1)!}(x-a)^{n+1}\]
%for some $z$ between $a$ and $x$.
%\end{theorem}
%
%The form for the error $R_{n+1}(x)$ is called the \dfont{Lagrange Formula} for the remainder. Notice that this error term looks quite similar to how we would expect the next term in the Taylor Polynomial of $f(x)$ to look. The main difference is that $f^{(n+1)}$ is evaluated at a point $z$ and not necessarily at the center $a$, and we do not know what $z$ might be.
%
%Notice as well that when $n=0$, Taylor's Theorem is precisely the Mean Value Theorem, which is mainly how we would prove Taylor's Theorem (but will not prove here). The following corollary is useful when applying the theorem.
%
%\begin{corollary}{Corollary to Taylor's Theorem}{TaylorTheoremCorollary}
%Let $\mu_{n+1}$ be the maximum of $|f^{(n+1)}(x)|$ on some interval containing the center $a$. Then for any $x$ in this interval, the error of the Taylor Polynomial is bounded as follows.
%\[\ds |f(x)-T_n(x)|\leq \frac{\mu_{n+1}}{(n+1)!}|x-a|^{n+1}.\]
%\end{corollary}
%
%Observe that if $f^{(n+1)}(x)=0$, then the Taylor Polynomial of degree $n$ is an exact approximation of the function $f$. Why? If the $(n+1)$\textsuperscript{th} derivative of a function is 0, then $f$ is simply a polynomial of degree at most $n$.
%
%Let's look at an example:
%
%\begin{example}{Approximate Square Root}{ApproxSquareRootTaylorTheorem}
%Approximate $\sqrt{11}$ to accuracy of at least 0.01.
%\end{example}
%\begin{solution}
%The 2nd degree Taylor Polynomial of $f(x)=\sqrt{x}$ centered at $x=9$ is given by:
%\[\ds T_2(x)=3+\frac{1}{6}(x-9)-\frac{1}{216}(x-9)^2.\]
%For our purposes a 2nd degree polynomial should be sufficient to produce our desired level of accuracy, which we will see shortly. If we require greater accuracy, we can simply increase the degree of the Taylor Polynomial. At $x=11$ this polynomial approximation gives $T_2(10)=3.3\overline{148}$.
%
%Now we must determine the accuracy of our approximation. We first find that the third derivative of $\sqrt{x}$ if $f'''(x)=\frac{3}{8}x^{-5/2}$, and that this function's maximum in the interval $[9,11]$ occurs at $x=9$ (since it is a positive decreasing function). Thus, the maximum value is $\mu_3=f'''(9)=\frac{3}{8}\cdot\frac{1}{243}=\frac{3}{1944}=\frac{1}{648}$. Applying the corollary we get
%\begin{align*}
%\left|\sqrt{11}-3.3\overline{148}\right|&\leq\frac{1}{648}\frac{1}{3!}(11-9)^3	\\
%&\leq\frac{1}{486}\approx 0.0020576
%\end{align*}
%Therefore, the approximation $3.3\overline{148}$ is guaranteed to be accurate to at least $\frac{1}{486}$, which is less than 0.01.
%\end{solution}


%%%%%%%%%%%%%%%%%%%%%%%%%%%%%%%%%%%%%%%%%%%%%%%
\Opensolutionfile{solutions}[ex]
\section*{Exercises for \ref{sec:Taylor}}

\begin{enumialphparenastyle}

%%%%%%%%%%
\begin{ex} 
Find the 5\textsuperscript{th} degree Taylor polynomial for $f(x)=\sin x$ around $a=0$.
\begin{enumerate}
	\item	Use this Taylor polynomial to approximate $\sin (0.1)$.
	\item	Use a calculator to find $\sin (0.1)$. How does this compare to our approximation in part (a)?
\end{enumerate}
\begin{sol}
$T_5(x)=x-\frac{x^3}{3!}+\frac{x^5}{5!}$
\begin{enumerate}
	\item	$\sin (0.1)\approx T_5(0.1)\approx 0.10016675$
	\item	$\sin (0.1)=0.0998334\ldots$ using a calculator. Our approximation is accurate to $0.10016675-0.0998334\ldots =0.000\bar{3}$.
\end{enumerate}
\end{sol}
\end{ex}


%%%%%%%%%%
\begin{ex}
Suppose that $f^{\prime\prime}$ exists and is continuous on $[1,2]$.
Suppose also that $\left\vert f^{\prime\prime}(x)\right\vert \leq \frac{1}{4}$
for all $x$ in $(1,2)$. Prove that if we use the linearization $y=L(x)$
of $y=f(x)$ at $x=1$ as an approximation of $y=f(x)$ near $x=1$,
then our estimated value of $f(1.2)$ is
guaranteed to have an accuracy of at least 0.01, i.e., our estimate will lie
within 0.01 units of the true value.
\end{ex}


%%%%%%%%%%
\begin{ex} 
Find the 3\textsuperscript{rd} degree Taylor polynomial for $f(x)=\frac{1}{1-x}-1$ around $a=0$. Explain why this approximation would not be useful for calculating $f(5)$.
\begin{sol}
	$T_3(x)=x+x^2+x^3$. The point $x=5$ is not close to $x=0$, and $f$ is not continuous at $x=1$.
\end{sol}
\end{ex}

%%%%%%%%%%
\begin{ex} 
Consider $f(x)=\ln x$ around $a=1$.
\begin{enumerate}
	\item	Find a general formula for $f^{(n)}(x)$ for $n\geq 1$.
	\item	Find a general formula for the Taylor Polynomial, $T_n(x)$.
\end{enumerate}
\begin{sol}
\begin{enumerate}
	\item	$f^{(n)}(x)=\frac{(-1)^{(n-1)}(n-1)!}{x^n}$
	\item	$T_n(x)=\ln (1)+\displaystyle\sum_{i=1}^{n} \frac{\big(\frac{(-1)^{(i-1)}(i-1)!}{1^n}\big)}{i!}(x-1)^i=\displaystyle\sum_{i=1}^{n} \bigg(\frac{(-1)^{(i-1)}(i-1)!}{i!}\bigg)(x-1)^i$ since $\ln (1)=0$ and $1^n=1$.
\end{enumerate}
\end{sol}
\end{ex}

%% % % % % % % % % %
%% Exercises for Taylor's Theorem
%% % % % % % % % % %
%\begin{ex}
%Approximate $\ln(1.3)$ to accuracy of at least 0.0001.
%\end{ex}
%
%\begin{ex}
%Determine a general inequality describing the error of Taylor Polynomial approximations of $f(x)=\sin(x)$ and $g(x)=\cos(x)$. \ifont{(Hint: Build the inequality in Corollary~\ref{cor:TaylorTheoremCorollary})}
%\begin{sol}
%	\begin{align*}
%	|\sin(x)-T_n(x)|&\leq\frac{1}{(n+1)!}x^{n+1}	\\
%	|\cos(x)-T_n(x)|&\leq\frac{1}{(n+1)!}x^{n+1}
%	\end{align*}
%\end{sol}
%\end{ex}
%
%% % % % % % % % % %

\end{enumialphparenastyle}