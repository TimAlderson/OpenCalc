\section{L'H\^opital's Rule}\label{sec:LH}
The following application of derivatives allows us to compute certain limits.

\begin{definition}{Limits of the Indeterminate Forms $\frac{0}{0}$ and $\frac{\infty}{\infty}$}{IndeterminateLimits}
A limit of a quotient $\lim\limits_{x\rightarrow a}\frac{f\left( x\right) }{%
	g\left( x\right) }$ is said to be an \textbf{indeterminate form of the type }%
$\frac{0}{0}$ if both $f\left( x\right) \rightarrow 0$ and $g\left( x\right)
\rightarrow 0$ as $x\rightarrow a.$ Likewise, it is said to be an \textbf{%
	indeterminate form of the type} $\frac{\infty }{\infty }$ if both $f\left(
x\right) \rightarrow \pm \infty $ and $g\left( x\right) \rightarrow \pm
\infty $ as $x\rightarrow a$ (Here, the two $\pm $ signs are independent of
each other).
\end{definition}

\begin{theorem}{L'H\^opital's Rule}{LHRule}
For a limit $\lim\limits_{x\rightarrow a}\frac{f\left( x\right) }{g\left(
	x\right) }$ of the indeterminate form $\frac{0}{0}$ or $\frac{\infty }{%
	\infty },$ $\lim\limits_{x\rightarrow a}\frac{f\left( x\right) }{g\left(
	x\right) }=\lim\limits_{x\rightarrow a}\frac{f^{\prime }\left( x\right) }{%
	g^{\prime }\left( x\right) }$ if $\lim\limits_{x\rightarrow a}\frac{%
	f^{\prime }\left( x\right) }{g^{\prime }\left( x\right) }$ exists or equals $%
\infty $ or $-\infty$.
\end{theorem}

This theorem is somewhat difficult to prove, in part because it
incorporates so many different possibilities, so we will not prove it
here.

We should also note that there may be instances where we would need to apply L'H\^opital's Rule multiple times, but we must confirm that $\lim_{x\to a}\frac{f'(x)}{g'(x)}$ is still indeterminate before we attempt to apply L'H\^opital's Rule again. Finally, we want to mention that L'H\^opital's rule is also valid for one-sided limits and limits at infinity.

\begin{example}{L'H\^opital's Rule}{lhrule0}
Compute $\ds\lim_{x\to \pi}\frac{x^2-\pi^2}{\sin x}$.
\end{example}

\begin{solution} 
We use L'H\^opital's Rule: Since the numerator and denominator
both approach zero,
$$\lim_{x\to \pi}\frac{x^2-\pi^2}{\sin x}=
\lim_{x\to \pi}\frac{2x}{\cos x},$$
provided the latter exists. But in fact this is an easy limit, since
the denominator now approaches $-1$, so 
$$\lim_{x\to \pi}\frac{x^2-\pi^2}{\sin x}=\frac{2\pi}{-1} = -2\pi.$$
\end{solution}

\begin{example}{L'H\^opital's Rule}{LHRule1}
Compute $\ds\lim_{x\to \infty}{2x^2-3x+7\over
x^2+47x+1}$.
\end{example}

\begin{solution} 
As $x$ goes to infinity, both the numerator and denominator go to
infinity, so we may apply L'H\^opital's Rule:
$$\lim_{x\to \infty}\frac{2x^2-3x+7}{x^2+47x+1}=
\lim_{x\to \infty}\frac{4x-3}{2x+47}.$$
In the second quotient, it is still the case that the numerator and
denominator both go to infinity, so we are allowed to use
L'H\^opital's Rule again:
$$\lim_{x\to \infty}\frac{4x-3}{2x+47}=\lim_{x\to \infty}\frac{4}{2}=2.$$
So the original limit is 2 as well.
\end{solution}

\begin{example} {L'H\^opital's Rule}{LHRule2}
Compute $\ds\lim_{x\to 0}\frac{\sec x - 1}{\sin x}$.
\end{example}

\begin{solution} 
Both the numerator and denominator approach zero, so applying 
L'H\^opital's Rule:
$$\lim_{x\to 0}\frac{\sec x - 1}{\sin x}=
\lim_{x\to 0}\frac{\sec x\tan x}{\cos x}=\frac{1\cdot 0}{1}=0.$$
\end{solution}

L'H\^{o}pital's rule concerns limits of a quotient that are indeterminate forms. But not all
functions are given in the form of a quotient. But all the same, nothing
prevents us from re-writing a given function in the form of a quotient.
Indeed, some functions whose given form involve either a product $f\left(
x\right) g\left( x\right) $ or a power $f\left( x\right) ^{g\left( x\right)
} $ carry indeterminacies such as $0\cdot \pm \infty $ and $1^{\pm \infty }.$
Something small times something numerically large (positive or negative)
could be anything. It depends on how small and how large each piece turns
out to be. A number close to 1 raised to a numerically large (positive or
negative) power could be anything. It depends on how close to 1 the base is,
whether the base is larger than or smaller than 1, and how large the
exponent is (and its sign). We can use suitable algebraic manipulations to
relate them to indeterminate quotients. We will illustrate with two
examples, first a product and then a power.

\begin{example}{L'H\^opital's Rule}{LHRule3}
Compute $\ds\lim_{x\to 0^+} x\ln x$.
\end{example}

\begin{solution} 
This doesn't appear to be suitable for L'H\^opital's Rule, but it also
is not ``obvious''. As $x$ approaches zero, $\ln x$ goes to $-\infty$,
so the product looks like:
$$(\hbox{something very small})\cdot(\hbox{something very large and negative}).$$
This could be anything: it depends on {\it how small} and
{\it how large} each piece of the function turns out to be. 
As defined earlier, this is a type of $\pm\mathquotes{0\cdot\infty}$, which is
indeterminate. So we can in fact apply L'H\^opital's Rule after re-writing
it in the form $\frac{\infty }{\infty }$:
$$x\ln x = \frac{\ln x}{1/x}=\frac{\ln x}{x^{-1}}.$$
Now as $x$ approaches zero, both the numerator and denominator
approach infinity (one $-\infty$ and one $+\infty$, but only the size
is important). Using  L'H\^opital's Rule:
$$\lim_{x\to 0^+} {\ln x\over x^{-1}}=
\lim_{x\to 0^+} {1/x\over -x^{-2}} =\lim_{x\to 0^+} {1\over x}(-x^2)=
\lim_{x\to 0^+} -x = 0.$$
One way to interpret this is that since $\ds\lim_{x\to
  0^+}x\ln x = 0$, the $x$ approaches zero much faster than the $\ln x$
approaches $-\infty$.
\end{solution}

Finally, we illustrate how a limit of the type $\mathquotes{1^\infty}$ can be indeterminate.

\begin{example} {L'H\^opital's Rule}{LHRule4}
Evaluate $\ds{\lim_{x\to 1^+}x^{1/(x-1)}}.$
\end{example}

\begin{solution} 
Plugging in $x=1$ (from the right) gives a limit of the type $\mathquotes{1^\infty}$.
To deal with this type of limit we will use logarithms.
Let
$$L=\lim_{x\to 1^+}x^{1/(x-1)}.$$
Now, take the natural log of both sides:
$$\ln L=\lim_{x\to 1^+}\ln\left(x^{1/(x-1)}\right).$$
Using log properties we have:
$$\ln L=\lim_{x\to 1^+}\frac{\ln x}{x-1}.$$
The right side limit is now of the type $0/0$, therefore, we can apply L'H\^opital's Rule:
$$\ln L=\lim_{x\to 1^+}\frac{\ln x}{x-1}=\lim_{x\to 1^+}\frac{1/x}{1}=1$$
Thus, $\ln L=1$ and hence, our original limit (denoted by $L$) is: $L=e^1=e$. That is,
$$L=\lim_{x\to 1^+}x^{1/(x-1)}=e.$$
In this case, even though our limit had a type of $\mathquotes{1^\infty}$, it actually had a value of $e$.
\end{solution}

%%%%%%%%%%%%%%%%%%%%%%%%%%%%%%%%%%%%%%%%%
\Opensolutionfile{solutions}[ex]
\section*{Exercises for \ref{sec:LH}}

\begin{enumialphparenastyle}

Compute the following limits.

%%%%%%%%%%
\begin{ex} 
$\ds\lim_{x\to 0} {\cos x -1\over \sin x}$
\begin{sol}
 $0$
\end{sol}
\end{ex}

%%%%%%%%%%
\begin{ex} 
$\ds\lim_{x\to \infty} {e^x\over x^3}$
\begin{sol}
 $\infty$
\end{sol}
\end{ex}

%%%%%%%%%%
\begin{ex} 
$\ds\lim_{x\to \infty} {\ln x\over x}$
\begin{sol}
 $0$
\end{sol}
\end{ex}

%%%%%%%%%%
\begin{ex} 
$\ds\lim_{x\to \infty} {\ln x\over \sqrt{x}}$
\begin{sol}
 $0$
\end{sol}
\end{ex}

%%%%%%%%%%
\begin{ex} 
$\ds\lim_{x\to0}{\sqrt{9+x}-3\over x}$
\begin{sol}
 $1/6$
\end{sol}
\end{ex}

%%%%%%%%%%
\begin{ex} 
$\ds\lim_{x\to2}{2-\sqrt{x+2}\over 4-x^2}$
\begin{sol}
 $1/16$
\end{sol}
\end{ex}

%%%%%%%%%%
\begin{ex} 
 $\ds\lim_{x\to1}{\sqrt{x}-1\over \root 1/3\of{x}-1}$
\begin{sol}
 $3/2$
\end{sol}
\end{ex}

%%%%%%%%%%
\begin{ex} 
 $\ds\lim_{x\to0}{(1-x)^{1/4}-1\over x}$
\begin{sol}
 $-1/4$
\end{sol}
\end{ex}

%%%%%%%%%%
\begin{ex} 
 $\ds\lim_{t\to 0}{\left(t+{1\over t}\right)((4-t)^{3/2}-8)}$
\begin{sol}
 $-3$
\end{sol}
\end{ex}

%%%%%%%%%%
\begin{ex} 
 $\ds\lim_{t\to 0^+}\left({1\over t}+{1\over\sqrt{t}}\right)
(\sqrt{t+1}-1)$
\begin{sol}
 $1/2$
\end{sol}
\end{ex}

%%%%%%%%%%
\begin{ex} 
 $\ds\lim_{x\to 0}{x^2\over\sqrt{2x+1}-1}$
\begin{sol}
 $0$
\end{sol}
\end{ex}

%%%%%%%%%%
\begin{ex} 
 $\ds\lim_{u\to 1}{(u-1)^3\over (1/u)-u^2+3/u-3}$
\begin{sol}
 $-1$
\end{sol}
\end{ex}

%%%%%%%%%%
\begin{ex} 
 $\ds\lim_{x\to 0}{2+(1/x)\over 3-(2/x)}$
\begin{sol}
 $-1/2$
\end{sol}
\end{ex}

%%%%%%%%%%
\begin{ex} 
 $\ds\lim_{x\to 0^+}{1+5/\sqrt{x}\over 2+1/\sqrt{x}}$
\begin{sol}
 $5$
\end{sol}
\end{ex}

%%%%%%%%%%
\begin{ex} 
 $\ds\lim_{x\to\pi/2}{\cos x\over (\pi/2)-x}$
\begin{sol}
 $1$
\end{sol}
\end{ex}

%%%%%%%%%%
\begin{ex} 
 $\ds\lim_{x\to0}{e^x-1\over x}$
\begin{sol}
 $1$
\end{sol}
\end{ex}

%%%%%%%%%%
\begin{ex} 
 $\ds\lim_{x\to0}{x^2\over e^x-x-1}$
\begin{sol}
 $2$
\end{sol}
\end{ex}

%%%%%%%%%%
\begin{ex} 
 $\ds\lim_{x\to1}{\ln x\over x-1}$
\begin{sol}
 $1$
\end{sol}
\end{ex}

%%%%%%%%%%
\begin{ex} 
 $\ds\lim_{x\to0}{\ln(x^2+1)\over x}$
\begin{sol}
 $0$
\end{sol}
\end{ex}

%%%%%%%%%%
\begin{ex} 
 $\ds\lim_{x\to1}{x\ln x\over x^2-1}$
\begin{sol}
 $1/2$
\end{sol}
\end{ex}

%%%%%%%%%%
\begin{ex} 
 $\ds\lim_{x\to0}{\sin(2x)\over\ln(x+1)}$
\begin{sol}
 $2$
\end{sol}
\end{ex}

%%%%%%%%%%
\begin{ex} 
 $\ds\lim_{x\to1}{x^{1/4}-1\over x}$
\begin{sol}
 $0$
\end{sol}
\end{ex}

%%%%%%%%%%
\begin{ex} 
 $\ds\lim_{x\to1}{\sqrt{x}-1\over x-1}$
\begin{sol}
 $1/2$
\end{sol}
\end{ex}

%%%%%%%%%%
\begin{ex} 
 $\ds\lim_{x\to0}{3x^2+x+2\over x-4}$
\begin{sol}
 $-1/2$
\end{sol}
\end{ex}

%%%%%%%%%%
\begin{ex} 
 $\ds\lim_{x\to0}{\sqrt{x+1}-1\over \sqrt{x+4}-2}$
\begin{sol}
 $2$
\end{sol}
\end{ex}

%%%%%%%%%%
\begin{ex} 
 $\ds\lim_{x\to0}{\sqrt{x+1}-1\over \sqrt{x+2}-2}$
\begin{sol}
 $0$
\end{sol}
\end{ex}

%%%%%%%%%%
\begin{ex} 
 $\ds\lim_{x\to0^+}{\sqrt{x+1}+1\over\sqrt{x+1}-1}$
\begin{sol}
 $\infty$
\end{sol}
\end{ex}

%%%%%%%%%%
\begin{ex} 
 $\ds\lim_{x\to0}{\sqrt{x^2+1}-1\over\sqrt{x+1}-1}$
\begin{sol}
 $0$
\end{sol}
\end{ex}

%%%%%%%%%%
\begin{ex} 
 $\ds\lim_{x\to1}{(x+5)\left({1\over 2x}+{1\over x+2}\right)}$
\begin{sol}
 $5$
\end{sol}
\end{ex}

%%%%%%%%%%
\begin{ex} 
 $\ds\lim_{x\to2}{x^3-6x-2\over x^3+4}$
\begin{sol}
 $-1/2$
\end{sol}
\end{ex}

\begin{ex}
Discuss what happens if we try to use L'H\^{o}pital's rule to find the limit $\lim\limits_{x\rightarrow \infty}\dfrac{x+\sin x}{x+1}$.
\end{ex}

\end{enumialphparenastyle}
