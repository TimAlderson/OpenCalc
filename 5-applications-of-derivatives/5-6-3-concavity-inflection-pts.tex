\subsection{Concavity and Inflection Points}\label{sec:Concavity}
We know that the sign of the derivative tells us whether a function is
increasing or decreasing; for example, when $f'(x)>0$,
$f(x)$ is increasing. The sign of the second derivative
$f''(x)$ tells us whether $f'$ is increasing or decreasing; we have
seen that if $f'$ is zero and increasing at a point then there is a
local minimum at the point. If $f'$ is zero and decreasing at a
point then there is a local maximum at the point. Thus, we extracted
information about $f$ from information about $f''$. 

We can get information from the sign of $f''$ even when $f'$ is not
zero. Suppose that $f''(a)>0$. This means that near $x=a$, $f'$ is
increasing. If $f'(a)>0$, this means that $f$ slopes up and is getting
steeper; if $f'(a)<0$, this means that $f$ slopes down and is getting
{\it less} steep. The two situations are shown in
figure~\ref{fig:concave up}. A curve that is shaped like this is
called \dfont{concave up}.


\figure[H]
\centerline{\vbox{\beginpicture
\normalgraphs
%\ninepoint
\setcoordinatesystem units <2truecm,2truecm>
\setplotarea x from 0 to 1, y from 0 to 1
\axis left shiftedto x=0 /
\axis bottom shiftedto y=0 ticks withvalues {$a$} / at 0.5 / /
\setquadratic
\plot 0.1 0.2 0.5 0.4 0.9 0.9 /
\setcoordinatesystem units <2truecm,2truecm> point at -2 0
\setplotarea x from 0 to 1, y from 0 to 1
\axis left shiftedto x=0 /
\axis bottom shiftedto y=0 ticks withvalues {$a$} / at 0.5 / /
\setquadratic
\plot 0.1 0.9 0.5 0.3 0.9 0.1 /
\endpicture}}
\caption{$f''(a)>0$: $f'(a)$ positive and increasing, $f'(a)$ negative and
  increasing. \label{fig:concave up}}
\endfigure

Now suppose that $f''(a)<0$. This means that near $x=a$, $f'$ is
decreasing. If $f'(a)>0$, this means that $f$ slopes up and is getting
less steep; if $f'(a)<0$, this means that $f$ slopes down and is getting
steeper. The two situations are shown in
figure~\ref{fig:concave down}. A curve that is shaped like this is
called \dfont{concave down}.


\figure[H]
\centerline{\vbox{\beginpicture
\normalgraphs
%\ninepoint
\setcoordinatesystem units <2truecm,2truecm>
\setplotarea x from 0 to 1, y from 0 to 1
\axis left shiftedto x=0 /
\axis bottom shiftedto y=0 ticks withvalues {$a$} / at 0.5 / /
\setquadratic
\plot 0.1 0.2 0.5 0.7 0.9 0.9 /
\setcoordinatesystem units <2truecm,2truecm> point at -2 0
\setplotarea x from 0 to 1, y from 0 to 1
\axis left shiftedto x=0 /
\axis bottom shiftedto y=0 ticks withvalues {$a$} / at 0.5 / /
\setquadratic
\plot 0.1 0.9 0.5 0.6 0.9 0.1 /
\endpicture}}
\caption{$f''(a)<0$: $f'(a)$ positive and decreasing, $f'(a)$ negative and
  decreasing. \label{fig:concave down}}
\endfigure

If we are trying to understand the shape of the graph of a function,
knowing where it is concave up and concave down helps us to get a more
accurate picture. Of particular interest are points at which the
concavity changes from up to down or down to up; such points are
called \dfont{inflection points}. If the
concavity changes from up to down at $x=a$, $f''$ changes from
positive to the left of $a$ to negative to the right of $a$, and
usually $f''(a)=0$. We can identify such points by first finding where
$f''(x)$ is zero and then checking to see whether $f''(x)$ does in
fact go from positive to negative or negative to positive at these
points. Note that it is possible that $f''(a)=0$ but the concavity is
the same on both sides; $\ds f(x)=x^4$ at $x=0$ is an example.

\begin{example}{Concavity}{concavity}
Describe the concavity of $\ds f(x)=x^3-x$.
\end{example}
\begin{solution}
The derivatives are $\ds f'(x)=3x^2-1$ and $f''(x)=6x$.
Since $f''(0)=0$, there is potentially an inflection point at
zero. Since $f''(x)>0$ when $x>0$ and $f''(x)<0$ when $x<0$ the
concavity does change from concave down to concave up at zero, and the curve is
concave down for all $x<0$ and concave up for all $x>0$.
\end{solution}

Note that we need to compute and analyze the second derivative to
understand concavity, so we may as well try to use the second
derivative test for maxima and minima. If for some reason this fails
we can then try one of the other tests.


%%%%%%%%%%%%%%%%%%%%%%%%%%%%%%%%%%%%%%%%%%%%
\Opensolutionfile{solutions}[ex]
\section*{Exercises for \ref{sec:Concavity}}

\begin{enumialphparenastyle}

Describe the concavity of the functions below.

%%%%%%%%%%
\begin{ex}
 $\ds y=x^2-x$ 
\begin{sol}
 concave up everywhere
\end{sol}
\end{ex}

%%%%%%%%%%
\begin{ex}
 $\ds y=2+3x-x^3$ 
\begin{sol}
 concave up when $x<0$, concave down when $x>0$
\end{sol}
\end{ex}

%%%%%%%%%%
\begin{ex}
 $\ds y=x^3-9x^2+24x$
\begin{sol}
 concave down when $x<3$, concave up when $x>3$
\end{sol}
\end{ex}

%%%%%%%%%%
\begin{ex}
 $\ds y=x^4-2x^2+3$ 
\begin{sol}
 concave up when $\ds x<-1/\sqrt3$ or $\ds x>1/\sqrt3$,
concave down when $\ds -1/\sqrt3<x<1/\sqrt3$
\end{sol}
\end{ex}

%%%%%%%%%%
\begin{ex}
 $\ds y=3x^4-4x^3$
\begin{sol}
 concave up when $x<0$ or $x>2/3$,
concave down when $0<x<2/3$
\end{sol}
\end{ex}

%%%%%%%%%%
\begin{ex}
 $\ds y=(x^2-1)/x$
\begin{sol}
 concave up when $x<0$, concave down when $x>0$
\end{sol}
\end{ex}

%%%%%%%%%%
\begin{ex}
 $\ds y=3x^2-(1/x^2)$ 
\begin{sol}
 concave up when $x<-1$ or $x>1$, concave down when
$-1<x<0$ or $0<x<1$
\end{sol}
\end{ex}

%%%%%%%%%%
\begin{ex}
 $y=\sin x + \cos x$ 
\begin{sol}
 concave down on $((8n-1)\pi/4,(8n+3)\pi/4)$,
concave up on $((8n+3)\pi/4,(8n+7)\pi/4)$, for integer $n$
\end{sol}
\end{ex}

%%%%%%%%%%
\begin{ex}
 $\ds y = 4x+\sqrt{1-x}$
\begin{sol}
 concave down everywhere
\end{sol}
\end{ex}

%%%%%%%%%%
\begin{ex}
 $\ds y = (x+1)/\sqrt{5x^2 + 35}$
\begin{sol}
 concave up on $\ds (-\infty,(21-\sqrt{497})/4)$ and 
$\ds (21+\sqrt{497})/4,\infty)$
\end{sol}
\end{ex}

%%%%%%%%%%
\begin{ex}
 $\ds y= x^5 - x$
\begin{sol}
 concave up on $(0,\infty)$
\end{sol}
\end{ex}

%%%%%%%%%%
\begin{ex}
 $\ds y = 6x + \sin 3x$
\begin{sol}
 concave down on $(2n\pi/3,(2n+1)\pi/3)$
\end{sol}
\end{ex}

%%%%%%%%%%
\begin{ex}
 $\ds y = x+ 1/x$
\begin{sol}
 concave up on $(0,\infty)$
\end{sol}
\end{ex}

%%%%%%%%%%
\begin{ex}
 $\ds y = x^2+ 1/x$
\begin{sol}
 concave up on $(-\infty,-1)$ and $(0,\infty)$
\end{sol}
\end{ex}

%%%%%%%%%%
\begin{ex}
 $\ds y = (x+5)^{1/4}$
\begin{sol}
 concave down everywhere
\end{sol}
\end{ex}

%%%%%%%%%%
\begin{ex}
 $\ds y = \tan^2 x$
\begin{sol}
 concave up everywhere
\end{sol}
\end{ex}

%%%%%%%%%%
\begin{ex}
 $\ds y =\cos^2 x - \sin^2 x$
\begin{sol}
 concave up on $(\pi/4+n\pi,3\pi/4+n\pi)$
\end{sol}
\end{ex}

%%%%%%%%%%
\begin{ex}
 $\ds y = \sin^3 x$
\begin{sol}
 inflection points at $n\pi$, $\ds \pm\arcsin(\sqrt{2/3})+n\pi$
\end{sol}
\end{ex}

%%%%%%%%%%
\begin{ex}
 Identify the intervals on which the graph of the function
$\ds f(x) = x^4-4x^3 +10$ is of one of these four
shapes: concave up and increasing; concave up and decreasing; concave
down and increasing; concave down and decreasing.
\begin{sol}
 up/incr: $(3,\infty)$, up/decr: $(-\infty,0)$, $(2,3)$,
down/decr: $(0,2)$
\end{sol}
\end{ex}

%%%%%%%%%%
\begin{ex}
 Describe the concavity of $\ds y =  x^3 + bx^2 + cx + d$.
You will need to consider different cases, depending on the values of
the coefficients.
\end{ex}

%%%%%%%%%%
\begin{ex}
 Let $n$ be an integer greater than or equal to
two, and suppose $f$ is a polynomial of degree $n$. How many inflection points
can $f$ have?  Hint: Use the second derivative test and the
fundamental theorem of algebra.
\end{ex}

\end{enumialphparenastyle}