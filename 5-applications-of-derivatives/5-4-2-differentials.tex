\subsection{Differentials}\label{sec:differentials}
Very much related to linear approximations are the {\em differentials} $dx$ and $dy$, used not to approximate values of $f$, but instead the change (or rise) in the values of $f$.

\begin{definition}{Differentials dx and dy}{dx}
Let $y=f(x)$ be a differentiable function. We define a new
  independent variable $dx$, and a new dependent variable
  $dy=f'(x)\,dx$. Notice that $dy$ is a function both of $x$ (since
  $f'(x)$ is a function of $x$) and of $dx$.  We call both $dx$ and
  $dy$ \deffont{differentials}.  
\end{definition}

Now fix a point $a$ and let $\Delta x =x-a$ and $\Delta y= f(x)-f(a)$.
If $x$ is near $a$ then $\Delta x$ is clearly small. If we set $dx=\Delta x$ then we obtain
\[ dy = f'(a)\,dx \approx \frac{\Delta y}{\Delta x}\Delta x = \Delta y.\]
Thus, $dy$ can be used to approximate $\Delta y$, the actual change in
the function $f$ between $a$ and $x$. This is exactly the
approximation given by the tangent line:
\[ dy = f'(a)(x-a) = f'(a)(x-a)+f(a)-f(a)=L(x)-f(a).\]
While $L(x)$ approximates $f(x)$, $dy$ approximates how $f(x)$ has
changed from $f(a)$.
Figure~\ref{fig:differentials} illustrates the relationships.

\figure[!ht]
\centerline{\vbox{\beginpicture
\normalgraphs
%\sevenpoint
\setcoordinatesystem units <2truecm,2truecm>
\setplotarea x from 0 to 4.5, y from 0 to 2.5
\axis left /
\axis bottom ticks withvalues {$a$} {$x$} / at 1 3 / /
\plot 0.500 0.707 0.588 0.766 0.675 0.822 0.762 0.873 0.850 0.922 
0.938 0.968 1.025 1.012 1.112 1.055 1.200 1.095 1.288 1.135 
1.375 1.173 1.462 1.209 1.550 1.245 1.638 1.280 1.725 1.313 
1.812 1.346 1.900 1.378 1.988 1.410 2.075 1.440 2.162 1.471 
2.250 1.500 2.338 1.529 2.425 1.557 2.512 1.585 2.600 1.612 
2.688 1.639 2.775 1.666 2.862 1.692 2.950 1.718 3.038 1.743 
3.125 1.768 3.212 1.792 3.300 1.817 3.388 1.841 3.475 1.864 
3.562 1.887 3.650 1.910 3.738 1.933 3.825 1.956 3.912 1.978 
4.000 2.000 /
\setlinear
\plot 1 1 3 2 3 1 1 1 /
\betweenarrows {$dx=\Delta x$} from 1 0.8 to 3 0.8
\betweenarrows {$\Delta y$} from 3.2 1 to 3.2 1.73
\betweenarrows {$dy$} from 3.6 1 to 3.6 2
\setdashes <2pt>
\putrule from 3 2 to 3.6 2
\putrule from 3 1 to 3.6 1
\putrule from 3 1.73 to 3.2 1.73
\endpicture}}
\caption{Differentials.\label{fig:differentials}}
\endfigure

Here is a concrete example.

\begin{example}{Rise of Natural Logarithm}{rise of natural logarithm}
Approximate the rise of $f(x)=\ln x$ from $x=1$ to $x=1.1$, using linear approximation.
\end{example}

\begin{solution}
Note that $\ln (1.1)$ is not readily calculated (without a calculator) hence why we wish to use linear approximation to approximate $f(1.1)-f(1)$.

We fix $a=1$ and as above we have $\Delta x=x-1$ and $\Delta y=f(x)-f(1)=\ln x$, and obtain
\[ dy=f'(1)dx\approx \frac{\Delta y}{\Delta x}\Delta x=\Delta y. \]
But $f'(x)=1/x$ and thus $f'(1)=1/1=1$, we obtain in this case
\[ dy=dx\approx\Delta y. \]
Finally for $x=1.1$, we can easily approximate the rise of $f$ as
\[ f(1.1)-f(1)=\Delta y\approx dy=1.1-1=0.1. \]
The correct value of $\ln (1.1)=\ln 1$ is 0.0953\ldots and thus we were relatively close.
\end{solution}


%%%%%%%%%%%%%%%%%%%%%%%%%%%%%%%%%%%%%%%%%%%%%%%
\Opensolutionfile{solutions}[ex]
\section*{Exercises for \ref{sec:differentials}}

\begin{enumialphparenastyle}
	
%%%%%%%%%%
\begin{ex} 
Let $\ds f(x) = x^4$. If $a=1$ and $dx= \Delta x =1/2$, 
what are $\Delta y$ and $dy$?
\begin{sol}
	$\Delta y=65/16$, $dy=2$
\end{sol}
\end{ex}

%%%%%%%%%%
\begin{ex} 
Let $\ds f(x) = \sqrt{x}$. If $a=1$ and $dx= \Delta x
=1/10$, what are $\Delta y$ and $dy$?
\begin{sol}
	$\ds \Delta y=\sqrt{11/10}-1$, $dy=0.05$
\end{sol}
\end{ex}

%%%%%%%%%%
\begin{ex} 
Let $f(x) = \sin (2x)$. If $a=\pi$ and $dx= \Delta x
=\pi/100$, what are $\Delta y$ and $dy$?
\begin{sol}
	$\ds \Delta y=\sin(\pi/50)$, $dy=\pi/50$
\end{sol}
\end{ex}

%%%%%%%%%%
\begin{ex} 
Use differentials to estimate the amount of paint needed to
 apply a coat of paint 0.02 cm thick to a sphere with diameter $40$
 meters. (Recall that the volume of a sphere of radius $r$ is $V
 =(4/3)\pi r^3$. Notice that you are given that $dr=0.02$.)
\begin{sol}
	$dV=8\pi/25$
\end{sol}
\end{ex}

\end{enumialphparenastyle}