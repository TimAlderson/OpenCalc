\begin{Answer}{1.1.1}
\end{Answer}
\begin{Answer}{1.1.2}
\begin{enumerate}
	\item	$\ds\frac{x}{y}$
	\item	$\sqrt{x}y$
	\item	$\ds\frac{2}{\sqrt[3]{x}}$
\end{enumerate}
\end{Answer}
\begin{Answer}{1.1.3}
	$a=2$, $b=-\frac{5}{3}$, $c=\frac{3}{2}$.
\end{Answer}
\begin{Answer}{1.1.4}
$x=-4$ and $x=6$.
\end{Answer}
\begin{Answer}{1.1.6}
\begin{enumerate}
	\item	$(5/3,\infty)$
	\item	$[1/7,2/7]$
	\item	$(-\infty,-3)\cup(-2,1]$
	\item	$(-\infty,\infty)$
	\item	No solution
	\item	$(-\infty,1)\cup(1,\infty)$
	\item	$(-2,0)\cup(2,\infty)$
	\item	$[4,\infty)\cup\{0\}$
	\item	$(0,\frac{1}{2})$
	\item	$(-2,-1]\cup(1,4]$
\end{enumerate}
\end{Answer}
\begin{Answer}{1.1.7}
$x=-\frac{1}{2}$ and $x=-\frac{1}{6}$.
\end{Answer}
\begin{Answer}{1.1.8}
\begin{multicols}{2}
\begin{enumerate}
	\item	$(-\infty,-2]\cup[2,\infty)$
	\item	$[2,4]$
	\item	$(-\infty,-9/2]\cup[-1/2,\infty)$
	\item	$(4,\infty)$
	\item	$(-\infty,\infty)$
	\item	$(-9,-6)\cup(4,7)$
\end{enumerate}
\end{multicols}
\end{Answer}
\begin{Answer}{1.2.1}
\begin{enumerate}
	\item	$(2/3)x+(1/3)$
	\item	$y=-2x$
	\item	$y=(-2/3)x+(1/3)$
	\item	$y=-x/3+17/3$
	\item	$y=-1/2x+5/2$
\end{enumerate}
\end{Answer}
\begin{Answer}{1.2.2}
\begin{enumerate}
	\item	$y=2x+2$, $2$, $-1$
	\item	$y=-x+6$, $6$, $6$
	\item	$y=x/2+1/2$, $1/2$, $-1$
	\item	$y=3/2$, $y$-intercept: $3/2$, no $x$-intercept
	\item	$y=(-2/3)x-2$, $-2$, $-3$
\end{enumerate}
\end{Answer}
\begin{Answer}{1.2.3}
Yes, the lines are parallel as they have the same slope of $-1/2$
\end{Answer}
\begin{Answer}{1.2.4}
$y=0$, $y=-2x+2$, $y=2x+2$
\end{Answer}
\begin{Answer}{1.2.5}
$y=(9/5)x+32$, $(-40,-40)$
\end{Answer}
\begin{Answer}{1.2.6}
$y=0.15x+10$
\end{Answer}
\begin{Answer}{1.2.7}
$0.03x+1.2$
\end{Answer}
\begin{Answer}{1.2.8}
(a) $P=-0.0001x+2$\\
(b) $x=-10000P+20000$
\end{Answer}
\begin{Answer}{1.2.9}
$(2/25)x-(16/5)$
\end{Answer}
\begin{Answer}{1.2.10}
	\begin{enumerate}
\item $d = 5$, $M = \left(-1, \frac{7}{2} \right)$
\item $d = 4 \sqrt{10}$, $M = \left(1, -4 \right)$
\item $d = \sqrt{26}$, $M = \left(1, \frac{3}{2} \right)$
\item $d= \frac{\sqrt{37}}{2}$, $M = \left(\frac{5}{6}, \frac{7}{4} \right)$
\item  $d = \sqrt{74}$, $M = \left(\frac{13}{10}, -\frac{13}{10} \right)$ \vphantom{$\left( \frac{\sqrt{3}}{2} \right)$}
\item $d= 3\sqrt{5}$, $M = \left(-\frac{\sqrt{2}}{2}, -\frac{\sqrt{3}}{2} \right)$
\item  $d = \sqrt{83}$, $M = \left(4 \sqrt{5}, \frac{5 \sqrt{3}}{2} \right)$
\item $d = \sqrt{x^2 + y^2}$, $M = \left( \frac{x}{2}, \frac{y}{2}\right)$ \vphantom{$\left( \frac{\sqrt{3}}{2} \right)$}
\end{enumerate}
\end{Answer}
\begin{Answer}{1.2.11}
$(3 + \sqrt{7}, -1)$, $(3-\sqrt{7}, -1)$
\end{Answer}
\begin{Answer}{1.2.12}
	$(0,3)$
\end{Answer}
\begin{Answer}{1.2.13}
$(-1+\sqrt{3},0)$, $(-1-\sqrt{3},0)$ \vphantom{$\left( \frac{\sqrt{3}}{2} \right)$}
\end{Answer}
\begin{Answer}{1.2.14}
	$\left(\frac{\sqrt{2}}{2},-\frac{\sqrt{2}}{2} \right)$, $\left(-\frac{\sqrt{2}}{2},\frac{\sqrt{2}}{2}\right)$
\end{Answer}
\begin{Answer}{1.2.15}
 	$(-3,-4), \, 5 \,\, km, \,\, (4, -4)$
 
\end{Answer}
\begin{Answer}{1.2.16}


\end{Answer}
\begin{Answer}{1.2.17}

\end{Answer}
\begin{Answer}{1.2.18}
	(a) The distance from $A$ to $B$ is $|AB| = \sqrt{13}$, the distance from $A$ to $C$ is $|AC| = \sqrt{52}$, and the distance from $B$ to $C$ is $|BC| = \sqrt{65}$.  Since $\left(\sqrt{13}\right)^2 + \left( \sqrt{52} \right)^2 = \left( \sqrt{65} \right)^2$, we are guaranteed by the \href{http://en.wikipedia.org/wiki/Pythagorean_theorem#Converse}{\underline{converse of the Pythagorean Theorem}} that the triangle is a right triangle.\\
	(b) Show that $|AC|^{2} + |BC|^{2} = |AB|^{2}$.
\end{Answer}
\begin{Answer}{1.2.19}

\end{Answer}
\begin{Answer}{1.2.20}
\begin{enumerate}
	\item	$x^2+y^2=9$
	\item	$(x-5)^2+(y-6)^2=9$
	\item	$(x+5)^2+(y+6)^2=9$
\end{enumerate}
\end{Answer}
\begin{Answer}{1.2.22}
\begin{enumerate}
	\item	circle
	\item	ellipse
	\item	horizontal parabola
\end{enumerate}
\end{Answer}
\begin{Answer}{1.2.23}
$(x+2/7)^2+(y-41/7)^2=1300/49$
\end{Answer}
\begin{Answer}{1.3.1}
$\begin{array}{cccc}
	(a) \hspace{1mm} \pi/2 & (b) \hspace{1mm} 3\pi/4 & (c) \hspace{1mm} 7\pi/6 & (d) \hspace{1mm} 5\pi/3
\end{array}$
\end{Answer}
\begin{Answer}{1.3.2}
	$\begin{array}{cccc}
	(a) \hspace{1mm} 60^{\circ} & (b) \hspace{1mm} 105^{\circ} & (c) \hspace{1mm} 270^{\circ} & (d) \hspace{1mm} 285^{\circ}
	\end{array}$
\end{Answer}
\begin{Answer}{1.3.5}
	$\ds{5\pi/6 \,\, cm}$
\end{Answer}
\begin{Answer}{1.3.4}
	$8 \,\,cm$
\end{Answer}
\begin{Answer}{1.3.5}
  $\begin{array}{llllll}
  (a) $0$ & (b) $-2\sqrt{3}/3$ & (c) $1/2$ & (d) $-2\sqrt{3}/3$ & (e) $-1$ & (f) $1$
\end{array}$
\end{Answer}
\begin{Answer}{1.3.6}
	$\ds{\sin \theta = \frac{\sqrt{21}}{5} ; \hspace{4mm} \tan \theta = \frac{\sqrt{21}}{2} ; \hspace{4mm} \csc \theta = \frac{5}{\sqrt{21}} ; \hspace{4mm} \sec \theta = \frac{5}{2} ; \hspace{4mm} \cot \theta = \frac{2}{\sqrt{21}}  }$
\end{Answer}
\begin{Answer}{1.3.7}
(a) $2n\pi-\pi/2$, any integer $n$; (b) $n\pi\pm\pi/6$, any integer $n$
\end{Answer}
\begin{Answer}{1.3.8}
$-\frac{5}{4}$
\end{Answer}
\begin{Answer}{1.3.9}
$\sin\theta=-x/\sqrt{x^{2}+1}$, $\cos\theta=-1/\sqrt{x^{2}+1}$.
\end{Answer}
\begin{Answer}{1.3.10}
$-\frac{2\pi}{7}$ is the unique answer.
\end{Answer}
\begin{Answer}{1.3.11}
(a) $(\sqrt{2}+\sqrt{6})/4$; \hspace{1cm} (b) $-(1+\sqrt3)/(1-\sqrt3)=2+\sqrt3$
\end{Answer}
\begin{Answer}{1.3.14}
 (a)  (b) (c) (d) (e) $t=\pi/2$
\end{Answer}
\begin{Answer}{1.4.1}
\begin{enumerate}
	\item	$\dfrac{\sqrt{2}}{2}$
	\item	$3(\sqrt{x+h+1}+\sqrt{x+1})$
\end{enumerate}
\end{Answer}
\begin{Answer}{1.4.2}
\begin{enumerate}
	\item	$-13/5$
	\item	$-1/2,3$
	\item	$(1\pm\sqrt{13})/2$
	\item	No real solutions
	\item	$\sqrt{3}$
\end{enumerate}
\end{Answer}
\begin{Answer}{1.4.3}
Counter-examples may vary.
\begin{enumerate}
	\item	$x=3$
	\item	$x=h=1$
	\item	$x=y=1$
\end{enumerate}
\end{Answer}
\begin{Answer}{1.4.4}
	$x+3y-13=0$, or equivalents such as $y=-\frac{1}{3}x+\frac{13}{3}$
\end{Answer}
\begin{Answer}{1.4.5}
	$(-\infty,0]\cup(\frac{1}{3},3]$
\end{Answer}
\begin{Answer}{1.4.6}
	It is impossible for both $x-2$ and $1-x$ to be non-negative
	for the same real number $x$.
\end{Answer}
\begin{Answer}{1.4.7}
	$6x+3h$
\end{Answer}
\begin{Answer}{1.4.8}
	$-1/\left[(2x+2h-1)(2x-1)\right]$
\end{Answer}
\begin{Answer}{1.4.9}
	$-3$
\end{Answer}
\begin{Answer}{1.4.10}
	$\pi /6,$ $5\pi /6$
\end{Answer}
\begin{Answer}{1.4.11}
	$2\pi/5$
\end{Answer}
\begin{Answer}{1.4.12}
	It is equal to 2 for all $x$ larger than 4.
\end{Answer}
\begin{Answer}{1.4.13}
	$(x+2)^2+(y-3)^2=25$.
\end{Answer}
\begin{Answer}{1.4.14}
	Centre is $(-3,2)$ and radius is 2.
\end{Answer}
\begin{Answer}{1.4.15}
	$y$ could be any real number greater than or equal to 6.
\end{Answer}
\begin{Answer}{1.4.16}
	$x^6 y^4/(36z^8)$
\end{Answer}
\begin{Answer}{1.4.17}
	$x=(y-2)/(3+4y)$
\end{Answer}
\begin{Answer}{1.4.18}
	$Q(x)=x+1$, $R=-7$
\end{Answer}
\begin{Answer}{2.1.1}
	Using the Vertical Line Test,
	(A) function, \hspace{3mm} (B) not a function, \hspace{3mm} (C) not a function, \hspace{3mm} (D) function, \hspace{3mm} (E) function, \hspace{3mm} (F) not a function.
\end{Answer}
\begin{Answer}{2.1.2}
\begin{multicols}{2}
	\begin{enumerate}
		\item $y = x^{3} - x$  function
		\item $y = \sqrt{x - 2}$ function
		\item $x^{3}y = -4$  function
		\item $x^{2} - y^{2} = 1$ not a function
		\item $y = \dfrac{x}{x^{2} - 9}$ function
		\item $x = -6$ not a function
		\item  $x = y^2 + 4$ not a function
		\item $y = x^2 + 4$ function
		\item $x^2 + y^2 = 4$ not a function
		\item $y = \sqrt{4-x^2}$ function
		\item $x^2 - y^2 = 4$not a function
		\item $x^3 + y^3 = 4$ function
		\item $2x + 3y = 4$ function
		\item $2xy = 4$ function
		\item $x^2 = y^2$ not a function
	\end{enumerate}
\end{multicols}
\end{Answer}
\begin{Answer}{2.1.3}
	$f(-4)$ is undefined, \hspace{3mm} $f(4)=-12$, \hspace{3mm} $f(8)=-22$
\end{Answer}
\begin{Answer}{2.1.4}
\begin{tabular}{ll}
(a) $f(2)=6$ & (b) $f(-2)=16$ \\
(c) $f(a)=2a^{2}-3a+2$  & (d)  $f(-a)=2a^{2}+3a+2$ \\
(e) $f(a+1)=2a^{2}+a+1$ & (f) $f(2a)=8a^{2}-6x+2$ \\
(g) $2f(a)=4x^{2}-6x+4$ & (h) $\displaystyle{\left[ f(a) \right]^{2}=4x^{4}-12x^{3}+17x^{2}-12x+4}$ \\
(i) $f(a+h)=2a^{2}+4ah+2h^{2}-3a-3h+2$ & \\
\end{tabular}
\end{Answer}
\begin{Answer}{2.1.5}
(a) \hspace{2mm} For $f(x) = 2x-5$ \\
$f(2) = -1$\\
$f(-2) = -9$\\
$f(2a) = 4a-5$	\\
$2 f(a) = 4a-10$\\
$f(a+2) = 2a-1$\\
$f(a) + f(2) = 2a-6$\\
$f \left( \frac{2}{a} \right) = \frac{4}{a} - 5$ \\
$\hphantom{f \left( \frac{2}{a} \right)} = \frac{4-5a}{a}$\\
$\frac{f(a)}{2} =\frac{2a-5}{2}$\\
$f(a + h) = 2a + 2h - 5$\\

\vspace{3mm}

(b) \space{2mm} For $f(x) = 5-2x$\\
$f(2) = 1$\\
$f(-2) = 9$\\
$f(2a) = 5-4a$	\\
$2 f(a) = 10-4a$\\
$f(a+2) = 1-2a$\\
$f(a) + f(2) = 6-2a$\\
$f \left( \frac{2}{a} \right) = 5 - \frac{4}{a}$ \\
$\hphantom{f \left( \frac{2}{a} \right)} = \frac{5a-4}{a}$\\
$\frac{f(a)}{2} = \frac{5-2a}{2}$\\
$f(a + h) = 5-2a-2h$\\

\vspace{3mm}

(c) \hspace{2mm} For $f(x) = 2x^2-1$\\
$f(2) = 7$\\
$f(-2) = 7$\\
$2 f(a) = 4a^2-2$\\
$f(a+2) = 2a^2+8a+7$\\
$f(a) + f(2) = 2a^2+6$\\
$f \left( \frac{2}{a} \right) = \frac{8}{a^2} - 1$ \\
$\hphantom{f \left( \frac{2}{a} \right)} = \frac{8-a^2}{a^2}$\\
$\frac{f(a)}{2} =  \frac{2a^2-1}{2}$\\
$f(a + h) = 2a^2+4ah+2h^2-1$\\

\vspace{2mm}

(d) \hspace{2mm} For $f(x) = 3x^2+3x-2$\\
$f(2) = 16$\\
$f(-2) = 4$\\
$f(2a) = 12a^2+6a-2$\\
$2 f(a) = 6a^2+6a-4$\\
$f(a+2) = 3a^2+15a+16$\\
\small $f(a) + f(2) = 3a^2+3a+14$ \normalsize\\
$f \left( \frac{2}{a} \right) = \frac{12}{a^2} + \frac{6}{a} - 2$ \\
$\hphantom{f \left( \frac{2}{a} \right)} = \frac{12+6a-2a^2}{a^2}$\\
$\frac{f(a)}{2} =  \frac{3a^2+3a-2}{2}$\\
$f(a + h) = 3a^2 + 6ah + 3h^2+3a+3h-2$\\

\vspace{3mm}

(e) \hspace{2mm} For $f(x) = \sqrt{2x+1}$\\
$f(2) = \sqrt{5}$\\
$f(-2)$ is not real \\
$f(2a) = \sqrt{4a+1}$\\
$2 f(a) = 2\sqrt{2a+1}$\\
$f(a+2) = \sqrt{2a+5}$\\
\small $f(a) + f(2) =\sqrt{2a+1} + \sqrt{5}$ \normalsize\\
$f \left( \frac{2}{a} \right) = \sqrt{\frac{4}{a} + 1}$ \\
$\hphantom{f \left( \frac{2}{a} \right)} = \sqrt{\frac{a+4}{a}}$\\
$\frac{f(a)}{2} = \frac{\sqrt{2a+1}}{2}$\\
$f(a + h) = \sqrt{2a+2h+1}$\\

\vspace{3mm}

(f) \hspace{2mm}  For $f(x) = -7$\\
$f(2) = -7$\\
$f(-2) = -7$\\
$f(2a) = -7$\\
$2 f(a) = -14$\\
$f(a+2) = -7$\\
$f(a) + f(2) = -14$\\
$f \left( \frac{2}{a} \right) = -7$ \\
$\frac{f(a)}{2} = \frac{-7}{2}$\\
$f(a + h) = -7$\\

\vspace{3mm}

For $f(x) = \frac{x}{2}$\\
$f(2) = 1$\\
$f(-2) = -1$\\
$f(2a) = a$\\
$2 f(a) = a$\\
$f(a+2) = \frac{a+2}{2}$\\
$f(a) + f(2) = \frac{a}{2}+ 1$ \\
$\hphantom{f(a) + f(2)} = \frac{a+2}{2}$\\
$f \left( \frac{2}{a} \right) = \frac{1}{a}$\\
$\frac{f(a)}{2} =  \frac{a}{4}$\\
$f(a + h) = \frac{a+h}{2}$\\

\vspace{3mm}

(g) For $f(x) = \frac{2}{x}$\\
$f(2) = 1$\\
$f(-2) = -1$\\
$f(2a) = \frac{1}{a}$\\
$2 f(a) = \frac{4}{a}$\\
$f(a+2) = \frac{2}{a+2}$\\
$f(a) + f(2) = \frac{2}{a}+1$ \\
$\hphantom{f(a)+f(2)}=\frac{a+2}{2}$\\
$f \left( \frac{2}{a} \right) = a$\\
$\frac{f(a)}{2} =  \frac{1}{a}$\\
$f(a + h) = \frac{2}{a+h}$\\

\end{Answer}
\begin{Answer}{2.1.6}
(a) \hspace{2mm} $S(10) = 17.5$, so it costs $\$ 17.50$ to ship 10 comic books. \\
(b) \hspace{2mm} There is free shipping on orders of $15$ or more comic books. \\

\end{Answer}
\begin{Answer}{2.1.7}

(a) \hspace{2mm} $C(750) = 25$, so it costs $\$ 25$ to talk 750 minutes per month with this plan. \\

(b) \hspace{2mm} Since $20 \, \text{hours} = 1200 \, \text{minutes}$, we substitute $m = 1200$ and get  $C(1200) = 45$.  It costs $\$ 45$ to talk 20 hours per month with this plan. \\

(c) \hspace{2mm}  It costs $\$25$ for up to $1000$ minutes and $10$ cents per minute for each minute over $1000$ minutes.


\end{Answer}
\begin{Answer}{2.1.8}
\begin{enumerate}
	\item	$\ds \{x\mid x\in \R\}$, i.e., all $x$
	\item	$\ds \{x\mid x\ge 3/2\}$
	\item	$\ds \{x\mid x\not=-1\}$
	\item	$\ds \{x\mid x\not=1 \hbox{ and } x\not=-1\}$
	\item	$\ds \{x\mid x<0\}$
	\item	$\ds \{t\mid t\in \R\}$, i.e., all $t$
	\item	$\ds \{x\mid x\not \pm 3\}$
	\item	$\ds \{x\mid -1\le x\le 1\}$
	\item	$\ds \{x\mid x\ge 1\}$
	\item	$\ds \{x\mid -1/3< x< 1/3\}$
	\item	$\ds \{s\mid s\ge0  \hbox{ and } s\not=1\}$
	\item	$\ds \{x\mid x\ge0  \hbox{ and } x\not=1\}$
	\item  $\ds \{x\mid x\not=0,\, 1\}$
	\item  $\ds \{x\mid x> 1/2\}$
\end{enumerate}
\end{Answer}
\begin{Answer}{2.1.9}
\begin{multicols}{2}
	\begin{enumerate}
		\item $(-\infty, \infty)$
		\item  $(-\infty, \infty)$
		\item $(-\infty, -1) \cup (-1, \infty)$
		\item  $(-\infty,-2) \cup (-2,1) \cup (1, \infty)$
		\item $(-\infty, \infty)$
		\item  $(-\infty, -\sqrt{3}) \cup (-\sqrt{3}, \sqrt{3}) \cup (\sqrt{3}, \infty)$
		\item  $(-\infty, -6) \cup (-6,6) \cup (6, \infty)$
		\item $(-\infty, 2) \cup (2, \infty)$
		\item  $(-\infty, 3]$
		\item $\left[-\frac{5}{2}, \infty \right)$
		\item  $[-3, \infty)$
		\item $(-\infty, 7]$
		\item    $\left[ \frac{1}{3}, \infty \right)$
		\item   $\left( \frac{1}{3}, \infty \right)$
		\item   $(-\infty, \infty)$
		\item   $\left[ \frac{1}{3}, 3 \right) \cup (3, \infty)$
		\item  $\left[ \frac{1}{3}, 6 \right) \cup (6, \infty)$
		\item   $(-\infty, \infty)$
		\item $(-\infty, 8) \cup (8, \infty)$
		\item $[0, 8) \cup (8, \infty)$
		\item $(8, \infty)$
		\item $[7, 9]$
		\item $(-\infty, 8) \cup (8, \infty)$
		\item $\left( -\infty, -\frac{1}{2} \right) \cup \left( -\frac{1}{2}, 0 \right) \cup \left(0, \frac{1}{2}$
		\item $[0, 5) \cup (5,\infty)$
		\item $[0, 25) \cup (25, \infty)$
\end{enumerate}
\end{multicols}

\end{Answer}
\begin{Answer}{2.1.10}
$A=x(500-2x)$, $\ds \{x\mid 0\le x\le 250\}$
\end{Answer}
\begin{Answer}{2.1.11}
$\ds V=r(50-\pi r^2)$, $\ds \{r\mid 0< r\le \sqrt{50/\pi}\}$
\end{Answer}
\begin{Answer}{2.1.12}
$\ds A=2\pi r^2+2000/r$, $\ds \{r\mid 0<r<\infty\}$
\end{Answer}
\begin{Answer}{2.2.1}
\end{Answer}
\begin{Answer}{2.2.3}
	\item For  $f(x) = 3x+1$ and $g(x) = 4-x$

	\begin{multicols}{3}
		\begin{itemize}

			\item  $(f+g)(2) = 9$
			\item  $(f-g)(-1) = -7$
			\item  $(g-f)(1) = -1$

		\end{itemize}
	\end{multicols}

	\begin{multicols}{3}
		\begin{itemize}

			\item  $(fg)\left(\frac{1}{2}\right) = \frac{35}{4}$
			\item  $\left(\frac{f}{g}\right)(0) = \frac{1}{4}$
			\item  $\left(\frac{g}{f}\right)\left(-2\right) = -\frac{6}{5}$

		\end{itemize}
	\end{multicols}

	\item For  $f(x) = x^2$ and $g(x) = -2x+1$

	\begin{multicols}{3}
		\begin{itemize}

			\item  $(f+g)(2) = 1$
			\item  $(f-g)(-1) = -2$
			\item  $(g-f)(1) = -2$

		\end{itemize}
	\end{multicols}

	\begin{multicols}{3}
		\begin{itemize}

			\item  $(fg)\left(\frac{1}{2}\right) = 0$
			\item  $\left(\frac{f}{g}\right)(0) = 0$
			\item  $\left(\frac{g}{f}\right)\left(-2\right) = \frac{5}{4}$

		\end{itemize}
	\end{multicols}

	\item For  $f(x) = x^2 - x$ and  $g(x) = 12-x^2$

	\begin{multicols}{3}
		\begin{itemize}

			\item  $(f+g)(2) = 10$
			\item  $(f-g)(-1) = -9$
			\item  $(g-f)(1) = 11$

		\end{itemize}
	\end{multicols}

	\begin{multicols}{3}
		\begin{itemize}

			\item  $(fg)\left(\frac{1}{2}\right) = -\frac{47}{16}$
			\item  $\left(\frac{f}{g}\right)(0) = 0$
			\item  $\left(\frac{g}{f}\right)\left(-2\right) = \frac{4}{3}$

		\end{itemize}
	\end{multicols}

	\item For $f(x) = 2x^3$ and  $g(x) = -x^2-2x-3$

	\begin{multicols}{3}
		\begin{itemize}

			\item  $(f+g)(2) = 5$
			\item  $(f-g)(-1) = 0$
			\item  $(g-f)(1) = -8$

		\end{itemize}
	\end{multicols}

	\begin{multicols}{3}
		\begin{itemize}

			\item  $(fg)\left(\frac{1}{2}\right) = -\frac{17}{16}$
			\item  $\left(\frac{f}{g}\right)(0) = 0$
			\item  $\left(\frac{g}{f}\right)\left(-2\right) = \frac{3}{16}$

		\end{itemize}
	\end{multicols}

	\item For $f(x) = \sqrt{x+3}$ and  $g(x) = 2x-1$

	\begin{multicols}{3}
		\begin{itemize}

			\item  $(f+g)(2) = 3+\sqrt{5}$
			\item  $(f-g)(-1) = 3+\sqrt{2}$
			\item  $(g-f)(1) = -1$

		\end{itemize}
	\end{multicols}

	\begin{multicols}{3}
		\begin{itemize}

			\item  $(fg)\left(\frac{1}{2}\right) = 0$
			\item  $\left(\frac{f}{g}\right)(0) = -\sqrt{3}$
			\item  $\left(\frac{g}{f}\right)\left(-2\right) = -5$

		\end{itemize}
	\end{multicols}

	\item For $f(x) = \sqrt{4-x}$ and $g(x) = \sqrt{x+2}$

	\begin{multicols}{3}
		\begin{itemize}

			\item  $(f+g)(2) = 2+\sqrt{2}$
			\item  $(f-g)(-1) = -1+\sqrt{5}$
			\item  $(g-f)(1) = 0$

		\end{itemize}
	\end{multicols}

	\begin{multicols}{3}
		\begin{itemize}

			\item  $(fg)\left(\frac{1}{2}\right) = \frac{\sqrt{35}}{2}$
			\item  $\left(\frac{f}{g}\right)(0) = \sqrt{2}$
			\item  $\left(\frac{g}{f}\right)\left(-2\right) = 0$

		\end{itemize}
	\end{multicols}

	\newpage

	\item For  $f(x) = 2x$ and  $g(x) = \frac{1}{2x+1}$

	\begin{multicols}{3}
		\begin{itemize}

			\item  $(f+g)(2) = \frac{21}{5}$
			\item  $(f-g)(-1) = -1$
			\item  $(g-f)(1) = -\frac{5}{3}$

		\end{itemize}
	\end{multicols}

	\begin{multicols}{3}
		\begin{itemize}

			\item  $(fg)\left(\frac{1}{2}\right) = \frac{1}{2}$
			\item  $\left(\frac{f}{g}\right)(0) = 0$
			\item  $\left(\frac{g}{f}\right)\left(-2\right) = \frac{1}{12}$

		\end{itemize}
	\end{multicols}

	\item For  $f(x) = x^2$ and $g(x) = \frac{3}{2x-3}$

	\begin{multicols}{3}
		\begin{itemize}

			\item  $(f+g)(2) = 7$
			\item  $(f-g)(-1) = \frac{8}{5}$
			\item  $(g-f)(1) = -4$

		\end{itemize}
	\end{multicols}

	\begin{multicols}{3}
		\begin{itemize}

			\item  $(fg)\left(\frac{1}{2}\right) = -\frac{3}{8}$
			\item  $\left(\frac{f}{g}\right)(0) = 0$
			\item  $\left(\frac{g}{f}\right)\left(-2\right) = -\frac{3}{28}$

		\end{itemize}
	\end{multicols}

	\item For  $f(x) = x^2$ and $g(x) = \frac{1}{x^2}$

	\begin{multicols}{3}
		\begin{itemize}

			\item  $(f+g)(2) =\frac{17}{4}$
			\item  $(f-g)(-1) = 0$
			\item  $(g-f)(1) = 0$

		\end{itemize}
	\end{multicols}

	\begin{multicols}{3}
		\begin{itemize}

			\item  $(fg)\left(\frac{1}{2}\right) =1$
			\item  $\left(\frac{f}{g}\right)(0)$ is undefined.
			\item  $\left(\frac{g}{f}\right)\left(-2\right) = \frac{1}{16}$

		\end{itemize}
	\end{multicols}

	\item For  $f(x) = x^2+1$ and $g(x) = \frac{1}{x^2+1}$

	\begin{multicols}{3}
		\begin{itemize}

			\item  $(f+g)(2) =\frac{26}{5}$
			\item  $(f-g)(-1) = \frac{3}{2}$
			\item  $(g-f)(1) = -\frac{3}{2}$

		\end{itemize}
	\end{multicols}

	\begin{multicols}{3}
		\begin{itemize}

			\item  $(fg)\left(\frac{1}{2}\right) =1$
			\item  $\left(\frac{f}{g}\right)(0) = 1$
			\item  $\left(\frac{g}{f}\right)\left(-2\right) = \frac{1}{25}$

		\end{itemize}
	\end{multicols}


\end{enumerate}

	
\end{Answer}
\begin{Answer}{2.2.4}

	\begin{enumerate}

		\item For $f(x) = 2x+1$ and $g(x) = x-2$

		\begin{multicols}{2}

			\begin{itemize}

				\item $(f+g)(x) = 3x-1$ \\
				Domain: $(-\infty, \infty)$

				\vfill

				\columnbreak

				\item $(f-g)(x) = x+3$ \\
				Domain:  $(-\infty, \infty)$


			\end{itemize}

		\end{multicols}

		\begin{multicols}{2}

			\begin{itemize}

				\item $(fg)(x) = 2x^2-3x-2$ \\
				Domain: $(-\infty, \infty)$

				\vfill

				\columnbreak

				\item $\left(\frac{f}{g}\right)(x) = \frac{2x+1}{x-2}$ \\
				Domain:  $(-\infty, 2) \cup (2, \infty)$


			\end{itemize}

		\end{multicols}

		\item For $f(x) = 1-4x$ and $g(x) = 2x-1$

		\begin{multicols}{2}

			\begin{itemize}

				\item $(f+g)(x) = -2x$ \\
				Domain: $(-\infty, \infty)$

				\vfill

				\columnbreak

				\item $(f-g)(x) = 2-6x$ \\
				Domain:  $(-\infty, \infty)$


			\end{itemize}

		\end{multicols}

		\begin{multicols}{2}

			\begin{itemize}

				\item $(fg)(x) = -8x^2+6x-1$ \\
				Domain: $(-\infty, \infty)$

				\vfill

				\columnbreak

				\item $\left(\frac{f}{g}\right)(x) = \frac{1-4x}{2x-1}$ \\
				Domain:  $\left(-\infty, \frac{1}{2} \right) \cup \left(\frac{1}{2}, \infty \right)$


			\end{itemize}

		\end{multicols}

		\newpage

		\item For $f(x) = x^2$ and $g(x) = 3x-1$

		\begin{multicols}{2}

			\begin{itemize}

				\item $(f+g)(x) = x^2+3x-1$ \\
				Domain: $(-\infty, \infty)$

				\vfill

				\columnbreak

				\item $(f-g)(x) = x^2-3x+1$ \\
				Domain:  $(-\infty, \infty)$


			\end{itemize}

		\end{multicols}

		\begin{multicols}{2}

			\begin{itemize}

				\item $(fg)(x) = 3x^3-x^2$ \\
				Domain: $(-\infty, \infty)$

				\vfill

				\columnbreak

				\item $\left(\frac{f}{g}\right)(x) = \frac{x^2}{3x-1}$ \\
				Domain:  $\left(-\infty, \frac{1}{3} \right) \cup \left(\frac{1}{3}, \infty \right)$


			\end{itemize}

		\end{multicols}

		\item For $f(x) = x^2-x$ and $g(x) = 7x$

		\begin{multicols}{2}

			\begin{itemize}

				\item $(f+g)(x) = x^2+6x$ \\
				Domain: $(-\infty, \infty)$

				\vfill

				\columnbreak

				\item $(f-g)(x) = x^2-8x$ \\
				Domain:  $(-\infty, \infty)$


			\end{itemize}

		\end{multicols}

		\begin{multicols}{2}

			\begin{itemize}

				\item $(fg)(x) = 7x^3-7x^2$ \\
				Domain: $(-\infty, \infty)$

				\vfill

				\columnbreak

				\item $\left(\frac{f}{g}\right)(x) = \frac{x-1}{7}$ \\
				Domain:  $\left(-\infty, 0 \right) \cup \left(0, \infty \right)$


			\end{itemize}

		\end{multicols}


		\item For $f(x) = x^2-4$ and $g(x) = 3x+6$

		\begin{multicols}{2}

			\begin{itemize}

				\item $(f+g)(x) = x^2+3x+2$ \\
				Domain: $(-\infty, \infty)$

				\vfill

				\columnbreak

				\item $(f-g)(x) = x^2-3x-10$ \\
				Domain:  $(-\infty, \infty)$


			\end{itemize}

		\end{multicols}

		\begin{multicols}{2}

			\begin{itemize}

				\item $(fg)(x) = 3x^3+6x^2-12x-24$ \\
				Domain: $(-\infty, \infty)$

				\vfill

				\columnbreak

				\item $\left(\frac{f}{g}\right)(x) = \frac{x-2}{3}$ \\
				Domain:  $\left(-\infty, -2 \right) \cup \left(-2, \infty \right)$


			\end{itemize}

		\end{multicols}

		\item For  $f(x) = \frac{x}{2}$ and $g(x) = \frac{2}{x}$

		\begin{multicols}{2}

			\begin{itemize}

				\item $(f+g)(x) = \frac{x^2+4}{2x}$ \\
				Domain: $(-\infty, 0) \cup (0, \infty)$

				\vfill

				\columnbreak

				\item $(f-g)(x) = \frac{x^2-4}{2x}$ \\
				Domain:  $(-\infty,0) \cup (0, \infty)$


			\end{itemize}

		\end{multicols}

		\begin{multicols}{2}

			\begin{itemize}

				\item $(fg)(x) = 1$ \\
				Domain: $(-\infty,0) \cup (0, \infty)$

				\vfill

				\columnbreak

				\item $\left(\frac{f}{g}\right)(x) = \frac{x^2}{4}$ \\
				Domain: $(-\infty,0) \cup (0, \infty)$


			\end{itemize}

		\end{multicols}

		\newpage

		\item For   $f(x) =x-1$ and $g(x) = \frac{1}{x-1}$

		\begin{multicols}{2}

			\begin{itemize}

				\item $(f+g)(x) = \frac{x^2-2x+2}{x-1}$ \\
				Domain: $(-\infty, 1) \cup (1, \infty)$

				\vfill

				\columnbreak

				\item $(f-g)(x) = \frac{x^2-2x}{x-1}$ \\
				Domain:  $(-\infty,1) \cup (1, \infty)$


			\end{itemize}

		\end{multicols}

		\begin{multicols}{2}

			\begin{itemize}

				\item $(fg)(x) = 1$ \\
				Domain: $(-\infty,1) \cup (1, \infty)$

				\vfill

				\columnbreak

				\item $\left(\frac{f}{g}\right)(x) =x^2-2x+1$ \\
				Domain: $(-\infty,1) \cup (1, \infty)$


			\end{itemize}

		\end{multicols}


		\item For   $f(x) =x$ and $g(x) = \sqrt{x+1}$

		\begin{multicols}{2}

			\begin{itemize}

				\item $(f+g)(x) = x+\sqrt{x+1}$ \\
				Domain: $[-1,\infty)$

				\vfill

				\columnbreak

				\item $(f-g)(x) = x-\sqrt{x+1}$ \\
				Domain: $[-1,\infty)$


			\end{itemize}

		\end{multicols}

		\begin{multicols}{2}

			\begin{itemize}

				\item $(fg)(x) = x\sqrt{x+1}$ \\
				Domain: $[-1,\infty)$

				\vfill

				\columnbreak

				\item $\left(\frac{f}{g}\right)(x) =\frac{x}{\sqrt{x+1}}$ \\
				Domain: $(-1,\infty)$


			\end{itemize}

		\end{multicols}

		\item For   $f(x) = \sqrt{x-5}$ and $g(x) = f(x) = \sqrt{x-5}$

		\begin{multicols}{2}

			\begin{itemize}

				\item $(f+g)(x) = 2\sqrt{x-5}$ \\
				Domain: $[5,\infty)$

				\vfill

				\columnbreak

				\item $(f-g)(x) =0$ \\
				Domain: $[5,\infty)$


			\end{itemize}

		\end{multicols}

		\begin{multicols}{2}

			\begin{itemize}

				\item $(fg)(x) =x-5$ \\
				Domain: $[5,\infty)$

				\vfill

				\columnbreak

				\item $\left(\frac{f}{g}\right)(x) =1$ \\
				Domain: $(5,\infty)$


			\end{itemize}

		\end{multicols}

    \end{enumerate}

	
\end{Answer}
\begin{Answer}{2.2.5}
	
\end{Answer}
\begin{Answer}{2.2.6}
	\begin{enumerate}

		\item \begin{itemize}

			\item  $C(0) = 26$, so the fixed costs are $\$26$.

			\item  $\overline{C}(10) = 4.6$, so when 10 shirts are produced, the cost per shirt is $\$4.60$.

			\item  $p(5) = 20$, so to sell $5$ shirts, set the price at $\$20$ per shirt.

			\item $R(x) = -2x^2+30x$, $0 \leq x \leq 15$

			\item  $P(x) = -2x^2+28x-26$, $0 \leq x \leq 15$

			\item  $P(x) = 0$ when $x = 1$ and $x=13$.  These are the `break even' points, so selling $1$ shirt or $13$ shirts will guarantee the revenue earned exactly recoups the cost of production.

		\end{itemize}


		\item \begin{itemize}

			\item  $C(0) = 100$, so the fixed costs are $\$100$.

			\item  $\overline{C}(10) = 20$, so when 10 bottles of tonic are produced, the cost per bottle is $\$20$.

			\item  $p(5) = 30$, so to sell $5$ bottles of tonic, set the price at $\$30$ per bottle.

			\item $R(x) = -x^2+35x$, $0 \leq x \leq 35$

			\item  $P(x) = -x^2+25x-100$, $0 \leq x \leq 35$

			\item  $P(x) = 0$ when $x = 5$ and $x=20$.  These are the `break even' points, so selling $5$ bottles of tonic or $20$ bottles of tonic will guarantee the revenue earned exactly recoups the cost of production.

		\end{itemize}


		\item \begin{itemize}

			\item  $C(0) = 240$, so the fixed costs are $240$\textcent \,  or $\$2.40$.

			\item  $\overline{C}(10) = 42$, so when 10 cups of lemonade are made, the cost per cup is $42$\textcent.

			\item  $p(5) = 75$, so to sell $5$ cups of lemonade, set the price at $75$\textcent \,  per cup.

			\item $R(x) = -3x^2+90x$, $0 \leq x \leq 30$

			\item  $P(x) = -3x^2+72x-240$, $0 \leq x \leq 30$

			\item  $P(x) = 0$ when $x = 4$ and $x=20$.  These are the `break even' points, so selling $4$ cups of lemonade or $20$ cups of lemonade will guarantee the revenue earned exactly recoups the cost of production.

		\end{itemize}


		\item  \begin{itemize}

			\item  $C(0) = 36$, so the daily fixed costs are $\$36$.

			\item  $\overline{C}(10) = 6.6$, so when 10 pies are made, the cost per pie is $\$6.60$.

			\item  $p(5) = 9.5$, so to sell $5$ pies a day, set the price at $\$9.50$  per pie.

			\item $R(x) = -0.5 x^2 + 12x$, $0 \leq x \leq 24$

			\item  $P(x) = -0.5 x^2+9x-36$, $0 \leq x \leq 24$

			\item  $P(x) = 0$ when $x = 6$ and $x=12$.  These are the `break even' points, so selling $6$ pies or $12$ pies a day will guarantee the revenue earned exactly recoups the cost of production.

		\end{itemize}

		\item  \begin{itemize}

			\item  $C(0) = 1000$, so the monthly fixed costs are $1000$ \textit{hundred} dollars, or $\$100,\!000$.

			\item  $\overline{C}(10) = 120$, so when 10 scooters are made, the cost per scooter is $120$ hundred dollars, or $\$12,\!000$.

			\item  $p(5) = 130$, so to sell $5$ scooters a month, set the price at $130$ hundred dollars, or $\$13,\!000$ per scooter.

			\item $R(x) = -2x^2+140x$, $0 \leq x \leq 70$

			\item  $P(x) = -2x^2+120x-1000$, $0 \leq x \leq 70$

			\item  $P(x) = 0$ when $x = 10$ and $x=50$.  These are the `break even' points, so selling $10$ scooters or $50$ scooters a month will guarantee the revenue earned exactly recoups the cost of production.

		\end{itemize}

	\end{enumerate}

	
\end{Answer}
\begin{Answer}{2.2.7}

\begin{multicols}{2}
	\begin{enumerate}

		\item $(f + g)(-3) = 2$
		\item $(f - g)(2) = 3$
		\item $(fg)(-1) = 0$
		\item $(g + f)(1) = 0$
		\item $(g - f)(3) = 3$
		\item $(gf)(-3) = -8$
		\item $\left(\frac{f}{g}\right)(-2)$ does not exist
		\item $\left(\frac{f}{g}\right)(-1) = 0$
		\item $\left(\frac{f}{g}\right)(2) = 4$
		\item $\left(\frac{g}{f}\right)(-1)$ does not exist
		\item $\left(\frac{g}{f}\right)(3) = -2$
		\item $\left(\frac{g}{f}\right)(-3) = -\frac{1}{2}$

	\end{enumerate}
\end{multicols}

\end{Answer}
\begin{Answer}{2.3.3}
			$\ds \{x\mid x\ge3\}$, $\{x\mid x\ge0\}$
		
\end{Answer}
\begin{Answer}{2.4.1}
				(a) \hspace{2mm} For $f(x) = 2x-1$,  $f(0) = -1$ and $f(x) = 0$ when $x = \frac{1}{2}$\\
				\\
				(b) \hspace{2mm}  For $f(x) =  3 - \frac{2}{5} x$, $f(0) = 3$ and $f(x) = 0$ when $x = \frac{15}{2}$\\
				\\
				(c) \hspace{2mm} For $f(x) =  2x^2-6$, $f(0) = -6$ and $f(x) = 0$ when $x = \pm \sqrt{3}$\\
				\\
				(d) \hspace{2mm} For $f(x) =  x^2-x-12$, $f(0) = -12$ and $f(x) = 0$ when $x = -3$ or $x=4$\\
				\\
				(e) \hspace{2mm} For $f(x) =  \sqrt{x+4}$, $f(0) = 2$ and $f(x) = 0$ when $x =-4$\\
				\\
				(f) \hspace{2mm} For $f(x) =  \sqrt{1-2x}$, $f(0) = 1$ and $f(x) = 0$ when $x = \frac{1}{2}$\\
				\\
				(g) \hspace{2mm} For $f(x) =   \frac{3}{4-x}$, $f(0) = \frac{3}{4}$ and $f(x)$ is never equal to $0$\\
				\\
				(h) \hspace{2mm} For $f(x) =   \frac{3x^2-12x}{4-x^2}$, $f(0) =0$ and $f(x) = 0$ when $x=0$ or $x=4$\\


			
\end{Answer}
\begin{Answer}{2.4.2}
				\includegraphics[scale=0.2]{images/ExPiecewise1}
			
\end{Answer}
\begin{Answer}{2.4.3}
            
\end{Answer}
\begin{Answer}{2.4.4}
				\begin{tabular}{ll}
					(a) $f(g(-1))=3$ \hspace{1mm} & (b) $g(f(-4))=2$ \hspace{1mm} \\
					(c) $f(g(-3))=-4$ & (d) $f(g(-2))=2$ \\
					(e) $g(g(-1))=-4$ &  (f) $f(g(0)-1)=2$ \\
					(e) $f(g(g(-2)))=3$ \hspace{1mm} & (f) $g(f(f(-4)))=2$ \\
				\end{tabular}

			
\end{Answer}
\begin{Answer}{2.4.5}
				(a) $f(-x)=-f(x)$ $\therefore \,\, f(x)$ is an odd function \hspace{3mm} (b)  $f(-x)=f(x)$ $\therefore \,\, f(x)$ is an even function \\
				(c)  $f(-x)\neq f(x) \neq -f(x)$ $\therefore \,\, f(x)$ is neither even nor odd \\
				(d)  $g(-x)=g(x)$ $\therefore \,\, g(x)$ is an even function \hspace{3mm} (e)  $f(-x)=f(x)$ $\therefore \,\, f(x)$ is an even function \hspace{3mm}  (f) $f(-x) \neq f(x) \neq -f(x)$ $\therefore \,\, f(x)$ is neither even nor odd.
			
\end{Answer}
\begin{Answer}{2.4.6}
		
\end{Answer}
\begin{Answer}{2.4.7}
				\includegraphics[scale=0.2]{images/ExSymmetrySOLN}
			
\end{Answer}
\begin{Answer}{2.4.8}

			
\end{Answer}
\begin{Answer}{2.4.9}

				
\end{Answer}
\begin{Answer}{2.5.1}
	\begin{multicols}{3}
		\begin{enumerate}

			\item $(2,0)$
			\item $(-1,-3)$
			\item $(2,-4)$

			\item $(3,-3)$
			\item $(2,-9)$
			\item $\left(\frac{2}{3}, -3\right)$

			\item $(2,3)$
			\item $(-2,-3)$
			\item $(5,-2)$

			\item $(1,-6)$
			\item $(2,13)$
			\item $y = (1,-10)$

			\item $\left(2, -\frac{3}{2}\right)$
			\item $\left(\frac{1}{2}, -12 \right)$
			\item $(-1,-7)$

			\item $\left(-\frac{1}{2}, -3\right)$
			\item $\left(\frac{2}{3}, -2 \right)$
			\item $(1,1)$
\end{enumerate}
\end{multicols}
\end{Answer}
\begin{Answer}{2.5.2}
\end{Answer}
\begin{Answer}{2.5.3}
\end{Answer}
\begin{Answer}{2.5.4}
	\begin{multicols}{2}
		\begin{enumerate}

			\item  $g(x) = \sqrt{x-2} - 3$
			\item  $g(x) = \sqrt{x-2} - 3$
			\item  $g(x) = -\sqrt{x} + 1$
			\item  $g(x) = -(\sqrt{x} + 1) = -\sqrt{x} - 1$
			\item  $g(x) = \sqrt{-x+1} + 2$
			\item  $g(x) = \sqrt{-(x+1)} + 2 = \sqrt{-x-1} + 2$
			\item  $g(x) = 2\sqrt{x+3} - 4$
			\item  $g(x) = 2\left(\sqrt{x+3} - 4\right) = 2\sqrt{x+3} - 8$
			\item  $g(x) = \sqrt{2x-3} + 1$
			\item  $g(x) = \sqrt{2(x-3)} + 1 = \sqrt{2x-6}+1$
			\item $g(x) = -2\sqrt[3]{x + 3} - 1$ or $g(x) = 2\sqrt[3]{-x - 3} - 1$

		\end{enumerate}
	\end{multicols}
\end{Answer}
\begin{Answer}{2.5.5}

\end{Answer}
\begin{Answer}{2.5.6}
\end{Answer}
\begin{Answer}{2.5.7}
	
\end{Answer}
\begin{Answer}{2.5.8}
	
\end{Answer}
\begin{Answer}{2.5.9}
	
\end{Answer}
\begin{Answer}{2.5.10}
	
\end{Answer}
\begin{Answer}{2.6.1}
	\begin{multicols}{2}
		\begin{enumerate}

			\item $y+1 = 3(x-3)$ \\ $y = 3x-10$
			\item $y-8 = -2(x+5)$ \\ $y = -2x-2$
			\item $y + 1 = -(x+7)$ \\ $y = -x-8$
			\item $y - 1 = \frac{2}{3} (x+2)$ \\ $y = \frac{2}{3} x + \frac{7}{3}$
			\item $y - 4 = -\frac{1}{5} (x-10)$ \\ $y = -\frac{1}{5} x + 6$
			\item $y - 4 = \frac{1}{7}(x + 1)$ \\ $y = \frac{1}{7}x + \frac{29}{7}$
			\item $y - 117 = 0$ \\ $y = 117$
			\item $y + 3 = -\sqrt{2}(x - 0)$ \\ $y = -\sqrt{2}x - 3$
			\item $y - 2\sqrt{3} = -5(x - \sqrt{3})$ \\ $y = -5x + 7\sqrt{3}$
			\item $y + 12 = 678(x + 1)$ \\ $y = 678x + 666$

		\end{enumerate}
	\end{multicols}
\end{Answer}
\begin{Answer}{2.6.2}
	\begin{multicols}{2}
		\begin{enumerate}

			\item $y = -\frac{5}{3}x$
			\item $y = -2$
			\item $y = \frac{8}{5}x - 8$
			\item $y = \frac{9}{4}x - \frac{47}{4}$
			\item $y = 5$
			\item $y = -8$
			\item $y = -\frac{5}{4} x + \frac{11}{8}$
			\item $y = 2x + \frac{13}{6}$
			\item $y = -x$
			\item $y = \frac{\sqrt{3}}{3} x$

		\end{enumerate}
	\end{multicols}

\end{Answer}
\begin{Answer}{2.6.3}

\end{Answer}
\begin{Answer}{2.6.4}
		$(-1,-1)$ and $(11/5, \, 27/5)$
	
\end{Answer}
\begin{Answer}{2.6.5}
  $d(t) = 3t$, $t \geq 0$.
	
\end{Answer}
\begin{Answer}{2.6.6}
	$C(x) = 45x+20$, $x \geq 0$.
	
\end{Answer}
\begin{Answer}{2.6.7}
		$C(p) = 0.035p + 1.5 \;$  The slope $0.035$ means it costs $3.5$\textcent \, per page.  $C(0) = 1.5$ means there is a fixed, or start-up, cost of $\$1.50$ to make each book.
	
\end{Answer}
\begin{Answer}{2.6.8}
		\begin{enumerate}

			\item $F(C) = \frac{9}{5}C + 32$
			\item $C(F) = \frac{5}{9}(F - 32) = \frac{5}{9}F - \frac{160}{9}$
			\item $F(-40) = -40 = C(-40)$.

		\end{enumerate}

	
\end{Answer}
\begin{Answer}{2.6.9}
		 ${\displaystyle C(p) = \left\{ \begin{array}{rcl} 6p + 1.5 & \mbox{ if } & 1 \leq p \leq 5 \\
			5.5p & \mbox{ if } & p\geq 6
			\end{array} \right. }$
	
\end{Answer}
\begin{Answer}{2.6.10}
		${\displaystyle T(n) = \left\{ \begin{array}{rcl} 15n & \mbox{ if } & 1 \leq n \leq 9 \\
			12.5n & \mbox{ if } & n \geq 10 \\
			\end{array} \right. }$
	
\end{Answer}
\begin{Answer}{2.6.11}
	\begin{enumerate}

		\item \[{\displaystyle D(d) = \left\{ \begin{array}{rcl} 8 & \mbox{ if } & 0 \leq d \leq 15 \\
			-\frac{1}{2} \, d + \frac{31}{2} & \mbox{ if } & 15 \leq d \leq 27 \\
			2 & \mbox{ if } & 27 \leq d \leq 37  \\
			\end{array} \right. }\]

		\item \[{\displaystyle D(s) = \left\{ \begin{array}{rcl} 2 & \mbox{ if } & 0 \leq s \leq 10 \\
			\frac{1}{2} \, s -3 & \mbox{ if } & 10 \leq s \leq 22 \\
			8 & \mbox{ if } & 22 \leq s \leq 37  \\
			\end{array} \right. }\]

		\item $$\includegraphics[scale=0.3]{images/Pool-ex1}$$
\end{enumerate}

\end{Answer}
\begin{Answer}{2.6.12}

		\begin{enumerate}

			\item $\dfrac{2^{3} - (-1)^{3}}{2 - (-1)} = 3$
			\item $\dfrac{\frac{1}{5} - \frac{1}{1}}{5 - 1} = -\dfrac{1}{5}$
			\item $\dfrac{\sqrt{16} - \sqrt{0}}{16 - 0} = \dfrac{1}{4}$
			\item $\dfrac{3^{2} - (-3)^{2}}{3 - (-3)} = 0$
			\item $\dfrac{\frac{7 + 4}{7 - 3} - \frac{5 + 4}{5 - 3}}{7 - 5} = -\dfrac{7}{8}$
			\item \scriptsize $\dfrac{(3(2)^{2}+2(2)-7)-(3(-4)^{2}+2(-4)-7)}{2-(-4)}=-4$ \normalsize
		\end{enumerate}

\end{Answer}
\begin{Answer}{2.6.13}
	\begin{enumerate}

		\item $3x^{2} + 3xh + h^{2}$
		\item $\dfrac{-1}{x(x + h)}$
		\item $\dfrac{-7}{(x - 3)(x + h - 3)}$
		\item $6x + 3h + 2$
\end{enumerate}

\end{Answer}
\begin{Answer}{2.6.14}
\end{Answer}
\begin{Answer}{2.6.15}
\end{Answer}
\begin{Answer}{2.6.16}
\end{Answer}
\begin{Answer}{2.6.17}
\end{Answer}
\begin{Answer}{2.6.18}
\end{Answer}
\begin{Answer}{2.6.19}
\end{Answer}
\begin{Answer}{2.6.20}
	
\end{Answer}
\begin{Answer}{2.6.21}
	
\end{Answer}
\begin{Answer}{2.6.22}
\end{Answer}
\begin{Answer}{2.6.23}
\end{Answer}
\begin{Answer}{2.6.24}
\end{Answer}
\begin{Answer}{2.6.25}
\end{Answer}
\begin{Answer}{2.6.26}
\end{Answer}
\begin{Answer}{2.6.27}
\end{Answer}
\begin{Answer}{2.6.28}
\end{Answer}
\begin{Answer}{2.6.29}
\end{Answer}
\begin{Answer}{2.7.1}
$y=2^x$
\end{Answer}
\begin{Answer}{2.7.2}
$y=7$
\end{Answer}
\begin{Answer}{2.7.3}
$y=2$
\end{Answer}
\begin{Answer}{2.7.4}
$x\neq 0$
\end{Answer}
\begin{Answer}{2.10.1}
\begin{enumerate}
	\item	$\pi/3$
	\item	$3\pi/4$
\end{enumerate}
\end{Answer}
\begin{Answer}{2.10.2}
\begin{multicols}{2}
\begin{enumerate}
	\item	$\pi/4$
	\item	$-\pi/3$
	\item	$1/3$
	\item	$-3/4$
\end{enumerate}
\end{multicols}
\end{Answer}
\begin{Answer}{2.10.3}
$\sqrt{1-x^2}/x$ with domain $[-1,0)\cup(0,1]$.
\end{Answer}
\begin{Answer}{2.11.1}
		\begin{enumerate}
		\item {\hfill$\begin{aligned}[t]
				\coth^2x-\csch^2x &= \left(\frac{e^x+e^{-x}}{e^x-e^{-x}}\right)^2 - \left(\frac{2}{e^x-e^{-x}}\right)^2 \\
													&= \frac{(e^{2x} + 2 + e^{-2x}) - (4)}{e^{2x} - 2 + e^{-2x}}\\
													&= \frac{e^{2x} - 2 + e^{-2x}}{e^{2x} - 2 + e^{-2x}}\\
													&= 1
		\end{aligned}$\hfill\null}
		\item {\hfill$\begin{aligned}[t]
				\cosh^2x+\sinh^2x &= \left(\frac{e^x+e^{-x}}{2}\right)^2 + \left(\frac{e^x-e^{-x}}{2}\right)^2 \\
													&= \frac{e^{2x} + 2 + e^{-2x}}{4} + \frac{e^{2x} - 2 + e^{-2x}}{4}\\
													&= \frac{2e^{2x}  + 2e^{-2x}}{4}\\
													&= \frac{e^{2x} + e^{-2x}}{2} \\
													&= \cosh 2x.
		\end{aligned}$\hfill\null}
		\item {\hfill$\begin{aligned}[t]
				\cosh^2x &= \left(\frac{e^x+e^{-x}}{2}\right)^2  \\
													&= \frac{e^{2x} + 2 + e^{-2x}}{4} \\
													&= \frac12\frac{(e^{2x}  + e^{-2x})+2}{2}\\
													&= \frac12\left(\frac{e^{2x}  + e^{-2x}}{2}+1\right)\\
													&= \frac{\cosh2x+1}{2}.
		\end{aligned}$\hfill\null}
		\item {\hfill$\begin{aligned}[t]
				\sinh^2x &= \left(\frac{e^x-e^{-x}}{2}\right)^2  \\
													&= \frac{e^{2x} - 2 + e^{-2x}}{4} \\
													&= \frac12\frac{(e^{2x}  + e^{-2x})-2}{2}\\
													&= \frac12\left(\frac{e^{2x}  + e^{-2x}}{2}-1\right)\\
													&= \frac{\cosh2x-1}{2}.
		\end{aligned}$\hfill\null}
		\end{enumerate}
	
\end{Answer}
\begin{Answer}{2.12.1}
	(d)
\end{Answer}
\begin{Answer}{2.12.2}
	$3/\left[5(x+3)\right]$
\end{Answer}
\begin{Answer}{2.12.3}
	$6+h$
\end{Answer}
\begin{Answer}{2.12.4}
\begin{enumerate}
	\item	$[2,3)\cup(3,\infty)$
	\item	$(-\infty,-3)\cup(-3,3)\cup(3,\infty)$
\end{enumerate}
\end{Answer}
\begin{Answer}{2.12.5}
	$\left\{x:x\neq 0\right\}$
\end{Answer}
\begin{Answer}{2.12.6}
	$g(x)=(5x+26)/3$
\end{Answer}
\begin{Answer}{2.12.7}
	(c)
\end{Answer}
\begin{Answer}{2.12.8}
	$f^{-1}(x)=f(x)=\ln\left(\dfrac{e^x}{e^x-1}\right)$ and its domain is $(0,\infty)$.
\end{Answer}
\begin{Answer}{2.12.9}
\begin{enumerate}
	\item	$2-\ln 3$
	\item	1,3
	\item	$(e^2-1)^2$
	\item	3
\end{enumerate}
\end{Answer}
\begin{Answer}{2.12.10}
	$-\pi$
\end{Answer}
\begin{Answer}{2.12.11}
	$2\pi/5$
\end{Answer}
\begin{Answer}{2.12.12}
	1
\end{Answer}
\begin{Answer}{3.2.1}
\begin{enumerate}
\item {Answers will vary.}
\item {An indeterminate form.}
\item {F}
\item {The function may approach different values from the left and right, the function may grow without bound, or the function might oscillate.}
\item {Answers will vary.}
\item {F}
\item {F}
\item {T}
\end{enumerate}
\end{Answer}
\begin{Answer}{3.2.2}
\begin{multicols}{3}
\begin{enumerate}
	\item	$8$
	\item	$6$
	\item	dne
	\item	$-2$
	\item	$-1$
	\item	$8$
	\item	$7$
	\item	$6$
	\item	$3$
	\item	$-3/2$
	\item	$6$
	\item	$2$
\end{enumerate}
\end{multicols}
\end{Answer}
\begin{Answer}{3.2.3}
\begin{enumerate}
\item {$\displaystyle \lim_{x\to 1} x^2+3x-5$}
\item {$-5$}
\item {Limit does not exist}
\item  {$2$}
\item {$1.5$}
\item
{Limit does not exist.}
\item
{Limit does not exist.}
\item
{$7$}
\item
{$1$}
\item
{Limit does not exist.}

\end{enumerate}
\end{Answer}
\begin{Answer}{3.2.4}
\begin{enumerate}
\item
{\begin{tabular}{cc}
$h$ & $\frac{f(a+h)-f(a)}{h}$\\ \hline
 $-0.1$ & $-7$ \\
 $-0.01$ & $-7$ \\
 $0.01$ & $-7$ \\
 $0.1$ & $-7$
\end{tabular}
The limit seems to be exactly 7.
}
\item
{\begin{tabular}{cc}
$h$ & $\frac{f(a+h)-f(a)}{h}$\\ \hline
 $-0.1$ & $9$ \\
 $-0.01$ & $9$ \\
 $0.01$ & $9$ \\
 $0.1$ & $9$
\end{tabular}
The limit seems to be exactly 9.
}
\item
{\begin{tabular}{cc}
$h$ & $\frac{f(a+h)-f(a)}{h}$\\ \hline
 $-0.1$ & $4.9$ \\
 $-0.01$ & $4.99$ \\
 $0.01$ & $5.01$ \\
 $0.1$ & $5.1$
\end{tabular}
The limit is approx. 5.
}
\item
{\begin{tabular}{cc}
$h$ & $\frac{f(a+h)-f(a)}{h}$\\ \hline
 $-0.1$ & $-0.114943$ \\
 $-0.01$ & $-0.111483$ \\
 $0.01$ & $-0.110742$ \\
 $0.1$ & $-0.107527$
\end{tabular}
The limit is approx. $-0.11$.
}
\item
{\begin{tabular}{cc}
$h$ & $\frac{f(a+h)-f(a)}{h}$\\ \hline
 $-0.1$ & $29.4$ \\
 $-0.01$ & $29.04$ \\
 $0.01$ & $28.96$ \\
 $0.1$ & $28.6$
\end{tabular}
The limit is approx. $29$.
}
\item
{\begin{tabular}{cc}
$h$ & $\frac{f(a+h)-f(a)}{h}$\\ \hline
 $ -0.1$ & $0.202027$ \\
 $-0.01$ & $0.2002$ \\
 $0.01$ & $0.1998$ \\
 $0.1$ & $0.198026$
\end{tabular}
The limit is approx. $0.2$.
}
\item
{\begin{tabular}{cc}
$h$ & $\frac{f(a+h)-f(a)}{h}$ \\ \hline
 $ -0.1$ & $-0.998334$ \\
 $-0.01$ & $-0.999983$ \\
 $0.01$ & $-0.999983$ \\
 $0.1$ & $-0.998334$
\end{tabular}
The limit is approx. $-1$.
}
\item {\begin{tabular}{cc}
$h$ & $\frac{f(a+h)-f(a)}{h}$\\ \hline
 $-0.1$ & $-0.0499583$ \\
 $-0.01$ & $-0.00499996$ \\
 $0.01$ & $0.00499996$ \\
 $0.1$ & $0.0499583$
\end{tabular}
The limit is approx. $0.005$.
}
\end{enumerate}
\end{Answer}
\begin{Answer}{3.2.5}
\begin{enumerate}
\item {\begin{enumerate}
\item		2
\item		2
\item		2
\item		1
\item	 	As $f$ is not defined for $x<0$, this limit is not defined.
\item		1
\end{enumerate}}
\item {\begin{enumerate}
\item		1
\item		2
\item		Does not exist.
\item		2
\item		0
\item	 	As $f$ is not defined for $x<0$, this limit is not defined.
\end{enumerate}
}
\item {\begin{enumerate}
\item		Does not exist.
\item		Does not exist.
\item		Does not exist.
\item		Not defined.
\item		0
\item	 	0
\end{enumerate}
}
\item {\begin{enumerate}
\item		2
\item		0
\item		Does not exist.
\item		1
\end{enumerate}
}
\item {\begin{enumerate}
\item		2
\item		2
\item		2
\item		2
\end{enumerate}
}
\item {\begin{enumerate}
\item		4
\item		$-4$
\item		Does not exist.
\item		0
\end{enumerate}
}
\item {\begin{enumerate}
\item		2
\item		2
\item		2
\item		0
\item		2
\item		2
\item		2
\item		Not defined

\end{enumerate}
}
\item  {\begin{enumerate}
\item		$a-1$
\item		$a$
\item		Does not exist.
\item		$a$
\end{enumerate}
}
\end{enumerate}
\end{Answer}
\begin{Answer}{3.2.6}
\begin{enumerate}
\item
\item
\item
\item
\end{enumerate}
\end{Answer}
\begin{Answer}{3.3.1}
\begin{enumerate}
\item {$\epsilon$ should be given first, and the restriction 	$|x-a|<\delta$ implies $|f(x)-K|< \epsilon$, not the other way around.}
\item {The $y$--tolerance.}
\item {T}
\item
{T}
\end{enumerate}
\end{Answer}
\begin{Answer}{3.3.2}
\begin{enumerate}
\item {Let $\epsilon >0$ be given. We wish to find $\delta >0$ such that when $|x-2|<\delta$, $|f(x)-5|<\epsilon$. However, since $f(x)=5$, a constant function, the latter inequality is simply $|5-5|<\epsilon$, which is always true. Thus we can choose any $\delta$ we like; we arbitrarily choose $\delta =\epsilon$.
}
\item {Let $\epsilon >0$ be given. We wish to find $\delta >0$ such that when $|x-5|<\delta$, $|f(x)-(-2)|<\epsilon$.

Consider $|f(x)-(-2)|<\epsilon$:
\begin{gather*}
|f(x) + 2 | < \epsilon \\
|(3-x) + 2 |<\epsilon \\
| 5-x | < \epsilon \\
-\epsilon < 5-x < \epsilon \\
-\epsilon < x-5 < \epsilon. \\
\end{gather*}
This implies we can let $\delta =\epsilon$. Then:
\begin{gather*}
|x-5|<\delta \\
-\delta < x-5 < \delta\\
-\epsilon < x-5 < \epsilon\\
-\epsilon < (x-3)-2 < \epsilon \\
-\epsilon < (-x+3)-(-2) < \epsilon \\
|3-x - (-2)| < \epsilon,
\end{gather*}
which is what we wanted to prove.
}

\item {Let $\epsilon >0$ be given. We wish to find $\delta >0$ such that when $|x-3|<\delta$, $|f(x)-6|<\epsilon$.

Consider $|f(x)-6|<\epsilon$, keeping in  mind we want to make a statement about $|x-3|$:
\begin{gather*}
|f(x) -6 | < \epsilon \\
|x^2-3 -6 |<\epsilon \\
| x^2-9 | < \epsilon \\
| x-3 |\cdot|x+3| < \epsilon \\
| x-3 | < \epsilon/|x+3| \\
\end{gather*}
Since $x$ is near 3, we can safely assume that, for instance, $2<x<4$. Thus
\begin{gather*}
2+3<x+3<4+3 \\
5 < x+3 < 7 \\
\frac{1}{7} < \frac{1}{x+3} < \frac{1}{5} \\
\frac{\epsilon}{7} < \frac{\epsilon}{x+3} < \frac{\epsilon}{5} \\
\end{gather*}
Let $\delta =\frac{\epsilon}{7}$. Then:
\begin{gather*}
|x-3|<\delta \\
|x-3| < \frac{\epsilon}7\\
|x-3| < \frac{\epsilon}{x+3}\\
|x-3|\cdot|x+3| < \frac{\epsilon}{x+3}\cdot|x+3|\\
\end{gather*}
Assuming $x$ is near 3, $x+3$ is positive and we can drop the absolute value signs on the right.
\begin{gather*}
|x-3|\cdot|x+3| < \frac{\epsilon}{x+3}\cdot(x+3)\\
|x^2-9| < \epsilon\\
|(x^2-3) - 6| < \epsilon,
\end{gather*}
which is what we wanted to prove.
}
\item
{Let $\epsilon >0$ be given. We wish to find $\delta >0$ such that when $|x-2|<\delta$, $|f(x)-7|<\epsilon$.

Consider $|f(x)-7|<\epsilon$, keeping in  mind we want to make a statement about $|x-2|$:
\begin{gather*}
|f(x) -7 | < \epsilon \\
|x^3-1 -7 |<\epsilon \\
| x^3-8 | < \epsilon \\
| x-2 |\cdot|x^2+2x+4| < \epsilon \\
| x-3 | < \epsilon/|x^2+2x+4| \\
\end{gather*}
Since $x$ is near 2, we can safely assume that, for instance, $1<x<3$. Thus
\begin{gather*}
1^2+2\cdot1+4<x^2+2x+4<3^2+2\cdot3+4 \\
7 < x^2+2x+4 < 19 \\
\frac{1}{19} < \frac{1}{x^2+2x+4} < \frac{1}{7} \\
\frac{\epsilon}{19} < \frac{\epsilon}{x^2+2x+4} < \frac{\epsilon}{7} \\
\end{gather*}
Let $\delta =\frac{\epsilon}{19}$. Then:
\begin{gather*}
|x-2|<\delta \\
|x-2| < \frac{\epsilon}{19}\\
|x-2| < \frac{\epsilon}{x^2+2x+4}\\
|x-2|\cdot|x^2+2x+4| < \frac{\epsilon}{x^2+2x+4}\cdot|x^2+2x+4|\\
\end{gather*}
Assuming $x$ is near 2, $x^2+2x+4$ is positive and we can drop the absolute value signs on the right.
\begin{gather*}
|x-2|\cdot|x^2+2x+4| < \frac{\epsilon}{x^2+2x+4}\cdot(x^2+2x+4)\\
|x^3-8| < \epsilon\\
|(x^3-1) - 7| < \epsilon,
\end{gather*}
which is what we wanted to prove.
}
\item {Let $\epsilon >0$ be given. We wish to find $\delta >0$ such that when $|x-0|<\delta$, $|f(x)-0|<\epsilon$.

Consider $|f(x)-0|<\epsilon$, keeping in  mind we want to make a statement about $|x-0|$ (i.e., $|x|$):
\begin{gather*}
|f(x) -0 | < \epsilon \\
|e^{2x}-1 |<\epsilon \\
-\epsilon< e^{2x}-1 < \epsilon \\
1-\epsilon< e^{2x} < 1+\epsilon \\
\ln (1-\epsilon) < 2x < \ln (1+\epsilon) \\
\frac{\ln (1-\epsilon)}{2} < x < \frac{\ln (1+\epsilon)}{2} \\
\end{gather*}
Let $\delta = \min\left\{\left|\frac{\ln(1-\epsilon)}{2}\right|,\frac{\ln(1+\epsilon)}{2}\right\}=\frac{\ln(1+\epsilon)}{2}.$

Thus:
\begin{gather*}
|x| < \delta \\
%|x| < \min\left\{\left|\frac{\ln(1-\epsilon)}{2}\right|,\left|\frac{\ln(1+\epsilon)}{2}\right|\right\} \\
|x| <\frac{\ln(1+\epsilon)}{2}<\left|\frac{\ln(1-\epsilon)}{2}\right| \\
\frac{\ln(1-\epsilon)}{2} < x < \frac{\ln(1+\epsilon)}{2}\\
\ln(1-\epsilon)< 2x < \ln(1+\epsilon)\\
1-\epsilon < e^{2x} < 1+\epsilon\\
-\epsilon < e^{2x}-1 < \epsilon\\
|e^{2x}-1-(0)| < \epsilon,
\end{gather*}
which is what we wanted to prove.
}
\item
{Let $\epsilon >0$ be given. We wish to find $\delta >0$ such that when $|x-0|<\delta$, $|f(x)-0|<\epsilon$. In simpler terms, we want to show that when $|x|<\delta$, $|\sin x| < \epsilon$.

Set $\delta = \epsilon$. We start with assuming that $|x|<\delta$. Using the hint, we have that $|\sin x | < |x| < \delta = \epsilon$. Hence if $|x|<\delta$, we know immediately that $|\sin x| < \epsilon$.
}
\item {Let $\epsilon >0$ be given. We wish to find $\delta >0$ such that when $|x-4|<\delta$, $|f(x)-15|<\epsilon$.

Consider $|f(x)-15|<\epsilon$, keeping in  mind we want to make a statement about $|x-4|$:
\begin{gather*}
|f(x) -15 | < \epsilon \\
|x^2+x-5 -15 |<\epsilon \\
| x^2+x-20 | < \epsilon \\
| x-4 |\cdot|x+5| < \epsilon \\
| x-4 | < \epsilon/|x+5| \\
\end{gather*}
Since $x$ is near 4, we can safely assume that, for instance, $3<x<5$. Thus
\begin{gather*}
3+5<x+5<5+5 \\
8 < x+5 < 10 \\
\frac{1}{10} < \frac{1}{x+5} < \frac{1}{8} \\
\frac{\epsilon}{10} < \frac{\epsilon}{x+5} < \frac{\epsilon}{8} \\
\end{gather*}
Let $\delta =\frac{\epsilon}{10}$. Then:
\begin{gather*}
|x-4|<\delta \\
|x-4| < \frac{\epsilon}{10}\\
|x-4| < \frac{\epsilon}{x+5}\\
|x-4|\cdot|x+5| < \frac{\epsilon}{x+5}\cdot|x+5|\\
\end{gather*}
Assuming $x$ is near 4, $x+5$ is positive and we can drop the absolute value signs on the right.
\begin{gather*}
|x-4|\cdot|x+5| < \frac{\epsilon}{x+5}\cdot(x+5)\\
|x^2+x-20| < \epsilon\\
|(x^2+x-5) -15| < \epsilon,
\end{gather*}
which is what we wanted to prove.
}

\end{enumerate}
\end{Answer}
\begin{Answer}{3.4.1}
\begin{multicols}{3}
\begin{enumerate}
	\item	$8$
	\item	$6$
	\item	dne
	\item	$-2$
	\item	$-1$
	\item	$8$
	\item	$7$
	\item	$6$
	\item	$3$
	\item	$-3/2$
	\item	$6$
	\item	$2$
\end{enumerate}
\end{multicols}
\end{Answer}
\begin{Answer}{3.5.1}
{Answers will vary.
}
\end{Answer}
\begin{Answer}{3.5.2}
{Answers will vary.
}
\end{Answer}
\begin{Answer}{3.5.3}
{Answers will vary.
}
\end{Answer}
\begin{Answer}{3.5.4}
\begin{enumerate}
\item
{9}
\item
{6}
\item
{0}
\item
{Limit does not exist.}
\item
{3}
\item
{Not possible to know.}
\item
{3}
\item
{$-45$}
\end{enumerate}
\end{Answer}
\begin{Answer}{3.5.5}
\begin{enumerate}
\item {$1$}
\item {$-1$}
\item {$0$}
\item {$\pi$}
\end{enumerate}
\end{Answer}
\begin{Answer}{3.5.6}
\begin{multicols}{2}
\begin{enumerate}
	\item	7
	\item	5
	\item	0
	\item {$\frac{3\pi+1}{1-\pi}$}
	\item {$\frac{\pi^2+3\pi+5}{5\pi^2-2\pi-3} \approx 0.6064$}
	\item 	{$-0.000000015\approx 0$}
	\item {$1/2$}
	\item 	{Limit does not exist}
	\item 	{$64$}
	\item	undefined
	\item	$1/6$
	\item	0
	\item	3
	\item	172
	\item	0
	\item	2
	\item	does not exist
	\item	$\ds \sqrt2$
	\item	$\ds 3a^2$
	\item	512
\end{enumerate}
\end{multicols}
\end{Answer}
\begin{Answer}{3.5.7}
	$L=0$ and $M=1.$ No.
\end{Answer}
\begin{Answer}{3.6.1}
\begin{enumerate}
	\item	$5$
	\item	$7/2$
	\item	$3/4$
	\item	$1$
	\item	$-\sqrt2/2$
\end{enumerate}
\end{Answer}
\begin{Answer}{3.6.2}
 $7$
\end{Answer}
\begin{Answer}{3.6.3}
 $2$
\end{Answer}
\begin{Answer}{3.6.4}
\begin{enumerate}
\item {$0$}
\item {$0$}
\item {$0$}
\item {$9$}
\end{enumerate}
\end{Answer}
\begin{Answer}{3.6.6}
	$ 3 $
\end{Answer}
\begin{Answer}{3.6.7}
\begin{enumerate}
\item {$3$}
\item {$5/8$}
\item {$1$}
\item {$\pi/180$}
\end{enumerate}
\end{Answer}
\begin{Answer}{3.7.1}
\begin{enumerate}
\item {Consider the function $h(x) = g(x) - f(x)$, and use the Bisection Method to find a root of $h$.}
\item {A root of a function $f$ is a value $c$ such that $f(c)=0$.}
\item {Answers will vary.}

\item {Answers will vary.}
\item {T}
\item {F}
\item {F}
\item {T}
\item {T}
\item {F}
\end{enumerate}
\end{Answer}
\begin{Answer}{3.7.2}
\begin{enumerate}
\item {No; $\ds \lim_{x\to 1} f(x) = 2$, while $f(1) = 1$.}
\item {No; $\ds \lim_{x\to 1} f(x)$ does not exist.}
\item {No; $f(1)$ does not exist.}
\item {Yes}
\item {Yes}
\item {Yes}
\item {\begin{enumerate}
\item		No; $\ds \lim_{x\to -2}f(x) \neq f(-2)$
\item		Yes
\item		No; $f(2)$ is not defined.
\end{enumerate}
}
\item No.
\end{enumerate}
\end{Answer}
\begin{Answer}{3.7.3}
\begin{enumerate}
\item {\begin{enumerate}
\item		Yes
\item		Yes
\end{enumerate}
}
\item {\begin{enumerate}
\item		Yes
\item		No; the left and right hand limits at 1 are not equal.
\end{enumerate}
}
\item {\begin{enumerate}
\item		Yes
\item		Yes
\end{enumerate}
}
\item {\begin{enumerate}
\item		Yes
\item		No. $\lim_{x\to 8} f(x) = 16/5 \neq f(8) = 5$.
\end{enumerate}
}
\end{enumerate}
\end{Answer}
\begin{Answer}{3.7.4}
\begin{enumerate}
\item {$(-\infty,\infty)$
}
\item {$(-\infty,-2]\cup [2,\infty)$
}
\item {$[-1,1]$
}
\item {$(-\infty,-\sqrt{6}]\cup [\sqrt{6},\infty)$
}
\item {$(-1,1)$
}
\item {$(-\infty,\infty)$
}
\item {$(-\infty,\infty)$
}
\item {$(0,\infty)$
}
\item {$(-\infty,\infty)$
}
\item
{$(-\infty,0]$
}
\item
{$(-\infty,\infty)$
}
\end{enumerate}
\end{Answer}
\begin{Answer}{3.7.8}
{Yes, by the Intermediate Value Theorem.
}
\end{Answer}
\begin{Answer}{3.7.9}
  {Yes, by the Intermediate Value Theorem. In fact, we can be more specific and state such a value $c$ exists in $(0,2)$, not just in $(-3,7)$.
  }
\end{Answer}
\begin{Answer}{3.7.10}
 {We cannot say; the Intermediate Value Theorem only applies to function values between $-10$ and 10; as 11 is outside this range, we do not know.
  }
\end{Answer}
\begin{Answer}{3.7.11}
 {We cannot say; the Intermediate Value Theorem only applies to continuous functions. As we do know know if $h$ is continuous, we cannot say.
  }
\end{Answer}
\begin{Answer}{3.7.12}
\begin{enumerate}
\item {Approximate root is $x=1.23$. The intervals used are:
$[1,1.5] \quad [1,1.25] \quad [1.125,1.25]$
$[1.1875,1.25]\quad [1.21875,1.25]\quad [1.234375,1.25]$
$[1.234375,1.2421875]\quad [1.234375,1.2382813]$
}
\item {Approximate root is $x=0.52$. The intervals used are:
$[0.5,0.55] \quad [0.5,0.525] \quad [0.5125,0.525]$
$[0.51875,0.525]\quad [0.521875,0.525]$
}
\item {Approximate root is $x=0.69$. The intervals used are:
$[0.65,0.7] \quad [0.675,0.7] \quad [0.6875,0.7]$
$[0.6875,0.69375]\quad [0.690625,0.69375]$
}
\item {Approximate root is $x=0.78$. The intervals used are:
$[0.7,0.8] \quad [0.75,0.8] \quad [0.775,0.8]$
$[0.775,0.7875]\quad [0.78125,0.7875]$

(A few more steps show that $0.79$ is better as the root is $\pi/4 \approx 0.78539$.)
}
\end{enumerate}
\end{Answer}
\begin{Answer}{3.7.13}
{\begin{enumerate}
\item		20
\item		25
\item		Limit does not exist
\item		25
\end{enumerate}
}

\end{Answer}
\begin{Answer}{3.7.14}


{\begin{tabular}{cc}
$x$ & $f(x)$ \\ \hline
$-0.81 $& $-2.34129$ \\
$ -0.801$ & $-2.33413$ \\
$ -0.79 $& $-2.32542 $\\
$ -0.799$ & $-2.33254$
\end{tabular}

The top two lines give an approximation of the limit from the left: $-2.33$. The bottom two lines give an approximation from the right: $-2.33$ as well.
}
\end{Answer}
\begin{Answer}{3.8.1}
\begin{multicols}{3}
\begin{enumerate}
	\item	$1$
	\item	$1$
	\item	$-\infty$
	\item	$1/3$
	\item	$0$
	\item	$\infty$
	\item	$\infty$
	\item	$2/7$
	\item	$2$
	\item	$-\infty$
	\item	$\infty$
	\item	$0$
	\item	$1/2$
	\item	$5$
	\item	$\ds 2\sqrt2$
	\item	$3/2$
	\item	$\infty$
	\item	does not exist
\end{enumerate}
\end{multicols}
\end{Answer}
\begin{Answer}{3.8.2}
$y=1$ and $y=-1$
\end{Answer}
\begin{Answer}{3.8.3}
	$x=0$ and $x=2$.
\end{Answer}
\begin{Answer}{3.8.5}
	$y=x+4$
\end{Answer}
\begin{Answer}{3.8.6}
		\begin{enumerate}
			\item	$-\infty$
			\item	$\pi/2$
			\item	0
			\item	$\infty$
			\item	$-5$
			\item	1/3
			\item	0
		\end{enumerate}
	
\end{Answer}
\begin{Answer}{4.1.1}
$-5$, $-2.47106145$, $-2.4067927$, $-2.400676$, $-2.4$
\end{Answer}
\begin{Answer}{4.1.2}
$-4/3$, $-24/7$, $7/24$, $3/4$
\end{Answer}
\begin{Answer}{4.1.3}
$-0.107526881$, $-0.11074197$, $-0.1110741$,
$\ds{-1\over3(3+\Delta x)}\rightarrow {-1\over9}$
\end{Answer}
\begin{Answer}{4.1.4}
$\ds{3+3\Delta x+\Delta x^2\over1+\Delta x}\rightarrow3$
\end{Answer}
\begin{Answer}{4.1.5}
$3.31$, $3.003001$, $3.0000$,\hfill\break
 $3+3\Delta x+\Delta x^2\rightarrow3$
\end{Answer}
\begin{Answer}{4.1.6}
$m$
\end{Answer}
\begin{Answer}{4.1.9}
$10$, $25/2$, $20$, $15$, $25$, $35$.
\end{Answer}
\begin{Answer}{4.1.10}
$5$, $4.1$, $4.01$, $4.001$, $4+\Delta t\rightarrow 4$
\end{Answer}
\begin{Answer}{4.1.11}
$-10.29$, $-9.849$, $-9.8049$, \hfill\break
$-9.8-4.9\Delta t\rightarrow -9.8$
\end{Answer}
\begin{Answer}{4.2.1}
\begin{enumerate}
	\item	$\ds -x/\sqrt{169-x^2}$
	\item	$-9.8t$
	\item	$\ds 2x+1/x^2$
	\item	$2ax+b$
	\item	$\ds 3x^2$
	\item	$\ds -2/(2x+1)^{3/2}$
	\item	$\ds 5/(t+2)^2$
\end{enumerate}
\end{Answer}
\begin{Answer}{4.2.4}
	$y=-13x+17$
\end{Answer}
\begin{Answer}{4.2.5}
	$-8$
\end{Answer}
\begin{Answer}{4.3.1}
\begin{multicols}{3}
\begin{enumerate}
	\item	$\ds 100x^{99}$
	\item	$\ds -100x^{-101}$
	\item	$\ds -5x^{-6}$
	\item	$\ds \pi x^{\pi-1}$
	\item	$\ds (3/4)x^{-1/4}$
	\item	$\ds -(9/7)x^{-16/7}$
	\item	$\ds 15x^2+24x$
	\item	$\ds -20x^4+6x+10/x^3$
	\item	$\ds -30x+25$
	\item	$\ds 3x^2+6x-1$
	\item	$\ds -\frac{x^2+2x+5}{(x^2+2x-3)^2}$
	\item	$\ds 3x^2(x^3-5x+10)+x^3(3x^2-5)$
	\item	$\ds x^4(7x^2+30x-15)$
	\item	$\ds\frac{-3x^2-20x+15}{x^6}$
	\item	$\ds -\frac{3x(5x+8)}{(5x^3+12x^2-15)^2}$
\end{enumerate}
\end{multicols}
\end{Answer}
\begin{Answer}{4.3.2}
$y=13x/4+5$
\end{Answer}
\begin{Answer}{4.3.3}
$\ds y=24x-48-\pi^3$
\end{Answer}
\begin{Answer}{4.3.4}
$-49t/5+5$, $-49/5$
\end{Answer}
\begin{Answer}{4.3.6}
$\ds\sum_{k=1}^n ka_kx^{k-1}$
\end{Answer}
\begin{Answer}{4.3.7}
$\ds x^3/16-3x/4+4$
\end{Answer}
\begin{Answer}{4.3.10}
$f'=4(2x-3)$, $y=4x-7$
\end{Answer}
\begin{Answer}{4.3.12}
$\ds {3x^2\over x^3-5x+10}-{x^3(3x^2-5)\over (x^3-5x+10)^2}$
\end{Answer}
\begin{Answer}{4.3.13}
$\ds {2x+5\over x^5-6x^3+3x^2-7x+1}-{(x^2+5x-3)(5x^4-18x^2+6x-7)\over(x^5-6x^3+3x^2-7x+1)^2}$
\end{Answer}
\begin{Answer}{4.3.14}
$\ds \frac{x-1250}{2(x-625)^{3/2}}$
\end{Answer}
\begin{Answer}{4.3.15}
$\ds \frac{200-39x}{2x^{21}\sqrt{x-5}}$
\end{Answer}
\begin{Answer}{4.3.16}
$\ds y=17x/4-41/4$
\end{Answer}
\begin{Answer}{4.3.17}
$y=11x/16-15/16$
\end{Answer}
\begin{Answer}{4.3.18}
$13/18$
\end{Answer}
\begin{Answer}{4.4.2}
$\ds \pi/6+2n\pi$, $5\pi/6+2n\pi$, any integer $n$
\end{Answer}
\begin{Answer}{4.5.1}
$\ds 4x^3-9x^2+x+7$
\end{Answer}
\begin{Answer}{4.5.2}
$\ds 3x^2-4x+2/\sqrt{x}$
\end{Answer}
\begin{Answer}{4.5.3}
$\ds 6(x^2+1)^2x$
\end{Answer}
\begin{Answer}{4.5.4}
$\ds \sqrt{169-x^2}-x^2/\sqrt{169-x^2}$
\end{Answer}
\begin{Answer}{4.5.5}
$\ds  (2x-4)\sqrt{25-x^2}-$\hfill\break$(x^2-4x+5)x/\sqrt{25-x^2}$
\end{Answer}
\begin{Answer}{4.5.6}
$\ds -x/\sqrt{r^2-x^2}$
\end{Answer}
\begin{Answer}{4.5.7}
$\ds 2x^3/\sqrt{1+x^4}$
\end{Answer}
\begin{Answer}{4.5.8}
$\ds{1\over 4\sqrt{x}(5-\sqrt{x})^{3/2}}$
\end{Answer}
\begin{Answer}{4.5.9}
$\ds  6+18x$
\end{Answer}
\begin{Answer}{4.5.10}
$\ds {2 x + 1\over1 - x }+{x^2  + x + 1\over(1 - x)^2}$
\end{Answer}
\begin{Answer}{4.5.11}
$\ds  -1/\sqrt{25-x^2}-\sqrt{25-x^2}/x^2$
\end{Answer}
\begin{Answer}{4.5.12}
$\ds{1\over2}\left({-169\over x^2}-1\right)\Big/\sqrt{{169\over x}-x}$
\end{Answer}
\begin{Answer}{4.5.13}
$ \ds{3x^2-2x+1/x^2\over 2\sqrt{x^3-x^2-(1/x)}}$
\end{Answer}
\begin{Answer}{4.5.14}
$ \ds{300 x \over(100-x^2)^{5/2}}$
\end{Answer}
\begin{Answer}{4.5.15}
$ \ds{ 1 + 3 x^2\over3(x+x^3)^{2/3}}$
\end{Answer}
\begin{Answer}{4.5.16}
$\ds \left(4x(x^2+1)+{4x^3+4x\over2\sqrt{1+(x^2+1)^2}}\right)\Big/$\hfill\break$2\sqrt{(x^2+1)^2+\sqrt{1+(x^2+1)^2}}$
\end{Answer}
\begin{Answer}{4.5.17}
$\ds 5(x+8)^4$
\end{Answer}
\begin{Answer}{4.5.18}
$\ds -3(4-x)^2$
\end{Answer}
\begin{Answer}{4.5.19}
$\ds 6x(x^2+5)^2$
\end{Answer}
\begin{Answer}{4.5.20}
$\ds -12x(6-2x^2)^2$
\end{Answer}
\begin{Answer}{4.5.21}
$\ds 24x^2(1-4x^3)^{-3}$
\end{Answer}
\begin{Answer}{4.5.22}
$\ds 5+5/x^2$
\end{Answer}
\begin{Answer}{4.5.23}
$\ds -8(4x-1)(2x^2-x+3)^{-3}$
\end{Answer}
\begin{Answer}{4.5.24}
$\ds 1/(x+1)^2$
\end{Answer}
\begin{Answer}{4.5.25}
$\ds 3(8x-2)/(4x^2-2x+1)^2$
\end{Answer}
\begin{Answer}{4.5.26}
$\ds -3x^2+5x-1$
\end{Answer}
\begin{Answer}{4.5.27}
$\ds 6x(2x-4)^3+6(3x^2+1)(2x-4)^2$
\end{Answer}
\begin{Answer}{4.5.28}
$\ds -2/(x-1)^2$
\end{Answer}
\begin{Answer}{4.5.29}
$\ds 4x/(x^2+1)^2$
\end{Answer}
\begin{Answer}{4.5.30}
$\ds (x^2-6x+7)/(x-3)^2$
\end{Answer}
\begin{Answer}{4.5.31}
$\ds -5/(3x-4)^2$
\end{Answer}
\begin{Answer}{4.5.32}
$\ds 60x^4+72x^3+18x^2+18x-6$
\end{Answer}
\begin{Answer}{4.5.33}
$\ds (5-4x)/((2x+1)^2(x-3)^2)$
\end{Answer}
\begin{Answer}{4.5.34}
$\ds 1/(2(2+3x)^2)$
\end{Answer}
\begin{Answer}{4.5.35}
$\ds 56x^6+72x^5+110x^4+100x^3+60x^2+28x+6$
\end{Answer}
\begin{Answer}{4.5.36}
$y=23x/96-29/96$
\end{Answer}
\begin{Answer}{4.5.37}
$y=3-2x/3$
\end{Answer}
\begin{Answer}{4.5.38}
$y=13x/2-23/2$
\end{Answer}
\begin{Answer}{4.5.39}
$y=2x-11$
\end{Answer}
\begin{Answer}{4.5.40}
$\ds y={20+2\sqrt5\over5\sqrt{4+\sqrt5}}\,x+{3\sqrt5\over5\sqrt{4+\sqrt5}}$
\end{Answer}
\begin{Answer}{4.5.41}
	$(f(g(1)))'=20$
\end{Answer}
\begin{Answer}{4.5.42}
	$g'(x)=2x(f(x^2)+x^2f'(x^2))$
\end{Answer}
\begin{Answer}{4.6.1}
	$\ds 2\ln(3)x3^{x^2}$
\end{Answer}
\begin{Answer}{4.6.2}
	$\ds {\cos x-\sin x \over e^x}$
\end{Answer}
\begin{Answer}{4.6.3}
	$\ds 2e^{2x}$
\end{Answer}
\begin{Answer}{4.6.4}
	$\ds e^x\cos(e^x)$
\end{Answer}
\begin{Answer}{4.6.5}
	$\ds  \cos (x) e^{\sin x}$
\end{Answer}
\begin{Answer}{4.6.6}
	$\ds x^{\sin x}\left(\cos x\ln x+{\sin x\over x}\right)$
\end{Answer}
\begin{Answer}{4.6.7}
	$\ds 3x^2e^x+x^3e^x$
\end{Answer}
\begin{Answer}{4.6.8}
	$\ds 1+2^x\ln(2)$
\end{Answer}
\begin{Answer}{4.6.9}
	$\ds -2x\ln(3)(1/3)^{x^2}$
\end{Answer}
\begin{Answer}{4.6.10}
	$\ds e^{4x}(4x-1)/x^2$
\end{Answer}
\begin{Answer}{4.6.11}
	$\ds (3x^2+3)/(x^3+3x)$
\end{Answer}
\begin{Answer}{4.6.12}
	$\ds -\tan(x)$
\end{Answer}
\begin{Answer}{4.6.13}
	$\ds (1-\ln(x^2))/(x^2\sqrt{\ln(x^2)})$
\end{Answer}
\begin{Answer}{4.6.14}
	$\ds \sec(x)$
\end{Answer}
\begin{Answer}{4.6.15}
	$\ds x^{\cos(x)}(\cos(x)/x-\cos(x)\ln(x))$
\end{Answer}
\begin{Answer}{4.6.19}
$e$
\end{Answer}
\begin{Answer}{4.7.1}
\begin{enumerate}
	\item	$\ds x/y$
	\item	$\ds -(2x+y)/(x+2y)$
	\item	$\ds (2xy-3x^2-y^2)/(2xy-3y^2-x^2)$
	\item	$\ds \sin(x)\sin(y)/(\cos(x)\cos(y))$
	\item	$\ds-\sqrt{y}/\sqrt{x}$
	\item	$\ds (y\sec^2(x/y)-y^2)/(x\sec^2(x/y)+y^2)$
	\item	$\ds (y-\cos(x+y))/(\cos(x+y)-x)$
	\item	$\ds -y^2/x^2$
\end{enumerate}
\end{Answer}
\begin{Answer}{4.7.2}
	$1$
\end{Answer}
\begin{Answer}{4.7.3}
	$y=2x\pm6$
\end{Answer}
\begin{Answer}{4.7.4}
	$y=x/2\pm3$
\end{Answer}
\begin{Answer}{4.7.6}
	$\ds (\sqrt3,2\sqrt3)$, $\ds (-\sqrt3,-2\sqrt3)$, $\ds (2\sqrt3,\sqrt3)$,
$\ds (-2\sqrt3,-\sqrt3)$
\end{Answer}
\begin{Answer}{4.7.7}
	$\ds y=7x/\sqrt3-8/\sqrt3$
\end{Answer}
\begin{Answer}{4.7.8}
	$\ds y=(-y_1^{1/3}x+y_1^{1/3}x_1+x_1^{1/3}y_1)/x_1^{1/3}$
\end{Answer}
\begin{Answer}{4.7.9}
	$\ds (y-y_1)/(x-x_1)=(2x_1^3+2x_1y_1^2-x_1)/(2y_1^3+2y_1x_1^2+y_1)$
\end{Answer}
\begin{Answer}{4.8.1}
	$1$
\end{Answer}
\begin{Answer}{4.10.1}
\begin{enumerate}
	\item	$4(2x+3)$
	\item	$\frac{3}{2}x^{1/2}$
\end{enumerate}
\end{Answer}
\begin{Answer}{4.10.2}
	3
\end{Answer}
\begin{Answer}{4.10.3}
\begin{enumerate}
	\item	$28x^3-\dfrac{1}{3\pi x^{4/3}}$
	\item	$-\dfrac{1}{\sqrt{x}(1+\sqrt{x})^2}$
	\item	$f^{\prime}(x)=\left\{
	\begin{array}{ll}
	-2 & \text{if }x<-2 \\
	0 & \text{if }-2<x<1 \\
	2 & \text{if }x>1%
	\end{array}%
	\right. $
	\item	$2x\sin x\cos x+x^2\cos^2 x-x^2\sin^2 x$
	\item	$\dfrac{(\sin x+x\cos x)(1+\sin x)-x\sin x\cos x}{(1+\sin x)^2}$
	\item	$-\dfrac{3}{4x^{3/2}}\left(2+\dfrac{3}{\sqrt{x}}\right)^{-1/2}$
	\item	$\frac{1}{3}(x^4+x^2+1)^{-2/3}(4x^3+2x)-\dfrac{5(3x^2-1)}{(x^3-x+4)^6}$
	\item	$3\sin^2 x\cos x-3x^2\cos(x^3)$
	\item	$4\sec^4 x\tan x + 4\tan^3 x\sec^2 x$
	\item	$\dfrac{4}{(1+x)^2}\cos\left(\dfrac{1-x}{1+x}\right)\sin\left(\dfrac{1-x}{1+x}\right)$
	\item	$(2x+2\sec^2 x\tan x)\sec^2(\sin(x^2+\sec^2 x))\cos(x^2+\sec^2 x)$
	\item	$-\dfrac{\pi\cos\frac{\pi}{x}}{x^2(2+\sin\frac{\pi}{x})^2}$
\end{enumerate}
\end{Answer}
\begin{Answer}{4.10.4}
\begin{enumerate}
	\item	$3e^{3x}-e^{-x}$
	\item	$2e^{2x}\cos 3x-3e^{2x}\sin 3x$
	\item	$(1+e^x)\sec^2(x+e^x)$
	\item	$2e^x/(e^x+2)^2$
	\item	$\dfrac{\cos x}{2+\sin x}-\dfrac{\cos(2+\ln	x)}{x}$
	\item	$e^{x^{\pi}}\cdot\pi x^{\pi-1}+\pi^e x^{\pi^e-1}+\pi^{e^x}\ln\pi\cdot e^x$
	\item	$\log_{a}b+(\log_{a}b) x^{(\log_{a}b)-1}$
	\item	$(x^2+1)^{x^3+1}\left(3x^2\ln(x^2+1)+\dfrac{2x(x^3+1)}{x^2+1}\right)$
	\item	$\dfrac{(x^2+e^x)^{1/\ln x}}{(\ln x)^2}\left(\dfrac{2x+e^x}{x^2+e^x}\ln	x-\dfrac{x^2+e^x}{x}\right)$
	\item	$\dfrac{x\sqrt{x^2+x+1}}{(2+\sin x)^4 (3x+5)^7} \left(\dfrac{1}{x}+\dfrac{2x+1}{2(x^2+x+1)}-\dfrac{4\cos x}{2+\sin x}-\dfrac{21}{3x+5}\right)$
\end{enumerate}
\end{Answer}
\begin{Answer}{4.10.5}
\begin{enumerate}
	\item	$-(2x+y)/(x+2y)$
	\item	$\dfrac{x-(2x^2+2y^2-x) (4x-1)}{4y(2x^2+2y^2-x)-y}$
	\item	$-\dfrac{\sin x+2x\sin y}{x^2\cos y+3y^2}$
	\item	$\dfrac{2x+e^y-e^x}{2-xe^y}$
\end{enumerate}
\end{Answer}
\begin{Answer}{4.10.6}
\begin{enumerate}
	\item	$\sin^{-1}x+x/\sqrt{1-x^2}$
	\item	$\dfrac{\cos^{-1}x+\sin^{-1}x}{(\cos^{-1}x)^2 \sqrt{1-x^2}}$
	\item	$a/(x^2+a^2)$
	\item	$\tan^{-1}x$
\end{enumerate}
\end{Answer}
\begin{Answer}{5.1.2}
	$1/(16\pi)$ cm/s
\end{Answer}
\begin{Answer}{5.1.3}
	$3/(1000\pi)$ meters/second
\end{Answer}
\begin{Answer}{5.1.4}
	$1/4$ m/s
\end{Answer}
\begin{Answer}{5.1.5}
	$-6/25$ m/s
\end{Answer}
\begin{Answer}{5.1.6}
	$80\pi$ mi/min
\end{Answer}
\begin{Answer}{5.1.7}
	$\ds 3\sqrt5$ ft/s
\end{Answer}
\begin{Answer}{5.1.8}
	$20/(3\pi)$ cm/s
\end{Answer}
\begin{Answer}{5.1.9}
	$13/20$ ft/s
\end{Answer}
\begin{Answer}{5.1.10}
	$\ds 5\sqrt{10}/2$ m/s
\end{Answer}
\begin{Answer}{5.1.11}
	$75/64$ m/min
\end{Answer}
\begin{Answer}{5.1.12}
	tip: 6 ft/s, length: $5/2$ ft/s
\end{Answer}
\begin{Answer}{5.1.13}
	tip: $20/11$ m/s, length: $9/11$ m/s
\end{Answer}
\begin{Answer}{5.1.14}
	$\ds 380/\sqrt3-150\approx 69.4$ mph
\end{Answer}
\begin{Answer}{5.1.15}
	$\ds 500/\sqrt3-200\approx 88.7$ km/hr
\end{Answer}
\begin{Answer}{5.1.16}
	$4000/49$ m/s
\end{Answer}
\begin{Answer}{5.2.1}
		min at $x=1/2$
	
\end{Answer}
\begin{Answer}{5.2.2}
		min at $x=-1$, max at $x=1$
	
\end{Answer}
\begin{Answer}{5.2.3}
		max at $x=2$, min at $x=4$
	
\end{Answer}
\begin{Answer}{5.2.4}
		min at $x=\pm 1$, max at $x=0$.
	
\end{Answer}
\begin{Answer}{5.2.5}
		min at $x=1$
	
\end{Answer}
\begin{Answer}{5.2.6}
		none
	
\end{Answer}
\begin{Answer}{5.2.7}
		none
	
\end{Answer}
\begin{Answer}{5.2.8}
		min at $x=7\pi/12+k\pi$, max at $x=-\pi/12+k\pi$, for integer $k$.
	
\end{Answer}
\begin{Answer}{5.2.9}
		local min at $x=49$
	
\end{Answer}
\begin{Answer}{5.2.12}
		one
	
\end{Answer}
\begin{Answer}{5.2.16}
		Absolute maximum $(3,7)$; Absolute minimum $(0,1)$.
	
\end{Answer}
\begin{Answer}{5.2.17}
		Absolute maximum $(3,7)$; Absolute minimum $(0,1)$.
	
\end{Answer}
\begin{Answer}{5.2.18}
		Absolute minimum $(\pi/2,1)$; No absolute maximum.
	
\end{Answer}
\begin{Answer}{5.2.19}
		Absolute minimum $(1,0)$; Absolute maximum $(e^{1/2},\frac{1}{2e})$.
	
\end{Answer}
\begin{Answer}{5.2.20}
		Absolute minimum $(1,0)$; Absolute maximum $(e^{1/2},\frac{1}{2e})$.
	
\end{Answer}
\begin{Answer}{5.2.21}
		Absolute minimum $(0,0)$; Absolute maximum $(2,2e^{1/8})$.
	
\end{Answer}
\begin{Answer}{5.2.22}
		Absolute minimum $(1/2,\frac{2-\pi}{4})$; Absolute maximum $(2,2-\tan^{-1}(4))$.
	
\end{Answer}
\begin{Answer}{5.2.23}
		Absolute maximum $(1,1/2)$; Absolute minimum $(-1,-1/2)$.
	
\end{Answer}
\begin{Answer}{5.3.1}
 $c=1/2$
\end{Answer}
\begin{Answer}{5.3.2}
 $\ds c=\sqrt{18}-2$
\end{Answer}
\begin{Answer}{5.3.6}
 $\ds x^3/3+47x^2/2-5x+k$
\end{Answer}
\begin{Answer}{5.3.7}
 $\arctan x + k$
\end{Answer}
\begin{Answer}{5.3.8}
 $\ds x^4/4 -\ln x +k$
\end{Answer}
\begin{Answer}{5.3.9}
 $-\cos(2x)/2 +k$
\end{Answer}
\begin{Answer}{5.4.1}
$L(x)=x$, $f(0.1)\approx L(0.1)=0.1$
\end{Answer}
\begin{Answer}{5.4.2}
Choose $f(x)=x^3$ and $a=2$, the closest integer to 1.9. The
linearization of $f$ at $a$ is $L(x)=12(x-2)+8$, and $(1.9)^3=f(1.9)\approx L(1.9)=12(1.9-2)+8=6.8$.
\end{Answer}
\begin{Answer}{5.4.4}
Choose $a=7$ since $f(7)=\sqrt[3]{7+1}=\sqrt[3]{8}=2$ is an integer
close to $\sqrt[3]{9}$. The linearization of $f$ at $a=7$ is
$L(x)=\sfrac{1}{12}(x-7)+2$. Then $f(8)=\sqrt[3]{8+1}=\sqrt[3]{9}\approx L(8)=\sfrac{1}{12}(8-7)+2=2.08\bar{3}$. We are over-estimating $\sqrt[3]{9}$ since $L(x)>f(x)$ for all $x$ around $a=7$.
\end{Answer}
\begin{Answer}{5.4.5}
	$\Delta y=65/16$, $dy=2$
\end{Answer}
\begin{Answer}{5.4.6}
	$\ds \Delta y=\sqrt{11/10}-1$, $dy=0.05$
\end{Answer}
\begin{Answer}{5.4.7}
	$\ds \Delta y=\sin(\pi/50)$, $dy=\pi/50$
\end{Answer}
\begin{Answer}{5.4.8}
	$dV=8\pi/25$
\end{Answer}
\begin{Answer}{5.4.9}
$T_5(x)=x-\frac{x^3}{3!}+\frac{x^5}{5!}$
\begin{enumerate}
	\item	$\sin (0.1)\approx T_5(0.1)\approx 0.10016675$
	\item	$\sin (0.1)=0.0998334\ldots$ using a calculator. Our approximation is accurate to $0.10016675-0.0998334\ldots =0.000\bar{3}$.
\end{enumerate}
\end{Answer}
\begin{Answer}{5.4.11}
	$T_3(x)=x+x^2+x^3$. The point $x=5$ is not close to $x=0$, and $f$ is not continuous at $x=1$.
\end{Answer}
\begin{Answer}{5.4.12}
\begin{enumerate}
	\item	$f^{(n)}(x)=\frac{(-1)^{(n-1)}(n-1)!}{x^n}$
	\item	$T_n(x)=\ln (1)+\displaystyle\sum_{i=1}^{n} \frac{\big(\frac{(-1)^{(i-1)}(i-1)!}{1^n}\big)}{i!}(x-1)^i=\displaystyle\sum_{i=1}^{n} \bigg(\frac{(-1)^{(i-1)}(i-1)!}{i!}\bigg)(x-1)^i$ since $\ln (1)=0$ and $1^n=1$.
\end{enumerate}
\end{Answer}
\begin{Answer}{5.4.13}
	Notice that $f(-2)=19$, $f(0)=-11$, and $f(5)=19$ and $f$ is a continuous function. By the
	Intermediate Value Theorem there exists a root in $[-2,0]$ and $[0,5]$. Choose $x_0=0$, then $x_4\approx -0.93242$.
	Choose $x_0=5$, then $x_4\approx 3.93242$.
\end{Answer}
\begin{Answer}{5.4.14}
\begin{enumerate}
	\item	$x_4\approx 1.00022\ldots$
	\item	$x=1$ is the root of $f$. Our approximation in part (a) was correct to 3 decimal places.
	\item	$x_1=1$. The root is found in one iteration of Newton's Method.
\end{enumerate}
\end{Answer}
\begin{Answer}{5.4.15}
	$\cos (\pi/2)=0$, so $x_1$ is undefined.
\end{Answer}
\begin{Answer}{5.5.1}
 $0$
\end{Answer}
\begin{Answer}{5.5.2}
 $\infty$
\end{Answer}
\begin{Answer}{5.5.3}
 $0$
\end{Answer}
\begin{Answer}{5.5.4}
 $0$
\end{Answer}
\begin{Answer}{5.5.5}
 $1/6$
\end{Answer}
\begin{Answer}{5.5.6}
 $1/16$
\end{Answer}
\begin{Answer}{5.5.7}
 $3/2$
\end{Answer}
\begin{Answer}{5.5.8}
 $-1/4$
\end{Answer}
\begin{Answer}{5.5.9}
 $-3$
\end{Answer}
\begin{Answer}{5.5.10}
 $1/2$
\end{Answer}
\begin{Answer}{5.5.11}
 $0$
\end{Answer}
\begin{Answer}{5.5.12}
 $-1$
\end{Answer}
\begin{Answer}{5.5.13}
 $-1/2$
\end{Answer}
\begin{Answer}{5.5.14}
 $5$
\end{Answer}
\begin{Answer}{5.5.15}
 $1$
\end{Answer}
\begin{Answer}{5.5.16}
 $1$
\end{Answer}
\begin{Answer}{5.5.17}
 $2$
\end{Answer}
\begin{Answer}{5.5.18}
 $1$
\end{Answer}
\begin{Answer}{5.5.19}
 $0$
\end{Answer}
\begin{Answer}{5.5.20}
 $1/2$
\end{Answer}
\begin{Answer}{5.5.21}
 $2$
\end{Answer}
\begin{Answer}{5.5.22}
 $0$
\end{Answer}
\begin{Answer}{5.5.23}
 $1/2$
\end{Answer}
\begin{Answer}{5.5.24}
 $-1/2$
\end{Answer}
\begin{Answer}{5.5.25}
 $2$
\end{Answer}
\begin{Answer}{5.5.26}
 $0$
\end{Answer}
\begin{Answer}{5.5.27}
 $\infty$
\end{Answer}
\begin{Answer}{5.5.28}
 $0$
\end{Answer}
\begin{Answer}{5.5.29}
 $5$
\end{Answer}
\begin{Answer}{5.5.30}
 $-1/2$
\end{Answer}
\begin{Answer}{5.5.31}
  {$-1$}
\end{Answer}
\begin{Answer}{5.5.32}
 {$-\sqrt{2}/2$}
\end{Answer}
\begin{Answer}{5.5.33}
  {$5$}
\end{Answer}
\begin{Answer}{5.5.34}
 {$\infty$}
\end{Answer}
\begin{Answer}{5.5.35}
  {$1/2$}
\end{Answer}
\begin{Answer}{5.5.36}
  {$0$}
\end{Answer}
\begin{Answer}{5.5.37}
  {$0$}
\end{Answer}
\begin{Answer}{5.5.38}
 {$\infty$}
\end{Answer}
\begin{Answer}{5.5.39}
 {$\infty$}
\end{Answer}
\begin{Answer}{5.5.40}
 {$0$}
\end{Answer}
\begin{Answer}{5.5.41}
 {$1$}
\end{Answer}
\begin{Answer}{5.5.42}
 {$1$}
\end{Answer}
\begin{Answer}{5.5.43}
 {$1$}
\end{Answer}
\begin{Answer}{5.5.44}
 {$1$}
\end{Answer}
\begin{Answer}{5.5.45}
{$1$}
\end{Answer}
\begin{Answer}{5.5.46}
 {$1$}
\end{Answer}
\begin{Answer}{5.5.47}
  {$1$}
\end{Answer}
\begin{Answer}{5.5.48}
 {$2$}
\end{Answer}
\begin{Answer}{5.5.49}
{$1/2$}
\end{Answer}
\begin{Answer}{5.5.50}
{$-\infty$}
\end{Answer}
\begin{Answer}{5.5.51}
{$1$}
\end{Answer}
\begin{Answer}{5.5.52}
{$0$}
\end{Answer}
\begin{Answer}{5.5.53}
  {$3$}
\end{Answer}
\begin{Answer}{5.5.54}
 {$\infty$}
\end{Answer}
\begin{Answer}{5.6.1}
 min at $x=1/2$
\end{Answer}
\begin{Answer}{5.6.2}
 min at $x=-1$, max at $x=1$
\end{Answer}
\begin{Answer}{5.6.3}
 max at $x=2$, min at $x=4$
\end{Answer}
\begin{Answer}{5.6.4}
 min at $x=\pm 1$, max at $x=0$.
\end{Answer}
\begin{Answer}{5.6.5}
 min at $x=1$
\end{Answer}
\begin{Answer}{5.6.6}
 none
\end{Answer}
\begin{Answer}{5.6.7}
 none
\end{Answer}
\begin{Answer}{5.6.8}
 min at $x=7\pi/12+k\pi$, max at $x=-\pi/12+k\pi$, for integer $k$.
\end{Answer}
\begin{Answer}{5.6.9}
 none
\end{Answer}
\begin{Answer}{5.6.10}
 max at $x=0$, min at $x=\pm 11$
\end{Answer}
\begin{Answer}{5.6.11}
 min at $x=-3/2$, neither at $x=0$
\end{Answer}
\begin{Answer}{5.6.12}
 min at $n\pi$, max at $\pi/2+n\pi$
\end{Answer}
\begin{Answer}{5.6.13}
 min at $2n\pi$, max at $(2n+1)\pi$
\end{Answer}
\begin{Answer}{5.6.14}
 min at $\pi/2+2n\pi$, max at $3\pi/2+2n\pi$
\end{Answer}
\begin{Answer}{5.6.17}
 min at $x=1/2$
\end{Answer}
\begin{Answer}{5.6.18}
 min at $x=-1$, max at $x=1$
\end{Answer}
\begin{Answer}{5.6.19}
 max at $x=2$, min at $x=4$
\end{Answer}
\begin{Answer}{5.6.20}
 min at $x=\pm 1$, max at $x=0$.
\end{Answer}
\begin{Answer}{5.6.21}
 min at $x=1$
\end{Answer}
\begin{Answer}{5.6.22}
 none
\end{Answer}
\begin{Answer}{5.6.23}
 none
\end{Answer}
\begin{Answer}{5.6.24}
 min at $x=7\pi/12+n\pi$, max at $x=-\pi/12+n\pi$, for integer $n$.
\end{Answer}
\begin{Answer}{5.6.25}
 max at $x=63/64$
\end{Answer}
\begin{Answer}{5.6.26}
 max at $x=7$
\end{Answer}
\begin{Answer}{5.6.27}
 max at $\ds -5^{-1/4}$, min at $\ds 5^{-1/4}$
\end{Answer}
\begin{Answer}{5.6.28}
 none
\end{Answer}
\begin{Answer}{5.6.29}
 max at $-1$, min at $1$
\end{Answer}
\begin{Answer}{5.6.30}
 min at $\ds 2^{-1/3}$
\end{Answer}
\begin{Answer}{5.6.31}
 none
\end{Answer}
\begin{Answer}{5.6.32}
 min at $n\pi$
\end{Answer}
\begin{Answer}{5.6.33}
 max at $n\pi$, min at $\pi/2+n\pi$
\end{Answer}
\begin{Answer}{5.6.34}
 max at $\pi/2+2n\pi$, min at $3\pi/2+2n\pi$
\end{Answer}
\begin{Answer}{5.6.35}
 concave up everywhere
\end{Answer}
\begin{Answer}{5.6.36}
 concave up when $x<0$, concave down when $x>0$
\end{Answer}
\begin{Answer}{5.6.37}
 concave down when $x<3$, concave up when $x>3$
\end{Answer}
\begin{Answer}{5.6.38}
 concave up when $\ds x<-1/\sqrt3$ or $\ds x>1/\sqrt3$,
concave down when $\ds -1/\sqrt3<x<1/\sqrt3$
\end{Answer}
\begin{Answer}{5.6.39}
 concave up when $x<0$ or $x>2/3$,
concave down when $0<x<2/3$
\end{Answer}
\begin{Answer}{5.6.40}
 concave up when $x<0$, concave down when $x>0$
\end{Answer}
\begin{Answer}{5.6.41}
 concave up when $x<-1$ or $x>1$, concave down when
$-1<x<0$ or $0<x<1$
\end{Answer}
\begin{Answer}{5.6.42}
 concave down on $((8n-1)\pi/4,(8n+3)\pi/4)$,
concave up on $((8n+3)\pi/4,(8n+7)\pi/4)$, for integer $n$
\end{Answer}
\begin{Answer}{5.6.43}
 concave down everywhere
\end{Answer}
\begin{Answer}{5.6.44}
 concave up on $\ds (-\infty,(21-\sqrt{497})/4)$ and
$\ds (21+\sqrt{497})/4,\infty)$
\end{Answer}
\begin{Answer}{5.6.45}
 concave up on $(0,\infty)$
\end{Answer}
\begin{Answer}{5.6.46}
 concave down on $(2n\pi/3,(2n+1)\pi/3)$
\end{Answer}
\begin{Answer}{5.6.47}
 concave up on $(0,\infty)$
\end{Answer}
\begin{Answer}{5.6.48}
 concave up on $(-\infty,-1)$ and $(0,\infty)$
\end{Answer}
\begin{Answer}{5.6.49}
 concave down everywhere
\end{Answer}
\begin{Answer}{5.6.50}
 concave up everywhere
\end{Answer}
\begin{Answer}{5.6.51}
 concave up on $(\pi/4+n\pi,3\pi/4+n\pi)$
\end{Answer}
\begin{Answer}{5.6.52}
 inflection points at $n\pi$, $\ds \pm\arcsin(\sqrt{2/3})+n\pi$
\end{Answer}
\begin{Answer}{5.6.53}
 up/incr: $(3,\infty)$, up/decr: $(-\infty,0)$, $(2,3)$,
down/decr: $(0,2)$
\end{Answer}
\begin{Answer}{5.7.1}
 $25\times 25$
\end{Answer}
\begin{Answer}{5.7.2}
 $P/4\times P/4$
\end{Answer}
\begin{Answer}{5.7.3}
 $\ds w=l=2\cdot 5^{2/3}$, $\ds h=5^{2/3}$, $\ds h/w=1/2$
\end{Answer}
\begin{Answer}{5.7.4}
 $\ds \root 3\of {100}\times\root 3\of {100}\times 2\root 3\of
{100}$, $h/s=2$
\end{Answer}
\begin{Answer}{5.7.5}
 $\ds w=l=2^{1/3}V^{1/3}$, $\ds h=V^{1/3}/2^{2/3}$, $h/w=1/2$
\end{Answer}
\begin{Answer}{5.7.6}
 $1250$ square feet
\end{Answer}
\begin{Answer}{5.7.7}
 $\ds l^2/8$ square feet
\end{Answer}
\begin{Answer}{5.7.8}
 \$5000
\end{Answer}
\begin{Answer}{5.7.9}
 $100$
\end{Answer}
\begin{Answer}{5.7.10}
 $\ds r^2$
\end{Answer}
\begin{Answer}{5.7.11}
 $h/r=2$
\end{Answer}
\begin{Answer}{5.7.12}
 $h/r=2$
\end{Answer}
\begin{Answer}{5.7.13}
 $r=5$, $h=40/\pi$, $h/r=8/\pi$
\end{Answer}
\begin{Answer}{5.7.14}
 $8/\pi$
\end{Answer}
\begin{Answer}{5.7.15}
 $4/27$
\end{Answer}
\begin{Answer}{5.7.16}
 (a) 2, (b) $7/2$
\end{Answer}
\begin{Answer}{5.7.17}
 $\ds{\sqrt3\over6}\times{\sqrt3\over6}+{1\over2}\times
{1\over4}-{\sqrt3\over 12}$
\end{Answer}
\begin{Answer}{5.7.18}
 (a) $a/6$, (b) $\ds (a+b-\sqrt{a^2-ab+b^2})/6$
\end{Answer}
\begin{Answer}{5.7.19}
 $1.5$ meters wide by $1.25$ meters tall
\end{Answer}
\begin{Answer}{5.7.20}
 If $k\le 2/\pi$ the ratio is $(2-k\pi)/4$; if $k\ge 2/\pi$,
the ratio is zero: the window should be semicircular with no
rectangular part.
\end{Answer}
\begin{Answer}{5.7.21}
 $a/b$
\end{Answer}
\begin{Answer}{5.7.22}
 $\ds 1/\sqrt3\approx 58\%$
\end{Answer}
\begin{Answer}{5.7.23}
 $18\times18\times36$
\end{Answer}
\begin{Answer}{5.7.24}
 $\ds r=5/(2\pi)^{1/3}\approx 2.7\hbox{ cm}$,\hfill\break
$\ds h=5\cdot2^{5/3}/\pi^{1/3}=4r\approx 10.8 \hbox{ cm}$
\end{Answer}
\begin{Answer}{5.7.25}
 $\ds h={750\over\pi}\left({2\pi^2\over 750^2}\right)^{1/3}$,
$\ds r=\left({750^2\over 2\pi^2}\right)^{1/6}$
\end{Answer}
\begin{Answer}{5.7.26}
 $\ds h/r=\sqrt2$
\end{Answer}
\begin{Answer}{5.7.27}
 $1/2$
\end{Answer}
\begin{Answer}{5.7.28}
 \$7000
\end{Answer}
\begin{Answer}{6.2.1}
 10
\end{Answer}
\begin{Answer}{6.2.2}
 $35/3$
\end{Answer}
\begin{Answer}{6.2.3}
 $\ds x^2$
\end{Answer}
\begin{Answer}{6.2.4}
 $\ds 2x^2$
\end{Answer}
\begin{Answer}{6.2.5}
 $\ds 2x^2-8$
\end{Answer}
\begin{Answer}{6.2.6}
 $\ds 2b^2-2a^2$
\end{Answer}
\begin{Answer}{6.2.7}
 4 rectangles: $41/4=10.25$,
8 rectangles: $183/16= 11.4375$
\end{Answer}
\begin{Answer}{6.2.8}
 $ 23/4$
\end{Answer}
\begin{Answer}{6.3.1}
 $87/2$
\end{Answer}
\begin{Answer}{6.3.2}
 $2$
\end{Answer}
\begin{Answer}{6.3.3}
 $\ln(10)$
\end{Answer}
\begin{Answer}{6.3.4}
 $\ds e^5-1$
\end{Answer}
\begin{Answer}{6.3.5}
 $\ds 3^4/4$
\end{Answer}
\begin{Answer}{6.3.6}
 $\ds 2^6/6 -1/6$
\end{Answer}
\begin{Answer}{6.3.7}
 $\ds x^2-3x$
\end{Answer}
\begin{Answer}{6.3.8}
 $\ds 2x(x^4-3x^2)$
\end{Answer}
\begin{Answer}{6.3.9}
 $\ds e^{x^2}$
\end{Answer}
\begin{Answer}{6.3.10}
 $\ds 2xe^{x^4}$
\end{Answer}
\begin{Answer}{6.3.11}
 $\ds \tan(x^2)$
\end{Answer}
\begin{Answer}{6.3.12}
 $\ds 2x\tan(x^4)-10\tan(100x^2)$
\end{Answer}
\begin{Answer}{6.3.13}
	31, 14
\end{Answer}
\begin{Answer}{6.3.14}
	5
\end{Answer}
\begin{Answer}{6.3.15}
\begin{enumerate}
	\item	2/3
	\item	24/5
\end{enumerate}
\end{Answer}
\begin{Answer}{6.4.1}
 $\ds (16/3)x^{3/2}+C$
\end{Answer}
\begin{Answer}{6.4.2}
 $\ds t^3+t+C$
\end{Answer}
\begin{Answer}{6.4.3}
 $\ds 8\sqrt{x}+C$
\end{Answer}
\begin{Answer}{6.4.4}
 $-2/z+C$
\end{Answer}
\begin{Answer}{6.4.5}
 $7\ln s+C$
\end{Answer}
\begin{Answer}{6.4.6}
 $\ds (5x+1)^3/15+C$
\end{Answer}
\begin{Answer}{6.4.7}
 $\ds (x-6)^3/3+C$
\end{Answer}
\begin{Answer}{6.4.8}
 $\ds 2x^{5/2}/5+C$
\end{Answer}
\begin{Answer}{6.4.9}
 $\ds -4/\sqrt{x}+C$
\end{Answer}
\begin{Answer}{6.4.10}
 $\ds 4t-t^2+C$, $t<2$; $\ds t^2-4t+8+C$, $t\ge 2$
\end{Answer}
\begin{Answer}{7.1.1}
 $\ds -(1-t)^{10}/10+C$
\end{Answer}
\begin{Answer}{7.1.2}
 $\ds x^5/5+2x^3/3+x+C$
\end{Answer}
\begin{Answer}{7.1.3}
 $\ds (x^2+1)^{101}/202+C$
\end{Answer}
\begin{Answer}{7.1.4}
 $\ds -3(1-5t)^{2/3}/10+C$
\end{Answer}
\begin{Answer}{7.1.5}
 $\ds (\sin^4x)/4+C$
\end{Answer}
\begin{Answer}{7.1.6}
 $\ds -(100-x^2)^{3/2}/3+C$
\end{Answer}
\begin{Answer}{7.1.7}
 $\ds \ds -2\sqrt{1-x^3}/3+C$
\end{Answer}
\begin{Answer}{7.1.8}
 $\ds \sin(\sin\pi t)/\pi+C$
\end{Answer}
\begin{Answer}{7.1.9}
 $\ds \ds 1/(2\cos^2 x)=(1/2)\sec^2x+C$
\end{Answer}
\begin{Answer}{7.1.10}
 $-\ln|\cos x|+C$
\end{Answer}
\begin{Answer}{7.1.11}
 $0$
\end{Answer}
\begin{Answer}{7.1.12}
 $\ds \tan^2(x)/2+C$
\end{Answer}
\begin{Answer}{7.1.13}
 $1/4$
\end{Answer}
\begin{Answer}{7.1.14}
 $-\cos(\tan x)+C$
\end{Answer}
\begin{Answer}{7.1.15}
 $1/10$
\end{Answer}
\begin{Answer}{7.1.16}
 $\ds \sqrt3/4$
\end{Answer}
\begin{Answer}{7.1.17}
 $\ds (27/8)(x^2-7)^{8/9}$
\end{Answer}
\begin{Answer}{7.1.18}
 $\ds -(3^7+1)/14$
\end{Answer}
\begin{Answer}{7.1.19}
 $0$
\end{Answer}
\begin{Answer}{7.1.20}
 $\ds f(x)^2/2$
\end{Answer}
\begin{Answer}{7.2.1}
 $x/2-\sin(2x)/4+C$
\end{Answer}
\begin{Answer}{7.2.2}
 $\ds -\cos x+(\cos^3x)/3+C$
\end{Answer}
\begin{Answer}{7.2.3}
 $3x/8-(\sin 2x)/4+(\sin 4x)/32+C$
\end{Answer}
\begin{Answer}{7.2.4}
 $\ds (\cos^5 x)/5-(\cos^3x)/3+C$
\end{Answer}
\begin{Answer}{7.2.5}
 $\ds \sin x-(\sin^3x)/3+C$
\end{Answer}
\begin{Answer}{7.2.6}
 $\ds (\sin^3x)/3-(\sin^5x)/5+C$
\end{Answer}
\begin{Answer}{7.2.7}
 $\ds -2(\cos x)^{5/2}/5+C$
\end{Answer}
\begin{Answer}{7.2.8}
 $\tan x-\cot x+C$
\end{Answer}
\begin{Answer}{7.2.9}
 $\ds (\sec^3x)/3-\sec x+C$
\end{Answer}
\begin{Answer}{7.2.10}
 $\ds -\cos x+\sin x+C$
\end{Answer}
\begin{Answer}{7.2.11}
 $\ds \frac{3}{2}\ln|\sec x+\tan x|+\tan x+\frac{1}{2}\sec x\tan x+C$
\end{Answer}
\begin{Answer}{7.2.12}
 $\ds \frac{\tan^5(x^2)}{10}+C$
\end{Answer}
\begin{Answer}{7.3.1}
 $\ds x\sqrt{x^2-1}/2-\ln|x+\sqrt{x^2-1}|/2+C$
\end{Answer}
\begin{Answer}{7.3.2}
 $\ds x\sqrt{9+4x^2}/2+\hbox{$\ds(9/4)\ln|2x+\sqrt{9+4x^2}|+C$}$
\end{Answer}
\begin{Answer}{7.3.3}
 $\ds -(1-x^2)^{3/2}/3+C$
\end{Answer}
\begin{Answer}{7.3.4}
 $\arcsin(x)/8-\sin(4\arcsin x)/32+C$
\end{Answer}
\begin{Answer}{7.3.5}
 $\ds \ln|x+\sqrt{1+x^2}|+C$
\end{Answer}
\begin{Answer}{7.3.6}
 $\ds (x+1)\sqrt{x^2+2x}/2-\hbox{$\ds\ln|x+1+\sqrt{x^2+2x}|/2+C$}$
\end{Answer}
\begin{Answer}{7.3.7}
 $-\arctan x - 1/x+C$
\end{Answer}
\begin{Answer}{7.3.8}
 $\ds 2\arcsin(x/2)-x\sqrt{4-x^2}/2+C$
\end{Answer}
\begin{Answer}{7.3.9}
 $\ds \arcsin(\sqrt{x})-\sqrt{x}\sqrt{1-x}+C$
\end{Answer}
\begin{Answer}{7.3.10}
 $\ds (2x^2+1)\sqrt{4x^2-1}/24+C$
\end{Answer}
\begin{Answer}{7.4.1}
 $\cos x+x\sin x+C$
\end{Answer}
\begin{Answer}{7.4.2}
 $\ds x^2\sin x-2 \sin x+2x\cos x +C$
\end{Answer}
\begin{Answer}{7.4.3}
 $\ds (x-1)e^x +C$
\end{Answer}
\begin{Answer}{7.4.4}
 $\ds (1/2)e^{x^2} +C$
\end{Answer}
\begin{Answer}{7.4.5}
 $(x/2)-\sin(2x)/4 +C=$\hfill\break$(x/2)-(\sin x\cos x)/2+C$
\end{Answer}
\begin{Answer}{7.4.6}
 $x\ln x-x +C$
\end{Answer}
\begin{Answer}{7.4.7}
 $\ds (x^2\arctan x +\arctan x -x)/2+C$
\end{Answer}
\begin{Answer}{7.4.8}
 $\ds -x^3\cos x+3x^2\sin x+6x\cos x-6\sin x+C$
\end{Answer}
\begin{Answer}{7.4.9}
 $\ds x^3\sin x+3x^2\cos x-6x\sin x-6\cos x+C$
\end{Answer}
\begin{Answer}{7.4.10}
 $\ds x^2/4-(\cos^2 x)/4-(x\sin x\cos x)/2+C$
\end{Answer}
\begin{Answer}{7.4.11}
 $\ds x/4-(x\cos^2 x)/2+(\cos x\sin x)/4+C$
\end{Answer}
\begin{Answer}{7.4.12}
 $x\arctan(\sqrt x)+\arctan(\sqrt x)-\sqrt{x}+C$
\end{Answer}
\begin{Answer}{7.4.13}
 $2\sin(\sqrt x)-2\sqrt x\cos(\sqrt x)+C$
\end{Answer}
\begin{Answer}{7.4.14}
 $\sec x\csc x-2\cot x+C$
\end{Answer}
\begin{Answer}{7.5.1}
 $-\ln|x-2|/4+\ln|x+2|/4+C$
\end{Answer}
\begin{Answer}{7.5.2}
 $\ds -x^3/3-4x-4\ln|x-2|+$\hfill\break$4\ln|x+2| +C$
\end{Answer}
\begin{Answer}{7.5.3}
 $-1/(x+5) +C$
\end{Answer}
\begin{Answer}{7.5.4}
 $-x-\ln|x-2|+\ln|x+2| +C$
\end{Answer}
\begin{Answer}{7.5.5}
 $\ds -4x+x^3/3+8\arctan(x/2) +C$
\end{Answer}
\begin{Answer}{7.5.6}
 $(1/2)\arctan(x/2+5/2) +C$
\end{Answer}
\begin{Answer}{7.5.7}
 $\ds x^2/2-2\ln(4+x^2) +C$
\end{Answer}
\begin{Answer}{7.5.8}
 $(1/4)\ln|x+3|-(1/4)\ln|x+7| +C$
\end{Answer}
\begin{Answer}{7.5.9}
 $(1/5)\ln|2x-3|-(1/5)\ln|1+x| +C$
\end{Answer}
\begin{Answer}{7.5.10}
 $(1/3)\ln|x|-(1/3)\ln|x+3| +C$
\end{Answer}
\begin{Answer}{7.6.1}
\begin{enumerate}
\item {\begin{enumerate}
\item		Area is $30.8667$ cm$^2$.
\item		Area is $308,667$ m$^2$.
\end{enumerate}
}

\item {\begin{enumerate}
\item		Area is 25.0667 cm$^2$
\item		Area is 250,667 yd$^2$
\end{enumerate}
}

\end{enumerate}
\end{Answer}
\begin{Answer}{7.6.2}
\begin{enumerate}
\item {\begin{enumerate}
\item		$3/4$
\item		$2/3$
\item		$2/3$
\end{enumerate}
}
\item {\begin{enumerate}
\item		$250$
\item		$250$
\item		$250$
\end{enumerate}
}
\item {\begin{enumerate}
\item		$\frac14(1+\sqrt2)\pi\approx 1.896$
\item		$\frac1{6}(1+2\sqrt{2})\pi\approx 2.005$
\item		$2$
\end{enumerate}
}
\item {\begin{enumerate}
\item		$2+\sqrt2+\sqrt3\approx 5.15$
\item		$2/3(3+\sqrt2+2\sqrt3)\approx 5.25$
\item		$16/3\approx 5.33$
\end{enumerate}
}
\item {\begin{enumerate}
\item		$38.5781$
\item		$147/4\approx 36.75$
\item		$147/4\approx 36.75$
\end{enumerate}
}
\item {\begin{enumerate}
\item		$0.2207$
\item		$0.2005$
\item		$1/5$
\end{enumerate}
}
\item {\begin{enumerate}
\item		$0$
\item		$0$
\item		$0$
\end{enumerate}
}
\item {\begin{enumerate}
\item		$9/2(1+\sqrt3)\approx 12.294$
\item		$3+6\sqrt3\approx 13.392$
\item		$9\pi/2\approx 14.137$
\end{enumerate}
}
\end{enumerate}
\end{Answer}
\begin{Answer}{7.6.3}
\begin{enumerate}
\item {Trapezoidal Rule: 	$0.9006$

Simpson's Rule: $0.90452$
}
\item {Trapezoidal Rule: 	$3.0241$

Simpson's Rule: $2.9315$
}
\item {Trapezoidal Rule: 	$13.9604$

Simpson's Rule: $13.9066$
}
\item {Trapezoidal Rule: 	$3.0695$

Simpson's Rule: $3.14295$
}
\item
{Trapezoidal Rule: 	$1.1703$

Simpson's Rule: $1.1873$
}
\item {Trapezoidal Rule: 	$2.52971$

Simpson's Rule: $2.5447$
}
\item {Trapezoidal Rule: 	$1.0803$

Simpson's Rule: $1.077$
}
\item {Trapezoidal Rule: 	$3.5472$

Simpson's Rule: $3.6133$
}
\end{enumerate}
\end{Answer}
\begin{Answer}{7.6.4}
\begin{enumerate}
\item {\begin{enumerate}
\item		$n=161$ (using $\max\big(\fpp(x)\big)=1$)
\item		$n=12$	(using $\max\big(f\,^{(4)}(x)\big)=1$)
\end{enumerate}
}
\item {\begin{enumerate}
\item		$n=150$ (using $\max\big(\fpp(x)\big)=1$)
\item		$n=18$	(using $\max\big(f\,^{(4)}(x)\big)=7$)
\end{enumerate}
}
\item {\begin{enumerate}
\item		$n=1004$ (using $\max\big(\fpp(x)\big)=39$)
\item		$n=62$	(using $\max\big(f\,^{(4)}(x)\big)=800$)
\end{enumerate}
}
\item {\begin{enumerate}
\item		$n=5591$ (using $\max\big(\fpp(x)\big)=300$)
\item		$n=46$	(using $\max\big(f\,^{(4)}(x)\big)=24$)
\end{enumerate}
}
\end{enumerate}
\end{Answer}
\begin{Answer}{7.6.5}
 T,S: $4\pm0$
\end{Answer}
\begin{Answer}{7.6.6}
 T: $9.28125\pm0.281125 $; S: $9\pm0$
\end{Answer}
\begin{Answer}{7.6.7}
 T: $60.75\pm1$; S: $60\pm0$
\end{Answer}
\begin{Answer}{7.6.8}
 T: $1.1167\pm 0.0833$; S: $1.1000\pm 0.0167$
\end{Answer}
\begin{Answer}{7.6.9}
 T: $0.3235\pm 0.0026$; S: $0.3217\pm 0.000065$
\end{Answer}
\begin{Answer}{7.6.10}
 T: $0.6478\pm 0.0052$; S: $0.6438\pm 0.000033$
\end{Answer}
\begin{Answer}{7.6.11}
 T: $2.8833\pm 0.0834$; S: $2.9000\pm 0.0167$
\end{Answer}
\begin{Answer}{7.6.12}
 T: $1.1170\pm 0.0077$; S: $1.1114\pm 0.0002$
\end{Answer}
\begin{Answer}{7.6.13}
 T: $1.097\pm 0.0147$; S: $1.089\pm 0.0003$
\end{Answer}
\begin{Answer}{7.6.14}
 T: $3.63\pm 0.087$; S: $3.62\pm 0.032$
\end{Answer}
\begin{Answer}{7.7.1}
		Converges to 1.
	
\end{Answer}
\begin{Answer}{7.7.2}
		Diverges.
	
\end{Answer}
\begin{Answer}{7.7.3}
		1/3
	
\end{Answer}
\begin{Answer}{7.7.4}
		Divergent.
	
\end{Answer}
\begin{Answer}{7.7.7}
		\begin{enumerate}
			\item	$\pi/2$
			\item	divergent (to $\infty$)
			\item	1
			\item	divergent (to $\infty$)
			\item	$\frac{5}{3}(4^{3/5})$
		\end{enumerate}
	
\end{Answer}
\begin{Answer}{7.7.9}
		$0<p<1$
	
\end{Answer}
\begin{Answer}{7.8.1}
		\begin{enumerate}
		\item {\hfill$\begin{aligned}[t]
				\frac{d}{dx}\left[\sech x\right] &= \frac{d}{dx}\left[\frac{2}{e^x+e^{-x}}\right]  \\
													&= \frac{-2(e^x-e^{-x})}{(e^x+e^{-x})^2} \\
													&= -\frac{2(e^x-e^{-x})}{(e^x+e^{-x})(e^x+e^{-x})} \\
													&= -\frac{2}{e^x+e^{-x}}\cdot \frac{e^x-e^{-x}}{e^x+e^{-x}}\\
													&= -\sech x\tanh x
		\end{aligned}$\hfill\null}
		\item
		{\hfill$\begin{aligned}[t]
				\frac{d}{dx}\left[\coth x\right] &= \frac{d}{dx}\left[\frac{e^x+e^{-x}}{e^x-e^{-x}}\right]  \\
													&= \frac{(e^x-e^{-x})(e^x-e^{-x})-(e^x+e^{-x})(e^x+e^{-x})}{(e^x-e^{-x})^2}\\
													&= \frac{e^{2x}+e^{-2x} - 2 - (e^{2x}+e^{-2x}+2)}{(e^x-e^{-x})^2}\\
													&= -\frac{4}{(e^x-e^{-x})^2}\\
													&=-\csch^2x
		\end{aligned}$\hfill\null}
		\item {$\ds	\int \tanh x\ dx = \int \frac{\sinh x}{\cosh x}\ dx$\\
		Let $u = \cosh x$; $du = (\sinh x) dx$\\
		\hfill$\begin{aligned}[t]
													&= \int \frac{1}{u}\ du \\
													&= \ln |u| + C \\
													&= \ln (\cosh x) + C.
		\end{aligned}$\hfill\null}
		\item {\hfill$\begin{aligned}[t]
				\int \coth x\ dx &= \int \frac{\cosh x}{\sinh x}\ dx\\
													\text{Let $u = \sinh x$; $du = (\cosh x) dx$} 	\\
													&= \int \frac{1}{u}\ du \\
													&= \ln |u| + C \\
													&= \ln |\sinh x| + C.
		\end{aligned}$\hfill\null}
		\end{enumerate}
	
\end{Answer}
\begin{Answer}{7.8.2}
	\begin{enumerate}
	\item {$2\sinh 2x$}
	\item {$2x\sec^2(x^2)$}
	\item {$\coth x$}
	\item {$\sinh^2x+\cosh^2x$}
	\item {$x\cosh x$}
	\item {$\frac{-2x}{(x^2)\sqrt{1-x^4}}$}
	\item {$\frac{3}{\sqrt{9x^2+1}}$}
	\item {$\frac{4x}{\sqrt{4x^4-1}}$}
	\item {$\frac{1}{1-(x+5)^2}$}
	\item {$-\csc x$}
	\item {$\sec x$}
	\end{enumerate}
	
\end{Answer}
\begin{Answer}{7.8.3}
	\begin{enumerate}
	\item {$y=x$}
	\item {$y=3/4(x-\ln 2)+5/4$}
	\item {$y=-72/125(x-\ln 3)+9/25$}
	\item {$y=x$}
	\item  {$y=(x-\sqrt{2})+\cosh^{-1}(\sqrt{2}) \approx (x-1.414)+0.881$}
	\end{enumerate}
	
\end{Answer}
\begin{Answer}{7.8.4}
	\begin{enumerate}
	\item {$1/2\ln (\cosh(2x))+C$}
	\item {$1/3\sinh(3x-7)+C$}
	\item {$1/2\sinh^2x+C$ or $1/2\cosh^2x+C$}
	\item {$x \sinh (x)-\cosh (x)+C$
	}
	\item {$x \cosh (x)-\sinh (x)+C$
	}
	\item {$\left\{\begin{array}{ccc} \frac13\tanh^{-1}\left(\frac x3\right)+C & & x^2<9 \\ \\
	\frac13\coth^{-1}\left(\frac x3\right)+C & & 9<x^2 \end{array}\right. = \frac12\ln |x+1| - \frac12\ln |x-1|+C$}
	\item  {$\cosh^{-1} (x^2/2) + C = \ln (x^2+\sqrt{x^4-4})+C$}
	\item {$2/3\sinh^{-1} x^{3/2} + C = 2/3\ln (x^{3/2}+\sqrt{x^3+1})+C$}
	\item {$\frac{1}{16}\tan^{-1}(x/2)+\frac{1}{32}\ln |x-2|+\frac1{32}\ln|x+2|+C$}
	\item {$\ln x- \ln|x+1|+C$}
	\item {$\tan^{-1}(e^x)+C$}
	\item {$x\sinh^{-1}x-\sqrt{x^2+1}+C$}
	\item {$x\tanh^{-1}x+1/2\ln|x^2-1|+C$}
	\item  {$\tan^{-1}(\sinh x)+C$}
	\end{enumerate}
	
\end{Answer}
\begin{Answer}{7.8.5}
	\begin{enumerate}
	\item {$0$}
	\item {$3/2$}

	\item {$2$}
	\end{enumerate}
	
\end{Answer}
\begin{Answer}{7.9.1}
 $\ds{(t+4)^4\over4}+C$
\end{Answer}
\begin{Answer}{7.9.2}
 $\ds{(t^2-9)^{5/2}\over5}+C$
\end{Answer}
\begin{Answer}{7.9.3}
 $\ds{(e^{t^2}+16)^2\over 4}+C$
\end{Answer}
\begin{Answer}{7.9.4}
 $\ds\cos t-{2\over3}\cos^3 t+C$
\end{Answer}
\begin{Answer}{7.9.5}
 $\ds{\tan^2 t\over 2}+C$
\end{Answer}
\begin{Answer}{7.9.6}
 $\ds\ln|t^2+t+3|+C$
\end{Answer}
\begin{Answer}{7.9.7}
 $\ds {1\over8} \ln|1-4/t^2|+C$
\end{Answer}
\begin{Answer}{7.9.8}
 $\ds{1\over25}\tan(\arcsin(t/5))+C={t\over25\sqrt{25-t^2}}+C$
\end{Answer}
\begin{Answer}{7.9.9}
 $\ds{2\over3}\sqrt{\sin 3t}+C$
\end{Answer}
\begin{Answer}{7.9.10}
 $\ds t\tan t+\ln|\cos t|+C$
\end{Answer}
\begin{Answer}{7.9.11}
 $\ds 2\sqrt{e^t+1}+C$
\end{Answer}
\begin{Answer}{7.9.12}
 $\ds{3t\over 8}+{\sin 2t\over4}+ {\sin 4t\over 32}+C$
\end{Answer}
\begin{Answer}{7.9.13}
 $\ds{\ln |t|\over 3} - {\ln |t+3|\over 3}+C$
\end{Answer}
\begin{Answer}{7.9.14}
 $\ds{-1\over \sin\arctan t}+C=-\sqrt{1+t^2}/t+C$
\end{Answer}
\begin{Answer}{7.9.15}
 $\ds{-1\over 2(1+\tan t)^2}+C$
\end{Answer}
\begin{Answer}{7.9.16}
 $\ds{(t^2+1)^{5/2}\over 5}-{(t^2+1)^{3/2}\over 3}+C$
\end{Answer}
\begin{Answer}{7.9.17}
 $\ds{e^t\sin t-e^t\cos t\over 2}+C$
\end{Answer}
\begin{Answer}{7.9.18}
 $\ds{(t^{3/2}+47)^4\over6}+C$
\end{Answer}
\begin{Answer}{7.9.19}
 $\ds{2\over 3(2-t^2)^{3/2}}-{1\over(2-t^2)^{1/2}}+C$
\end{Answer}
\begin{Answer}{7.9.20}
 $\ds{\ln|\sin(\arctan(2t/3))|\over9}+C =
(\ln(4t^2)-\ln(9+4t^2))/18 + C$
\end{Answer}
\begin{Answer}{7.9.21}
 $\ds{(\arctan(2t))^2\over4}+C$
\end{Answer}
\begin{Answer}{7.9.22}
 $\ds{3\ln|t+3|\over 4}+{\ln|t-1|\over4}+C$
\end{Answer}
\begin{Answer}{7.9.23}
 $\ds{\cos^7 t\over 7}-{\cos^5 t\over 5}+C$
\end{Answer}
\begin{Answer}{7.9.24}
 $\ds{-1\over t-3}+C$
\end{Answer}
\begin{Answer}{7.9.25}
 $\ds{-1\over \ln t}+C$
\end{Answer}
\begin{Answer}{7.9.26}
 $\ds{t^2(\ln t)^2\over 2}-{t^2\ln t\over 2}+{t^2\over4}+C$
\end{Answer}
\begin{Answer}{7.9.27}
 $\ds(t^3-3t^2+6t-6)e^t+C$
\end{Answer}
\begin{Answer}{7.9.28}
 $\ds{5+\sqrt5\over10}
\ln(2t+1-\sqrt5)+{5-\sqrt5\over10}\ln(2t+1+\sqrt5)+C$
\end{Answer}
\begin{Answer}{8.1.1}
 It rises until $t=100/49$, then falls. The position of the
object at time $t$ is $\ds s(t)=-4.9t^2+20t+k$. The net distance traveled
is $-45/2$, that is, it ends up $45/2$ meters below where it started.
The total distance traveled is $6205/98$ meters.
\end{Answer}
\begin{Answer}{8.1.2}
 $\ds\int_0^{2\pi}\sin t\,dt=0$
\end{Answer}
\begin{Answer}{8.1.3}
 net: $2\pi$, total: $\ds 2\pi/3+4\sqrt3$
\end{Answer}
\begin{Answer}{8.1.4}
 $8$
\end{Answer}
\begin{Answer}{8.1.5}
 $17/3$
\end{Answer}
\begin{Answer}{8.1.6}
 $A=18$, $B=44/3$, $C=10/3$
\end{Answer}
\begin{Answer}{8.2.1}
\begin{enumerate}
\item {$\pi$}
\item {$4\pi+\pi^2\approx 22.436$}
\item {$16/3$}
\item {$\pi$}
\item {$1/2$}
\item {$2\sqrt{2}$}
\item {$1/\ln 4$}
\end{enumerate}
\end{Answer}
\begin{Answer}{8.2.2}
\begin{enumerate}
\item {$1$}
\item {$5/3$}
\item {$9/2$}
\item {$9/4$}
\item {$1/12(9-2\sqrt{2})\approx 0.514$}
\end{enumerate}
\end{Answer}
\begin{Answer}{8.2.3}
\begin{enumerate}
\item $ 1 $
\item $ 5 $
\item $ 4 $
\item $ \frac{133}{20} $
\end{enumerate}
\end{Answer}
\begin{Answer}{8.2.4}
 $\ds 8\sqrt2/15$
\end{Answer}
\begin{Answer}{8.2.5}
 $1/12$
\end{Answer}
\begin{Answer}{8.2.6}
 $9/2$
\end{Answer}
\begin{Answer}{8.2.7}
 $4/3$
\end{Answer}
\begin{Answer}{8.2.8}
 $2/3-2/\pi$
\end{Answer}
\begin{Answer}{8.2.9}
 $\ds 3/\pi - 3\sqrt3/(2\pi)-1/8$
\end{Answer}
\begin{Answer}{8.2.10}
 $1/3$
\end{Answer}
\begin{Answer}{8.2.11}
 $\ds 10\sqrt{5}/3-6$
\end{Answer}
\begin{Answer}{8.2.12}
 $500/3$
\end{Answer}
\begin{Answer}{8.2.13}
 $2$
\end{Answer}
\begin{Answer}{8.2.14}
 $1/5$
\end{Answer}
\begin{Answer}{8.2.15}
 $1/6$
\end{Answer}
\begin{Answer}{8.2.16}
{219,000 m$^2$}
\end{Answer}
\begin{Answer}{8.2.17}
{623,333 m$^2$}

\end{Answer}
\begin{Answer}{8.3.5}
 $8\pi/3$
\end{Answer}
\begin{Answer}{8.3.6}
 $\pi/30$
\end{Answer}
\begin{Answer}{8.3.7}
 $\pi(\pi/2-1)$
\end{Answer}
\begin{Answer}{8.3.8}
\begin{multicols}{2}
\begin{enumerate}
	\item	$114\pi/5$
	\item	$74\pi/5$
	\item	$20\pi$
	\item	$4\pi$
\end{enumerate}
\end{multicols}
\end{Answer}
\begin{Answer}{8.3.9}
 $16\pi$, $24\pi$
\end{Answer}
\begin{Answer}{8.3.11}
 $\ds \pi h^2(3r-h)/3$
\end{Answer}
\begin{Answer}{8.3.13}
 $2\pi$
\end{Answer}
\begin{Answer}{8.4.1}
 $2/\pi$; $2/\pi$; $0$
\end{Answer}
\begin{Answer}{8.4.2}
 $4/3$
\end{Answer}
\begin{Answer}{8.4.3}
 $1/A$
\end{Answer}
\begin{Answer}{8.4.4}
 $\pi/4$
\end{Answer}
\begin{Answer}{8.4.5}
 $-1/3$, $1$
\end{Answer}
\begin{Answer}{8.4.6}
 $\ds -4\sqrt{1224}$ ft/s; $\ds -8\sqrt{1224}$ ft/s
\end{Answer}
\begin{Answer}{8.5.1}
 $\approx 5,305,028,516$ N-m
\end{Answer}
\begin{Answer}{8.5.2}
 $\approx 4,457,854,041$ N-m
\end{Answer}
\begin{Answer}{8.5.3}
 $367,500 \pi$ N-m
\end{Answer}
\begin{Answer}{8.5.4}
 $49000\pi + 196000/3$ N-m
\end{Answer}
\begin{Answer}{8.5.5}
 $2450\pi$ N-m
\end{Answer}
\begin{Answer}{8.5.6}
 $0.05$ N-m
\end{Answer}
\begin{Answer}{8.5.7}
 $6/5$ N-m
\end{Answer}
\begin{Answer}{8.5.8}
 $3920$ N-m
\end{Answer}
\begin{Answer}{8.5.9}
 $23520$ N-m
\end{Answer}
\begin{Answer}{8.5.10}
 $12740$ N-m
\end{Answer}
\begin{Answer}{8.6.1}
$15/2$
\end{Answer}
\begin{Answer}{8.6.2}
$5$
\end{Answer}
\begin{Answer}{8.6.3}
$16/5$
\end{Answer}
\begin{Answer}{8.6.5}
$\ds \bar x=45/28$, $\ds \bar y = 93/70$
\end{Answer}
\begin{Answer}{8.6.6}
$\ds \bar x=0$, $\ds \bar y=4/(3\pi)$
\end{Answer}
\begin{Answer}{8.6.7}
$\ds \bar x=1/2$, $\ds \bar y=2/5$
\end{Answer}
\begin{Answer}{8.6.8}
$\ds \bar x=0$, $\ds \bar y=8/5$
\end{Answer}
\begin{Answer}{8.6.9}
$\ds \bar x=4/7$, $\ds \bar y=2/5$
\end{Answer}
\begin{Answer}{8.6.10}
$\bar x=\bar y=1/5$
\end{Answer}
\begin{Answer}{8.6.11}
$\ds \bar x=0$, $\ds \bar y=28/(9\pi)$
\end{Answer}
\begin{Answer}{8.6.12}
$\ds \bar x=\bar y=28/(9\pi)$
\end{Answer}
\begin{Answer}{8.6.13}
$\ds \bar x=0$, $\ds\bar y=244/(27\pi)\approx 2.88$
\end{Answer}
\begin{Answer}{8.7.1}
\begin{enumerate}
\item {$\sqrt{2}$}
\item {$6$}
\item {$4/3$}
\item {$6$}
\item {$109/2$}
\item {$3/2$}
\item {$12/5$}
\item {$79953333/400000 \approx 199.883$}
\item {$-\ln (2-\sqrt{3}) \approx 1.31696$}
\item {$\sinh^{-1} 1$}
\end{enumerate}
\end{Answer}
\begin{Answer}{8.7.2}
\begin{enumerate}
\item {$\int_0^1 \sqrt{1+4x^2}\ dx$}
\item {$\int_0^1 \sqrt{1+100x^{18}}\ dx$}
\item {$\int_1^e \sqrt{1+\frac1{x^2}}\ dx$}
\item {$\int_0^1 \sqrt{1+\frac{1}{4x}}\ dx$}
\item {$\int_{-1}^1 \sqrt{1+\frac{x^2}{1-x^2}}\ dx$}
\item
{$\int_{-3}^3 \sqrt{1+\frac{x^2}{81-9x^2}}\ dx$}

\item {$\int_{1}^2 \sqrt{1+\frac1{x^4}}\ dx$}
\item {$\int_{-\pi/4}^{\pi/4} \sqrt{1+\sec^2x\tan^2x}\ dx$}

\end{enumerate}
\end{Answer}
\begin{Answer}{8.7.3}
\begin{enumerate}
\item {$1.4790$}
\item {$1.8377$}
\item {$2.1300$}
\item
\item
\item
\item {$1.4058$}
\item {$1.7625$}
\end{enumerate}
\end{Answer}
\begin{Answer}{8.7.4}
 $\ds (22\sqrt{22}-8)/27$
\end{Answer}
\begin{Answer}{8.7.5}
 $\ln(2)+3/8$
\end{Answer}
\begin{Answer}{8.7.6}
 $\ds a+a^3/3$
\end{Answer}
\begin{Answer}{8.7.7}
 $\ds \ln((\sqrt2+1)/\sqrt3)$
\end{Answer}
\begin{Answer}{8.7.9}
 $3/4$
\end{Answer}
\begin{Answer}{8.7.10}
 $\approx 3.82$
\end{Answer}
\begin{Answer}{8.7.11}
 $\approx 1.01$
\end{Answer}
\begin{Answer}{8.7.12}
 $\ds \sqrt{1+e^2}-\sqrt2+
{1\over2}\ln\left({\sqrt{1+e^2}-1\over\sqrt{1+e^2}+1}\right)+
{1\over2}\ln(3+2\sqrt2)$
\end{Answer}
\begin{Answer}{8.8.1}
 $\ds 8\pi\sqrt3-{16\pi\sqrt2\over 3}$
\end{Answer}
\begin{Answer}{8.8.3}
 $\ds {730\pi\sqrt{730}\over27}-{10\pi\sqrt{10}\over 27}$
\end{Answer}
\begin{Answer}{8.8.4}
 $\ds \pi +2\pi e+ {1\over4}\pi e^2-{\pi\over4e^2}-{2\pi\over e}$
\end{Answer}
\begin{Answer}{8.8.6}
 $8\pi^2$
\end{Answer}
\begin{Answer}{8.8.7}
 $\ds 2\pi+{8\pi^2\over 3\sqrt{3}}$
\end{Answer}
\begin{Answer}{8.8.8}
 $a>b$: $\ds 2\pi b^2+$\hfill\break
\hbox{\hskip1cm}$\ds {2\pi a^2b\over\sqrt{a^2-b^2}}
  \arcsin(\sqrt{a^2-b^2}/a)$,\hfill\break
$a<b$: $\ds 2\pi b^2+ $\hfill\break
\hbox{\hskip1cm}$\ds {2\pi a^2b\over\sqrt{b^2-a^2}}
  \ln\left({b\over a}+{\sqrt{b^2-a^2}\over a}\right)$
\end{Answer}
\begin{Answer}{9.1.1}
$1$
\end{Answer}
\begin{Answer}{9.1.3}
$0$
\end{Answer}
\begin{Answer}{9.1.4}
$1$
\end{Answer}
\begin{Answer}{9.1.5}
$1$
\end{Answer}
\begin{Answer}{9.1.6}
$0$
\end{Answer}
\begin{Answer}{9.2.1}
$\ds\lim_{n\to\infty} n^2/(2n^2+1)=1/2$
\end{Answer}
\begin{Answer}{9.2.2}
$\ds\lim_{n\to\infty} 5/(2^{1/n}+14)=1/3$
\end{Answer}
\begin{Answer}{9.2.3}
$\sum_{n=1}^\infty {1\over n}$ diverges, so $\ds\sum_{n=1}^\infty 3{1\over n}$ diverges
\end{Answer}
\begin{Answer}{9.2.4}
$-3/2$
\end{Answer}
\begin{Answer}{9.2.5}
$11$
\end{Answer}
\begin{Answer}{9.2.6}
$20$
\end{Answer}
\begin{Answer}{9.2.7}
$3/4$
\end{Answer}
\begin{Answer}{9.2.8}
$3/2$
\end{Answer}
\begin{Answer}{9.2.9}
$3/10$
\end{Answer}
\begin{Answer}{9.3.1}
diverges
\end{Answer}
\begin{Answer}{9.3.2}
diverges
\end{Answer}
\begin{Answer}{9.3.3}
converges
\end{Answer}
\begin{Answer}{9.3.4}
converges
\end{Answer}
\begin{Answer}{9.3.5}
converges
\end{Answer}
\begin{Answer}{9.3.6}
converges
\end{Answer}
\begin{Answer}{9.3.7}
diverges
\end{Answer}
\begin{Answer}{9.3.8}
converges
\end{Answer}
\begin{Answer}{9.3.9}
$N=5$
\end{Answer}
\begin{Answer}{9.3.10}
$N=10$
\end{Answer}
\begin{Answer}{9.3.11}
$N=1687$
\end{Answer}
\begin{Answer}{9.3.12}
any integer greater than $\ds e^{200}$
\end{Answer}
\begin{Answer}{9.4.1}
converges
\end{Answer}
\begin{Answer}{9.4.2}
converges
\end{Answer}
\begin{Answer}{9.4.3}
diverges
\end{Answer}
\begin{Answer}{9.4.4}
converges
\end{Answer}
\begin{Answer}{9.4.5}
$0.90$
\end{Answer}
\begin{Answer}{9.4.6}
$0.95$
\end{Answer}
\begin{Answer}{9.5.1}
converges
\end{Answer}
\begin{Answer}{9.5.2}
converges
\end{Answer}
\begin{Answer}{9.5.3}
converges
\end{Answer}
\begin{Answer}{9.5.4}
diverges
\end{Answer}
\begin{Answer}{9.5.5}
diverges
\end{Answer}
\begin{Answer}{9.5.6}
diverges
\end{Answer}
\begin{Answer}{9.5.7}
converges
\end{Answer}
\begin{Answer}{9.5.8}
diverges
\end{Answer}
\begin{Answer}{9.5.9}
converges
\end{Answer}
\begin{Answer}{9.5.10}
diverges
\end{Answer}
\begin{Answer}{9.6.1}
converges absolutely
\end{Answer}
\begin{Answer}{9.6.2}
diverges
\end{Answer}
\begin{Answer}{9.6.3}
converges conditionally
\end{Answer}
\begin{Answer}{9.6.4}
converges absolutely
\end{Answer}
\begin{Answer}{9.6.5}
converges conditionally
\end{Answer}
\begin{Answer}{9.6.6}
converges absolutely
\end{Answer}
\begin{Answer}{9.6.7}
diverges
\end{Answer}
\begin{Answer}{9.6.8}
converges conditionally
\end{Answer}
\begin{Answer}{9.7.5}
\begin{enumerate}
	\item converges
	\item converges
	\item converges
	\item diverges
\end{enumerate}
\end{Answer}
\begin{Answer}{9.8.1}
\begin{enumerate}
	\item $R=1$, $I=(-1,1)$
	\item $R=\infty$, $I=(-\infty,\infty)$
	\item $R=e$, $I=(2-e,2+e)$
	\item $R=0$, converges only when $x=2$
	\item $R=1$, $I=[-6,-4]$
\end{enumerate}
\end{Answer}
\begin{Answer}{9.8.2}
$R=e$
\end{Answer}
\begin{Answer}{9.9.1}
the alternating harmonic series
\end{Answer}
\begin{Answer}{9.9.2}
$\ds\sum_{n=0}^\infty (n+1)x^n$
\end{Answer}
\begin{Answer}{9.9.3}
$\ds\sum_{n=0}^\infty (n+1)(n+2)x^n$
\end{Answer}
\begin{Answer}{9.9.4}
$\ds\sum_{n=0}^\infty {(n+1)(n+2)\over 2}x^n$, $R=1$
\end{Answer}
\begin{Answer}{9.9.5}
$\ds C+\sum_{n=0}^\infty {-1\over (n+1)(n+2)}x^{n+2}$
\end{Answer}
\begin{Answer}{9.10.1}
\begin{enumerate}
\item
{$p_3(x) = 1-x+\frac12x^3-\frac16x^3$
}
\item
{$p_8(x) = x-\frac16x^3+\frac1{120}x^5-\frac1{5040}x^7$
}
\item
{$p_8(x) = x+x^2+\frac12x^3+\frac16x^4+\frac1{24}x^5$
}
\item
{$p_6(x) = \frac{2 x^5}{15}+\frac{x^3}{3}+x$
}
\item
{$p_4(x) = \frac{2 x^4}{3}+\frac{4 x^3}{3}+2 x^2+2 x+1$
}
\item
{$p_4(x) = x^4+x^3+x^2+x+1$
}
\item
{$p_4(x) = x^4-x^3+x^2-x+1$
}
\item
{$p_7(x) = -\frac{x^7}{7}+\frac{x^5}{5}-\frac{x^3}{3}+x$
}
\end{enumerate}
\end{Answer}
\begin{Answer}{9.10.2}
\begin{enumerate}
\item
{$p_4(x) = 1+\frac{1}{2} (-1+x)-\frac{1}{8} (-1+x)^2+\frac{1}{16}
   (-1+x)^3-\frac{5}{128} (-1+x)^4$
}
\item
{$p_4(x) = \ln (2)+\frac{1}{2} (-1+x)-\frac{1}{8}
   (-1+x)^2+\frac{1}{24} (-1+x)^3-\frac{1}{64} (-1+x)^4$
}
\item
{$p_6(x) = \frac{1}{\sqrt{2}}-\frac{-\frac{\pi
   }{4}+x}{\sqrt{2}}-\frac{\left(-\frac{\pi
   }{4}+x\right)^2}{2 \sqrt{2}}+\frac{\left(-\frac{\pi
   }{4}+x\right)^3}{6 \sqrt{2}}+\frac{\left(-\frac{\pi
   }{4}+x\right)^4}{24 \sqrt{2}}-\frac{\left(-\frac{\pi
   }{4}+x\right)^5}{120 \sqrt{2}}-\frac{\left(-\frac{\pi
   }{4}+x\right)^6}{720 \sqrt{2}}$
}
\item
{$p_5(x) = \frac{1}{2}+\frac{1}{2} \sqrt{3} \left(-\frac{\pi
   }{6}+x\right)-\frac{1}{4} \left(-\frac{\pi
   }{6}+x\right)^2-\frac{\left(-\frac{\pi }{6}+x\right)^3}{4
   \sqrt{3}}+\frac{1}{48} \left(-\frac{\pi
   }{6}+x\right)^4+\frac{\left(-\frac{\pi
   }{6}+x\right)^5}{80 \sqrt{3}}$
}
\item
{$p_5(x) = \frac{1}{2}-\frac{x-2}{4}+\frac{1}{8} (x-2)^2-\frac{1}{16}
   (x-2)^3+\frac{1}{32} (x-2)^4-\frac{1}{64} (x-2)^5$
}
\item
{$p_8(x) = 1-2 (-1+x)+3 (-1+x)^2-4 (-1+x)^3+5 (-1+x)^4-6 (-1+x)^5+7
   (-1+x)^6-8 (-1+x)^7+9 (-1+x)^8$
}
\item
{$p_3(x) =\frac{1}{2}+\frac{1+x}{2}+\frac{1}{4} (1+x)^2$
}
\item
{$p_2(x) =-\pi ^2-2 \pi  (x-\pi)+\frac{1}{2} \left(\pi ^2-2\right)
   (x-\pi)^2$
}
\end{enumerate}
\end{Answer}
\begin{Answer}{9.10.3}
\begin{enumerate}
\item
{$p_3(x) =x-\frac{x^3}{6}$; $p_3(0.1) = 0.09983$. Error is bounded by $\pm \frac{1}{4!}\cdot0.1^4 \approx \pm 0.000004167$.
}
\item
{$p_4(x) =1-\frac{x^2}{2}+\frac{x^4}{24}$; $p_4(1) = 13/24\approx 0.54167$. Error is bounded by $\pm \frac{1}{5!}\cdot1^5 \approx \pm 0.00833$
}
\item
{$p_2(x) =3+\frac{1}{6} (-9+x)-\frac{1}{216} (-9+x)^2$; $p_2(10) =  3.16204$. The third derivative of $f(x) =\sqrt x$ is bounded on $(8,11)$ by $0.003$. Error is bounded by $\pm \frac{0.003}{3!}\cdot1^3 = \pm 0.0005.$
}
\item
{$p_3(x) =-1+x-\frac{1}{2} (-1+x)^2+\frac{1}{3} (-1+x)^3$; $p_3(1.5) =  0.41667$. The fourth derivative of $f(x) =\ln x$ is bounded on $(.9,2)$ by $10$. Error is bounded by $\pm \frac{10}{4!}\cdot.5^4 = \pm 0.026.$
}
\end{enumerate}
\end{Answer}
\begin{Answer}{9.10.4}
\begin{enumerate}
\item
{The $n^\text{th}$ derivative of $f(x)=e^x$ is bounded by $3$ on intervals containing $0$ and 1. Thus $|R_n(1)|\leq \frac{3}{(n+1)!}1^{(n+1)}$. When $n=7$, this is less than $0.0001$.
}
\item
{The $n^\text{th}$ derivative of $f(x)=\sqrt x$ is bounded by $0.1$ on intervals containing $3$ and $4$. Thus $|R_n(\pi)|\leq \frac{0.1}{(n+1)!}(1)^{(n+1)}$. When $n=4$, this is less than $0.0001$.
}
\item
{The $n^\text{th}$ derivative of $f(x)=\cos x$ is bounded by $1$ on intervals containing $0$ and $\pi/3$. Thus $|R_n(\pi/3)|\leq \frac{1}{(n+1)!}(\pi/3)^{(n+1)}$. When $n=7$, this is less than $0.0001$. Since the Maclaurin polynomial of $\cos x$ only uses even powers, we can actually just use $n=6$.
}
\item
{The $n^\text{th}$ derivative of $f(x)=\sin x$ is bounded by $1$ on intervals containing $0$ and $\pi$. Thus $|R_n(\pi)|\leq \frac{1}{(n+1)!}(\pi)^{(n+1)}$. When $n=12$, this is less than $0.0001$. Since the Maclaurin polynomial of $\sin x$ only uses odd powers, we can actually just use $n=11$.
}
\end{enumerate}
\end{Answer}
\begin{Answer}{9.10.5}
\begin{enumerate}
\item
{The $n^\text{th}$ term is $\frac{1}{n!}x^n$.
}
\item
{The $n^\text{th}$ term is: when $n$ is even,  $\frac{(-1)^{n/2}}{n!}x^n$; when $n$ is odd, $0$.
}
\item
{The $n^\text{th}$ term is $x^n$.
}
\item
{The $n^\text{th}$ term is $(-1)^nx^n$.
}
\item
{The $n^\text{th}$ term is $(-1)^n\frac{(x-1)^n}{n}$.
}
\end{enumerate}
\end{Answer}
\begin{Answer}{9.10.6}
\begin{enumerate}
\item
{$\ds 1+x+\frac12x^2+\frac16x^3+\frac1{24}x^4$
}
\item
{$\ds 3+15x+\frac{75}{2}x^2+\frac{375}{6}x^3+\frac{1875}{24}x^4$
}
\item
{$\ds 1+2x-2x^2+4x^3-10x^4$
}

\end{enumerate}
\end{Answer}
\begin{Answer}{9.11.1}
\begin{enumerate}
\item
{All derivatives of $e^x$ are $e^x$ which evaluate to 1 at $x=0$.

The Taylor series starts $1+x+\frac12x^2+\frac{1}{3!}x^3+\frac{1}{4!}x^4+\cdots$;

the Taylor series is $\ds \sum_{n=0}^\infty \frac{x^n}{n!}$
}
\item
{All derivatives of $\sin x$ are either $\pm\cos x$ or $\pm \sin x$, which evaluate to $\pm 1$ or $0$ at $x=0$. The Taylor series starts $0+x+0x^2-\frac16x^3+0x^4+\frac1{120}x^5$;

the Taylor series is $\ds \sum_{n=0}^\infty (-1)^n\frac{x^{2n+1}}{(2n+1)!}$
}
\item
{The $n^\text{th}$ derivative of $1/(1-x)$ is $f\,^{(n)}(x) = (n)!/(1-x)^{n+1}$, which evaluates to $n!$ at $x=0$.

The Taylor series starts $1+x+x^2+x^3+\cdots$;

the Taylor series is $\ds \sum_{n=0}^\infty x^n$
}
\item
{The derivative of $\tan^{-1}x$ is $1/(1+x^2)$. Taking successive derivatives using the Quotient Rule, the derivatives of $\tan^{-1}x$ fall into two categories in terms of their evaluation at $x=0$.

When $n$ is even, $\ds f\,^{(n)}(x) = (-1)^{(n-1)/2}\frac{p(x)}{(1+x^2)^n}$, where $p(x)$ is a polynomial such that $p(0) = 0$. Hence $f\,^{(n)}(0) = 0$ when $n$ is even.

When $n$ is odd, $\ds f\,^{(n)}(x) = (-1)^{(n-1)/2}\frac{p(x)}{(1+x^2)^n}$, where $p(x)$ is a polynomial such that $p(0) = (n-1)!$. Hence $f\,^{(n)}(0) = (-1)^{(n-1)/2}(n-1)!$ when $n$ is odd. (The unusual power of $(-1)$ is such that every other odd term is negative.)

The Taylor series starts $x-\frac13x^3+\frac15x^5+\cdots$; by reindexing to only obtain odd powers of $x$, we get

the Taylor series is $\ds \sum_{n=0}^\infty (-1)^n\frac{x^{2n+1}}{2n+1}$.
}
\end{enumerate}
\end{Answer}
\begin{Answer}{9.11.2}
\begin{enumerate}
\item
{The Taylor series starts $0-(x-\pi/2)+0x^2+\frac16(x-\pi/2)^3+0x^4-\frac1{120}(x-\pi/2)^5$;

the Taylor series is $\ds \sum_{n=0}^\infty (-1)^{n+1}\frac{(x-\pi/2)^{2n+1}}{(2n+1)!}$
}
\item
{The Taylor series starts $1-(x-1)+(x-1)^2-(x-1)^3+(x-1)^4-(x-1)^5$;

the Taylor series is $\ds \sum_{n=0}^\infty (-1)^{n}(x-1)^n$
}
\item
{$f\,^{(n)}(x) = (-1)^ne^{-x}$; at $x=0$, $f\,^{(n)}(0)=-1$ when $n$ is odd and $f\,^{(n)}(0)=1$ when $n$ is even.

The Taylor series starts $1-x+\frac12x^2-\frac1{3!}x^3+\cdots$;

the Taylor series is $\ds \sum_{n=0}^\infty (-1)^n\frac{x^n}{n!}$.
}
\item
{$f\,^{(n)}(x) = (-1)^{n+1}\frac{(n-1)!}{(1+x)^n}$; at $x=0$, $f\,^{(n)}(0)=(-1)^{n+1}(n-1)!$

The Taylor series starts $x-\frac{x^2}2+\frac{x^3}3-\frac{x^4}4+\cdots$;

the Taylor series is $\ds \sum_{n=1}^\infty (-1)^{n+1}\frac{x^n}{n}$.
}
\item
{$f\,^{(n)}(x) = (-1)^{n+1}\frac{n!}{(x+1)^{n+1}}$; at $x=1$, $f\,^{(n)}(1)=(-1)^{n+1}\frac{n!}{2^{n+1}}$

The Taylor series starts $\frac12+\frac14(x-1)-\frac18(x-1)^2+\frac1{16}(x-1)^3\cdots$;

the Taylor series is $\ds \sum_{n=0}^\infty (-1)^{n+1}\frac{(x-1)^n}{2^{n+1}}$.
}
\item
{The derivatives of $\sin x$ are $\pm \cos x$ and $\pm \sin x$; at $x=\pi/4$, these derivatives evaluate to $\pm \sqrt{2}/2$.

The Taylor series starts $\frac{\sqrt{2}}2+\frac{\sqrt{2}}2(x-\pi/4) - \frac{\sqrt{2}}2\frac{(x-\pi/4)^2}{2}-\frac{\sqrt{2}}2\frac{(x-\pi/4)^3}{3!}+\frac{\sqrt{2}}2\frac{(x-\pi/4)^4}{4!}+\frac{\sqrt{2}}2\frac{(x-\pi/4)^5}{5!}\cdots$. Note how the signs are ``even, even, odd, odd, even, even, odd, odd,$\ldots$ We saw signs like these in Example \ref{ex_seq1} of Section \ref{sec:sequences}; one way of producing such signs is to raise $(-1)$ to a special quadratic power. While many possibilities exist,
one such quadratic is $(n+3)(n+4)/2$.

Thus the Taylor series is $\ds \sum_{n=0}^\infty (-1)^{\frac{(n+3)(n+4)}{2}}\frac{\sqrt2}{2}\frac{(x-\pi/4)^n}{n!}$.
}
\end{enumerate}
\end{Answer}
\begin{Answer}{9.11.3}
\begin{enumerate}
\item
{The following argument is essentially the same as that given for $f(x) = \cos x$ in Example \ref{ex_ts3}.

Given a value $x$, the magnitude of the error term $R_n(x)$ is bounded by
$$ \big|R_n(x)\big| \leq \frac{\max\left|\,f\,^{(n+1)}(z)\right|}{(n+1)!}\big|x^{(n+1)}\big|.$$
Since all derivatives of $\sin x$ are $\pm \cos x$ or $\pm\sin x$, whose magnitudes are bounded by $1$, we can state
$$ \big|R_n(x)\big| \leq \frac{1}{(n+1)!}\big|x^{(n+1)}\big|.$$
For any $x$, $\ds\lim_{n\to\infty} \frac{x^{n+1}}{(n+1)!} = 0$. Thus by the Squeeze Theorem, we conclude that $\ds \lim_{n\to\infty} R_n(x) = 0$ for all $x$, and hence
$$\sin x = \sum_{n=0}^\infty (-1)^{n}\frac{x^{2n+1}}{(2n+1)!}\quad \text{for all $x$}.$$
}
\item
{Given a value $x$, the magnitude of the error term $R_n(x)$ is bounded by
$$ \big|R_n(x)\big| \leq \frac{\max\left|\,f\,^{(n+1)}(z)\right|}{(n+1)!}\big|x^{(n+1)}\big|,$$
where $z$ is between $0$ and $x$.

If $x>0$, then $z<x$ and $f\,^{(n+1)}(z) =e^z<e^x$. If $x<0$, then $x<z<0$ and $f\,^{(n+1)}(z) =e^z<1$. So given a fixed $x$ value, let $M = \max\{e^x,1\}$; $f\,^{(n)}(z)<M.$ This allows us to state

$$ \big|R_n(x)\big| \leq \frac{M}{(n+1)!}\big|x^{(n+1)}\big|.$$
For any $x$, $\ds\lim_{n\to\infty} \frac{M}{(n+1)!}\big|x^{(n+1)}\big|= 0$. Thus by the Squeeze Theorem, we conclude that $\ds \lim_{n\to\infty} R_n(x) = 0$ for all $x$, and hence
$$e^x = \sum_{n=0}^\infty \frac{x^{n}}{n!}\quad \text{for all $x$}.$$
}
\item
{Given a value $x$, the magnitude of the error term $R_n(x)$ is bounded by
$$ \big|R_n(x)\big| \leq \frac{\max\left|\,f\,^{(n+1)}(z)\right|}{(n+1)!}\big|(x-1)^{(n+1)}\big|,$$
where $z$ is between $1$ and $x$.

Note that $\big|f\,^{(n+1)}(x)\big| = \frac{n!}{x^{n+1}}$.

We consider the cases when $x>1$ and when $x<1$ separately.

If $x>1$, then $1<z<x$ and $f\,^{(n+1)}(z) =\frac{n!}{z^{n+1}}<n!$. Thus
$$ \big|R_n(x)\big| \leq \frac{n!}{(n+1)!}\big|(x-1)^{(n+1)}\big|= \frac{(x-1)^{n+1}}{n+1}.$$
For a fixed $x$,
$$\lim_{n\to\infty} \frac{(x-1)^{n+1}}{n+1}=0.$$


If $0<x<1$, then $x<z<1$ and $f\,^{(n+1)}(z) =\frac{n!}{z^{n+1}}<\frac{n!}{x^{n+1}}$. Thus
$$ \big|R_n(x)\big| \leq \frac{n!/x^{n+1}}{(n+1)!}\big|(x-1)^{(n+1)}\big| = \frac{x^{n+1}}{n+1}(1-x)^{n+1}.$$
Since $0<x<1$, $x^{n+1}<1$ and $(1-x)^{n+1}<1$. We can then extend the inequality from above to state
$$\big|R_n(x)\big| \leq \frac{x^{n+1}}{n+1}(1-x)^{n+1}<\frac1{n+1}.$$

As $n\to\infty$, $1/(n+1)\to0$. Thus by the Squeeze Theorem, we conclude that $\ds \lim_{n\to\infty} R_n(x) = 0$ for all $x$, and hence
$$\ln x = \sum_{n=1}^\infty (-1)^{n+1}\frac{(x-1)^{n}}{n}\quad \text{for all $0<x\leq 2$}.$$
}
\item
{Given a value $x$, the magnitude of the error term $R_n(x)$ is bounded by
$$ \big|R_n(x)\big| \leq \frac{\max\left|\,f\,^{(n+1)}(z)\right|}{(n+1)!}\big|x^{(n+1)}\big|,$$
where $z$ is between $0$ and $x$.

Note that $\big|f\,^{(n+1)}(x)\big| = \frac{(n+1)!}{(1-x)^{n+2}}$.

%We consider the cases when $x>0$ and when $x<0$ separately.

If $0<x<1$, then $0<z<x$ and $f\,^{(n+1)}(z) =\frac{(n+1)!}{(1-z)^{n+2}}<\frac{(n+1)!}{(1-x)^{n+2}}$.
Thus
$$ \big|R_n(x)\big| \leq \frac{(n+1)!}{(1-x)^{n+2}}\frac{1}{(n+1)!}\big|x^{n+1}\big|= \frac{(x-1)^{n+1}}{n+1}.$$
For a fixed $x$,
$$\lim_{n\to\infty} \frac{(x-1)^{n+1}}{n+1}=0,$$
hence
$$\frac{1}{1-x} = \sum_{n=0}^\infty x^n \text{ on } (-1,0).$$

%If $-1<x<0$, then $x<z<0$ and $f\,^{(n+1)}(z) =\frac{(n+1)!}{(1-z)^{n+2}}<(n+1)!$.
%Thus
%$$ \big|R_n(x)\big| \leq \frac{(n+1)!}{(n+1)!}\big|x^{n+1}\big|= \big|x^{n+1}\big|.$$
%For a fixed $x$,
%$$\lim_{n\to\infty} \big|x^{n+1}\big|=0 \text{ as } |x|<1.$$



%As $n\to\infty$, $1/(n+1)\to0$. Thus by the Squeeze Theorem, we conclude that $\ds \lim_{n\to\infty} R_n(x) = 0$ for all $x$, and hence
%$$\ln x = \sum_{n=1}^\infty (-1)^{n+1}\frac{(x-1)^{n}}{n}\quad \text{for all $0<x\leq 2$}.$$
}
\end{enumerate}
\end{Answer}
\begin{Answer}{9.11.4}
\begin{enumerate}
\item
{Given $\ds \cos x = \sum_{n=0}^\infty (-1)^n\frac{x^{2n}}{(2n)!}$,

$\ds\cos (-x) = \sum_{n=0}^\infty (-1)^n\frac{(-x)^{2n}}{(2n)!}=\sum_{n=0}^\infty (-1)^n\frac{x^{2n}}{(2n)!}=\cos x$, as all powers in the series are even.
}
\item
{Given $\ds \sin x = \sum_{n=0}^\infty (-1)^n\frac{x^{2n+1}}{(2n+1)!}$,

$\ds\sin (-x) = \sum_{n=0}^\infty (-1)^n\frac{(-x)^{2n+1}}{(2n+1)!}=\sum_{n=0}^\infty (-1)^n\frac{-x^{2n+1}}{(2n+1)!}=-\sin x$, as all powers in the series are odd.
}
\item
{Given $\ds \sin x = \sum_{n=0}^\infty (-1)^n\frac{x^{2n+1}}{(2n+1)!}$,

$\ds\frac{d}{dx}\big(\sin x\big)  = \frac{d}{dx}\left(\sum_{n=0}^\infty (-1)^n\frac{x^{2n+1}}{(2n+1)!}\right)=\sum_{n=0}^\infty (-1)^n\frac{(2n+1)x^{2n}}{(2n+1)!}=\sum_{n=0}^\infty (-1)^n\frac{x^{2n}}{(2n)!}=\cos x$. (The summation still starts at $n=0$ as there was no constant term in the expansion of $\sin x$).
}
\item
{Given $\ds \cos x = \sum_{n=0}^\infty (-1)^n\frac{x^{2n}}{(2n)!}$,

$\ds\frac{d}{dx}\big(\cos x\big)  = \frac{d}{dx}\left(\sum_{n=0}^\infty (-1)^n\frac{x^{2n}}{(2n)!}\right)=\sum_{n=1}^\infty (-1)^n\frac{(2n)x^{2n-1}}{(2n)!}=\sum_{n=1}^\infty (-1)^n\frac{x^{2n-1}}{(2n-1)!}$. We can re-index this summation to start at $n=0$ by replacing $n$ with $n+1$ in the summation:
$$ \sum_{n=1}^\infty (-1)^n\frac{x^{2n-1}}{(2n-1)!} =\sum_{n=0}^\infty (-1)^{n+1}\frac{x^{2n+1}}{(2n+1)!}.$$

Note that this series has the opposite sign of the Taylor series for $\sin x$; thus $\frac{d}{dx}(\cos x) = -\sin x$.
}
\end{enumerate}
\end{Answer}
\begin{Answer}{9.11.5}
\begin{enumerate}
\item
{$\ds \sum_{n=0}^\infty (-1)^n\frac{(x^2)^{2n}}{(2n)!} = \sum_{n=0}^\infty (-1)^n\frac{x^{4n}}{(2n)!}.$
}
\item
{$\ds \sum_{n=0}^\infty \frac{(-x)^n}{n!}.$
}
\item
{$\ds \sum_{n=0}^\infty (-1)^n\frac{(2x+3)^{2n+1}}{(2n+1)!}.$
}
\item
{$\ds \sum_{n=0}^\infty (-1)^n\frac{(x/2)^{2n+1}}{(2n+1)}.$
}
\item
{$\ds x+x^2+\frac{x^3}{3}-\frac{x^5}{30}$
}
\item
{$\ds 1+\frac x2-\frac{5x^2}{8}-\frac{3x^3}{16}$
}
\end{enumerate}
\end{Answer}
\begin{Answer}{9.11.6}
\begin{enumerate}
\item
{$\ds 1+\frac x2-\frac{x^2}{8}+\frac{x^3}{16}-\frac{5x^4}{128}$
}
\item
{$\ds 1-\frac x2+\frac{3x^2}{8}-\frac{5x^3}{16}+\frac{35x^4}{128}$
}
\item
{$\ds 1+\frac x3-\frac{x^2}{9}+\frac{5x^3}{81}-\frac{10x^4}{243}$
}
\item
{$\ds 1+4x+6x^2+4x^3+x^4$ (note the series is finite, and the formula still applies)
}
\end{enumerate}
\end{Answer}
\begin{Answer}{9.11.7}
\begin{enumerate}
\item
{$\ds \int_0^{\sqrt{\pi}} \sin \big(x^2\big)\ dx \approx \int_0^{\sqrt{\pi}} \left(x^2-\frac{x^6}6+\frac{x^{10}}{120}-\frac{x^{14}}{5040}\right) dx = 0.8877$
}
\item
{$\ds \int_0^{\pi^2/4} \cos \big(\sqrt{x}\big)\ dx \approx \int_0^{\pi^2/4} \left(1-\frac x2+\frac{x^2}{24}-\frac{x^3}{720}\right)\ dx = 1.1412$. (Actual answer: $\pi-2$)
}
\end{enumerate}
\end{Answer}
\begin{Answer}{9.11.8}
\begin{enumerate}
	\item $\ds\sum_{n=0}^\infty (-1)^n x^{2n}/(2n)!$, $R=\infty$
	\item $\ds\sum_{n=0}^\infty x^n/n!$, $R=\infty$
	\item $\ds\sum_{n=0}^\infty (-1)^n{(x-5)^n\over 5^{n+1}}$, $R=5$
	\item $\ds\sum_{n=1}^\infty (-1)^{n-1}{(x-1)^n\over n}$, $R=1$
	\item $\ds\ln(2)+\sum_{n=1}^\infty (-1)^{n-1}{(x-2)^n\over n 2^n}$, $R=2$
	\item $\ds\sum_{n=0}^\infty (-1)^n(n+1)(x-1)^n$, $R=1$
	\item $\ds1+\sum_{n=1}^\infty {1\cdot3\cdot5\cdots(2n-1)\over
		n!2^n} x^n=1+\sum_{n=1}^\infty {(2n-1)!\over 2^{2n-1}(n-1)!\,n!}x^n$, $R=1$
	\item $\ds x+x^3/3$
	\item $\ds\sum_{n=0}^\infty (-1)^n x^{4n+1}/(2n)!$
	\item $\ds\sum_{n=0}^\infty (-1)^n x^{n+1}/n!$
\end{enumerate}
\end{Answer}
\begin{Answer}{10.1.2}
 $\ds y=\arctan t + C$
\end{Answer}
\begin{Answer}{10.1.3}
 $\ds y={t^{n+1}\over n+1}+1$
\end{Answer}
\begin{Answer}{10.1.4}
 $\ds y=t\ln t-t+C$
\end{Answer}
\begin{Answer}{10.1.5}
 $y=n\pi$, for any integer $n$.
\end{Answer}
\begin{Answer}{10.1.6}
 none
\end{Answer}
\begin{Answer}{10.1.7}
 $\ds y=\pm\sqrt{t^2+C}$
\end{Answer}
\begin{Answer}{10.1.8}
 $\ds y=\pm 1$, $\ds y=(1+Ae^{2t})/(1-Ae^{2t})$
\end{Answer}
\begin{Answer}{10.1.9}
 $\ds y^4/4-5y=t^2/2+C$
\end{Answer}
\begin{Answer}{10.1.10}
 $\ds y=(2t/3)^{3/2}$
\end{Answer}
\begin{Answer}{10.1.11}
 $\ds y=M+Ae^{-kt}$
\end{Answer}
\begin{Answer}{10.1.12}
 $\ds {10\ln(15/2)\over\ln 5}\approx 2.52$ minutes
\end{Answer}
\begin{Answer}{10.1.13}
 $\ds y={M\over 1+Ae^{-Mkt}}$
\end{Answer}
\begin{Answer}{10.1.14}
 $\ds y=2e^{3t/2}$
\end{Answer}
\begin{Answer}{10.1.15}
 $\ds t=-{\ln 2\over k}$
\end{Answer}
\begin{Answer}{10.1.16}
 $\ds 600e^{-6\ln 2/5}\approx 261$ mg; $\ds {5\ln
  300\over\ln2}\approx 41$ days
\end{Answer}
\begin{Answer}{10.1.17}
 $\ds 100e^{-200\ln 2/191}\approx 48$ mg; $\ds {5730\ln
  50\over\ln2}\approx 32339$ years
\end{Answer}
\begin{Answer}{10.1.18}
 $\ds y=y_0e^{t\ln 2}$
\end{Answer}
\begin{Answer}{10.1.19}
 $\ds 500e^{-5\ln2/4}\approx 210$ g
\end{Answer}
\begin{Answer}{10.2.1}
 $\ds y=Ae^{-5t}$
\end{Answer}
\begin{Answer}{10.2.2}
 $\ds y=Ae^{2t}$
\end{Answer}
\begin{Answer}{10.2.3}
 $\ds y=Ae^{-\arctan t}$
\end{Answer}
\begin{Answer}{10.2.4}
 $\ds y=Ae^{-t^3/3}$
\end{Answer}
\begin{Answer}{10.2.5}
 $\ds y=4e^{-t}$
\end{Answer}
\begin{Answer}{10.2.6}
 $\ds y=-2e^{3t-3}$
\end{Answer}
\begin{Answer}{10.2.7}
 $\ds y=e^{1+\cos t}$
\end{Answer}
\begin{Answer}{10.2.8}
 $\ds y=e^2e^{-e^t}$
\end{Answer}
\begin{Answer}{10.2.9}
 $\ds y=0$
\end{Answer}
\begin{Answer}{10.2.10}
 $\ds y=0$
\end{Answer}
\begin{Answer}{10.2.11}
 $\ds y=4t^2$
\end{Answer}
\begin{Answer}{10.2.12}
 $\ds y=-2e^{(1/t)-1}$
\end{Answer}
\begin{Answer}{10.2.13}
 $\ds y=e^{1-t^{-2}}$
\end{Answer}
\begin{Answer}{10.2.14}
 $\ds y=0$
\end{Answer}
\begin{Answer}{10.2.15}
 $k=\ln 5$, $\ds y=100e^{-t\ln 5}$
\end{Answer}
\begin{Answer}{10.2.16}
 $k=-12/13$, $\ds y=\exp(-13 t^{1/13})$
\end{Answer}
\begin{Answer}{10.2.17}
 $\ds y=10^6e^{t\ln(3/2)}$
\end{Answer}
\begin{Answer}{10.2.18}
 $\ds y=10e^{-t\ln(2)/6}$
\end{Answer}
\begin{Answer}{10.3.1}
 $\ds y=Ae^{-4t}+2$
\end{Answer}
\begin{Answer}{10.3.2}
 $\ds y=Ae^{2t}-3$
\end{Answer}
\begin{Answer}{10.3.3}
 $\ds y=Ae^{-(1/2)t^2}+5$
\end{Answer}
\begin{Answer}{10.3.4}
 $\ds y=Ae^{-e^t}-2$
\end{Answer}
\begin{Answer}{10.3.5}
 $\ds y=Ae^{t}-t^2-2t-2$
\end{Answer}
\begin{Answer}{10.3.6}
 $\ds y=Ae^{-t/2}+t-2$
\end{Answer}
\begin{Answer}{10.3.7}
 $\ds y=At^2-{1\over3t}$
\end{Answer}
\begin{Answer}{10.3.8}
 $\ds y={c\over t}+{2\over3}\sqrt t$
\end{Answer}
\begin{Answer}{10.3.9}
 $\ds y= A\cos t+\sin t$
\end{Answer}
\begin{Answer}{10.3.10}
 $\ds y= {A\over\sec t+\tan t}+1-{t\over\sec t+\tan t}$
\end{Answer}
\begin{Answer}{10.4.1}
 $y(1)\approx 1.355$
\end{Answer}
\begin{Answer}{10.4.2}
 $y(1)\approx 40.31$
\end{Answer}
\begin{Answer}{10.4.3}
 $y(1)\approx 1.05$
\end{Answer}
\begin{Answer}{10.4.4}
 $y(1)\approx 2.30$
\end{Answer}
\begin{Answer}{10.5.1}
 $\ds {\omega+1\over2\omega}e^{\omega t}+
{\omega-1\over2\omega}e^{-\omega t}$
\end{Answer}
\begin{Answer}{10.5.2}
 $\ds 2\cos(3t)+5\sin(3t)$
\end{Answer}
\begin{Answer}{10.5.3}
 $\ds -(1/4)e^{-5t}+(5/4)e^{-t}$
\end{Answer}
\begin{Answer}{10.5.4}
 $\ds-2e^{-3t}+2e^{4t}$
\end{Answer}
\begin{Answer}{10.5.5}
 $\ds 5e^{-6t}+20te^{-6t}$
\end{Answer}
\begin{Answer}{10.5.6}
 $\ds (16t-3)e^{4t}$
\end{Answer}
\begin{Answer}{10.5.7}
 $\ds -2\cos(\sqrt5t)+\sqrt{5}\sin(\sqrt{5}t)$
\end{Answer}
\begin{Answer}{10.5.8}
 $\ds -\sqrt2\cos t+\sqrt2\sin t$
\end{Answer}
\begin{Answer}{10.5.9}
 $\ds e^{-6t}\left(4\cos t+24\sin t\right)$
\end{Answer}
\begin{Answer}{10.5.10}
 $\ds 2e^{-3t}\sin(3t)$
\end{Answer}
\begin{Answer}{10.5.11}
 $\ds 2\cos(2t-\pi/6)$
\end{Answer}
\begin{Answer}{10.5.12}
 $\ds 5\sqrt2\cos(10t-\pi/4)$
\end{Answer}
\begin{Answer}{10.5.13}
 $\ds \sqrt2 e^{-2t}\cos(3t-\pi/4)$
\end{Answer}
\begin{Answer}{10.5.14}
 $\ds 5e^{4t}\cos(3t+\arcsin(4/5))$
\end{Answer}
\begin{Answer}{10.5.15}
 $\ds (2\cos(5t)+\sin(5t))e^{-2t}$
\end{Answer}
\begin{Answer}{10.5.16}
 $\ds-(1/2)e^{-2t}\sin(2t)$
\end{Answer}
\begin{Answer}{10.6.1}
 $Ae^{5t}+Bte^{5t}+(6/169)\cos t-(5/338)\sin t$
\end{Answer}
\begin{Answer}{10.6.2}
 $\ds Ae^{-\sqrt2t}+Bte^{-\sqrt2t}+5$
\end{Answer}
\begin{Answer}{10.6.3}
 $\ds A\cos(4t)+B\sin(4t)+ (1/2)t^2+(3/16)t-5/16$
\end{Answer}
\begin{Answer}{10.6.4}
 $\ds A\cos(\sqrt2t)+B\sin(\sqrt2t)-(\cos(5t)+\sin(5t))/23$
\end{Answer}
\begin{Answer}{10.6.5}
 $\ds e^{t}(A\cos t+B\sin t)+e^{2t}/2$
\end{Answer}
\begin{Answer}{10.6.6}
 $\ds Ae^{\sqrt6t}+Be^{-\sqrt6t}+2-t/3-e^{-t}/5$
\end{Answer}
\begin{Answer}{10.6.7}
 $\ds Ae^{-3t}+Be^{2t}-(1/5)te^{-3t}$
\end{Answer}
\begin{Answer}{10.6.8}
 $\ds Ae^t+Be^{3t}+(1/2)te^{3t}$
\end{Answer}
\begin{Answer}{10.6.9}
 $\ds A\cos(4t)+B\sin(4t)+(1/8)t\sin(4t)$
\end{Answer}
\begin{Answer}{10.6.10}
 $\ds A\cos(3t)+B\sin(3t)-(1/2)t\cos(3t)$
\end{Answer}
\begin{Answer}{10.6.11}
 $\ds Ae^{-6t}+Bte^{-6t}+3t^2e^{-6t}$
\end{Answer}
\begin{Answer}{10.6.12}
 $\ds Ae^{4t}+Bte^{4t}-t^2e^{4t}$
\end{Answer}
\begin{Answer}{10.6.13}
 $\ds Ae^{-t}+Be^{-5t}+(4/5)$
\end{Answer}
\begin{Answer}{10.6.14}
 $\ds Ae^{4t}+Be^{-3t}+(1/144)-(t/12)$
\end{Answer}
\begin{Answer}{10.6.15}
 $\ds A\cos(\sqrt5t)+B\sin(\sqrt5t)+8\sin(2t)$
\end{Answer}
\begin{Answer}{10.6.16}
 $\ds Ae^{2t}+Be^{-2t}+te^{2t}$
\end{Answer}
\begin{Answer}{10.6.17}
 $\ds 4e^{t}+e^{-t}-3t-5$
\end{Answer}
\begin{Answer}{10.6.18}
 $\ds -(4/27)\sin(3t)+(4/9)t$
\end{Answer}
\begin{Answer}{10.6.19}
 $\ds e^{-6t}(2\cos t+20\sin t)+2e^{-4t}$
\end{Answer}
\begin{Answer}{10.6.20}
 $\ds
\left(-{23\over 325}\cos(3t)+{592\over 975}\sin(3t)\right)+
{23\over325}\cos t-{11\over325}\sin t$
\end{Answer}
\begin{Answer}{10.6.21}
 $\ds e^{-2t}(A\sin(5t)+B\cos(5t))+8\sin(2t)+25\cos(2t)$
\end{Answer}
\begin{Answer}{10.6.22}
 $\ds e^{-2t}(A\sin(2t)+B\cos(2t))+(14/195)\sin t-(8/195)\cos t$
\end{Answer}
\begin{Answer}{10.7.1}
 $\ds A\sin(t)+B\cos(t)-\hfill\break\cos t\ln|\sec t+\tan t|$
\end{Answer}
\begin{Answer}{10.7.2}
 $\ds A\sin(t)+B\cos(t)+{1\over5}e^{2t}$
\end{Answer}
\begin{Answer}{10.7.3}
 $\ds A\sin(2t)+B\cos(2t)+\cos t-\sin t\cos t\ln|\sec t+\tan t|$
\end{Answer}
\begin{Answer}{10.7.4}
 $\ds A\sin(2t)+B\cos(2t)+{1\over2}\sin(2t)\sin^2(t)+
{1\over2}\sin(2t)\ln|\cos t|-{t\over2}\cos(2t)+{1\over4}\sin(2t)\cos(2t)$
\end{Answer}
\begin{Answer}{10.7.5}
 $\ds Ae^{2t}+Be^{-3t}+{t^3\over15}e^{2t}-\left({t^2\over5}
-{2t\over25}+{2\over125}\right){e^{2t}\over5}$
\end{Answer}
\begin{Answer}{10.7.6}
 $\ds Ae^{t}\sin t+Be^{t}\cos t-e^t\cos t\ln|\sec t+\tan t|$
\end{Answer}
\begin{Answer}{10.7.7}
 $\ds Ae^{t}\sin t+Be^{t}\cos t-
{1\over10}\cos t(\cos^3 t+3\sin^3 t-2\cos t-\sin t)+
{1\over10}\sin t(\sin^3 t-3\cos^3 t-2\sin t+\cos t)=
{1\over10}\cos(2t)-{1\over20}\sin(2t)$
\end{Answer}
\begin{Answer}{11.1.2}
\begin{multicols}{3}
\begin{enumerate}
	\item	$\ds \theta=\arctan(3)$
	\item	$\ds r=-4\csc\theta$
	\item	$\ds r=\sec\theta\csc^2\theta$
	\item	$\ds r=\sqrt{5}$
	\item	$\ds r^2=\sin\theta\sec^3\theta$
	\item	$\ds r\sin\theta=\sin(r\cos\theta)$
	\item	$\ds r=2/(\sin\theta-5\cos\theta)$
	\item	$\ds r=2\sec\theta$
	\item	$\ds 0=r^2\cos^2\theta-r\sin\theta+1$
\end{enumerate}
\end{multicols}
\end{Answer}
\begin{Answer}{11.1.4}
\begin{multicols}{2}
\begin{enumerate}
	\item	$\ds (x^2+y^2)^2=4x^2y-(x^2+y^2)y$
	\item	$\ds (x^2+y^2)^{3/2}=y^2$
	\item	$\ds x^2+y^2=x^2y^2$
	\item	$\ds x^4+x^2y^2=y^2$
\end{enumerate}
\end{multicols}
\end{Answer}
\begin{Answer}{11.2.1}
\begin{enumerate}
	\item	$(\theta\cos\theta+\sin\theta)/(-\theta\sin\theta+\cos\theta)$,
	$\ds (\theta^2+2)/(-\theta\sin\theta+\cos\theta)^3$
	\item	$\ds {\cos\theta+2\sin\theta\cos\theta\over
		\cos^2\theta-\sin^2\theta-\sin\theta}$,
	$\ds {3(1+\sin\theta)\over(\cos^2\theta-\sin^2\theta-\sin\theta)^3}$
	\item	$\ds (\sin^2\theta-\cos^2\theta)/(2\sin\theta\cos\theta)$,
	$\ds -1/(4\sin^3\theta\cos^3\theta)$
	\item	$\ds {2\sin\theta\cos\theta\over\cos^2\theta-\sin^2\theta}$,
	$\ds {2\over(\cos^2\theta-\sin^2\theta)^3}$
	\item	undefined
	\item	$\ds {2\sin\theta-3\sin^3\theta\over3\cos^3\theta-2\cos\theta}$,
	$\ds {3\cos^4\theta-3\cos^2\theta+2\over2\cos^3\theta(3\cos^2\theta-2)^3}$
\end{enumerate}
\end{Answer}
\begin{Answer}{11.3.1}
\begin{multicols}{2}
\begin{enumerate}
	\item	$1$
	\item	$9\pi/2$
	\item	$\ds \sqrt3/3$
	\item	$\ds \pi/12+\sqrt3/16$
	\item	$\ds \pi a^2/4$
	\item	$41\pi/2$
\end{enumerate}
\end{multicols}
\end{Answer}
\begin{Answer}{11.3.2}
 $2-\pi/2$
\end{Answer}
\begin{Answer}{11.3.3}
 $\pi/12$
\end{Answer}
\begin{Answer}{11.3.4}
 $3\pi/16$
\end{Answer}
\begin{Answer}{11.3.5}
 $\ds \pi/4-3\sqrt3/8$
\end{Answer}
\begin{Answer}{11.3.6}
 $\ds \pi/2+3\sqrt3/8$
\end{Answer}
\begin{Answer}{11.3.7}
 $1$
\end{Answer}
\begin{Answer}{11.3.8}
 $3/2-\pi/4$
\end{Answer}
\begin{Answer}{11.3.9}
 $\ds \pi/3+\sqrt3/2$
\end{Answer}
\begin{Answer}{11.3.10}
 $\ds \pi/3-\sqrt3/4$
\end{Answer}
\begin{Answer}{11.3.11}
 $\ds 4\pi^3/3$
\end{Answer}
\begin{Answer}{11.3.12}
 $\ds \pi^2$
\end{Answer}
\begin{Answer}{11.3.13}
 $\ds 5\pi/24-\sqrt3/4$
\end{Answer}
\begin{Answer}{11.3.14}
 $\ds 7\pi/12-\sqrt3$
\end{Answer}
\begin{Answer}{11.3.15}
 $\ds 4\pi-\sqrt{15}/2-7\arccos(1/4)$
\end{Answer}
\begin{Answer}{11.3.16}
 $\ds 3\pi^3$
\end{Answer}
\begin{Answer}{11.4.6}
 $\ds x=t-{\sin(t)\over2}$, $\ds t=1-{\cos(t)\over2}$
\end{Answer}
\begin{Answer}{11.5.1}
 There is a horizontal tangent at all multiples of $\pi$.
\end{Answer}
\begin{Answer}{11.5.2}
 $9\pi/4$
\end{Answer}
\begin{Answer}{11.5.3}
 $\ds \int_0^{2\pi}{1\over2} \sqrt{5-4\cos t}\;dt$
\end{Answer}
\begin{Answer}{12.1.6}
$3$, $\sqrt{26}$, $\sqrt{29}$
\end{Answer}
\begin{Answer}{12.1.7}
$\sqrt{14}$, $2\sqrt{14}$, $3\sqrt{14}$.
\end{Answer}
\begin{Answer}{12.1.8}
$(x-1)^2+(y-1)^2+(z-1)^2=4$.
\end{Answer}
\begin{Answer}{12.1.9}
$(x-2)^2+(y+1)^2+(z-3)^2=25$.
\end{Answer}
\begin{Answer}{12.1.11}
$(x-2)^2+(y-1)^2+(z+1)^2=16$,
$(y-1)^2+(z+1)^2=12$
\end{Answer}
\begin{Answer}{12.2.6}
$\ds \sqrt{10}$, $\langle 0,-2\rangle$, $\langle 2,8\rangle$
2, $\ds 2\sqrt{17}$, $\langle -2,-6\rangle$
\end{Answer}
\begin{Answer}{12.2.7}
$\ds \sqrt{14}$, $\langle 0,4,0\rangle$, $\langle 2,0,6\rangle$
4, $\ds 2\sqrt{10}$, $\langle -2,-4,-6\rangle$
\end{Answer}
\begin{Answer}{12.2.8}
$\ds \sqrt{2}$, $\langle 0,-2,3\rangle$, $\langle 2,2,-1\rangle$
$\ds \sqrt{13}$, $3$, $\langle -2, 0, -2\rangle$
\end{Answer}
\begin{Answer}{12.2.9}
$\ds \sqrt{3}$, $\langle 1,-1,4\rangle$, $\langle 1,-1,-2\rangle$
$\ds 3\sqrt{2}$, $\ds \sqrt{6}$, $\langle -2, 2, -2\rangle$
\end{Answer}
\begin{Answer}{12.2.10}
$\ds \sqrt{14}$, $\langle 2,1,0\rangle$, $\langle 4,3,2\rangle$
$\ds \sqrt{5}$, $\ds \sqrt{29}$, $\langle -6,-4, -2\rangle$
\end{Answer}
\begin{Answer}{12.2.11}
$\langle -3, -3, -11\rangle$,
$\langle -3/\sqrt{139},-3/\sqrt{139},-11/\sqrt{139}\rangle$
$\langle -12/\sqrt{139},-12/\sqrt{139},-44/\sqrt{139}\rangle$
\end{Answer}
\begin{Answer}{12.2.12}
$\langle 0,0,0\rangle$
\end{Answer}
\begin{Answer}{12.2.13}
$\vect{0}$; $\langle -r\sqrt3/2,r/2\rangle$; $\langle
0,-12r\rangle$; where $r$ is the radius of the clock
\end{Answer}
\begin{Answer}{12.3.1}
	$3$
\end{Answer}
\begin{Answer}{12.3.2}
	$0$
\end{Answer}
\begin{Answer}{12.3.3}
	$2$
\end{Answer}
\begin{Answer}{12.3.4}
	$-6$
\end{Answer}
\begin{Answer}{12.3.5}
	$42$
\end{Answer}
\begin{Answer}{12.3.6}
	$\sqrt6/\sqrt7$, $\approx 0.39$
\end{Answer}
\begin{Answer}{12.3.7}
	$-11\sqrt{14}\sqrt{29}/406$, $\approx 2.15$
\end{Answer}
\begin{Answer}{12.3.8}
	$0$, $\pi/2$
\end{Answer}
\begin{Answer}{12.3.9}
	$1/2$, $\pi/3$
\end{Answer}
\begin{Answer}{12.3.10}
	$-1/\sqrt3$, $\approx 2.19$
\end{Answer}
\begin{Answer}{12.3.11}
	$\arccos(1/\sqrt3)\approx 0.96$
\end{Answer}
\begin{Answer}{12.3.12}
	$\sqrt{5}$, $\langle 1,2,0\rangle$.
\end{Answer}
\begin{Answer}{12.3.13}
	$3\sqrt{14}/7$, $\langle 9/7,6/7,3/7\rangle$.
\end{Answer}
\begin{Answer}{12.3.14}
	$\langle 0,5\rangle$, $\langle 5\sqrt3,0\rangle$
\end{Answer}
\begin{Answer}{12.3.15}
	$\langle 0,15\sqrt2/2\rangle$,$\langle 15\sqrt2/2,0\rangle$
\end{Answer}
\begin{Answer}{12.3.16}
	Any vector of the form $\langle a, -7a/2, -2a\rangle$
\end{Answer}
\begin{Answer}{12.3.17}
	$\langle 1/\sqrt3,-1/\sqrt3,1/\sqrt3\rangle$
\end{Answer}
\begin{Answer}{12.3.18}
	No.
\end{Answer}
\begin{Answer}{12.3.19}
	Yes.
\end{Answer}
\begin{Answer}{12.4.1}
	$\langle 1,-2,1\rangle$
\end{Answer}
\begin{Answer}{12.4.2}
	$\langle 4,-6,-2\rangle$
\end{Answer}
\begin{Answer}{12.4.3}
	$\langle -7,13,-9\rangle$
\end{Answer}
\begin{Answer}{12.4.4}
	$\langle 0,-1,0\rangle$
\end{Answer}
\begin{Answer}{12.4.5}
	$3$
\end{Answer}
\begin{Answer}{12.4.6}
	$21\sqrt2/2$
\end{Answer}
\begin{Answer}{12.4.7}
	$1$
\end{Answer}
\begin{Answer}{12.5.1}
	$(x-6)+(y-2)+(z-1)=0$
\end{Answer}
\begin{Answer}{12.5.2}
	$4(x+1)+5(y-2)-(z+3)=0$
\end{Answer}
\begin{Answer}{12.5.3}
	$(x-1)-(y-2)=0$
\end{Answer}
\begin{Answer}{12.5.4}
	$-2(x-1)+3y-2z=0$
\end{Answer}
\begin{Answer}{12.5.5}
	$4(x-1)-6y = 0$
\end{Answer}
\begin{Answer}{12.5.6}
	$x+3y=0$
\end{Answer}
\begin{Answer}{12.5.7}
	$\langle 1,0,3\rangle+t\langle 0,2,1\rangle$
\end{Answer}
\begin{Answer}{12.5.8}
	$\langle 1,0,3\rangle+t\langle 1,2,-1\rangle$
\end{Answer}
\begin{Answer}{12.5.9}
	$t\langle 1,1,-1\rangle$
\end{Answer}
\begin{Answer}{12.5.10}
	$-2/5$, $13/5$
\end{Answer}
\begin{Answer}{12.5.12}
	neither
\end{Answer}
\begin{Answer}{12.5.13}
	parallel
\end{Answer}
\begin{Answer}{12.5.14}
	intersect at $(3,6,5)$
\end{Answer}
\begin{Answer}{12.5.15}
	same line
\end{Answer}
\begin{Answer}{12.5.19}
	$7/\sqrt3$
\end{Answer}
\begin{Answer}{12.5.20}
	$4/\sqrt{14}$
\end{Answer}
\begin{Answer}{12.5.21}
	$\sqrt{131}/\sqrt{14}$
\end{Answer}
\begin{Answer}{12.5.22}
	$\sqrt{68}/3$
\end{Answer}
\begin{Answer}{12.5.23}
	$\sqrt{42}/7$
\end{Answer}
\begin{Answer}{12.5.24}
	$\sqrt{21}/6$
\end{Answer}
\begin{Answer}{12.6.1}
\begin{enumerate}
	\item	$\ds (\sqrt2,\pi/4,1)$, $\ds (\sqrt3,\pi/4,\arccos(1/\sqrt3))$
	\item	$\ds (7\sqrt2,7\pi/4,5)$, $\ds (\sqrt{123},7\pi/4,\arccos(5/\sqrt{123})$
	\item	$(1,1,1)$, $\ds (\sqrt2,1,\pi/4)$
	\item	$(0,0,-\pi)$, $(\pi,0,\pi)$
\end{enumerate}
\end{Answer}
\begin{Answer}{12.6.2}
	$r^2+z^2=4$
\end{Answer}
\begin{Answer}{12.6.3}
	$r\cos\theta=0$
\end{Answer}
\begin{Answer}{12.6.4}
	$r^2+2z^2+2z-5=0$
\end{Answer}
\begin{Answer}{12.6.5}
	$z=e^{-r^2}$
\end{Answer}
\begin{Answer}{12.6.6}
	$z=r$
\end{Answer}
\begin{Answer}{12.6.7}
	$\sin\theta=0$
\end{Answer}
\begin{Answer}{12.6.8}
	$1=\rho\cos\phi$
\end{Answer}
\begin{Answer}{12.6.9}
	$\rho=2\sin\theta\sin\phi$.
\end{Answer}
\begin{Answer}{12.6.10}
	$\rho\sin\phi=2$
\end{Answer}
\begin{Answer}{12.6.11}
	$\cos\phi=1/\sqrt2$
\end{Answer}
\begin{Answer}{12.6.13}
	$z=mr$; $\cot\phi=m$ if $m\neq0$, $\phi=0$ if $m=0$
\end{Answer}
\begin{Answer}{12.6.14}
	A sphere with radius $1/2$, center at $(0,1/2,0)$
\end{Answer}
\begin{Answer}{12.6.15}
	$0<\theta<\pi/2$, $0<\phi<\pi/2$, $\rho>0$;
$0<\theta<\pi/2$, $r>0$, $z>0$
\end{Answer}
\begin{Answer}{13.1.1}
$z=y^2$, $z=x^2$, $z=0$, lines of slope 1
\end{Answer}
\begin{Answer}{13.1.2}
$z=|y|$, $z=|x|$, $z=2|x|$, diamonds
\end{Answer}
\begin{Answer}{13.1.3}
$z=e^{-y^2}\sin(y^2)$, $z=e^{-x^2}\sin(x^2)$,
$z=e^{-2x^2}\sin(2x^2)$, circles
\end{Answer}
\begin{Answer}{13.1.4}
$z=-\sin(y)$, $z=\sin(x)$,
$z=0$, lines of slope 1
\end{Answer}
\begin{Answer}{13.1.5}
$z=y^4$, $z=x^4$,
$z=0$, hyperbolas
\end{Answer}
\begin{Answer}{13.1.6}
\begin{enumerate}
	\item	$\{(x,y)\mid |x|\le3\ \hbox{and}\ |y|\ge2\}$
	\item	$\{(x,y)\mid 1\le x^2+y^2\le3\}$
	\item	$\{(x,y)\mid x^2+4y^2\le16\}$
\end{enumerate}
\end{Answer}
\begin{Answer}{13.2.1}
No limit; use $x=0$ and $y=0$.
\end{Answer}
\begin{Answer}{13.2.2}
No limit; use $x=0$ and $x=y$.
\end{Answer}
\begin{Answer}{13.2.3}
No limit; use $x=0$ and $x=y$.
\end{Answer}
\begin{Answer}{13.2.4}
Limit is zero.
\end{Answer}
\begin{Answer}{13.2.5}
Limit is 1.
\end{Answer}
\begin{Answer}{13.2.6}
Limit is zero.
\end{Answer}
\begin{Answer}{13.2.7}
Limit is $-1$.
\end{Answer}
\begin{Answer}{13.2.8}
Limit is zero.
\end{Answer}
\begin{Answer}{13.2.9}
No limit; use $x=0$ and $y=0$.
\end{Answer}
\begin{Answer}{13.2.10}
Limit is zero.
\end{Answer}
\begin{Answer}{13.2.11}
Limit is $-1$.
\end{Answer}
\begin{Answer}{13.2.12}
Limit is zero.
\end{Answer}
\begin{Answer}{13.3.1}
$-2xy\sin(x^2y)$, $-x^2\sin(x^2y)+3y^2$
\end{Answer}
\begin{Answer}{13.3.2}
$(y^2-x^2y)/(x^2+y)^2$, $x^3/(x^2+y)^2$
\end{Answer}
\begin{Answer}{13.3.3}
$2xe^{x^2+y^2}$, $2ye^{x^2+y^2}$
\end{Answer}
\begin{Answer}{13.3.4}
$y\ln(xy)+y$, $x\ln(xy)+x$
\end{Answer}
\begin{Answer}{13.3.5}
$-x/\sqrt{1-x^2-y^2}$, $-y/\sqrt{1-x^2-y^2}$
\end{Answer}
\begin{Answer}{13.3.6}
$\tan y$, $x\sec^2 y$
\end{Answer}
\begin{Answer}{13.3.7}
$-1/(x^2y)$, $-1/(xy^2)$
\end{Answer}
\begin{Answer}{13.3.8}
$z=-2(x-1)-3(y-1)-1$
\end{Answer}
\begin{Answer}{13.3.9}
$z=1$
\end{Answer}
\begin{Answer}{13.3.10}
$z=6(x-3)+3(y-1)+10$
\end{Answer}
\begin{Answer}{13.3.11}
$z=(x-2)+4(y-1/2)$
\end{Answer}
\begin{Answer}{13.3.12}
${\bf r}(t)=\langle 2,1,4\rangle+t\langle 2,4,-1\rangle$
\end{Answer}
\begin{Answer}{13.3.16}
		height
	
\end{Answer}
\begin{Answer}{13.4.1}
$4xt\cos(x^2+y^2)+6yt^2\cos(x^2+y^2)$
\end{Answer}
\begin{Answer}{13.4.2}
$2xy\cos t+2x^2t$
\end{Answer}
\begin{Answer}{13.4.3}
$2xyt\cos(st)+2x^2s$, $2xys\cos(st)+2x^2t$
\end{Answer}
\begin{Answer}{13.4.4}
$2xy^2t-4yx^2s$, $2xy^2s+4yx^2t$
\end{Answer}
\begin{Answer}{13.4.5}
$x/z$, $3y/(2z)$
\end{Answer}
\begin{Answer}{13.4.6}
$-2x/z$, $-y/z$
\end{Answer}
\begin{Answer}{13.4.7}
\begin{enumerate}
	\item	$\ds V'=(nR-0.2V)/P$
	\item	$\ds P'=(nR+0.6P)/2V$
	\item	$\ds T' = (3P-0.4V)/(nR)$
\end{enumerate}
\end{Answer}
\begin{Answer}{13.5.1}
$9\sqrt5/5$
\end{Answer}
\begin{Answer}{13.5.2}
$\sqrt2\cos3$
\end{Answer}
\begin{Answer}{13.5.3}
$e\sqrt2(\sqrt3-1)/4$
\end{Answer}
\begin{Answer}{13.5.4}
$\sqrt3+5$
\end{Answer}
\begin{Answer}{13.5.5}
$-\sqrt6(2+\sqrt3)/72$
\end{Answer}
\begin{Answer}{13.5.6}
$-1/5$, $0$
\end{Answer}
\begin{Answer}{13.5.7}
$4(x-2)+8(y-1)=0$
\end{Answer}
\begin{Answer}{13.5.8}
$2(x-3)+3(y-2)=0$
\end{Answer}
\begin{Answer}{13.5.9}
$\langle -1,-1-\cos1,-\cos1\rangle$, $-\sqrt{2+2\cos1+2\cos^21}$
\end{Answer}
\begin{Answer}{13.5.10}
Any direction perpendicular to $\nabla T=\langle
1,1,1\rangle$,
for example, $\langle -1,1,0\rangle$
\end{Answer}
\begin{Answer}{13.5.11}
$2(x-1)-6(y-1)+6(z-3)=0$
\end{Answer}
\begin{Answer}{13.5.12}
$6(x-1)+3(y-2)+2(z-3)=0$
\end{Answer}
\begin{Answer}{13.5.13}
$\langle 2+4t,-3-12t,-1-8t\rangle$
\end{Answer}
\begin{Answer}{13.5.14}
$\langle 4+8t,2+4t,-2-36t\rangle$
\end{Answer}
\begin{Answer}{13.5.15}
$\langle 4+8t,2+20t,6-12t\rangle$
\end{Answer}
\begin{Answer}{13.5.16}
$\langle 0,1\rangle$, $\langle 4/5,-3/5\rangle$
\end{Answer}
\begin{Answer}{13.5.18}
\begin{enumerate}
	\item	$\langle 4,9\rangle$
	\item	$\langle -81,2\rangle$ or $\langle 81,-2\rangle$
\end{enumerate}
\end{Answer}
\begin{Answer}{13.5.19}
in the direction of $\langle 8,1\rangle$
\end{Answer}
\begin{Answer}{13.5.20}
$\ds \nabla g(-1,3)=\langle 2,1\rangle$
\end{Answer}
\begin{Answer}{13.6.1}
$f_{xx}=(2x^3y-6xy^3)/(x^2+y^2)^3$,
$f_{yy}=(2xy^3-6x^3y)/(x^2+y^2)^3$
\end{Answer}
\begin{Answer}{13.6.2}
$f_x=3x^2y^2$, $f_y=2x^3y+5y^4$,
$f_{xx}=6xy^2$, $f_{yy}=2x^3+20y^3$, $f_{xy}=6x^2y$
\end{Answer}
\begin{Answer}{13.6.3}
$f_x=12x^2+y^2$, $f_y=2xy$, \hfill\break
$f_{xx}=24x$, $f_{yy}=2x$, $f_{xy}=2y$
\end{Answer}
\begin{Answer}{13.6.4}
$f_x=\sin y$, $f_y=x\cos y$, $f_{xx}=0$, $f_{yy}=-x\sin y$,
$f_{xy}=\cos y$
\end{Answer}
\begin{Answer}{13.6.5}
$\ds f_x=3\cos(3x)\cos(2y)$,\hfill\break
$\ds f_y=-2\sin(3x)\sin(2y)$,\hfill\break
$\ds f_{xy}=-6\cos(3x)\sin(2y)$,\hfill\break
$\ds f_{yy}=-4\sin(3x)\cos(2y)$,\hfill\break
$\ds f_{xx}=-9\sin(3x)\cos(2y)$
\end{Answer}
\begin{Answer}{13.6.6}
$\ds f_x=e^{x+y^2}$, $\ds f_y=2ye^{x+y^2}$,\hfill\break
$\ds f_{xx}=e^{x+y^2}$,\hfill\break
$\ds f_{yy}=4y^2e^{x+y^2}+2e^{x+y^2}$,\hfill\break
$\ds f_{xy}=2ye^{x+y^2}$
\end{Answer}
\begin{Answer}{13.6.7}
$\ds f_x={3x^2\over2(x^3+y^4)}$,
$\ds f_y={2y^3\over x^3+y^4}$,
$\ds f_{xx}={3x\over x^3+y^4}-{9x^4\over 2(x^3+y^4)^2}$,
$\ds f_{yy}={6y^2\over x^3+y^4}-{8y^6\over (x^3+y^4)^2}$,\hfill\break
$\ds f_{xy}={-6x^2y^3\over (x^3+y^4)^2}$
\end{Answer}
\begin{Answer}{13.6.8}
$\ds z_x={-x\over16z}$,
$\ds z_y={-y\over4z}$,\hfill\break
$\ds z_{xx}=-{16z^2+x^2\over16^2z^3}$,\hfill\break
$\ds z_{yy}=-{4z^2+y^2\over16z^3}$,\hfill\break
$\ds z_{xy}={-xy\over64z^3}$
\end{Answer}
\begin{Answer}{13.6.9}
$\ds z_x=-{y+z\over x+y}$,
$\ds z_y=-{x+z\over x+y}$,\hfill\break
$\ds z_{xx}=2{y+z\over(x+y)^2}$,
$\ds z_{yy}=2{x+z\over(x+y)^2}$,\hfill\break
$\ds z_{xy}={2z\over(x+y)^2}$
\end{Answer}
\begin{Answer}{13.7.1}
minimum at $(1,-1)$
\end{Answer}
\begin{Answer}{13.7.2}
none
\end{Answer}
\begin{Answer}{13.7.3}
none
\end{Answer}
\begin{Answer}{13.7.4}
maximum at $(1,-1/6)$
\end{Answer}
\begin{Answer}{13.7.5}
none
\end{Answer}
\begin{Answer}{13.7.6}
minimum at $(2,-1)$
\end{Answer}
\begin{Answer}{13.7.7}
$f(2,2)=-2$, $f(2,0)=4$
\end{Answer}
\begin{Answer}{13.7.8}
a cube $1/\root 3 \of {2}$ on a side
\end{Answer}
\begin{Answer}{13.7.9}
$65/3\times 65/3\times 130/3$
\end{Answer}
\begin{Answer}{13.7.10}
It has a square base, and is one and one half times as tall as wide.
If the volume is $V$ the dimensions are $\root 3 \of {2V/3}\times
\root 3 \of {2V/3}\times \root 3\of {9V/4}$.
\end{Answer}
\begin{Answer}{13.7.11}
$\sqrt{100/3}$
\end{Answer}
\begin{Answer}{13.7.12}
$|ax_0+by_0+cz_0-d|/\sqrt{a^2+b^2+c^2}$
\end{Answer}
\begin{Answer}{13.7.13}
The sides and bottom should all be $2/3$ meter, and the sides
should be bent up at angle $\pi/3$.
\end{Answer}
\begin{Answer}{13.7.14}
$(3,4/3)$
\end{Answer}
\begin{Answer}{13.7.16}
$|b|$ if $b\le1/2$, otherwise $\ds\sqrt{b-1/4}$
\end{Answer}
\begin{Answer}{13.7.17}
$|b|$ if $b\le1/2$, otherwise $\ds\sqrt{b-1/4}$
\end{Answer}
\begin{Answer}{13.7.19}
$\ds 1024/\sqrt3$
\end{Answer}
\begin{Answer}{13.8.1}
a cube, $\root 3 \of {1/2}\times\root 3 \of {1/2}\times\root 3 \of {1/2}$
\end{Answer}
\begin{Answer}{13.8.2}
$65/3\cdot 65/3\cdot 130/3=2\cdot 65^3/27$
\end{Answer}
\begin{Answer}{13.8.3}
It has a square base, and is one and one half times as tall as wide.
If the volume is $V$ the dimensions are $\root 3 \of {2V/3}\times
\root 3 \of {2V/3}\times \root 3\of {9V/4}$.
\end{Answer}
\begin{Answer}{13.8.4}
$|ax_0+by_0+cz_0-d|/\sqrt{a^2+b^2+c^2}$
\end{Answer}
\begin{Answer}{13.8.5}
$(0,0,1)$, $(0,0,-1)$
\end{Answer}
\begin{Answer}{13.8.6}
$\root 3\of{4V}\times\root 3\of{4V}\times\root 3\of{V/16}$
\end{Answer}
\begin{Answer}{13.8.7}
Farthest: $(-\sqrt2,\sqrt2,2+2\sqrt2)$; closest:
$(2,0,0)$, $(0,-2,0)$
\end{Answer}
\begin{Answer}{13.8.8}
$x=y=z=16$
\end{Answer}
\begin{Answer}{13.8.9}
$(1,2,2)$
\end{Answer}
\begin{Answer}{13.8.10}
$\ds (\sqrt{5},0,0)$, $\ds (-\sqrt{5},0,0)$
\end{Answer}
\begin{Answer}{13.8.11}
standard \$65, deluxe \$75
\end{Answer}
\begin{Answer}{13.8.12}
$x=9$, $\phi=\pi/3$
\end{Answer}
\begin{Answer}{13.8.13}
$35$, $-35$
\end{Answer}
\begin{Answer}{13.8.14}
maximum $e^4$, no minimum
\end{Answer}
\begin{Answer}{13.8.15}
$5$, $-9/2$
\end{Answer}
\begin{Answer}{13.8.16}
$3$, $3$, $3$
\end{Answer}
\begin{Answer}{13.8.17}
a cube of side length $\ds 2/\sqrt{3}$
\end{Answer}
\begin{Answer}{14.1.1}
$16$
\end{Answer}
\begin{Answer}{14.1.2}
$4$
\end{Answer}
\begin{Answer}{14.1.3}
$15/8$
\end{Answer}
\begin{Answer}{14.1.4}
$1/2$
\end{Answer}
\begin{Answer}{14.1.5}
$5/6$
\end{Answer}
\begin{Answer}{14.1.6}
$12-65/(2e)$.
\end{Answer}
\begin{Answer}{14.1.7}
$1/2$
\end{Answer}
\begin{Answer}{14.1.8}
$\pi/64$
\end{Answer}
\begin{Answer}{14.1.9}
$(2/9)2^{3/2}-(2/9)$
\end{Answer}
\begin{Answer}{14.1.10}
$(1-\cos(1))/4$
\end{Answer}
\begin{Answer}{14.1.11}
$(2\sqrt2-1)/6$
\end{Answer}
\begin{Answer}{14.1.12}
$\pi-2$
\end{Answer}
\begin{Answer}{14.1.13}
$(e^9-1)/6$
\end{Answer}
\begin{Answer}{14.1.14}
$\ds {4\over15}-{\pi\over4}$
\end{Answer}
\begin{Answer}{14.1.15}
$1/3$
\end{Answer}
\begin{Answer}{14.1.16}
$448$
\end{Answer}
\begin{Answer}{14.1.17}
$4/5$
\end{Answer}
\begin{Answer}{14.1.18}
$8\pi$
\end{Answer}
\begin{Answer}{14.1.19}
$2$
\end{Answer}
\begin{Answer}{14.1.20}
$5/3$
\end{Answer}
\begin{Answer}{14.1.21}
$81/2$
\end{Answer}
\begin{Answer}{14.1.22}
$2a^3/3$
\end{Answer}
\begin{Answer}{14.1.23}
$4\pi$
\end{Answer}
\begin{Answer}{14.1.24}
$\pi/32$
\end{Answer}
\begin{Answer}{14.1.25}
$31/8$
\end{Answer}
\begin{Answer}{14.1.26}
$128/15$
\end{Answer}
\begin{Answer}{14.1.27}
$1800\pi$ ${\rm m}^3$
\end{Answer}
\begin{Answer}{14.1.28}
$\ds{(e^2+8e+16)\over15}\sqrt{e+4}-{5\sqrt5\over3}-{e^{5/2}\over15}
+{1\over15}$
\end{Answer}
\begin{Answer}{14.1.30}
$16-8\sqrt{2}$
\end{Answer}
\begin{Answer}{14.2.1}
$4\pi$
\end{Answer}
\begin{Answer}{14.2.2}
$32\pi/3-4\sqrt3\pi$
\end{Answer}
\begin{Answer}{14.2.3}
$(2-\sqrt2)\pi/3$
\end{Answer}
\begin{Answer}{14.2.4}
$4/9$
\end{Answer}
\begin{Answer}{14.2.5}
$5\pi/3$
\end{Answer}
\begin{Answer}{14.2.6}
$\pi/6$
\end{Answer}
\begin{Answer}{14.2.7}
$\pi/2$
\end{Answer}
\begin{Answer}{14.2.8}
$\pi/2-1$
\end{Answer}
\begin{Answer}{14.2.9}
$\sqrt3/4+\pi/6$
\end{Answer}
\begin{Answer}{14.2.10}
$8+\pi$
\end{Answer}
\begin{Answer}{14.2.11}
$\pi/12$\
\end{Answer}
\begin{Answer}{14.2.12}
$(1-\cos(9))\pi/2$
\end{Answer}
\begin{Answer}{14.2.13}
$-a^5/15$
\end{Answer}
\begin{Answer}{14.2.14}
$12\pi$
\end{Answer}
\begin{Answer}{14.2.15}
$\pi$
\end{Answer}
\begin{Answer}{14.2.16}
$16/3$
\end{Answer}
\begin{Answer}{14.2.17}
$21\pi$
\end{Answer}
\begin{Answer}{3}
		$2\pi$
		
\end{Answer}
\begin{Answer}{14.3.1}
$\bar x=\bar y=2/3$
\end{Answer}
\begin{Answer}{14.3.2}
$\bar x=4/5$, $\bar y=8/15$
\end{Answer}
\begin{Answer}{14.3.3}
$\bar x=0$, $\bar y=3\pi/16$
\end{Answer}
\begin{Answer}{14.3.4}
$\bar x=0$, $\bar y=16/(15\pi)$
\end{Answer}
\begin{Answer}{14.3.5}
$\bar x=3/2$, $\bar y=9/4$
\end{Answer}
\begin{Answer}{14.3.6}
$\bar x=6/5$, $\bar y=12/5$
\end{Answer}
\begin{Answer}{14.3.7}
$\bar x=14/27$, $\bar y=28/55$
\end{Answer}
\begin{Answer}{14.3.8}
$(3/4,2/5)$
\end{Answer}
\begin{Answer}{14.3.9}
$\ds\left({81\sqrt3\over80\pi},0\right)$
\end{Answer}
\begin{Answer}{14.3.10}
$\bar x=\pi/2$, $\bar y=\pi/8$
\end{Answer}
\begin{Answer}{14.3.11}
$\ds M=\int_0^{2\pi} \int_0^{1+\cos\theta} (2+\cos\theta)r\,dr\,d\theta$,
\hfill\break
$\ds M_x=\int_0^{2\pi} \int_0^{1+\cos\theta} \sin\theta(2+\cos\theta)r^2\,dr\,d\theta$,
\hfill\break
$\ds M_y=\int_0^{2\pi} \int_0^{1+\cos\theta} \cos\theta(2+\cos\theta)r^2\,dr\,d\theta$.
\end{Answer}
\begin{Answer}{14.3.12}
$\ds M=\int_{-\pi/2}^{\pi/2} \int_0^{\cos\theta} (r+1)r\,dr\,d\theta$,
\hfill\break
$\ds M_x=\int_{-\pi/2}^{\pi/2} \int_0^{\cos\theta} \sin\theta(r+1)r^2\,dr\,d\theta$,
\hfill\break
$\ds M_y=\int_{-\pi/2}^{\pi/2} \int_0^{\cos\theta} \cos\theta(r+1)r^2\,dr\,d\theta$.
\end{Answer}
\begin{Answer}{14.3.13}
$\ds M= \int_{-\pi/2}^{\pi/2}\int_{\cos\theta}^{1+\cos\theta}
r\,dr\,d\theta + \int_{\pi/2}^{3\pi/2}\int_0^{1+\cos\theta}r\,dr\,d\theta$,
\hfill\break
$\ds M_x=\int_{-\pi/2}^{\pi/2}\int_{\cos\theta}^{1+\cos\theta}
r^2\sin\theta\,dr\,d\theta + \int_{\pi/2}^{3\pi/2}\int_0^{1+\cos\theta}r^2\sin\theta\,dr\,d\theta$,
\hfill\break
$\ds M_y=\int_{-\pi/2}^{\pi/2}\int_{\cos\theta}^{1+\cos\theta}
r^2\cos\theta\,dr\,d\theta + \int_{\pi/2}^{3\pi/2}\int_0^{1+\cos\theta}r^2\cos\theta\,dr\,d\theta$.
\end{Answer}
\begin{Answer}{14.4.1}
$\pi a\sqrt{h^2+a^2}$
\end{Answer}
\begin{Answer}{14.4.2}
$\pi a^2\sqrt{m^2+1}$
\end{Answer}
\begin{Answer}{14.4.3}
$\sqrt3/2$
\end{Answer}
\begin{Answer}{14.4.4}
$\pi\sqrt2$
\end{Answer}
\begin{Answer}{14.4.5}
$\pi\sqrt2/8$
\end{Answer}
\begin{Answer}{14.4.6}
$\pi/2-1$
\end{Answer}
\begin{Answer}{14.4.7}
$\ds {d^2\sqrt{a^2+b^2+c^2}\over 2abc}$
\end{Answer}
\begin{Answer}{14.4.8}
$8\sqrt{3}\pi/3$
\end{Answer}
\begin{Answer}{14.5.1}
$11/24$
\end{Answer}
\begin{Answer}{14.5.2}
$623/60$
\end{Answer}
\begin{Answer}{14.5.3}
$-3e^2/4+2e-3/4$
\end{Answer}
\begin{Answer}{14.5.4}
$1/20$
\end{Answer}
\begin{Answer}{14.5.5}
$\pi/48$
\end{Answer}
\begin{Answer}{14.5.6}
$11/84$
\end{Answer}
\begin{Answer}{14.5.7}
$151/60$
\end{Answer}
\begin{Answer}{14.5.8}
$\pi$
\end{Answer}
\begin{Answer}{14.5.10}
$\ds {3\pi\over16}$
\end{Answer}
\begin{Answer}{14.5.11}
$32$
\end{Answer}
\begin{Answer}{14.5.12}
$64/3$
\end{Answer}
\begin{Answer}{14.5.13}
$\bar x=\bar y=0$, $\bar z=16/15$
\end{Answer}
\begin{Answer}{14.5.14}
$\bar x=\bar y=0$, $\bar z=1/3$
\end{Answer}
\begin{Answer}{14.6.1}
$\pi/12$
\end{Answer}
\begin{Answer}{14.6.2}
$\pi(1-\sqrt2/2)$
\end{Answer}
\begin{Answer}{14.6.3}
$5\pi/4$
\end{Answer}
\begin{Answer}{14.6.4}
$0$
\end{Answer}
\begin{Answer}{14.6.5}
$5\pi/4$
\end{Answer}
\begin{Answer}{14.6.6}
$4/5$
\end{Answer}
\begin{Answer}{14.6.7}
$256\pi/15$
\end{Answer}
\begin{Answer}{14.6.8}
$4\pi^2$
\end{Answer}
\begin{Answer}{14.6.9}
$\ds {3\pi\over16}$
\end{Answer}
\begin{Answer}{14.6.10}
$\pi kh^2a^2/12$
\end{Answer}
\begin{Answer}{14.6.11}
$\pi kha^3/6$
\end{Answer}
\begin{Answer}{14.6.12}
$\pi^2/4$
\end{Answer}
\begin{Answer}{14.6.13}
$4\pi/5$
\end{Answer}
\begin{Answer}{14.6.14}
$15\pi$
\end{Answer}
\begin{Answer}{14.6.15}
$9k\pi(5\sqrt2-2\sqrt5)/20$
\end{Answer}
\begin{Answer}{14.7.1}
$\ds 4\pi\sqrt3/3$
\end{Answer}
\begin{Answer}{14.7.2}
$0$
\end{Answer}
\begin{Answer}{14.7.3}
$2/3$
\end{Answer}
\begin{Answer}{14.7.4}
$\ds {e^2-1\over 2e^2}$
\end{Answer}
\begin{Answer}{14.7.5}
$36$
\end{Answer}
\begin{Answer}{14.7.6}
$32(\sqrt2+\ln(1+\sqrt2))/3$
\end{Answer}
\begin{Answer}{14.7.7}
$3\cos(1)-3\cos(4)$
\end{Answer}
\begin{Answer}{14.7.8}
$\pi(1-\cos(1))/24$
\end{Answer}
\begin{Answer}{14.7.10}
$(4/3)\pi abc$
\end{Answer}
\begin{Answer}{15.1.5}
	$\langle 3\cos t, 3\sin t, 2-3\sin t\rangle$
\end{Answer}
\begin{Answer}{15.1.6}
	$\langle 0,t\cos t,t\sin t\rangle$
\end{Answer}
\begin{Answer}{15.2.1}
	$\langle 2t,0,1\rangle$, $\vect{r}'/\sqrt{1+4t^2}$
\end{Answer}
\begin{Answer}{15.2.2}
	$\langle -\sin t, 2\cos 2t,2t\rangle$,
	$\vect{r}'/\sqrt{\sin^2t + 4\cos^2(2t)+4t^2}$
\end{Answer}
\begin{Answer}{15.2.3}
 $\langle -e^t\sin(e^t),e^t\cos(e^t),\cos t\rangle$,
$\vect{r}'/\sqrt{e^{2t}+\cos^2 t}$
\end{Answer}
\begin{Answer}{15.2.4}
 $\langle \sqrt2/2,\sqrt2/2,\pi/4\rangle+
t\langle -\sqrt2/2,\sqrt2/2,1\rangle$
\end{Answer}
\begin{Answer}{15.2.5}
 $\langle 1/2,\sqrt3/2,-1/2\rangle+
t\langle -\sqrt3/2,1/2,2\sqrt3\rangle$
\end{Answer}
\begin{Answer}{15.2.6}
 $2/\sqrt5/\sqrt{4+\pi^2}$
\end{Answer}
\begin{Answer}{15.2.7}
 $7\sqrt{5}\sqrt{17}/85$, $-9\sqrt{5}\sqrt{17}/85$
\end{Answer}
\begin{Answer}{15.2.9}
 $\langle 0,t\cos t,t\sin t\rangle$,
$\langle 0,\cos t-t\sin t,\sin t+t\cos t\rangle$,
$\vect{r}'/\sqrt{1+t^2}$, $\sqrt{1+t^2}$
\end{Answer}
\begin{Answer}{15.2.10}
 $\langle \sin t,1-\cos t,t^2/2\rangle$
\end{Answer}
\begin{Answer}{15.2.11}
 $t=4$
\end{Answer}
\begin{Answer}{15.2.12}
 $37$, $1$
\end{Answer}
\begin{Answer}{15.2.13}
 $\langle t^2/2,t^3/3,\sin t\rangle$
\end{Answer}
\begin{Answer}{15.2.16}
 $(1,1,1)$ when $t=1$ and $s=0$; $\theta=\arccos(3/\sqrt{14})$; no
\end{Answer}
\begin{Answer}{15.2.17}
 $-6x+(y-\pi)=0$
\end{Answer}
\begin{Answer}{15.2.18}
 $\ds -x/\sqrt2+y/\sqrt2+6z=0$
\end{Answer}
\begin{Answer}{15.2.19}
 $(-1,-3,1)$
\end{Answer}
\begin{Answer}{15.2.20}
 $\langle 1/\sqrt2,1/\sqrt2,0\rangle+t\langle -1,1,6\sqrt2\rangle$
\end{Answer}
\begin{Answer}{15.3.1}
	$2\pi\sqrt{13}$
\end{Answer}
\begin{Answer}{15.3.2}
	$(-8+13\sqrt{13})/27$
\end{Answer}
\begin{Answer}{15.3.3}
	$\sqrt5/2+\ln(\sqrt5+2)/4$
\end{Answer}
\begin{Answer}{15.3.4}
	$(85\sqrt{85}-13\sqrt{13})/27$
\end{Answer}
\begin{Answer}{15.3.5}
	$\int_0^5 \sqrt{1+e^{2t}}\,dt$
\end{Answer}
\begin{Answer}{15.4.1}
	$2\sqrt2/(2+4t^2)^{3/2}$
\end{Answer}
\begin{Answer}{15.4.2}
	$2\sqrt2/(1+8t^2)^{3/2}$
\end{Answer}
\begin{Answer}{15.4.3}
	$\sqrt{3600t^{10}+400t^6+36t^2}/(1+9t^4+25t^8)^{3/2}$
\end{Answer}
\begin{Answer}{15.4.4}
	$12\sqrt{17}/289$
\end{Answer}
\begin{Answer}{15.5.1}
	$\langle 5t^4, 4t, 1\rangle$,
	$\langle 20t^3, 4,0\rangle$,
	$a_T=100t^7+16t/ \sqrt{25t^8 + 16t^2}$, $a_N=\sqrt{3600t^8 + 400t^6 -16}/ \sqrt{25t^8 + 16t^2}$
\end{Answer}
\begin{Answer}{15.5.2}
	$\langle -\sin t,\cos t,2t\rangle$,
	$\langle -\cos t, -\sin t,2\rangle$,
	$4t/\sqrt{4t^2+1}$, $\sqrt{4t^2+5}/\sqrt{4t^2+1}$
\end{Answer}
\begin{Answer}{15.5.3}
	$\langle -\sin t,\cos t,e^t\rangle$,
	$\langle -\cos t, -\sin t,e^t\rangle$,
	$e^{2t}/\sqrt{e^{2t}+1}$, $\sqrt{2e^{2t}+1}/\sqrt{e^{2t}+1}$
\end{Answer}
\begin{Answer}{15.5.4}
	$\langle e^t,\cos t,e^t\rangle$,
	$\langle e^t, -\sin t,e^t\rangle$,
	$(2e^{2t}-\cos t\sin t)/\sqrt{2e^{2t}+\cos^2 t}$,
	$\sqrt{2}e^t|\cos t+\sin t|/\sqrt{2e^{2t}+\cos^2 t}$
\end{Answer}
\begin{Answer}{15.5.5}
	$\langle -3\sin t,2\cos t,0\rangle$,
	$\langle 3\cos t, 2\sin t,0\rangle$
\end{Answer}
\begin{Answer}{15.5.6}
	$\langle -3\sin t,2\cos t+0.1,0\rangle$,
	$\langle 3\cos t, 2\sin t+t/10,0\rangle$
\end{Answer}
\begin{Answer}{15.5.7}
	$\langle -3\sin t,2\cos t,1\rangle$,
	$\langle 3\cos t, 2\sin t,t\rangle$
\end{Answer}
\begin{Answer}{15.5.8}
	$\langle -3\sin t,2\cos t+1/10,1\rangle$,
	$\langle 3\cos t, 2\sin t+t/10,t\rangle$
\end{Answer}
\begin{Answer}{16.2.1}
	$-1$, $0$
\end{Answer}
\begin{Answer}{16.2.2}
	$0$, $a+b$
\end{Answer}
\begin{Answer}{16.2.3}
	$(2b-a)/3$, $0$
\end{Answer}
\begin{Answer}{16.2.4}
	$0$, $1$
\end{Answer}
\begin{Answer}{16.2.5}
	$-2\pi$, $0$
\end{Answer}
\begin{Answer}{16.2.6}
	$0$, $2\pi$
\end{Answer}
\begin{Answer}{16.3.1}
	$13\sqrt{11}/4$
\end{Answer}
\begin{Answer}{16.3.2}
	$0$
\end{Answer}
\begin{Answer}{16.3.3}
	$3\sin(4)/2$
\end{Answer}
\begin{Answer}{16.3.4}
	$2e^3$
\end{Answer}
\begin{Answer}{16.3.5}
	$128$
\end{Answer}
\begin{Answer}{16.3.6}
	$(9e-3)/2$
\end{Answer}
\begin{Answer}{16.3.7}
	$e^{e+1}-e^e-e^{1/e-1}+e^{1/e}+e^4/4-e^{-4}/4$
\end{Answer}
\begin{Answer}{16.3.8}
	$1+\sin(1)-\cos(1)$
\end{Answer}
\begin{Answer}{16.3.9}
	$3\ln3-2\ln2$
\end{Answer}
\begin{Answer}{16.3.10}
	$3/20+10\ln(2)/7$
\end{Answer}
\begin{Answer}{16.3.11}
	$2\ln5-2\ln2+15/32$
\end{Answer}
\begin{Answer}{16.3.12}
	$1$
\end{Answer}
\begin{Answer}{16.3.13}
	$0$
\end{Answer}
\begin{Answer}{16.3.14}
	$21+\cos(1)-\cos(8)$
\end{Answer}
\begin{Answer}{16.3.15}
	$(\ln29-\ln2)/2$
\end{Answer}
\begin{Answer}{16.3.16}
	$2\ln2+\pi/4-2$
\end{Answer}
\begin{Answer}{16.3.17}
	$1243/3$
\end{Answer}
\begin{Answer}{16.3.18}
	$\ln2+11/3$
\end{Answer}
\begin{Answer}{16.3.19}
	$3\cos(1)-\cos(2)-\cos(4)-\cos(8)$
\end{Answer}
\begin{Answer}{16.3.20}
	$-10/3$
\end{Answer}
\begin{Answer}{16.3.22}
	no $f$
\end{Answer}
\begin{Answer}{16.3.23}
	$x^4/4-y^5/5$
\end{Answer}
\begin{Answer}{16.3.24}
	no $f$
\end{Answer}
\begin{Answer}{16.3.25}
	no $f$
\end{Answer}
\begin{Answer}{16.3.26}
	$y\sin x$
\end{Answer}
\begin{Answer}{16.3.27}
	no $f$
\end{Answer}
\begin{Answer}{16.3.28}
	$xyz$
\end{Answer}
\begin{Answer}{16.3.29}
	414
\end{Answer}
\begin{Answer}{16.3.30}
	$6$
\end{Answer}
\begin{Answer}{16.3.31}
	$1/e-\sin3$
\end{Answer}
\begin{Answer}{16.3.32}
	$1/\sqrt{77}-1/\sqrt3$
\end{Answer}
\begin{Answer}{16.4.1}
	$1$
\end{Answer}
\begin{Answer}{16.4.2}
	$0$
\end{Answer}
\begin{Answer}{16.4.3}
	$1/(2e)-1/(2e^7)+e/2-e^7/2$
\end{Answer}
\begin{Answer}{16.4.4}
	$1/2$
\end{Answer}
\begin{Answer}{16.4.5}
	$-1/6$
\end{Answer}
\begin{Answer}{16.4.6}
	$(2\sqrt3-10\sqrt5+8\sqrt6)/3-2\sqrt2/5+1/5$
\end{Answer}
\begin{Answer}{16.4.7}
	$11/2-\ln(2)$
\end{Answer}
\begin{Answer}{16.4.8}
	$2-\pi/2$
\end{Answer}
\begin{Answer}{16.4.9}
	$-17/12$
\end{Answer}
\begin{Answer}{16.4.10}
	$0$
\end{Answer}
\begin{Answer}{16.4.11}
	$-\pi/2$
\end{Answer}
\begin{Answer}{16.4.12}
	$12\pi$
\end{Answer}
\begin{Answer}{16.4.13}
	$2\cos(1)-2\sin(1)-1$
\end{Answer}
\begin{Answer}{16.5.1}
	both are $-45\pi/4$
\end{Answer}
\begin{Answer}{16.5.2}
	$a^2bc+ab^2c+abc^2$
\end{Answer}
\begin{Answer}{16.5.3}
	$e^2-2e+7/2$
\end{Answer}
\begin{Answer}{16.5.4}
	$3$
\end{Answer}
\begin{Answer}{16.5.5}
	$384\pi/5$
\end{Answer}
\begin{Answer}{16.5.6}
	$\pi/3$
\end{Answer}
\begin{Answer}{16.5.7}
	$10\pi$
\end{Answer}
\begin{Answer}{16.5.8}
	$\pi/2$
\end{Answer}
\begin{Answer}{16.6.2}
	$25\sqrt{21}/4$
\end{Answer}
\begin{Answer}{16.6.3}
	$\pi\sqrt{21}$
\end{Answer}
\begin{Answer}{16.6.4}
	$\pi(5\sqrt5-1)/6$
\end{Answer}
\begin{Answer}{16.6.5}
	$4\pi\sqrt2$
\end{Answer}
\begin{Answer}{16.6.6}
	$\pi a^2/2$
\end{Answer}
\begin{Answer}{16.6.7}
	$2\pi a(a-\sqrt{a^2-b^2})$
\end{Answer}
\begin{Answer}{16.6.8}
	$\pi((1+4a^2)^{3/2}-1)/6$
\end{Answer}
\begin{Answer}{16.6.9}
	$\pi a^2-2a^2$
\end{Answer}
\begin{Answer}{16.6.10}
	$\pi a^2\sqrt{1+k^2}/4$
\end{Answer}
\begin{Answer}{16.6.11}
	$A\sqrt{1+a^2+b^2}$
\end{Answer}
\begin{Answer}{16.6.12}
	$A\sqrt{k^2+1}$
\end{Answer}
\begin{Answer}{16.6.13}
	$8a^2$
\end{Answer}
\begin{Answer}{16.7.1}
	$(0,0,3/8)$
\end{Answer}
\begin{Answer}{16.7.2}
	$(11/20,11/20,3/10)$
\end{Answer}
\begin{Answer}{16.7.3}
	$(0,0,1364/425)$
\end{Answer}
\begin{Answer}{16.7.4}
	on center axis, $h/3$ above the base
\end{Answer}
\begin{Answer}{16.7.5}
	$16$
\end{Answer}
\begin{Answer}{16.7.6}
	$7$
\end{Answer}
\begin{Answer}{16.7.7}
	$-\pi$
\end{Answer}
\begin{Answer}{16.7.8}
	$-2/e$
\end{Answer}
\begin{Answer}{16.7.9}
	$\pi b^2(-4b^4-3b^2+6a^2b^2+6a^2)/6$
\end{Answer}
\begin{Answer}{16.7.10}
	$9280$ kg/s
\end{Answer}
\begin{Answer}{16.7.11}
	$24\epsilon_0$
\end{Answer}
\begin{Answer}{16.8.1}
	$-3\pi$
\end{Answer}
\begin{Answer}{16.8.2}
	$0$
\end{Answer}
\begin{Answer}{16.8.3}
	$-4\pi$
\end{Answer}
\begin{Answer}{16.8.4}
	$3\pi$
\end{Answer}
\begin{Answer}{16.8.5}
	$A(p(c-b)+q(a-c)+a-b)$
\end{Answer}
