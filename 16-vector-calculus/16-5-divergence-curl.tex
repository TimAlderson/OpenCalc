\section{Divergence and Curl}\label{sec:DivergenceCurl}

Divergence and curl are two measurements of vector fields that are
very useful in a variety of applications. Both are most easily
understood by thinking of the vector field as representing a flow of a
liquid or gas;
that is, each vector in the vector field should be interpreted as a
velocity vector. 
Roughly speaking, divergence
measures the tendency of 
the fluid to collect or disperse at a point, and curl measures the
tendency of the fluid to swirl around the point. Divergence is a
scalar, that is, a single number, while curl is itself a vector. The
magnitude of the curl measures how much the fluid is swirling, the
direction indicates the axis around which it tends to swirl. These
ideas are somewhat subtle in practice, and are beyond the scope of
this course.
%You can find additional information on the web, for
%example at 
%\texonly
%\url{http://mathinsight.org/curl_idea}%
%\vb|http://mathinsight.org/curl_idea|\endurl\ 
%and 
%\url{http://mathinsight.org/divergence_idea}%
%\vb|http://mathinsight.org/divergence_idea|\endurl\ 
%\endtexonly
%\htmlonly
%<center>
%<a href="http://mathinsight.org/curl_idea">http://mathinsight.org/curl_idea</a>
%</center>
%and
%<center>
%<a href="http://mathinsight.org/divergence_idea">http://mathinsight.org/divergence_idea</a>
%</center>
%\endhtmlonly
%and in
%many books including {\em
%Div, Grad, Curl, and All That: An Informal Text on Vector Calculus},
%by H. M. Schey.

Recall that if $f$ is a function, the gradient of $f$
is given by 
$$\nabla f=\left\langle {\partial f\over\partial x},{\partial
  f\over\partial y},{\partial f\over\partial z}\right\rangle.$$
A useful mnemonic for this (and for the divergence and curl, as it
turns out) is to let
$$\nabla = \left\langle{\partial \over\partial x},{\partial
  \over\partial y},{\partial \over\partial z}\right\rangle,$$
that is, we pretend that $\nabla$ is a vector with rather odd looking
entries. Recalling that $\langle u,v,w\rangle a=\langle ua,va,wa\rangle$,
we can then think of the gradient as
$$\nabla f=\left\langle{\partial \over\partial x},{\partial
  \over\partial y},{\partial \over\partial z}\right\rangle f = 
\left\langle {\partial f\over\partial x},{\partial
  f\over\partial y},{\partial f\over\partial z}\right\rangle,$$
that is, we simply multiply the $f$ into the vector.

The divergence and curl can now be defined in terms of this same
vector $\nabla$ by using the cross product and dot product.

\begin{formulabox}[Divergence]
The divergence of a vector field $\vect{f}=\langle f_1, f_2, f_3 \rangle$ is
$$\nabla \cdot \vect{f} =
\left\langle{\partial \over\partial x},{\partial
  \over\partial y},{\partial \over\partial z}\right\rangle\cdot
\langle f_1,f_2,f_3\rangle
= {\partial f_1 \over\partial x}+{\partial
  f_2 \over\partial y}+{\partial f_3\over\partial z}.$$\index{divergence}\index{vector field!divergence}
\end{formulabox}

\begin{formulabox}[Curl]
The curl of $\vect{f}$ is
$$\nabla\times\vect{f} = \left|
\begin{matrix}
\vect{i}	&	\vect{j}	&	\vect{k}	\\
{\partial \over\partial x}	&	{\partial \over\partial y}	&	{\partial \over\partial z}	\\
f_1	&	f_2	&	f_3
\end{matrix}
\right| = 
\left\langle {\partial f_3\over\partial y}-{\partial f_2\over\partial z},
{\partial f_1\over\partial z}-{\partial f_3\over\partial x},
{\partial f_2\over\partial x}-{\partial f_1\over\partial y}\right\rangle.$$\index{curl}\index{vector field!curl}
\end{formulabox}

Here are two simple but useful facts about divergence and curl.

\begin{theorem}{Divergence of Curl is Zero}{div of curl is zero}
$\nabla\cdot(\nabla\times\vect{f})=0$.
\end{theorem}

In words, this says that the divergence of the curl is zero.

\begin{theorem}{Curl of Gradient is Zero}{curl of gradient is zero}
$\nabla\times(\nabla f) = \vect{0}$.
\end{theorem}

That is, the curl of a gradient is the zero vector. Recalling that
gradients are conservative vector fields, this says that the curl of a
conservative vector field is the zero vector. Under suitable
conditions, it is also true that if the curl of $\vect{f}$ is $\vect{0}$
then $\vect{f}$ is conservative. (Note that this is exactly the same test
that we discussed at the end of Section ~\ref{page:test for conservative vector field}.)

\begin{example}{}{conservative}
Let $\vect{f} = \langle 2e^x - y, -x ,e^z\rangle$. Show that $\vect{f}$ is conservative and find the function $f$ such that $\vect{f} = \langle f_x, f_y, f_z \rangle$. 
\end{example}

\begin{solution}
If $\vect{f} = \langle 2e^x - y, -x ,e^z \rangle$ then $\nabla\times\vect{f} = \langle 0, 0, (-1)-(-1) \rangle = \vect{0}$.
Thus, $\vect{f}$ is conservative, and we can exhibit this directly by
finding the corresponding $f$.

Since $f_x=2e^x-y$, $f=2e^x-xy+g(y,z)$. Since $f_y=-x$, it must be that
$g_y=0$, so $g(y,z)= C + h(z)$. Thus $f=2e^x-xy+ C + h(z)$ and 
$$f_z = h'(z) = e^z,$$
so $h(z)=e^z$. This leaves $f=2e^x -xy + e^z + C$.
\end{solution}


%%%%%%%%%%%%%%%%%%%%%%%%%%%%%%%%%%%%%%%%%%%%
\Opensolutionfile{solutions}[ex]
\section*{Exercises for \ref{sec:DivergenceCurl}}

\begin{enumialphparenastyle}

\begin{ex}
Let $\vect{f}=\langle xy,-xy\rangle$ and 
let $D$ be given by $0\le x\le 1$, $0\le y\le 1$.
Compute $\ds\int_{\partial D} \vect{f}\cdot d\vect{r}$ and
$\ds\int_{\partial D} \vect{f}\cdot\vect{N}\,ds$.
\begin{sol}
	$-1$, $0$
\end{sol}
\end{ex}

\begin{ex}
Let $\vect{f}=\langle ax^2,by^2\rangle$ and 
let $D$ be given by $0\le x\le 1$, $0\le y\le 1$.
Compute $\ds\int_{\partial D} \vect{f}\cdot d\vect{r}$ and
$\ds\int_{\partial D} \vect{f}\cdot\vect{N}\,ds$.
\begin{sol}
	$0$, $a+b$
\end{sol}
\end{ex}

\begin{ex}
Let $\vect{f}=\langle ay^2,bx^2\rangle$ and 
let $D$ be given by $0\le x\le 1$, $0\le y\le x$.
Compute $\ds\int_{\partial D} \vect{f}\cdot d\vect{r}$ and
$\ds\int_{\partial D} \vect{f}\cdot\vect{N}\,ds$.
\begin{sol}
	$(2b-a)/3$, $0$
\end{sol}
\end{ex}

\begin{ex}
Let $\vect{f}=\langle \sin x\cos y,\cos x\sin y\rangle$ and 
let $D$ be given by $0\le x\le \pi/2$, $0\le y\le x$.
Compute $\ds\int_{\partial D} \vect{f}\cdot d\vect{r}$ and
$\ds\int_{\partial D} \vect{f}\cdot\vect{N}\,ds$.
\begin{sol}
	$0$, $1$
\end{sol}
\end{ex}

\begin{ex}
Let $\vect{f}=\langle y,-x\rangle$ and 
let $D$ be given by $x^2+y^2\le 1$.
Compute $\ds\int_{\partial D} \vect{f}\cdot d\vect{r}$ and
$\ds\int_{\partial D} \vect{f}\cdot\vect{N}\,ds$.
\begin{sol}
	$-2\pi$, $0$
\end{sol}
\end{ex}

\begin{ex}
Let $\vect{f}=\langle x,y\rangle$ and 
let $D$ be given by $x^2+y^2\le 1$.
Compute $\ds\int_{\partial D} \vect{f}\cdot d\vect{r}$ and
$\ds\int_{\partial D} \vect{f}\cdot\vect{N}\,ds$.
\begin{sol}
	$0$, $2\pi$
\end{sol}
\end{ex}

\begin{ex}
Prove Theorem~\ref{thm:div of curl is zero}.
\end{ex}

\begin{ex}
Prove Theorem~\ref{thm:curl of gradient is zero}.
\end{ex}

\begin{ex}
If $\nabla \cdot \vect{f}=0$, $\vect{f}$ is said to be \dfont{incompressible}.  Show that any vector field
of the form $\vect{f}(x,y,z) = \langle f(y,z),g(x,z),h(x,y)\rangle$ is
incompressible.  Give a non-trivial example.
\end{ex}

\end{enumialphparenastyle}
