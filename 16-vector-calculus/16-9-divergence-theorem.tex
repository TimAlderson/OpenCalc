\section{The Divergence Theorem}\label{sec:DivergenceTheorem}

The third version of Green's Theorem (Equation~\ref{eq:greens theorem third form}) we saw was:
$$\int_{\partial D} \vect{f}\cdot\vect{N}\,ds=\iint_{D} \nabla\cdot\vect{f}\,dA.$$ With minor changes this turns into another equation, the
Divergence Theorem:

\begin{theorem}{Divergence Theorem}{}
Under suitable conditions, if $E$ is a
region of three dimensional space and $D$ is its boundary surface,
oriented outward, then
$$\iint_{D} \vect{f}\cdot\vect{N}\,dS=\iiint_{E} \nabla\cdot\vect{f}\,dV.$$\index{divergence!theorem}
\end{theorem}
\goodbreak
\begin{proof}
Again this theorem is too difficult to prove here, but a special case
is easier. In the proof of a special case of Green's Theorem, we
needed to know that we could describe the region of integration in
both possible orders, so that we could set up one double integral
using $dx\,dy$ and another using $dy\,dx$. Similarly here, we need to
be able to describe the three-dimensional region $E$ in different
ways.

We start by rewriting the triple integral:
$$\iiint_{E} \nabla\cdot\vect{f}\,dV = \iiint_{E} ({\partial f_1 \over \partial x}+{\partial f_2 \over \partial y}+{\partial f_3 \over \partial z})\,dV
=\iiint_{E} {\partial f_1 \over \partial x}\,dV + \iiint_{E} {\partial f_2 \over \partial y}\,dV + \iiint_{E} {\partial f_3 \over \partial z}\,dV.$$
The double integral may be rewritten:
$$\iint_{D} \vect{f}\cdot\vect{N}\,dS
=\iint_{D} (f_1\vect{i}+f_2\vect{j}+f_3\vect{k})\cdot\vect{N}\,dS
=\iint_{D} f_1\vect{i}\cdot\vect{N}\,dS+\iint_{D} f_2\vect{j}\cdot\vect{N}\,dS+
\iint_{D} f_3\vect{k}\cdot\vect{N}\,dS.$$
To prove that these give the same value it is sufficient to prove that
\begin{align}\label{eq:divergence proof}
\iint_{D} f_1\vect{i}\cdot\vect{N}\,dS&=\iiint_{E} {\partial f_1 \over \partial x}\,dV,	\\
\iint_{D} f_2 \vect{j}\cdot\vect{N}\,dS&=\iiint_{E} {\partial f_2 \over \partial y}\,dV,\;\hbox{and}	\\
\iint_{D} f_3 \vect{k}\cdot\vect{N}\,dS&=\iiint_{E} {\partial f_3 \over \partial z}\,dV.
\end{align}
% $$
% \iint{D} P\vect{i}\cdot\vect{N}\,dS=\iiint{E} P_x\,dV,\;
% \iint{D} Q\vect{j}\cdot\vect{N}\,dS=\iiint{E} Q_y\,dV,\;\hbox{and}\;
% \iint{D} R\vect{k}\cdot\vect{N}\,dS=\iiint{E} R_z\,dV.$$
Not surprisingly, these are all pretty much the same; we'll do the
first one.

We set the triple integral up with $dx$ innermost:
$$\iiint_{E} {\partial f_1 \over \partial x}\,dV=\iint_{B}\int_{g_1(y,z)}^{g_2(y,z)} {\partial f_1 \over \partial x}\,dx\,dA=
\iint_{B} f_1(g_2(y,z),y,z)-f_1(g_1(y,z),y,z)\,dA,$$
where $B$ is the region in the $y$-$z$ plane over which we integrate.
The boundary surface of $E$ consists of a ``top'' $x=g_2(y,z)$, a
``bottom'' $x=g_1(y,z)$, and a ``wrap-around side'' that is vertical
to the $y$-$z$ plane. To integrate over the entire boundary surface,
we can integrate over each of these (top, bottom, side) and add the
results. Over the side surface, the vector $\vect{N}$ is perpendicular to
the vector $\vect{i}$, so
$$
\iint_{\hbox{\scriptsize side}} f_1\vect{i}\cdot\vect{N}\,dS = 
\iint_{\hbox{\scriptsize side}}
0\,dS=0.$$
Thus, we are left with just the surface integral over the top plus the
surface integral over the bottom. For the top, we use the vector
function
$\vect{r}=\langle g_2(y,z),y,z\rangle$ which gives 
$\vect{r}_y\times\vect{r}_z=\langle 1,-g_{2y},-g_{2z}\rangle$; the dot
product of this with $\vect{i}=\langle 1,0,0\rangle$ is 1. Then
$$
\iint_{\hbox{\scriptsize top}} f_1\vect{i}\cdot\vect{N}\,dS
=\iint_{B} f_1(g_2(y,z),y,z)\,dA.$$
In almost identical fashion we get
$$
\iint_{\hbox{\scriptsize bottom}} f_1\vect{i}\cdot\vect{N}\,dS
=-\iint_{B} f_1(g_1(y,z),y,z)\,dA,$$
where the negative sign is needed to make $\vect{N}$ point in the
negative $x$ direction. Now
$$\iint_{D} f_1\vect{i}\cdot\vect{N}\,dS
=\iint_{B} f_1(g_2(y,z),y,z)\,dA-\iint_{B}
    f_1(g_1(y,z),y,z)\,dA,$$ which is the same as the value of the
    triple integral above.
\end{proof}

It is worth noting that this theorem is also referred to as \textit{Gauss' Theorem}. We now compute an example. \index{Gauss' Theorem}

\begin{example}{}{}
Let $\vect{f}=\langle 2x,3y,z^2\rangle$, and consider the
three-dimensional volume inside the cube with faces parallel to the
principal planes and opposite corners at
$(0,0,0)$ and $(1,1,1)$. Compute the two integrals of the
divergence theorem.
\end{example}

\begin{solution}
The triple integral is the easier of the two:
$$\int_0^1\int_0^1\int_0^1 2+3+2z\,dx\,dy\,dz=6.$$
The surface integral must be separated into six parts, one for each
face of the cube. One face is $z=0$ or $\vect{r}=\langle u,v,0\rangle$, $0\le
u,v\le 1$. Then $\vect{r}_u=\langle 1,0,0\rangle$, 
$\vect{r}_v=\langle 0,1,0\rangle$, and $\vect{r}_u\times\vect{r}_v=
\langle 0,0,1\rangle$. We need this to be oriented downward (out of
the cube), so we use
$\langle 0,0,-1\rangle$ and the corresponding integral is
$$\int_0^1\int_0^1 -z^2\,du\,dv=\int_0^1\int_0^1 0\,du\,dv=0.$$

Another face is $y=1$ or $\vect{r}=\langle u,1,v\rangle$. Then $\vect{r}_u=\langle 1,0,0\rangle$, $\vect{r}_v=\langle 0,0,1\rangle$, and
$\vect{r}_u\times\vect{r}_v= \langle 0,-1,0\rangle$. We need a normal in
the positive $y$ direction, so we convert this to $\langle
0,1,0\rangle$, and the corresponding integral is
$$\int_0^1\int_0^1 3y\,du\,dv=\int_0^1\int_0^1 3\,du\,dv=3.$$

The remaining four integrals have values 0, 0, 2, and 1, and the sum
of these is 6, in agreement with the triple integral.
\end{solution}

\begin{example}{}{}
Let $\vect{f}=\langle x^3,y^3,z^2\rangle$, and consider the
cylindrical volume $x^2+y^2\le9$, $0\le z\le2$.
Compute the two integrals of the divergence theorem.
\end{example}

\begin{solution}
The triple integral (using cylindrical coordinates) is 
$$\int_0^{2\pi}\int_0^3\int_0^2 (3r^2+2z)r\,dz\,dr\,d\theta=279\pi.$$

For the surface we need three integrals. The top of the cylinder can
be represented by
$\vect{r}=\langle v\cos u,v\sin u,2\rangle$; 
$\vect{r}_u\times\vect{r}_v=\langle 0,0,-v\rangle$, which points down
into the cylinder,
so we convert it to $\langle 0,0,v\rangle$. Then
$$\int_0^{2\pi}\int_0^3 \langle v^3\cos^3u,v^3\sin^3u,4\rangle\cdot
\langle 0,0,v\rangle\,dv\,du=
\int_0^{2\pi}\int_0^3 4v\,dv\,du=36\pi.$$
The bottom is 
$\vect{r}=\langle v\cos u,v\sin u,0\rangle$; 
$\vect{r}_u\times\vect{r}_v=\langle 0,0,-v\rangle$ and
$$\int_0^{2\pi}\int_0^3 \langle v^3\cos^3u,v^3\sin^3u,0\rangle\cdot
\langle 0,0,-v\rangle\,dv\,du=
\int_0^{2\pi}\int_0^3 0\,dv\,du=0.$$
The side of the cylinder is $\vect{r}=\langle 3\cos u,3\sin u,v\rangle$;
$\vect{r}_u\times\vect{r}_v=\langle 3\cos u,3\sin u,0\rangle$ which does
point outward, so
\begin{align*}
\int_0^{2\pi}\int_0^2 &\langle 27\cos^3 u,27\sin^3 u,v^2\rangle\cdot
\langle 3\cos u,3\sin u,0\rangle \,dv\,du	\\
&=\int_0^{2\pi}\int_0^2 81\cos^4 u+81\sin^4u\,dv\,du=243\pi.
\end{align*}
The total surface integral is thus $36\pi+0+243\pi=279\pi$.
\end{solution}



%%%%%%%%%%%%%%%%%%%%%%%%%%%%%%%%%%%%%%%%%%%%
\Opensolutionfile{solutions}[ex]
\section*{Exercises for \ref{sec:DivergenceTheorem}}

\begin{enumialphparenastyle}

\begin{ex}
Using $\ds\vect{f}=\langle 3x,y^3,-2z^2\rangle$ and the
region bounded by $\ds x^2+y^2=9$, $z=0$, and $z=5$, compute both
integrals from the Divergence Theorem.
\begin{sol}
	both are $-45\pi/4$
\end{sol}
\end{ex}

\begin{ex}
Let $E$ be the volume described by 
$0\le x\le a$, $0\le y\le b$, $0\le z\le c$, and 
$\vect{f}= \langle x^2,y^2,z^2\rangle$. Compute
$\ds\iint_{\partial E} \vect{f}\cdot \vect{N}\,dS$.
\begin{sol}
	$a^2bc+ab^2c+abc^2$
\end{sol}
\end{ex}

\begin{ex}
Let $E$ be the volume described by 
$0\le x\le 1$, $0\le y\le 1$, $0\le z\le 1$, and 
$\vect{f}= \langle 2xy,3xy,ze^{x+y}\rangle$. Compute
$\ds\iint_{\partial E} \vect{f}\cdot \vect{N}\,dS$.
\begin{sol}
	$e^2-2e+7/2$
\end{sol}
\end{ex}

\begin{ex}
Let $E$ be the volume described by 
$0\le x\le 1$, $0\le y\le x$, $0\le z\le x+y$, and 
$\vect{f}= \langle x,2y,3z\rangle$. Compute
$\ds\iint_{\partial E} \vect{f}\cdot \vect{N}\,dS$.
\begin{sol}
	$3$
\end{sol}
\end{ex}

\begin{ex}
Let $E$ be the volume described by 
$x^2+y^2+z^2\le 4$, and 
$\vect{f}= \langle x^3,y^3,z^3\rangle$. Compute
$\ds\iint_{\partial E} \vect{f}\cdot \vect{N}\,dS$.
\begin{sol}
	$384\pi/5$
\end{sol}
\end{ex}

\begin{ex}
Let $E$ be the hemisphere described by 
$0\le z\le \sqrt{1-x^2-y^2}$, and 
$\vect{f}= \langle \sqrt{x^2+y^2+z^2},\sqrt{x^2+y^2+z^2},
\sqrt{x^2+y^2+z^2}\rangle$. Compute
$\ds\iint_{\partial E} \vect{f}\cdot \vect{N}\,dS$.
\begin{sol}
	$\pi/3$
\end{sol}
\end{ex}

\begin{ex}
Let $E$ be the volume described by 
$x^2+y^2\le1$, $0\le z\le4$, and 
$\vect{f}= \langle xy^2,yz,x^2z\rangle$. Compute
$\ds\iint_{\partial E} \vect{f}\cdot \vect{N}\,dS$.
\begin{sol}
	$10\pi$
\end{sol}
\end{ex}

\begin{ex}
Let $E$ be the solid cone above the $x$-$y$ plane and
inside $z=1-\sqrt{x^2+y^2}$, and 
$\vect{f}= \langle x\cos^2 z,y\sin^2z,\sqrt{x^2+y^2}z\rangle$. Compute
$\ds\iint_{\partial E} \vect{f}\cdot \vect{N}\,dS$.
\begin{sol}
	$\pi/2$
\end{sol}
\end{ex}

\begin{ex}
Prove the other two equations in the 
display~\ref{eq:divergence proof}.
\end{ex}

\begin{ex}
Suppose $D$ is a closed surface, and that $D$ and $F$ are
sufficiently nice. Show that 
$$\iint_{D} (\nabla \times \vect{f}) \cdot \vect{N}\, dS = 0$$
where $\vect{N}$ is the outward pointing unit normal.
\end{ex}

\begin{ex}
Suppose $D$ is a closed surface, $D$ is sufficiently nice,
and $F=\langle a,b,c\rangle$ is a constant vector field.
Show that 
$$\iint_{D} \vect{f} \cdot \vect{N}\, dS = 0$$
where $\vect{N}$ is the outward pointing unit normal.
\end{ex}

\begin{ex}
We know that the volume of a region $E$ may often be computed as
$\ds \iiint_{E}dx\,dy\,dz$. Show that this volume may also be computed as
$\ds{1\over3} \iint_{\partial E} \langle x,y,z\rangle \cdot \vect{N}\,dS$
where $\vect{N}$ is the outward pointing unit normal to $\partial E$.
\end{ex}

\end{enumialphparenastyle}
