\section{Stokes' Theorem}\label{sec:StokesTheorem}

Recall that one version of Green's Theorem (see
Equation~\ref{eq:greens theorem second form}) is
$$\int_{\partial D} \vect{f}\cdot d\vect{r}
=\iint_{D}(\nabla\times \vect{f})\cdot\vect{k}\,dA.
$$
Here $D$ is a region in the $x$-$y$ plane and $\vect{k}$ is a unit normal
to $D$ at every point. If $D$ is instead an orientable surface in
space, there is an obvious way to alter this equation, and it turns
out still to be true:

\begin{theorem}{Stokes' Theorem}{}
Provided that the quantities involved are
sufficiently nice, and in particular if $D$ is orientable, 
$$\int_{\partial D} \vect{f}\cdot d\vect{r}
=\iint_{D}(\nabla\times \vect{f})\cdot\vect{N}\,dS,$$
if $\partial D$ is oriented counter-clockwise relative to $\vect{N}$.\index{Stokes' theorem}
\end{theorem}

The proof of Stokes' Theorem will follow a discussion and several examples of the Theorem in use. 

Note how little has changed: $\vect{k}$ becomes $\vect{N}$, a unit normal to
the surface, and $dA$ becomes $dS$, since this is now a general
surface integral. The phrase ``counter-clockwise relative to $\vect{N}$''
means that if we take the direction of $\vect{N}$ to be ``up'', then we
go around the boundary counter-clockwise when viewed from ``above''.

\begin{example}{}{}
Let $\vect{f}=\langle  y^2z,x^2z,xy^2\rangle$ 
and the surface $D$ be $x=\sqrt{1-y^2-z^2}$, oriented in
the positive $x$ direction.
It quickly becomes apparent that the surface integral in Stokes'
Theorem is intractable, so compute the line integral.
\end{example}

\begin{solution}
The boundary of
$D$ is the unit circle in the $y$-$z$ plane, $\vect{r}=\langle 0,\cos
u,\sin u\rangle$, $0\le u\le 2\pi$. The integral is
$$\int_0^{2\pi} \langle  y^2z,x^2z,xy^2 \rangle\cdot
\langle 0,-\sin u,\cos u\rangle\,du=
\int_0^{2\pi} 0\,du = 0,$$
because $x=0$.
\end{solution}

An interesting consequence of Stokes' Theorem is that if $D$ and $E$
are two orientable surfaces with the same boundary, then
$$
\iint_{D}(\nabla\times \vect{f})\cdot\vect{N}\,dS
=\int_{\partial D} \vect{f}\cdot d\vect{r}
=\int_{\partial E} \vect{f}\cdot d\vect{r}
=\iint_{E}(\nabla\times \vect{f})\cdot\vect{N}\,dS.
$$
Sometimes both of the integrals 
$$\iint_{D}(\nabla\times \vect{f})\cdot\vect{N}\,dS
\qquad\hbox{and}\qquad\int_{\partial D} \vect{f}\cdot d\vect{r}$$
are difficult, but you may be able to find a second surface $E$ so
that
$$\iint_{E}(\nabla\times \vect{f})\cdot\vect{N}\,dS$$
has the same value but is easier to compute.

In the previous example, the line integral was easy to
compute. But we might also notice that another surface $E$ with the
same boundary is the flat disk $y^2+z^2\le 1$. 

\begin{example}{}{}
Let $\vect{f}=\langle  y^2z,x^2z,xy^2 \rangle$ and the surface $E$ be $y^2+z^2\le 1$. Compute the surface integral. 
\end{example}

\begin{solution} 
The unit normal $\vect{N}$ for this surface is simply $\vect{i}=\langle 1,0,0\rangle$. We compute
the curl:
$$\nabla\times\vect{f}=\langle 2xy-x^2, 0, 2xz-2yz \rangle.$$ 
Since $x=0$ everywhere on the surface,
$$(\nabla\times\vect{f})\cdot \vect{N}=
\langle 0, 0, 2xz-2yz \rangle\cdot\langle 1,0,0\rangle=0,$$
so the surface integral is
$$\iint_{E}0\,dS=0,$$
as before. In this case, of course, it is still somewhat easier to
compute the line integral, avoiding $\nabla\times\vect{f}$ entirely.
\end{solution}

%\begin{example}{}{}
%Let $\vect{f}=\langle -y^2,x,z^2\rangle$, and let the curve $C$
%be the intersection of the cylinder $x^2+y^2=1$ with the plane
%$y+z=2$, oriented counter-clockwise when viewed from above.
%Compute $\ds\int_C \vect{f}\cdot d\vect{r}$.
%\end{example}

%\begin{solution}
%We compute $\ds\int_C \vect{f}\cdot d\vect{r}$ in two ways.

%First we do it directly: a vector function for $C$ is
%$\vect{r}=\langle \cos u,\sin u, 2-\sin u\rangle$, so
%$\vect{r}'=\langle -\sin u,\cos u,-\cos u\rangle$, and the integral is then
%$$\int_0^{2\pi} y^2\sin u+x\cos u-z^2\cos u\,du
%=\int_0^{2\pi} \sin^3 u+\cos^2 u-(2-\sin u)^2\cos u\,du
%=\pi.$$

%To use Stokes' Theorem, we pick a surface with $C$ as the boundary;
%the simplest such surface is that portion of the plane $y+z=2$ inside
%the cylinder. This has vector equation $\vect{r}=\langle
%v\cos u,v\sin u,2-v\sin u\rangle$. We compute
%$\vect{r}_u= \langle -v\sin u,v\cos u,-v\cos u\rangle$,
%$\vect{r}_v= \langle \cos u,\sin u, -\sin u\rangle$, and 
%$\vect{r}_u\times\vect{r}_v=\langle 0,-v,-v\rangle$. To match the
%orientation of $C$ we need to use the normal $\langle
%0,v,v\rangle$. The curl of $\vect{f}$ is $\langle 0,0,1+2y\rangle=
%\langle 0,0,1+2v\sin u\rangle$, and
%the surface integral from Stokes' Theorem is
%$$\int_0^{2\pi}\int_0^1 (1+2v\sin u)v\,dv\,du=\pi.$$
%In this case the surface integral was more work to set up, but the
%resulting integral is somewhat easier.
%\end{solution}

Now let's look at the proof of Stokes' Theorem.

\begin{proof}
We can prove here a special case of Stokes' Theorem, which perhaps
not too surprisingly uses Green's Theorem.

Suppose the surface $D$ of interest can be expressed in the form
$z=g(x,y)$, and let $\vect{f}=\langle f_1,f_2,f_3\rangle$. Using the vector
function $\vect{r}=\langle x,y,g(x,y)\rangle$ for the surface we get the
surface integral
\begin{align*}
\iint_{D} \nabla\times\vect{f}\cdot d\vect{S}&=
\iint_{E} \langle {\partial f_3 \over \partial y}-{\partial f_2 \over \partial z},{\partial f_1 \over \partial z}-{\partial f_3 \over \partial x},{\partial f_2 \over \partial x}-{\partial f_1 \over \partial y}\rangle\cdot
\langle -g_x,-g_y,1\rangle\,dA	\\
&=\iint_{E} -{\partial f_3 \over \partial x}g_x+{\partial f_2 \over \partial z}g_x-{\partial f_1 \over \partial z}g_y+{\partial f_3 \over \partial x}g_y+{\partial f_2 \over \partial x}-{\partial f_1 \over \partial y}\,dA.
\end{align*}
Here $E$ is the region in the $x$-$y$ plane directly below the surface
$D$. 

For the line integral, we need a vector function for $\partial D$. If 
$\langle x(t),y(t)\rangle$ is a vector function for 
$\partial E$ then we may use $\vect{r}(t)=\langle x(t),y(t),g(x(t),y(t))\rangle$
to represent $\partial D$. Then
$$\int_{\partial D}\vect{f}\cdot d\vect{r}
=\int_a^b f_1{dx\over dt}+f_2{dy\over dt}+f_3{dz\over dt}\,dt
=\int_a^b f_1{dx\over dt}+f_2{dy\over dt}+f_2\left({\partial z\over\partial
    x}{dx\over dt}+{\partial z\over\partial y}{dy\over dt}\right)\,dt.$$
using the chain rule for $dz/dt$. Now we continue to manipulate this:
\begin{align*}
\int_a^b f_1{dx\over dt}+f_2{dy\over dt}+&f_3\left({\partial z\over\partial
    x}{dx\over dt}+{\partial z\over\partial y}{dy\over dt}\right)\,dt	\\
&=\int_a^b \left[\left(f_1+f_3{\partial z\over\partial x}\right){dx\over dt}+
\left(f_2+f_3{\partial z\over\partial y}\right){dy\over dt}\right]\,dt	\\
&=\int_{\partial E} \left(f_1+f_3{\partial z\over\partial x}\right)\,dx+
\left(f_2+f_3{\partial z\over\partial y}\right)\,dy,
\end{align*}
which now looks just like the line integral of Green's Theorem, except
that the functions $f_1$ and $f_2$ of Green's Theorem have been replaced
by the more complicated $f_1+f_3(\partial z/\partial x)$
and $f_2+f_3(\partial z/\partial y)$. We can apply Green's Theorem to get
$$\int_{\partial E} \left(f_1+f_3{\partial z\over\partial x}\right)\,dx+
\left(f_2+f_3{\partial z\over\partial y}\right)\,dy=
\iint_{E} {\partial\over \partial x}\left(f_2+f_3{\partial z\over\partial y}\right)
-{\partial\over \partial y}\left(f_1+f_3{\partial z\over\partial x}\right)\,dA.$$
Now we can use the chain rule again to evaluate the derivatives
inside this integral, and it becomes
\begin{align*}
\iint_{E} &{\partial f_2 \over \partial x}+{\partial f_2 \over \partial z}g_x+{\partial f_3 \over \partial x}g_y+{\partial f_3 \over \partial z}g_xg_y+f_3g_{yx}-
\left({\partial f_1 \over \partial y}+{\partial f_1 \over \partial z}g_y+{\partial f_3 \over \partial y}g_x+{\partial f_3 \over \partial z}g_yg_x+f_3g_{xy}\right)\,dA	\\
&=\iint_{E} {\partial f_2 \over \partial x}+{\partial f_2 \over \partial z}g_x+{\partial f_3 \over \partial x}g_y-{\partial f_1 \over \partial y}-{\partial f_1 \over \partial z}g_y-{\partial f_3 \over \partial y}g_x\,dA,
\end{align*}
which is the same as the expression we obtained for the surface
integral.
\end{proof}


%%%%%%%%%%%%%%%%%%%%%%%%%%%%%%%%%%%%%%%%%%%%
\Opensolutionfile{solutions}[ex]
\section*{Exercises for \ref{sec:StokesTheorem}}

\begin{enumialphparenastyle}

\begin{ex}
Let $\vect{f}=\langle z,x,y\rangle$.
The plane $z=2x+2y-1$ and the paraboloid $z=x^2+y^2$ intersect in a
closed curve. Stokes' Theorem implies that
$$\iint_{D_1} (\nabla\times\vect{f})\cdot \vect{N}\,dS=
\oint_C \vect{f}\cdot d\vect{r}=
\iint_{D_2} (\nabla\times\vect{f})\cdot \vect{N}\,dS,
$$
where the line integral is computed over the intersection $C$ of the plane
and the paraboloid, and the two surface integrals are computed over
the portions of the two surfaces that have boundary $C$ (provided, of
course, that the orientations all match). Compute all three integrals.
\begin{sol}
	$-3\pi$
\end{sol}
\end{ex}

\begin{ex}
Let $D$ be the portion of $z=1-x^2-y^2$ above the $x$-$y$
plane, oriented up, and let $\vect{f}=\langle
xy^2,-x^2y,xyz\rangle$. Compute $\ds\iint_{D} (\nabla\times\vect{f})\cdot \vect{N}\,dS$.
\begin{sol}
	$0$
\end{sol}
\end{ex}

\begin{ex}
Let $D$ be the portion of $z=2x+5y$ inside $x^2+y^2=1$,
oriented up, and
let $\vect{f}=\langle y,z,-x\rangle$. Compute
$\ds\int_{\partial D} \vect{f}\cdot d\vect{r}$.
\begin{sol}
	$-4\pi$
\end{sol}
\end{ex}

\begin{ex}
Compute $\ds\oint_C x^2z\,dx + 3x\,dy - y^3\,dz$, where $C$
is the unit circle $\ds x^2+y^2=1$ oriented counter-clockwise.
\begin{sol}
	$3\pi$
\end{sol}
%/Albert
\end{ex}

\begin{ex}
Let $D$ be the portion of $z=px+qy+r$ over a region in the
$x$-$y$ plane that has area $A$, oriented up, and 
let $\vect{f}=\langle ax+by+cz,ax+by+cz,ax+by+cz\rangle$. Compute
$\ds\int_{\partial D} \vect{f}\cdot d\vect{r}$.
\begin{sol}
	$A(p(c-b)+q(a-c)+a-b)$
\end{sol}
\end{ex}

\begin{ex}
Let $D$ be any surface and 
let $\vect{f}=\langle f_1(x),f_2(y),f_3(z)\rangle$ ($f_1$ depends only on $x$,
$f_2$ only on $y$, and $f_3$ only on $z$). Show that
$\ds\int_{\partial D} \vect{f}\cdot d\vect{r}=0$.
\end{ex}

\begin{ex}
Show that $\ds\int_C f\nabla g+g\nabla f\cdot d\vect{r}=0$, where
$\vect{r}$ describes a closed curve $C$ to which Stokes' Theorem
applies.
\end{ex}

\end{enumialphparenastyle}
