\section{Cylindrical and Spherical Coordinates}\label{sec:CylSphCoords}

We have seen that sometimes double integrals are simplified by doing
them in polar coordinates; not surprisingly, triple integrals are
sometimes simpler in cylindrical coordinates or spherical coordinates.
To set up integrals in polar coordinates, we had to understand the
shape and area of a typical small region into which the region of
integration was divided. We need to do the same thing here, for three
dimensional regions.\index{cylindrical coordinates!triple integral}\index{spherical coordinates!triple integral}

The cylindrical coordinate system is the simplest, since it is just
the polar coordinate system plus a $z$ coordinate. A typical small
unit of volume is the shape shown in Figure~\ref{fig:polarcoordinatesregions}
``fattened up'' in the $z$ direction, so its
volume is $r\Delta r\Delta \theta\Delta z$, or in the limit, 
$r\,dr\,d\theta\,dz$. 

\begin{example}{Finding Volume}{FindVolumeExample}
Find the volume under $z=\sqrt{4-r^2}$ 
above the quarter circle inside $x^2+y^2=4$
in the first quadrant.
\end{example}
\begin{solution}
We could of course do this with a double integral, but we'll use a
triple integral:
$$\int_0^{\pi/2}\int_0^2\int_0^{\sqrt{4-r^2}} r\,dz\,dr\,d\theta=
\int_0^{\pi/2}\int_0^2 \sqrt{4-r^2}\; r\,dr\,d\theta=
{4\pi\over3}.$$
Compare this to Example~\ref{exa:integrationinpolarcoordinates}.
\end{solution}

\begin{example}{Mass using Cylindrical Coordinates}{MassCylCoords}
An object occupies the space inside both the cylinder
$x^2+y^2=1$ and the sphere $x^2+y^2+z^2=4$, and has density $x^2$ at
$(x,y,z)$. Find the total mass.
\end{example}
\begin{solution}
We set this up in cylindrical coordinates, recalling that
$x=r\cos\theta$: 
\begin{align*}
\int_0^{2\pi}\int_0^1\int_{-\sqrt{4-r^2}}^{\sqrt{4-r^2}}
r^3\cos^2(\theta)\,dz\,dr\,d\theta
&=\int_0^{2\pi}\int_0^1 2\sqrt{4-r^2}\;r^3\cos^2(\theta)\,dr\,d\theta	\\
&=\int_0^{2\pi}\left({128\over15}-{22\over5}\sqrt3\right)\cos^2(\theta)\,d\theta	\\
&=\left({128\over15}-{22\over5}\sqrt3\right)\pi
\end{align*}
\end{solution}

Spherical coordinates are somewhat more difficult to understand. The
small volume we want will be defined by $\Delta\rho$, $\Delta\phi$,
and $\Delta\theta$, as pictured in Figure~\ref{fig:sphericalvolumeunit}.
The small volume is nearly box shaped, with 4 flat sides and two sides
formed from bits of concentric spheres. When $\Delta\rho$, $\Delta\phi$,
and $\Delta\theta$ are all very small, the volume of this little
region will be nearly the volume we get by treating it as a box.
One dimension of the box is simply $\Delta\rho$, the change in distance
from the origin. The other two dimensions are the lengths of small
circular arcs, so they are $r\Delta\alpha$ for some suitable
$r$ and $\alpha$, just as in the polar coordinates case.

\begin{figure}[H]
\centerline{
\vbox{\beginpicture
\normalgraphs
\setcoordinatesystem units <1.5truecm,1.5truecm>
\setplotarea x from 0 to 2.1, y from -1.1 to 1.1
\put {\hbox{\epsfxsize9cm\epsfbox{images/spherical_volume_unit.eps}}} at 0 0
\endpicture}}
\caption{A small unit of volume for spherical coordinates.}
\label{fig:sphericalvolumeunit}
\end{figure}

The easiest of these to understand is the arc corresponding to a
change in $\phi$, which is nearly identical to the derivation for
polar coordinates, as shown in the left graph in Figure~\ref{fig:intspherical}.
In that graph we are looking ``face on'' at the side of
the box we are interested in, so the small angle pictured is
precisely $\Delta\phi$, the vertical axis really is the $z$ axis, but
the horizontal axis is \emph{not} a real axis---it is just some line
in the $x$-$y$ plane.
Because the other arc is governed by $\theta$, we need
to imagine looking straight down the $z$ axis, so that the apparent
angle we see is $\Delta\theta$. In this view, the axes really are the
$x$ and $y$ axes.
In this graph, the apparent distance from the
origin is not $\rho$ but $\rho\sin\phi$, as indicated in the left
graph. 

\begin{figure}[H]
\centerline{
\vbox{\beginpicture
\normalgraphs
\setcoordinatesystem units <1.5truecm,1.5truecm>
\setplotarea x from 0 to 3.5, y from 0 to 3
\axis left  /
\axis bottom  /
\circulararc 15 degrees from 3.17 1.48 center at 0 0
\circulararc 15 degrees from 2.266 1.057 center at 0 0
\setlinear
\plot 3.17 1.48 0 0 2.68 2.25 /
\betweenarrows {$\rho\sin\phi$} [t] <0pt,-3pt> from 0 0 to 3.17 0
\setdashes
\plot 3.17 1.48 3.17 0 /
\put {$z$} [b] <0pt,3pt> at 0 3
\put {$\Delta \rho$} [tl] <0pt,-3pt> at 2.72 1.27
\put {$\rho\Delta \phi$} [bl] <2pt,2pt> at 2.95 1.88
\put {$\Delta\phi$} at 1.01 0.645
\setsolid
\setcoordinatesystem units <1.5truecm,1.5truecm> point at -5 0
\setplotarea x from 0 to 3.5, y from 0 to 3
\axis left  /
\axis bottom  /
\circulararc 15 degrees from 2.87 1.34 center at 0 0
\circulararc 15 degrees from 2.05 0.955 center at 0 0
\setlinear
\plot 2.87 1.34 0 0 2.43 2.04 /
\put {$x$} [l] <3pt,0pt> at 3.5 0
\put {$y$} [b] <0pt,3pt> at 0 3
\put {$\rho\sin\phi\Delta \theta$} [bl] <2pt,2pt> at 2.67 1.7
\put {$\Delta\theta$} at 1.01 0.645
\endpicture}}
\caption{Setting up integration in spherical coordinates.}
\label{fig:intspherical}
\end{figure}

The upshot is that the volume of the little box is approximately
$\Delta\rho(\rho\Delta\phi)(\rho\sin\phi\Delta\theta)
=\rho^2\sin\phi\Delta\rho\Delta\phi\Delta\theta$, or in the limit
$\rho^2\sin\phi\,d\rho\,d\phi\,d\theta$.

\begin{example}{Average Temperature in a Unit Sphere}{AvgTempUnitSphere}
Suppose the temperature at $(x,y,z)$ is
$T=1/(1+x^2+y^2+z^2)$. Find the average temperature in the unit sphere
centered at the origin.
\end{example}
\begin{solution}
In two dimensions we add up the temperature at ``each'' point and
divide by the area; here we add up the temperatures and divide by the
volume, $(4/3)\pi$:
\[{3\over4\pi}\int_{-1}^1\int_{-\sqrt{1-x^2}}^{\sqrt{1-x^2}}
\int_{-\sqrt{1-x^2-y^2}}^{\sqrt{1-x^2-y^2}}
{1\over1+x^2+y^2+z^2}\,dz\,dy\,dx\]
This looks quite messy; since everything in the problem is closely
related to a sphere, we'll convert to spherical coordinates.
\[{3\over4\pi}\int_0^{2\pi}\int_0^\pi\int_0^1{1\over1+\rho^2}\,\rho^2\sin\phi\,d\rho\,d\phi\,d\theta
={3\over4\pi}(4\pi -\pi^2)=3-{3\pi\over4}.\]
\end{solution}


%%%%%%%%%%%%%%%%%%%%%%%%%%%%%%%%%%%%%%%%%%%%
\Opensolutionfile{solutions}[ex]
\section*{Exercises for \ref{sec:CylSphCoords}}

\begin{enumialphparenastyle}

\begin{ex}
Evaluate $\ds\int_{0}^{1}\int_{0}^{x}\int_{0}^{\sqrt{x^2+y^2}}
{(x^2+y^2)^{3/2}\over x^2+y^2+z^2}\,dz\,dy\,dx$.
\begin{sol}
$\pi/12$
\end{sol}
\end{ex}

\begin{ex}
Evaluate $\ds\int_{-1}^{1}\int_{0}^{\sqrt{1-x^2}}
\int_{\sqrt{x^2+y^2}}^{\sqrt{2-x^2-y^2}}\sqrt{x^2+y^2+z^2}\,dz\,dy\,dx$.
\begin{sol}
$\pi(1-\sqrt2/2)$
\end{sol}
\end{ex}

\begin{ex}
Evaluate $\ds\int\int\int x^2\,dV$
over the interior of the cylinder $x^2+y^2=1$ between $z=0$ and $z=5$.
\begin{sol}
$5\pi/4$
\end{sol}
\end{ex}

\begin{ex}
Evaluate $\ds\int\int\int xy\,dV$
over the interior of the cylinder $x^2+y^2=1$ between $z=0$ and $z=5$.
\begin{sol}
$0$
\end{sol}
\end{ex}

\begin{ex}
Evaluate $\ds\int\int\int z\,dV$
over the region above the $x$-$y$ plane, inside $x^2+y^2-2x=0$ and
under $x^2+y^2+z^2=4$.
\begin{sol}
$5\pi/4$
\end{sol}
\end{ex}

\begin{ex}
Evaluate $\ds\int\int\int yz\,dV$
over the region in the first octant, inside $x^2+y^2-2x=0$ and 
under $x^2+y^2+z^2=4$.
\begin{sol}
$4/5$
\end{sol}
\end{ex}

\begin{ex}
Evaluate $\ds\int\int\int x^2+y^2\,dV$
over the interior of $x^2+y^2+z^2=4$.
\begin{sol}
$256\pi/15$
\end{sol}
\end{ex}

\begin{ex}
Evaluate $\ds\int\int\int \sqrt{x^2+y^2}\,dV$
over the interior of $x^2+y^2+z^2=4$.
\begin{sol}
$4\pi^2$
\end{sol}
\end{ex}

\begin{ex}
Compute $\ds\int\int\int
x+y+z\,dV$ over the region inside
$x^2+y^2+z^2 = 1$ in the first octant.
\begin{sol}
$\ds {3\pi\over16}$
\end{sol}
\end{ex}

\begin{ex}
Find the mass of a right circular cone of height $h$ and
base radius $a$ if the density is proportional to the distance from
the base.
\begin{sol}
$\pi kh^2a^2/12$
\end{sol}
\end{ex}

\begin{ex}
Find the mass of a right circular cone of height $h$ and
base radius $a$ if the density is proportional to the distance from
its axis of symmetry.
\begin{sol}
$\pi kha^3/6$
\end{sol}
\end{ex}

\begin{ex}
An object occupies the region inside the unit sphere at the
origin, and has density equal to the distance from the $x$-axis. Find
the mass.
\begin{sol}
$\pi^2/4$
\end{sol}
\end{ex}

\begin{ex}
An object occupies the region inside the unit sphere at the
origin, and has density equal to the square of the distance from the
origin. Find the mass.
\begin{sol}
$4\pi/5$
\end{sol}
\end{ex}

\begin{ex}
An object occupies the region between the unit sphere at the
origin and a sphere of radius 2 with center at the origin, and has
density equal to the distance from the origin. Find the mass.
\begin{sol}
$15\pi$
\end{sol}
\end{ex}

\begin{ex}
An object occupies the region in the first octant bounded by
the cones $\phi = \pi/4$ and $\phi = \arctan 2$, and the sphere $\rho
= \sqrt{6}$, and has density proportional to the distance from the
origin. Find the mass.
\begin{sol}
$9k\pi(5\sqrt2-2\sqrt5)/20$
\end{sol}
\end{ex}

\end{enumialphparenastyle}
