\section{First Order Homogeneous Linear Equations}\label{sec:first order homogeneous linear}
A simple, but important and useful, type of separable equation is the
\dfont{first order homogeneous linear equation}:

\begin{definition}{First Order Homogeneous Linear Equation}{First Order Homogeneous Linear Equation}\label{First Order Homogeneous Linear Equation}
A first order homogeneous linear differential equation
is one of
the form $\ds y' + p(t)y=0$
or equivalently
$\ds y' = -p(t)y$.
\end{definition}

``Homogeneous'' refers to the zero on the right side of the equation, provided that $y'$ and $y$ are on the left. ``Linear'' in this definition indicates that both $y'$ and $y$ appear independently and explicitly; we don't see $y'$ or $y$ to any power greater than 1, or multiplied by each other (i.e. $y'y$).

\begin{example}{Linear Examples}{Linear Examples}\label{Linear Examples}
 The equation $\ds y' = 2t(25-y)$ can be written
$\ds y' + 2ty= 50t$. This is linear, but not homogeneous. The
equation $\ds y'=ky$, or $\ds y'-ky=0$ is linear and
homogeneous, with a particularly simple $p(t)=-k$.
The equation $y'+y^2=0$ is homogeneous, but not linear.
\end{example}

Since first order homogeneous linear equations are separable, we can
solve them in the usual way:
\begin{eqnarray*}
y' &=& -p(t)y\cr
\int {1\over y}\,dy &=& \int -p(t)\,dt\cr
\ln|y| &=& P(t)+C\cr
y&=&\pm\,e^{P(t)}\cr
y&=&Ae^{P(t)},
\end{eqnarray*}
where $P(t)$ is an anti-derivative of $-p(t)$. As in previous
examples, if we allow $A=0$ we get the constant solution $y=0$.

\begin{example}{Solving an IVP}{Solving an IVP}\label{Solving an IVP}
 Solve the initial value problem 
$$\ds y' + y\cos t =0,$$
subject to $y(0)=1/2$ and $y(2)=1/2$.
\end{example}

\begin{solution}
We start with
$$P(t)=\int -\cos t\,dt = -\sin t,$$
so the general solution to the differential equation is
$$y=Ae^{-\sin t}.$$
To compute $A$ we substitute:
$$ {1\over 2} = Ae^{-\sin 0} = A,$$
so the solutions is 
$$ y = {1\over 2} e^{-\sin t}.$$
For the second problem,
\begin{eqnarray*}
{1\over 2} &=& Ae^{-\sin 2}\cr
A &=& {1\over 2}e^{\sin 2}
\end{eqnarray*}
so the solution is 
$$ y = {1\over 2}e^{\sin 2}e^{-\sin t}.$$
\vskip-15pt\end{solution}

\begin{example}{}{}\label{}
 Solve the initial value problem $ty'+3y=0$, $y(1)=2$,
assuming $t>0$.
\end{example}

\begin{solution}
We write the equation in standard form: $y'+3y/t=0$. Then
$$P(t)=\int -{3\over t}\,dt=-3\ln t$$
and 
$$ y=Ae^{-3\ln t}=At^{-3}.$$
Substituting to find $A$:
$\ds 2=A(1)^{-3}=A$, so the solution is $\ds y=2t^{-3}$.
\end{solution}


%%%%%%%%%%%%%%%%%%%%%%%%%%%%%%%%%%%%%%%%%%%%
\Opensolutionfile{solutions}[ex]
\section*{Exercises for \ref{sec:first order homogeneous linear}}

\begin{enumialphparenastyle}

Find the general solution of each equation in the following exercises.

\begin{multicols}{2}
%%%%%%%%%%
\begin{ex}
 $\ds y'+5y=0$
\begin{sol}
 $\ds y=Ae^{-5t}$
\end{sol}
\end{ex}


%%%%%%%%%%
\begin{ex}
 $\ds y'-2y=0$
\begin{sol}
 $\ds y=Ae^{2t}$
\end{sol}
\end{ex}


%%%%%%%%%%
\begin{ex}
 $\ds y'+{y\over 1+t^2}=0$
\begin{sol}
 $\ds y=Ae^{-\arctan t}$
\end{sol}
\end{ex}


%%%%%%%%%%
\begin{ex}
 $\ds y'+t^2y=0$
\begin{sol}
 $\ds y=Ae^{-t^3/3}$
\end{sol}
\end{ex}

\end{multicols}

In the following exercises, solve the initial value problem.

\begin{multicols}{2}

%%%%%%%%%%
\begin{ex}
 $\ds y' + y=0$, $y(0)=4$
\begin{sol}
 $\ds y=4e^{-t}$
\end{sol}
\end{ex}


%%%%%%%%%%
\begin{ex}
 $\ds y' -3y=0$, $y(1)=-2$
\begin{sol}
 $\ds y=-2e^{3t-3}$
\end{sol}
\end{ex}


%%%%%%%%%%
\begin{ex}
 $\ds y' + y\sin t = 0$, $y(\pi)=1$
\begin{sol}
 $\ds y=e^{1+\cos t}$
\end{sol}
\end{ex}


%%%%%%%%%%
\begin{ex}
 $\ds y' +ye^t=0$, $y(0)=e$
\begin{sol}
 $\ds y=e^2e^{-e^t}$
\end{sol}
\end{ex}


%%%%%%%%%%
\begin{ex}
 $\ds y' +y\sqrt{1+t^4}=0$, $y(0)=0$
\begin{sol}
 $\ds y=0$
\end{sol}
\end{ex}


%%%%%%%%%%
\begin{ex}
 $\ds y' + y\cos(e^t)=0$, $y(0)=0$
\begin{sol}
 $\ds y=0$
\end{sol}
\end{ex}


%%%%%%%%%%
\begin{ex}
 $\ds ty' - 2y = 0$, $y(1)=4$
\begin{sol}
 $\ds y=4t^2$
\end{sol}
\end{ex}


%%%%%%%%%%
\begin{ex}
 $\ds t^2y' + y = 0$, $y(1)=-2$, $t>0$
\begin{sol}
 $\ds y=-2e^{(1/t)-1}$
\end{sol}
\end{ex}


%%%%%%%%%%
\begin{ex}
 $\ds t^3y' = 2y$, $y(1)=1$, $t>0$
\begin{sol}
 $\ds y=e^{1-t^{-2}}$
\end{sol}
\end{ex}


%%%%%%%%%%
\begin{ex}
 $\ds t^3y' = 2y$, $y(1)=0$, $t>0$
\begin{sol}
 $\ds y=0$
\end{sol}
\end{ex}

\end{multicols}


%%%%%%%%%%
\begin{ex}
 A function $y(t)$ is a solution of $\ds y' +
ky=0$. Suppose that $y(0)=100$ and $y(2)=4$. Find $k$ and find $y(t)$.
\begin{sol}
 $k=\ln 5$, $\ds y=100e^{-t\ln 5}$
\end{sol}
\end{ex}


%%%%%%%%%%
\begin{ex}
 A function $y(t)$ is a solution of $\ds y' +
t^ky=0$. Suppose that $y(0)=1$ and $y(1)=e^{-13}$. Find $k$ and find
$y(t)$. 
\begin{sol}
 $k=-12/13$, $\ds y=\exp(-13 t^{1/13})$
\end{sol}
\end{ex}


%%%%%%%%%%
\begin{ex}
 A bacterial culture grows at a rate proportional to its
population. If the population is one million at $t=0$ and 1.5
million at $t=1$ hour, find the population as a function of time.
\begin{sol}
 $\ds y=10^6e^{t\ln(3/2)}$
\end{sol}
\end{ex}


%%%%%%%%%%
\begin{ex}
 A radioactive element decays with a half-life of 6 years. If
a mass of the element weighs ten pounds at $t=0$, find the amount of
the element at time $t$.
\begin{sol}
 $\ds y=10e^{-t\ln(2)/6}$
\end{sol}
\end{ex}

\end{enumialphparenastyle}