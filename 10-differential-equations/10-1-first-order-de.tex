\section{First Order Differential Equations}\label{sec:first order differential equations}
We start by considering equations in which only the first derivative
of the function appears. 

\begin{definition}{First Order Differential Equation}{First Order Differential Equation}\label{First Order Differential Equation}
A \deffont{first order differential equation} is an equation of
the form $F(t, y, y')=0$.
A solution of a first order differential equation is a
function $f(t)$ that makes $\ds F(t,f(t),f'(t))=0$ for every value of $t$.
\end{definition}

Here, $F$ is a function of three
variables which we label $t$, $y$, and $y'$. It is understood
that $y' $ will explicitly appear in the equation although $t$
and $y$ need not. The term ``first order'' means that the first
derivative of $y$ appears, but no higher order derivatives do.

\begin{example}{Newton's Law of Cooling}{Newton's Law of Cooling}\label{Newton's Law of Cooling}
 The equation from Newton's law of cooling,
$y'=k(y-T)$ is a first order
differential equation; $F(t,y,y')=k(y-T)-y'$.
\end{example}

\begin{example}{A First Order Differential Equation}{A First Order Differential Equation}\label{A First Order Differential Equation}
 $\ds y'=t^2+1$ is a first order differential
equation; $\ds F(t,y,y')= y'-t^2-1$. All solutions to this
equation are of the form $\ds t^3/3+t+C$. 
\end{example}

\begin{definition}{First Order Initial Value Problem}{First Order Initial Value Problem}\label{First Order Initial Value Problem}
A \deffont{first order initial value problem} is a system of
equations of the form
$F(t, y, y')=0$, $y(t_0)=y_0$. Here $t_0 $ is a fixed time
and $y_0$ is a number.
A solution of an initial value problem is a solution $f(t)$ of
the differential equation that also satisfies the 
\deffont{initial condition} $f(t_0) = y_0$.
\end{definition}

\begin{example}{An Initial Value Problem}{An Initial Value Problem}\label{An Initial Value Problem}
Verify that the initial value problem $\ds y'=t^2+1$, $y(1)=4$
has solution $\ds f(t)=t^3/3+t+8/3$.
\end{example}

\begin{solution}
Observe that $f'(t)=t^2+1$ and $f(1)=1^3/2+1+8/3=4$ as required.
\end{solution}

The general first order equation is too general, so
we can't describe methods that will work on them all, or even a large
portion of them. We can make progress with specific kinds of
first order differential equations.
For example, much can be said about equations of the form
$\ds y' = \phi (t, y)$ where $\phi $
is a function of the two variables $t$ and $y$.
Under reasonable conditions on $\phi$, such an
equation has a solution and the corresponding 
initial value problem has a unique solution.
However, in general, these equations can be very difficult or
impossible to solve explicitly.

A special case for which we do have a well defined method is that of separable differential equations.

\subsection{Separable Differential Equations}




\begin{definition}{Separable Differential Equations}{Separable Differential Equations}\label{Separable Differential Equations}
A first order differential equation is 
\deffont{separable} if it can be written in the form
$$y' = f(t) g(y) \;\;\text{ or, }\;\;\frac{dy}{dt} = f(t) g(y).$$
\end{definition}

For example, the differential equation 
\[
  \frac{\; d y}{\; d x} = \sin(x) \bigl(1+y^2\bigr)
\]
is separable, and one has $F(x) = \sin x$ and $G(y) = 1+y^2$.
On the other hand, the differential equation
\[
  \frac{\; d y}{\; d x} = x+y
\]
is not separable.


The general approach to separable equations is as follows:

Suppose we wish to solve $y' =
f(t) g(y) $ where $f$ and $g$ are continuous functions. If $g(a)=0$
for some $a$ then $y(t)=a$ is a constant solution of the equation,
since in this case $y' = 0 = f(t)g(a)$.  For example, $y'
=y^2 -1$ has constant solutions $y(t)=1$ and $y(t)=-1$.

Such constant solutions to a differential equation are called \textit{equilibrium solutions}. To find the nonconstant solutions, we divide by $g(y)$ to get
\begin{equation}
  \label{eq:separated}
  \frac{1}{ g(y)} \frac{\; d y}{\; d t} = f(t). 
\end{equation}
Next find a function $H(y)$ whose derivative with respect to $y$ is
\begin{equation}\label{eq:separable-3}
  H'(y) = \frac{1}{g(y)}
  \quad\left(\text{solution: } H(y) = \int {\frac{dy}{g(y)}}.\right)
\end{equation}
Then the chain rule implies that the left hand side in (\ref{eq:separated}) can be written as
\[
  \frac{1}{ g(y)} \frac{\; d y}{\; d t} = H'(y) \frac{\; d y}{\; d t} =
  \frac{\; d H(y)}{\; d t}.
\]
Thus \eqref{eq:separated} is equivalent with
\[
\frac{\; d H(y)}{\; d t} = f(t).
\]
In words: $H(y)$ is an antiderivative of $f(t)$, which means we can find $H(y)$
by integrating $f(t)$:
\begin{equation}
  \label{eq:separable-solution}
  H(y) = \int f(t) dt +C. 
\end{equation}
Once we have found the integral of $f(t)$ this gives us $y(t)$ in implicit form: the
equation (\ref{eq:separable-solution}) gives us $y(t)$ as an \textit{implicit
  function} of $t$.  To get $y(t)$ itself we must solve the equation
(\ref{eq:separable-solution}) for $y(t)$.

A quick way of organizing the calculation goes like this:
\begin{quote}
  To solve \( \ds \frac{dy}{ dt} = f(t)g(y)\) we first \textit{separate the
    variables},
  \[
  \frac{d y}{g(y)} = f(t)\,d t,
  \]
  and then integrate,
  \[
  \int\frac{d y}{g(y)} = \int f(t)\, dt.
  \]
  The result is an implicit equation for the solution $y$ with one undetermined
  integration constant.
\end{quote}

This technique is called \dfont{separation of variables}. 

As we have seen so far, a differential equation typically has an infinite number of solutions. Such a solution is called a \dfont{general solution}.  A corresponding initial value problem will give rise to just one
solution. Such a solution in which there are no unknown constants remaining is called a \dfont{particular solution}.


\begin{example}{}{}
Find all functions $y$ that are solutions to the differential equation 
$$\frac{dy}{dt}= \frac{t}{y^2}.$$
\end{example}


\begin{solution}
We begin by separating the variables and writing
    $$
    y^2 dy  = t\; dt.
    $$
Integrating both sides of the equation with respect to the independent
    variable $t$ shows that
    $$
    \int y^2\frac{dy}{dt}~dt = \int t~dt.
    $$
Next, we notice that the left-hand side allows us to change
    the variable of antidifferentiation\footnote{This is why we required that the left-hand side be written as a
    product in which $dy/dt$ is one of the terms.} from $t$ to $y$.  In
    particular,
    $dy = \frac{dy}{dt}~dt$, so we now have
    $$
    \int y^2 ~dy = \int t~dt.
    $$
   This most recent equation says that two families of antiderivatives are
    equal to one another.  Therefore, when we find representative
    antiderivatives of both sides, we know they must differ by
    arbitrary constant $C$.  Antidifferentiating and including the integration constant $C$ on the right, we find that
    $$
    \frac{y^3}{3} = \frac{t^2}{2} + C.
    $$
    Again, note that it is not necessary to include an arbitrary constant on both sides 
    of the equation;  we know that $y^3/3$ and $t^2/2$ are in the same
    family of antiderivatives and must therefore differ by a single
    constant.

Finally, we may now solve the last equation above for $y$ as a function of $t$, which gives
    $$
    y(t) = \sqrt[3]{\frac 32 \thinspace t^2 + 3C}.
    $$
    Of course, the term $3C$ on the right-hand side represents
    3 times an unknown constant.  It is, therefore, still an unknown
    constant, which we will rewrite as $C$.  We thus conclude that the funtion
    $$
    y(t) = \sqrt[3]{\frac 32 \thinspace t^2 + C}
    $$
is a solution to the original differential equation for any value of $C$.
\end{solution}



Notice that because this solution depends on the arbitrary constant $C$, we have found an infinite family of
solutions.  This makes sense because we expect to find a unique solution that corresponds to any given
 initial value.

For example, if we want to solve the initial value problem
$$
  \frac{dy}{dt} = \frac{t}{y^2}, \
  y(0) = 2,
$$
we know that the solution has the form $y(t) = \sqrt[3]{\frac32\thinspace
  t^2 + C}$ for some constant $C$.  We therefore must find the appropriate
value for $C$ that gives the initial value $y(0)=2$.  Hence,
$$
  2 = y(0)  \sqrt[3]{\frac 32 \thinspace 0^2 + C} = \sqrt[3]{C},
  $$
which shows that $C = 2^3 = 8$.  The solution to the initial value problem is then
$$
y(t) = \sqrt[3]{\frac32\thinspace t^2+8}.
$$





\begin{example}{Solving an IVP}{Solving an IVP}\label{Solving an IVP}
 Solve the IVP: $\ds y' = 2t(25-y)$, $ y(0)= 20 $.
\end{example}
 
\begin{solution}
We begin by finding the general solution to the differential equation. 
This is almost identical to the previous example. As before, $y(t)=25$
is a solution. If $y\not=25$,
\begin{eqnarray}
\int {1\over 25-y}\,dy &=& \int 2t\,dt\cr
(-1)\ln|25-y| &=& t^2+C_0\cr
\ln|25-y| &=& -t^2 - C_0 = -t^2 + C\cr
|25-y| &=& e^{-t^2+C}=e^{-t^2} e^C\cr
y-25 &=& \pm\, e^C e^{-t^2} \cr
y &=& 25 \pm e^C e^{-t^2} =25+Ae^{-t^2}. \label{eqn:solveIVP}
\end{eqnarray}
As before, all solutions are represented by $\ds y=25+Ae^{-t^2}$,
allowing $A$ to be zero.

To solve the IVP, we let $ y= 20$, and $ t=0 $ in Equation \ref{eq:solveIVP} to get
$$
20=25+A
$$
which immediately gives $ A=-20 $. So the particular solution to the IVP is
\[
y=25-20e^{-t^2}
\]
\end{solution}


One application often discussed when introducing Separable Equations is that of \textbf{mixing problems}.
A typical mixing problems involves: A tank of fixed capacity; a completely mixed solution of some substance in the tank; a solution of a certain concentration entering the tank at a (usually) fixed rate; the solution immediately becomes completely stirred; and the mixture leaves at the other end at a (usually fixed) rate. We illustrate with an example.

\begin{example}
A tank contains 20 kg of salt dissolved in 5000 L of water.  Brine that contains 0.03 kg of salt per liter of water enters the tank at a rate of 25 L/min.  The solution is kept thoroughly mixed and drains from the tank at the same rate. {How much salt is in the tank after half an hour?}
\end{example}


\begin{solution}
Let $y(t)$ denote the amount of salt (kg) in the tank after $t$ minutes.

{Given: $y(0) =  {20.}$} {We want to know: $ {y(30).}$}

 
\[
\frac{d  y}{d t} = \textrm{(rate in) $-$ (rate out)}%
\]


\[
\textrm{rate in} = \textrm{( {concentration in})( {rate of volume in})}%
=\left( { {0.03\; \frac{\textrm{kg}}{\textrm{L}}}}\right)\left( { {25\; \frac{\textrm{L}}{\textrm{min}}}}\right)%
 { = }  
0.75\;\; \frac{\textrm{kg}}{\textrm{min}}%
\]

\[
\textrm{rate out}%
  = \textrm{( {concentration out})( {rate of volume out})}%
=\left( { {\frac{y(t)}{5000}\; \frac{\textrm{kg}}{\textrm{L}}}}\right)\left( { {25\; \frac{\textrm{L}}{\textrm{min}}}}\right)%
 =\frac{y(t)}{200} \;\; \frac{\textrm{kg}}{\textrm{min}}%
\]

Therefore we have 

\[
\frac{dy}{dt}= \frac{150 - y(t)}{200} 
\]

Separating variables we get

\[
\int \frac{1}{150-y} \; dy =\int \frac{1}{200}\; dt 
\]

which gives

\[
-\ln |150 - y| = t /200 {+ C}%
\]
 $ y(0) = 20 $,  so   $ C = {-\ln 130} $. Also observe that since $ y<150 (= 0.3\cdot 5000)$, so $ |150-y|=150-y$,  so after simplification we get


\[
y=150 - 130e^{-t/200}
\]

and therefore $ y(30)=150 - 130e^{-30/200} \approx 38.1 $kg.
\end{solution}








\begin{example}{}{}
Solve the differential equation
$$\frac{dy}{dt} =3y.$$
\end{example}


\begin{solution}
Following the same strategy as in Example~\ref{Ex:7.4.1}, we have
$$  \frac 1y \frac{dy}{dt} = 3. $$
Integrating both sides with respect to $t$,
$$  \int \frac 1y\frac{dy}{dt}~dt = \int 3~dt,$$
and thus 
$$ \int \frac 1y~dy =  \int 3~dt.$$
Antidifferentiating and including the integration constant, we find that
$$  \ln|y| = 3t + C_1$$
where $ C_1 $ is an arbitrary constant. Finally, we need to solve for $y$.  Here, one point deserves careful
attention.  By the definition of the natural logarithm function, it follows that
$$
|y| = e^{3t+C_1} = e^{3t}e^{C_1}.
$$
Since $C$ is an unknown constant, $e^C$ is as well, though we do know
that it is positive (because $e^x$ is positive for any $x$).
When we remove the absolute value in order to solve for $y$ we obtain 
$$
y = \pm e^{C_1} e^{3t}.
$$
As $ \pm e^{C_1} $ may be either positive or
negative, we will denote this  by $C$ to obtain
$$
y(t) = Ce^{3t}.
$$

There is one technical point to make here.  Notice that $y=0$
is an equilibrium  solution to this differential equation.  In solving
the equation above, we begin by dividing both sides by $y$, which
is not allowed if $y=0$.  To be perfectly careful, therefore, we will typically
consider these equilibrium solutions separately.  In this case, notice that the final
form of our solution captures the equilibrium solution by allowing
$C=0$.
\end{solution}


\subsection{Exponential Growth and Decay}

The differential equation in the previous example ($ y'=3y $) describes a quantity $ y $ whose rate of change is directly proportional to the quantity itself. Such a differential equation is said to model exponential growth. 






\begin{example}{Population Growth and Radioactive Decay}{Population Growth and Radioactive Decay}\label{Population Growth and Radioactive Decay}
Analyze the differential equation $y'=ky$.
\end{example}

\begin{solution}
When $k>0$, this describes certain simple cases of (exponential) population growth:
It says that the change in the population $y$ is proportional to the
population. The underlying assumption is that each organism in the
current population reproduces at a fixed rate, so the larger the
population the more new organisms are produced. While this is too
simple to model most real populations, it is useful in some cases over
a limited time.  The parameter $ k $ is called the \textit{proportionality constant}. 
When $k<0$, the differential equation describes a
quantity that decreases in proportion to the current value (exponential decay); this can
be used to model radioactive decay.

The constant solution is $y(t)=0$; of course this will not be the
solution to any interesting initial value problem. 
For the non-constant solutions, we proceed much as before:
\begin{eqnarray*}
\int {1\over y}\,dy&=&\int k\,dt\cr
\ln|y| &=& kt+C\cr
|y| &=& e^{kt} e^C\cr
y &=& \pm \,e^C e^{kt} \cr
y&=& Ae^{kt}.
\end{eqnarray*}
Again, if we allow $A=0$ this includes the equilibrium solution, and we
can simply say that $\ds y=Ae^{kt}$ is the general solution. With an
initial value we can easily solve for $A$ to get the solution of the
initial value problem. In particular, if the initial value is
given for time $t=0$, $y(0)=y_0$, then $A=y_0$ and the solution
is $\ds y= y_0 e^{kt}$.
\end{solution}

In general, the work in the previous example shows the following to hold true.

\begin{formulabox}[\label{expDE} ]
The solution of the initial value problem
\[
\frac{dy}{dt}=ky,\;\;\;\;y(0)=y_0
\]
is $ \ds y=y_0e^{kt} $
\end{formulabox}


\begin{example}{Global Population Growth}{}
Assuming that the growth rate is proportional to population size,
 use the fact the world population in 1900 is 1650 million and the 1910 is 1750
 million to estimate population in the year 2000.
\end{example}


\begin{solution}
Since growth rate is proportional to population, we 
know that the population  $P(t)$ will be given by a function of the form:
\[
P=P_0e^{kt}
\] 
taking $t$ to be the number of years after 1900.  
We are asked to find the population in the year 2000, in other words, find $P(100)$.  We know $P_0=1650$ (in millions), so 
\begin{equation} \label{eq:pop}
P=1650e^{kt}
\end{equation}


So we must solve for the growth constant $k$. In 1910, (when $t=10$) the population was 1750 (million), so  $ P(10)=1750 $:
\[
1750=1650e^{10k}
\]
which gives

\[
k=\frac{1}{10}\ln\left(\frac{175}{165}\right)
\]
Substituting into equation \ref{eq:pop} and simplifying gives 
\[
P=1650\left(\frac{175}{165}\right)^{\frac{t}{10}}
\]

Therefore, after 100 years, the population will be
\[
P(100)=1650\left(\frac{175}{165}\right)^{10}
\approx 2972 \textrm{ million.}
\]

\end{solution}

As mentioned previously, radioactive decay also follows an exponential model, $ y=y_0e^{kt} $ (where $ k<0 $). The   \textit{half-life} of a material  is the time required for half of a given amount to decay. That is, the time for which $ \frac12y_0 = y_0e^{kt} $. Solving for $ t $ gives $ t=-\frac{\ln(2)}{k} $.

 \begin{formulabox}[\label{halflife}Half Life ]
 Radioactive decay of a material with decay constant $ k$ is modelled by $ y=y_0e^{kt} $, and has a half-life of $ \ds -\frac{ln(2)}{k} $    
 \end{formulabox} 


\begin{example}{}{}
The half-life of radium-226 is 1590 years.  A sample of radium has a mass of 100 mg.
\begin{enumerate}
\item Find a formula for the mass of radium after $ t $ years. 
\item Find the mass after 1000 years.

\item When will the mass be reduced to 30mg?
\end{enumerate}

\end{example}


\begin{solution}
\begin{enumerate}
\item As this model is one of exponential decay, we know the formula for the mass after $t$ years will have the form:
\[
y=y_0e^{kt},
\]
where $k<0$. We are told $100$mg are initially present, so $ y_0=100$. To determine the decay constant, we use the given half-life with the formula in Key Idea \ref{halflife}:
\[
k =  -\frac{ln(2)}{1590}.
\] 
Therefore, after simplifications we have
\[
y=100e^{t\cdot -\frac{ln(2)}{1590}} = 100\cdot \left(\frac12\right)^\frac{t}{1590}.
\]
\item From part (a) we see that after $ 1000 $  years the amount remaining will be
\[
y(1000)=100\cdot \left(\frac12\right)^{\frac{1000}{1590}}\approx 64.67\textrm{mg}.
\]
 
\item  We wish to find $t$ when $y=30$, so we solve: $\ds 30=100\cdot \left(\frac12\right)^{\frac{t}{1590}}$. Dividing by $ 100 $, taking the natural logarithm of both sides, and solving for $ t $ gives
\[
\ln\left(\frac{3}{10}\right)= \frac{t}{1590} \ln\left(\frac12\right)
\to\;t= 1590\cdot\frac{\ln\left(\frac{3}{10}\right)}{\ln\left(\frac12\right)}\approx 2762 \textrm{ years.}
\]

\end{enumerate}

\end{solution}

More generally, a quantity $y$ may grow (or shrink) with rate of change proportional to a difference $y-b$. Such is the case with Newton's Law of Cooling. 

\begin{formulabox}[\label{NLC} Newton's Law of Cooling ]
The rate of cooling of an object is directly proportional to the difference between the temperature $y(t)$ of the object and the ambient temperature $T$ ( i.e. the temperature $T$ of its surroundings.)
  \[
  \frac{dy}{dt}=k(y-T)
  \]                  
where $ k $ is called the cooling constant (in units of $ (\text{time})^{-1} $), and depends on the physical properties of the materials involved. This differential equation may be solved in the same manner as in Example \ref{exa:Population Growth and Radioactive Decay} to give
\[
y= T+y_0e^{kt}
\]                   
\end{formulabox}


More generally, if $\frac{dy}{dt} = k(y-b) $ for some constant $ b $, then $ y=b+Ce^{kt} $, where $ C=y(0)$.


\begin{example}{IVP for Newton's Law of Cooling}{IVP for Newton's Law of Cooling}\label{IVP for Newton's Law of Cooling}
 Consider this specific example of an initial value problem
for Newton's law of cooling: $y' = -2(y-25)$, $y(0)=40$.
Discuss the solutions for this initial value problem.
\end{example}

\begin{solution}
We first note the zero of the equation: If $y = 25$, the right hand side of the differential
equation is zero, and so the constant function $y(t)=25$ is a solution
to the differential equation. It is not a solution to the initial
value problem, since $y(0)\neq 25$.  (The physical interpretation of
this constant solution is that if a liquid is at the same temperature
as its surroundings, then the liquid will stay at that temperature.)

At this point we may appeal to the key idea \ref{Newton's Law of Cooling}, taking $ T=25 $, and $ k=-2 $ to solve the differential equation.  However, just for practice we will derive the result directly. 

Separating variables, so long as $y\ne 25$,  we can rewrite the differential equation as
\begin{eqnarray*}
{dy\over dt}{1\over 25-y}&=&2\cr
{1\over 25-y}\,dy&=&2\,dt,
\end{eqnarray*}
so 
$$\int {1\over 25-y}\,dy = \int 2\,dt,$$
We can calculate these anti-derivatives and 
rearrange the results:
\begin{eqnarray*}
\int {1\over 25-y}\,dy &=& \int 2\,dt\cr
(-1)\ln|25-y| &=& 2t+C_0\cr
\ln|25-y| &=& -2t - C_0 = -2t + C_1\cr
|25-y| &=& e^{-2t+C_1}=e^{-2t} e^{C_1}\cr
y-25 &=& \pm\, e^{C_1} e^{-2t} \cr
y &=& 25 \pm e^{C_1} e^{-2t} =25+Ce^{-2t}.
\end{eqnarray*}
Here $\ds C = \pm\, e^{C_1} = \pm\, e^{-C_0}$ 
is some non-zero constant. Note that this agrees with the solution we would have obtained directly from Key Idea \ref{Newton's Law of Cooling}.

Since we require $y(0)=40$, we substitute and solve for $C$:
\begin{eqnarray*}
40&=&25+Ce^0\cr
15&=&C,
\end{eqnarray*}
and so $\ds y=25+15 e^{-2t}$ is a solution to the initial value
problem. 

Note that $y$ is never $ 25 $, so this makes sense for all values
of $t$. However, if we allow $C=0$ we get the solution
$y=25$ to the differential equation, which would be the solution to
the initial value problem if we were to require $y(0)=25$. Thus, 
$\ds y=25+Ce^{-2t}$ describes all solutions to the differential
equation $\ds y' = 2(25-y)$, and all solutions to the associated
initial value problems. 
\end{solution}


 
\begin{example}{}{}
If an object takes $ 40 $ minutes to cool from $ 30 $ degrees to $ 24 $ degrees in a $ 20 $
degree room, how long will it take the object to cool to $ 21 $ degrees?                                   
\end{example}


\begin{solution}
From the above discussion we know that the model for the temperature $ y $, $ t $ minutes after the first temperature measurement is given by
\[
y=20+10e^{kt}.
\]
To solve for $ k $ we use the fact that $ y(40) =24$. Substituting $ t=40, $ and $ y=24 $ into the last equation, and simplifying gives
\[
24=20 10e^{40k}\;\;\to\;\; k= \ln\left[\left(\frac{2}{5}\right)^{\frac{1}{40}}\right]
\]
Therefore, 
\[
y=10e^{t\cdot \ln\left[\left(\frac{2}{5}\right)^{\frac{1}{40}}\right]}+20 =10e^{\ln\left[\left(\frac{2}{5}\right)^{\frac{t}{40}}\right]}+20=10\left(\frac{2}{5}\right)^{\frac{t}{40}}+20
\]

So when the temperature is $ 21 $ degrees, we have
\[
21=10\left(\frac{2}{5}\right)^{\frac{t}{40}}+20\Rightarrow 
t= \frac{-40\ln(10)}{\ln\left( \frac{2}{5} \right)} \approx 100.52 \text{min.}
\]
Therefore, even though it took only $ 40 $ minutes to cool from $ 30 $ degrees to $ 24 $ degrees 
(a difference of $ 6 $ degrees),  it will take over $ 100 $ minutes to cool to $ 21 $ degrees
 (the last $ 3 $ degrees add more than an hour to the time required!).

\end{solution}




                                   




%%%%%%%%%%%%%%%%%%%%%%%%%%%%%%%%%%%%%%%%%%%%
\Opensolutionfile{solutions}[ex]
\section*{Exercises for \ref{sec:first order differential equations}}

\begin{enumialphparenastyle}

%%%%%%%%%%
\begin{ex}
 Which of the following equations are separable?

\begin{enumerate}
	\item $\ds y' = \sin (ty)$
	\item $\ds y' = e^t e^y $
	\item $\ds yy' = t $
	\item $\ds y' = (t^3 -t) \arcsin(y)$
	\item $\ds y' = t^2 \ln y + 4t^3 \ln y $
\end{enumerate}
\end{ex}

%%%%%%%%%%
\begin{ex}
 Solve $\ds y' = 1/(1+t^2)$.
\begin{sol}
 $\ds y=\arctan t + C$
\end{sol}
\end{ex}

%%%%%%%%%%
\begin{ex}
 Solve the initial value problem $y' = t^n$ with
$y(0)=1$ and $n\ge 0$.
\begin{sol}
 $\ds y={t^{n+1}\over n+1}+1$
\end{sol}
\end{ex}

%%%%%%%%%%
\begin{ex}
 Solve $y' = \ln t$. 
\begin{sol}
 $\ds y=t\ln t-t+C$
\end{sol}
\end{ex}

%%%%%%%%%%
\begin{ex}
 Identify the constant solutions (if any) of $y' =t\sin y$.
\begin{sol}
 $y=n\pi$, for any integer $n$.
\end{sol}
\end{ex}

%%%%%%%%%%
\begin{ex}
 Identify the constant solutions (if any) of $\ds y'=te^y$.
\begin{sol}
 none
\end{sol}
\end{ex}

%%%%%%%%%%
\begin{ex}
 Solve $y' = t/y$.
\begin{sol}
 $\ds y=\pm\sqrt{t^2+C}$
\end{sol}
\end{ex}

%%%%%%%%%%
\begin{ex}
 Solve $\ds y' = y^2 -1$.
\begin{sol}
 $\ds y=\pm 1$, $\ds y=(1+Ae^{2t})/(1-Ae^{2t})$
\end{sol}
\end{ex}

%%%%%%%%%%
\begin{ex}
 Solve $\ds y' = t/(y^3 - 5)$. You may leave
your solution in implicit form: that is, you may stop once you have
done the integration, without solving for $y$.
\begin{sol}
 $\ds y^4/4-5y=t^2/2+C$
\end{sol}
\end{ex}

%%%%%%%%%%
\begin{ex}
 Find a non-constant solution of the initial value problem 
$y' = y^{1/3}$, $y(0)=0$, using
 separation of variables. Note that the constant function $y(t)=0 $
 also solves the initial value problem. This shows that an initial value
 problem can have more than one solution.
\begin{sol}
 $\ds y=(2t/3)^{3/2}$
\end{sol}
\end{ex}

%%%%%%%%%%
\begin{ex}
 Solve the equation for Newton's law of cooling leaving $M$
and $k$ unknown.
\begin{sol}
 $\ds y=M+Ae^{-kt}$
\end{sol}
\end{ex}


%%%%%%%%%%
\begin{ex}
 After 10 minutes in Jean-Luc's room, his tea has
cooled to $40^\circ $ Celsius from $100^\circ$ Celsius. 
The room temperature is $25^\circ$
Celsius. How much longer will it take to cool to $35^\circ$?
\begin{sol}
 $\ds {10\ln(15/2)\over\ln 5}\approx 2.52$ minutes
\end{sol}
\end{ex} 


%%%%%%%%%%
\begin{ex}
 Solve the \dfont{logistic equation} $y' = ky(M-y)$. (This is a somewhat more
reasonable population model in most cases than the simpler
$y'=ky$.) Sketch the
graph of the solution to this equation when 
$M=1000$, $k=0.002$, $y(0)=1$.
\begin{sol}
 $\ds y={M\over 1+Ae^{-Mkt}}$
\end{sol}
\end{ex}


%%%%%%%%%%
\begin{ex}
 Suppose that $y' = ky$, $y(0)=2$, and $y'(0)=3$. 
What is $y$?
\begin{sol}
 $\ds y=2e^{3t/2}$
\end{sol}
\end{ex}


%%%%%%%%%%
\begin{ex}
 A radioactive substance obeys the equation
$y' =ky$ where $k< 0 $ and $y$ is the mass of the
substance at time $t$. Suppose that initially, the mass of the
substance is $y(0)=M>0$. At what time does half of the mass remain?
(This is known as the half life. Note that the half life depends on
$k$ but not on $M$.)
\begin{sol}
 $\ds t=-{\ln 2\over k}$
\end{sol}
\end{ex}


%%%%%%%%%%
\begin{ex}
 Bismuth-210 has a half life of five days. If there is
initially 600 milligrams, how much is left after 6 days? When will
there be only 2 milligrams left?
\begin{sol}
 $\ds 600e^{-6\ln 2/5}\approx 261$ mg; $\ds {5\ln
  300\over\ln2}\approx 41$ days
\end{sol}
\end{ex}


%%%%%%%%%%
\begin{ex}
 The half life of carbon-14 is 5730 years. If one starts
with 100 milligrams of carbon-14, how much is left after 6000
years? How long do we have to wait before there is less than 2
milligrams?
\begin{sol}
 $\ds 100e^{-200\ln 2/191}\approx 48$ mg; $\ds {5730\ln
  50\over\ln2}\approx 32339$ years
\end{sol}
\end{ex}


%%%%%%%%%%
\begin{ex}
 A certain species of bacteria doubles its population
(or its mass)
every hour in the lab. 
The differential equation that models this phenomenon
is $y' =ky$, where $k>0 $ and $y$
is the population of bacteria at time $t$. What is $y$?
\begin{sol}
 $\ds y=y_0e^{t\ln 2}$
\end{sol}
\end{ex}


%%%%%%%%%%
\begin{ex}
 If a certain microbe doubles its population every 4
hours and after 5 hours the total population has mass 500 grams,
what was the initial mass?
\begin{sol}
 $\ds 500e^{-5\ln2/4}\approx 210$ g
\end{sol}
\end{ex}

\end{enumialphparenastyle}