\section{Second Order Linear Equations - Variation of Parameters}{}{}\label{sec:2nd order differential equations two} 
The method of the last section works only when the function $f(t)$ in
$\ds ay''+by'+cy=f(t)$ has a particularly nice form, namely,
when the derivatives of $f$ look much like $f$ itself. In other cases
we can try variation of parameters as we did in the first order case.

Since as before
$a\not=0$, we can always divide by $a$ to make the coefficient of
$\ds y''$ equal to 1. Thus, to simplify the discussion, we assume $a=1$. 
We know that the differential equation $\ds y''+by'+cy=0$
has a general solution $\ds y=Ay_1+By_2$. As before, we guess a
particular solution to $\ds y''+by'+cy=f(t)$; this time we use
the guess $\ds y=u(t)y_1+v(t)y_2$. Compute the derivatives:
\begin{eqnarray*}
y'&=&u'y_1+uy'_1+v'y_2+vy'_2\cr
y''&=&u''y_1+u'y'_1+u'y'_1+uy''_1+v''y_2+v'y'_2+v'y'_2+vy''_2.
\end{eqnarray*}
Now substituting:
\begin{eqnarray*}
y''+by'+cy&=&
u''y_1+u'y'_1+u'y'_1+uy''_1+v''y_2+v'y'_2+v'y'_2+vy''_2\cr
&&\qquad + bu'y_1+buy'_1+bv'y_2+bvy'_2+cuy_1+cvy_2\cr
&=&(uy''_1+buy'_1+cuy_1)+(vy''_2+bvy'_2+cvy_2)\cr
&&\qquad + b(u'y_1+v'y_2) + (u''y_1+u'y'_1+v''y_2+v'y'_2)+
(u'y'_1+v'y'_2)\cr
&=&0+0+ b(u'y_1+v'y_2) + (u''y_1+u'y'_1+v''y_2+v'y'_2)+
(u'y'_1+v'y'_2).
\end{eqnarray*}
The first two terms in parentheses are zero because $y_1$ and $y_2$
are solutions to the associated homogeneous equation. Now we engage in
some wishful thinking. If $\ds u'y_1+v'y_2=0$, then we also have
$\ds u''y_1+u'y'_1+v''y_2+v'y'_2=0$ by taking derivatives of both sides. This reduces the
entire expression to $\ds u'y'_1+v'y'_2=0$. We want this
to be $f(t)$, that is, we need 
$\ds u'y'_1+v'y'_2=f(t)$.
So we would very much like these equations to be true:
\begin{eqnarray*}
u'y_1+v'y_2&=&0\cr
u'y'_1+v'y'_2&=&f(t).
\end{eqnarray*}
This is a system of two equations in the two unknowns $\ds u'$ and
$\ds v'$, so we can solve as usual to get $\ds u'=g(t)$ and
$\ds v'=h(t)$. Then we can find $u$ and $v$ by computing
antiderivatives. This is of course the sticking point in the whole
plan, since the antiderivatives may be impossible to
find. Nevertheless, this sometimes works out and is worth a try.

\begin{example}{Variation of Parameters}{Variation of Parameters}\label{Variation of Parameters}
Consider the equation $\ds y''-5y'+6y=\sin t$. 
Solve using variation of parameters. 
\end{example}

\begin{solution}
The solution to the homogeneous equation is
$\ds Ae^{2t}+Be^{3t}$, so the 
simultaneous equations to be solved are
\begin{eqnarray*}
u'e^{2t}+v'e^{3t}&=&0\cr
2u'e^{2t}+3v'e^{3t}&=&\sin t.
\end{eqnarray*}
If we multiply the first equation by 2 and subtract it from the second
equation we get
\begin{eqnarray*}
v'e^{3t}&=&\sin t\cr
v'&=&e^{-3t}\sin t\cr
v&=&-{1\over 10}(3\sin t+\cos t)e^{-3t},
\end{eqnarray*}
using integration by parts. Then from the first equation:
\begin{eqnarray*}
u'&=&-e^{-2t}v'e^{3t}=-e^{-2t}e^{-3t}\sin(t)e^{3t}=-e^{-2t}\sin
t\cr
u&=&{1\over 5}(2\sin t+\cos t)e^{-2t}.
\end{eqnarray*}
Now the particular solution we seek is
\begin{eqnarray*}
ue^{2t}+ve^{3t}&=&{1\over 5}(2\sin t+\cos t)e^{-2t}e^{2t}
-{1\over 10}(3\sin t+\cos t)e^{-3t}e^{3t}\cr
&=&{1\over 5}(2\sin t+\cos t)-{1\over 10}(3\sin t+\cos t)\cr
&=&{1\over 10}(\sin t+\cos t),
\end{eqnarray*}
and the solution to the differential equation is
$\ds Ae^{2t}+Be^{3t}+(\sin t+\cos t)/10$. For comparison (and
practice) you might want to solve this using the method of
undetermined coefficients---both techniques should yield the same result.
\end{solution}

\begin{example}{Variation of Parameters}{Variation of Parameters 2}\label{Variation of Parameters 2}
 The differential equation $\ds y''-5y'+6y=e^t\sin t$
can be solved using the method of undetermined coefficients, though we
have not seen any examples of such a solution. Again, we will solve it
by variation of parameters.
\end{example}

\begin{solution}
The equations to be solved are 
\begin{eqnarray*}
u'e^{2t}+v'e^{3t}&=&0\cr
2u'e^{2t}+3v'e^{3t}&=&e^t\sin t.
\end{eqnarray*}
If we multiply the first equation by 2 and subtract it from the second
equation we get
\begin{eqnarray*}
v'e^{3t}&=&e^t\sin t\cr
v'&=&e^{-3t}e^t\sin t=e^{-2t}\sin t\cr
v&=&-{1\over 5}(2\sin t+\cos t)e^{-2t}.
\end{eqnarray*}
Then substituting we get
\begin{eqnarray*}
u'&=&-e^{-2t}v'e^{3t}=-e^{-2t}e^{-2t}\sin(t)e^{3t}=-e^{-t}\sin
t\cr
u&=&{1\over 2}(\sin t+\cos t)e^{-t}.
\end{eqnarray*}
The particular solution is
\begin{eqnarray*}
ue^{2t}+ve^{3t}&=&{1\over 2}(\sin t+\cos t)e^{-t}e^{2t}
-{1\over 5}(2\sin t+\cos t)e^{-2t}e^{3t}\cr
&=&{1\over 2}(\sin t+\cos t)e^t-{1\over 5}(2\sin t+\cos t)e^t\cr
&=&{1\over 10}(\sin t+3\cos t)e^t,
\end{eqnarray*}
and the solution to the differential equation is
$\ds Ae^{2t}+Be^{3t}+e^t(\sin t+3\cos t)/10$.
\end{solution}

\begin{example}{Solving a DE}{Solving a DE}\label{Solving a DE}
 The differential equation $\ds y'' -2y'+y=e^t/t^2$ is
not of the form amenable to the method of undetermined
coefficients. Solve it using variation of parameters.
\end{example}

\begin{solution}
The solution to the homogeneous equation is
$\ds Ae^t+Bte^t$ and so the simultaneous equations are
\begin{eqnarray*}
u'e^{t}+v'te^{t}&=&0\cr
u'e^{t}+v'te^{t}+v'e^t&=&{e^t\over t^2}.
\end{eqnarray*}
Subtracting the equations gives
\begin{eqnarray*}
v'e^{t}&=&{e^t\over t^2}\cr
v'&=&{1\over t^2}\cr
v&=&-{1\over t}.
\end{eqnarray*}
Then substituting we get
\begin{eqnarray*}
u'e^t&=&-v'te^t=-{1\over t^2}te^t\cr
u'&=&-{1\over t}\cr
u&=&-\ln t.
\end{eqnarray*}
The solution is $\ds Ae^t+Bte^t-e^t\ln t-e^t$.
\end{solution}


%%%%%%%%%%%%%%%%%%%%%%%%%%%%%%%%%%%%%%%%%%%%
\Opensolutionfile{solutions}[ex]
\section*{Exercises for \ref{sec:2nd order differential equations two}}

\begin{enumialphparenastyle}

Find the general solution to the differential equation using variation
of parameters.

%%%%%%%%%%
\begin{ex}
 $\ds y''+y=\tan x$
\begin{sol}
 $\ds A\sin(t)+B\cos(t)-\hfill\break\cos t\ln|\sec t+\tan t|$
\end{sol}
\end{ex}


%%%%%%%%%%
\begin{ex}
 $\ds y''+y=e^{2t}$
\begin{sol}
 $\ds A\sin(t)+B\cos(t)+{1\over5}e^{2t}$
\end{sol}
\end{ex}


%%%%%%%%%%
\begin{ex}
 $\ds y''+4y=\sec x$
\begin{sol}
 $\ds A\sin(2t)+B\cos(2t)+\cos t-\sin t\cos t\ln|\sec t+\tan t|$
\end{sol}
\end{ex}


%%%%%%%%%%
\begin{ex}
 $\ds y''+4y=\tan x$
\begin{sol}
 $\ds A\sin(2t)+B\cos(2t)+{1\over2}\sin(2t)\sin^2(t)+
{1\over2}\sin(2t)\ln|\cos t|-{t\over2}\cos(2t)+{1\over4}\sin(2t)\cos(2t)$
\end{sol}
\end{ex}


%%%%%%%%%%
\begin{ex}
 $\ds y''+y'-6y=t^2e^{2t}$
\begin{sol}
 $\ds Ae^{2t}+Be^{-3t}+{t^3\over15}e^{2t}-\left({t^2\over5}
-{2t\over25}+{2\over125}\right){e^{2t}\over5}$
\end{sol}
\end{ex}


%%%%%%%%%%
\begin{ex}
 $\ds y''-2y'+2y=e^{t}\tan(t)$
\begin{sol}
 $\ds Ae^{t}\sin t+Be^{t}\cos t-e^t\cos t\ln|\sec t+\tan t|$
\end{sol}
\end{ex}


%%%%%%%%%%
\begin{ex}
 $\ds y''-2y'+2y=\sin(t)\cos(t)$ (This is rather messy
when done by variation of parameters; compare to undetermined coefficients.)
\begin{sol}
 $\ds Ae^{t}\sin t+Be^{t}\cos t-
{1\over10}\cos t(\cos^3 t+3\sin^3 t-2\cos t-\sin t)+
{1\over10}\sin t(\sin^3 t-3\cos^3 t-2\sin t+\cos t)=
{1\over10}\cos(2t)-{1\over20}\sin(2t)$
\end{sol}
\end{ex}

\end{enumialphparenastyle}