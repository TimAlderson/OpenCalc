\section{Second Order Homogeneous Equations}{}{}\label{sec:second order homogeneous}
A second order differential equation is one containing the second
derivative $y''$. These are in general quite complicated, but one fairly
simple type is useful: The second order linear equation with constant
coefficients. 

\begin{example}{Second Order Homogeneous Equation}{Second Order Homogeneous Equation}\label{Second Order Homogeneous Equation}
Analyze the intial value problem $y''-y'-2y=0$,
$y(0)=5$, $y'(0)=0$. 
\end{example}

\begin{solution}
We make an inspired guess: might there be a
solution of the form $\ds e^{rt}$? This seems at least plausible,
since in this case $\ds y''$, $\ds y'$, and $y$ all
involve $\ds e^{rt}$. 

If such a function is a solution then
\begin{eqnarray*}
r^2 e^{rt}-r e^{rt}-2e^{rt}&=&0\cr
e^{rt}(r^2-r-2)&=&0\cr
(r^2-r-2)&=&0\cr
(r-2)(r+1)&=&0,
\end{eqnarray*}
so $r$ is $2$ or $-1$. Not only are $\ds f=e^{2t}$ and $\ds g=e^{-t}$
solutions, but notice that $\ds y=Af+Bg$ is also, for any constants $A$
and $B$:
\begin{eqnarray*}
(Af+Bg)''-(Af+Bg)'-2(Af+Bg)&=&Af''+Bg''-Af'-Bg'-2Af-2Bg\cr
&=&A(f''-f'-2f)+B(g''-g'-2g)\cr
&=&A(0)+B(0)=0.
\end{eqnarray*}
Can we find $A$ and $B$ so that this is a solution to the initial
value problem? Let's substitute:
$$
5=y(0)=Af(0)+Bg(0)=Ae^0+Be^0=A+B
$$
and 
$$0=y'(0)=Af'(0)+Bg'(0)=A2e^{0}+B(-1)e^0=2A-B.$$
So we need to solve this system of \underline{two} equations with \underline{two} unknowns:
\[
\left\{
\begin{array}{ll}
A+B	&	=5	\\
2A-B	&	=0
\end{array}
\right.
\]
Let $B=2A$, substitute into the first equation to get $5=A+2A=3A$. Then $A=5/3$ and $B=10/3$, and the
desired solution is $\ds (5/3)e^{2t}+(10/3)e^{-t}$. You now see why
the initial condition in this case included both $y(0)$ and $y'(0)$:
We needed two equations in the two unknowns $A$ and $B$
\end{solution}

You should of course wonder whether there might be other solutions, but as it turns out,
the answer is no. We will not prove this, but here is the theorem that
tells us what we need to know:

\begin{theorem}{Solutions to Second Order Homogeneous}{Solutions to Second Order Homogeneous}\label{Solutions to Second Order Homogeneous}
Given the differential equation $\ds ay''+by'+cy=0$, $a\not=0$,
consider the quadratic polynomial $ar^2+br+c=0$, called the
\dfont{characteristic polynomial}. Using the quadratic formula, this polynomial
always has one or two roots, call them $r_1$ and $r_2$.  The general
solution of the differential equation is:

\begin{enumerate}[(a)]
\item	$\ds y=Ae^{r_1t}+Be^{r_2t}$, if the roots $r_1$ and $r_2$ are real
  numbers, and $r_1\not=r_2$.
\item	$\ds y=Ae^{r_1t}+Bte^{r_2t}$, if $r_1=r_2$ is a real, repeated root.
\item	$\ds y=A\cos(\beta t)e^{\alpha t}+B\sin(\beta t)e^{\alpha t}$, 
if the roots are complex numbers, $r=\alpha\pm\beta i$.
\end{enumerate}
\end{theorem}

\begin{example}{}{}
	Solve the differential equation $y''+A^2y=0$.
\end{example}
\begin{solution}
	First we write the characteristic equation, $r^2+A^2=0$. Then we find the roots of the characteristic equation: 
	\[r^2=-A^2\implies r=\pm Ai\]
	These are imaginary roots, so the solution of the differential equation is in the form:
	\[y=c_1\cos(At)+c_2\sin(At)\]
\end{solution}

\begin{example}{}{}
	Solve the differential equation $y''-A^2y=0$.
\end{example}
\begin{solution}
	First we write the characteristic equation, $r^2-A^2=0$. Then we find the roots of the characteristic equation: 
	\[r^2=A^2\implies r=\pm A\]
	These are imaginary roots, so the solution of the differential equation is in the form:
	\[y=c_1e^{At}+c_2e^{-At}=c_1e^{At}+\frac{c_2}{e^{At}}\]
\end{solution}

\begin{example}{Damped Spring Oscillation}{Damped Spring Oscillation}\label{Damped Spring Oscillation}
Use a differential equation to describe the position of a mass hung on a spring.
\end{example}

\begin{solution}
 Suppose a mass $m$ is hung on a spring with spring
constant $k$. If the spring is compressed or stretched and then
released, the mass will oscillate up and down. Due to friction,
the oscillation will be damped: Eventually the motion will cease. The
damping will depend on the amount of friction; for example, if the
system is suspended in oil the motion will cease sooner than if the
system is in air. Using some simple physics, it is not hard to see
that the position of the mass is described by the differential
equation:
$\ds my''+by'+ky=0$. Using $m=1$, $b=4$, and $k=5$ we find the
motion of the mass. The characteristic polynomial is 
$r^2+4r+5=0$, with roots $r=(-4\pm\sqrt{16-20})/2=-2\pm i$. Thus the
general solution is
$\ds y=A\cos(t)e^{-2t}+B\sin(t)e^{-2t}$.
Suppose we know that $y(0)=1$ and $y'(0)=2$. Then as before we
form two simultaneous equations: From $y(0)=1$ we get
$1=A\cos(0)e^0+B\sin(0)e^0=A$. For the second we compute
$$y''=-2Ae^{-2t}\cos(t)+Ae^{-2t}(-\sin(t))-2Be^{-2t}\sin(t)+
Be^{-2t}\cos(t),$$
and then
$$2=-2Ae^0\cos(0)-Ae^0\sin(0)-2Be^0\sin(0)+Be^0\cos(0)
=-2A+B.$$
So we get $A=1$, $B=4$, and $\ds y=\cos(t)e^{-2t}+4\sin(t)e^{-2t}$.

Here is a useful trick that makes this easier to understand: We have
$\ds y=(\cos t+4\sin t)e^{-2t}$. The expression $\cos t+4 \sin t$ is a
bit reminiscent of the trigonometric formula
$\cos(\alpha-\beta)=\cos(\alpha)\cos(\beta)+\sin(\alpha)\sin(\beta)$
with $\alpha=t$.
Let's rewrite it a bit as
$$\sqrt{17}\left({1\over\sqrt{17}}\cos t + {4\over\sqrt{17}}\sin t\right).$$
Note that $\ds (1/\sqrt{17})^2+(4/\sqrt{17})^2=1$, 
which means that there is an angle
$\beta$ with $\ds \cos\beta=1/\sqrt{17}$ and 
$\ds \sin\beta=4/\sqrt{17}$ (of course, $\beta$ may not be a ``nice'' angle). Then
$$\cos t+4\sin t = \sqrt{17}\left(\cos t\cos \beta+\sin\beta\sin t\right)
=\sqrt{17}\cos(t-\beta).$$
Thus, the solution may also be written
$\ds y=\sqrt{17}e^{-2t}\cos(t-\beta)$.
This is a cosine curve  that has been shifted $\beta$ to the
right; the $\ds \sqrt{17}e^{-2t}$ has the effect of diminishing the
amplitude of the cosine as $t$ increases.
%; see figure~\ref{fig:damped oscillation}. The oscillation is damped very
%quickly, so in the first graph it is not clear that this is an
%oscillation. The second graph shows a restricted range for $t$.
\end{solution}

Other physical systems that oscillate can also be described by such
differential equations. Some electric circuits, for example, generate
oscillating current.

%\figure[!ht]
%%\texonly
%\hbox to \hsize{\hfill
%\def\yarrow{-- +(-1.5pt,-3pt) +(0pt,0pt) -- +(1.5pt,-3pt) +(0pt,0pt)}
%\def\xarrow{-- +(-3pt,-1.3pt) +(0pt,0pt) -- +(-3pt,1.5pt) +(0pt,0pt) }
%\tikzpicture[domain=0:3,x=1.2cm,y=3cm]
%\draw (0,0) -- (5.2,0) \xarrow node [right] {$x$};
%\draw (0,0) -- (0,1.3) \yarrow node [above] {$y$};
%\gpad
%\draw[color=black] plot[smooth,id=\the\gpnum,domain=0:5] function{sqrt(17)*exp(-2*x)*cos(x-asin(4/sqrt(17)))};
%\foreach \x in {1,2,3,4,5} \draw (\x,0) -- (\x,-2pt) node[anchor=north] {$\x$};
%\foreach \y in {0,1} \draw (0,\y) -- (-2pt,\y) node[anchor=east]
         %{$\y$};
%\endtikzpicture
%\hfill
%\tikzpicture[domain=2.5:5,x=1.6cm,y=120cm]
%\draw (2.5,0) -- (5.2,0) \xarrow node [right] {$x$};
%\draw (2.5,0) -- (2.5,0.03) \yarrow node [above] {$y$};
%\gpad
%\draw[color=black] plot[smooth,id=\the\gpnum,domain=2.5:5] function{sqrt(17)*exp(-2*x)*cos(x-asin(4/sqrt(17)))};
%\foreach \x in {3,4,5} \draw (\x,0) -- (\x,-2pt) node[anchor=north] {$\x$};
%\foreach \y in {0,0.01,0.02} \draw (2.5,\y) -- (2.4,\y) node[anchor=east] {$\y$};
%\endtikzpicture
%\hfill}%\endtexonly
%%\figrdef{fig:damped oscillation}
%%\htmlfigure{DE-damped_oscillation.html}
%\caption{\label{fig:damped oscillation}
%Graph of a damped oscillation.}
%%\endcaption
%\endfigure

\begin{example}{}{}\label{}
 Find the solution to the intial value problem
$\ds y''-4y'+4y=0$, $y(0)=-3$, $y'(0)=1$.
\end{example}

\begin{solution}
The characteristic polynomial is $r^2-4r+4=(r-2)^2$, so there is one root,
$r=2$, 
and the general solution is $\ds Ae^{2t}+Bte^{2t}$. Substituting
$t=0$ we get $-3=A+0=A$. The first derivative is
$\ds 2Ae^{2t}+2Bte^{2t}+Be^{2t}$; substituting $t=0$ gives
$1=2A+0+B=2A+B=2(-3)+B=-6+B$, so $B=7$. The solution is
$\ds -3e^{2t}+7te^{2t}$.
\end{solution}


%%%%%%%%%%%%%%%%%%%%%%%%%%%%%%%%%%%%%%%%%%%%
\Opensolutionfile{solutions}[ex]
\section*{Exercises for \ref{sec:second order homogeneous}}

\begin{enumialphparenastyle}

%%%%%%%%%%%
%\begin{ex}
 %Verify that the function in part (a) of
%theorem~\xrefn{thm:solns to second order homogeneous} is a solution to
%the differential equation $\ds ay''+by'+cy=0$.
%\end{ex}
%
%%%%%%%%%%%
%\begin{ex}
 %Verify that the function in part (b) of
%theorem~\xrefn{thm:solns to second order homogeneous} is a solution to
%the differential equation $\ds ay''+by'+cy=0$.
%\end{ex}
%
%%%%%%%%%%%
%\begin{ex}
 %Verify that the function in part (c) of
%theorem~\xrefn{thm:solns to second order homogeneous} is a solution to
%the differential equation $\ds ay''+by'+cy=0$.
%\end{ex}

%%%%%%%%%%
\begin{ex}
 Solve the initial value problem $\ds y''-\omega^2y=0$,
$y(0)=1$, $\ds y'(0)=1$, assuming $\omega\not=0$.
\begin{sol}
 $\ds {\omega+1\over2\omega}e^{\omega t}+
{\omega-1\over2\omega}e^{-\omega t}$
\end{sol}
\end{ex}


%%%%%%%%%%
\begin{ex}
 Solve the initial value problem $\ds2y''+18y=0$,
$y(0)=2$, $\ds y'(0)=15$.
\begin{sol}
 $\ds 2\cos(3t)+5\sin(3t)$
\end{sol}
\end{ex}


%%%%%%%%%%
\begin{ex}
 Solve the initial value problem 
$\ds y''+6y' +5y=0$,
$y(0)=1$, $\ds y'(0)=0$.
\begin{sol}
 $\ds -(1/4)e^{-5t}+(5/4)e^{-t}$
\end{sol}
\end{ex}


%%%%%%%%%%
\begin{ex}
 Solve the initial value problem 
$\ds y''-y'-12y=0$,
$y(0)=0$, $\ds y'(0)=14$.
\begin{sol}
 $\ds-2e^{-3t}+2e^{4t}$
\end{sol}
\end{ex}


%%%%%%%%%%
\begin{ex}
 Solve the initial value problem 
$\ds y''+12y'+36y=0$,
$y(0)=5$, $\ds y'(0)=-10$.
\begin{sol}
 $\ds 5e^{-6t}+20te^{-6t}$
\end{sol}
\end{ex}


%%%%%%%%%%
\begin{ex}
 Solve the initial value problem 
$\ds y''-8y'+16y=0$,
$y(0)=-3$, $\ds y'(0)=4$.
\begin{sol}
 $\ds (16t-3)e^{4t}$
\end{sol}
\end{ex}


%%%%%%%%%%
\begin{ex}
 Solve the initial value problem 
$\ds y''+5y=0$,
$y(0)=-2$, $\ds y'(0)=5$.
\begin{sol}
 $\ds -2\cos(\sqrt5t)+\sqrt{5}\sin(\sqrt{5}t)$
\end{sol}
\end{ex}


%%%%%%%%%%
\begin{ex}
 Solve the initial value problem 
$\ds y''+y=0$,
$y(\pi/4)=0$, $\ds y'(\pi/4)=2$.
\begin{sol}
 $\ds -\sqrt2\cos t+\sqrt2\sin t$
\end{sol}
\end{ex}


%%%%%%%%%%
\begin{ex}
 Solve the initial value problem 
$\ds y''+12y'+37y=0$,
$y(0)=4$, $\ds y'(0)=0$.
\begin{sol}
 $\ds e^{-6t}\left(4\cos t+24\sin t\right)$
\end{sol}
\end{ex}


%%%%%%%%%%
\begin{ex}
 Solve the initial value problem 
$\ds y''+6y'+18y=0$,
$y(0)=0$, $\ds y'(0)=6$.
\begin{sol}
 $\ds 2e^{-3t}\sin(3t)$
\end{sol}
\end{ex}


%%%%%%%%%%
\begin{ex}
 Solve the initial value problem 
$\ds y''+4y=0$,
$y(0)=\sqrt3$, $\ds y'(0)=2$. 
%Put your answer in the form developed
%at the end of exercise~\xrefn{example:damped spring oscillation}.
\begin{sol}
 $\ds 2\cos(2t-\pi/6)$
\end{sol}
\end{ex}


%%%%%%%%%%
\begin{ex}
 Solve the initial value problem 
$\ds y''+100y=0$,
$y(0)=5$, $\ds y'(0)=50$. 
%Put your answer in the form developed
%at the end of exercise~\xrefn{example:damped spring oscillation}.
\begin{sol}
 $\ds 5\sqrt2\cos(10t-\pi/4)$
\end{sol}
\end{ex}


%%%%%%%%%%
\begin{ex}
 Solve the initial value problem 
$\ds y''+4y'+13y=0$,
$y(0)=1$, $\ds y'(0)=1$. 
%Put your answer in the form developed
%at the end of exercise~\xrefn{example:damped spring oscillation}.
\begin{sol}
 $\ds \sqrt2 e^{-2t}\cos(3t-\pi/4)$
\end{sol}
\end{ex}


%%%%%%%%%%
\begin{ex}
 Solve the initial value problem 
$\ds y''-8y'+25y=0$,
$y(0)=3$, $\ds y'(0)=0$. 
%Put your answer in the form developed
%at the end of exercise~\xrefn{example:damped spring oscillation}.
\begin{sol}
 $\ds 5e^{4t}\cos(3t+\arcsin(4/5))$
\end{sol}
\end{ex}


%%%%%%%%%%
\begin{ex}
 A mass-spring system $\ds my''+by'+kx$ has
$k=29$, $b=4$, and $m=1$. At time $t=0$ the position is $y(0)=2$ and
the velocity is $y'(0)=1$. Find $y(t)$.
\begin{sol}
 $\ds (2\cos(5t)+\sin(5t))e^{-2t}$
\end{sol}
\end{ex}


%%%%%%%%%%
\begin{ex}
 A mass-spring system $\ds my''+by'+kx$ has
$k=24$, $b=12$, and $m=3$. At time $t=0$ the position is $y(0)=0$ and
the velocity is $y'(0)=-1$. Find $y(t)$.
\begin{sol}
 $\ds-(1/2)e^{-2t}\sin(2t)$
\end{sol}
\end{ex}


%%%%%%%%%%
\begin{ex}
 Consider 
%\exrdef{exer:second order really first order}
the differential equation $\ds ay'' + by'=0$,
with $a$ and $b$ both non-zero. Find the general solution by the
method of this section. Now let $\ds g=y'$; the equation may be
written as $\ds ag'+bg=0$, a first order linear homogeneous
equation. Solve this for $g$, then use the relationship $\ds g=y'$ to
find $y$.
\end{ex}


%%%%%%%%%%
\begin{ex}
 Suppose that $y(t)$ is a solution to $\ds ay''+by'+cy=0$, $y(t_0)=0$, $\ds y'(t_0)=0$. Show that $y(t)=0$.
\end{ex}

\end{enumialphparenastyle}