\section{Second Order Linear Equations - Method of Undetermined Coefficients}{}{}\label{sec:second order linear equations}
Now we consider second order equations of the form $\ds ay''+by'+cy=f(t)$,
with $a$, $b$, and $c$ constant. Of course, if $a=0$ this
is really a first order equation, so we assume $a\not=0$.
%Also, much
%as in exercise~\xrefn{exer:second order really first order} of
%section~\xrefn{sec:second order homogeneous}, 
Also, if $c=0$ we can solve
the related first order equation $\ds ah'+bh=f(t)$, and then solve
$\ds h=y'$ for $y$. So we will only examine examples in which
$c\not=0$.

Suppose that
$\ds y_1(t)$ and $\ds y_2(t)$ are solutions to $\ds ay''+by'+cy=f(t)$,
and consider the function $\ds h=y_1-y_2$. We substitute
this function into the left hand side of the differential equation and
simplify: 
$$
a(y_1-y_2)''+b(y_1-y_2)'+c(y_1-y_2)=ay_1''+by_1'+cy_1 -
(ay_2''+by_2'+cy_2)=f(t)-f(t)=0.
$$ 
So $h$ is a solution to the homogeneous equation $\ds ay''+by'+cy=0$.
Since we know how to find all such $h$, then with
just one particular solution $\ds y_2$ we can express all possible
solutions $\ds y_1$, namely, $\ds y_1=h+y_2$, where now $h$ is the
general solution to the homogeneous equation. Of course, this is
exactly how we approached the first order linear equation.

To make use of this observation we need a method to find a single
solution $y_2$. This turns out to be somewhat more difficult than the
first order case, but if $f(t)$ is of a certain simple form, we can
find a solution using the 
\dfont{method of undetermined coefficients}, sometimes 
more whimsically called the
\dfont{method of judicious guessing}.

\begin{example}{Second Order Linear Equation}{Second Order Linear Equation 1}\label{Second Order Linear Equation 1}
 Solve the differential equation 
$y''-y'-6y=18t^2+5.$
\end{example}

\begin{solution}
The general solution of the homogeneous equation is
$\ds Ae^{3t}+Be^{-2t}$. We guess that a solution to the
non-homogeneous equation might look like $f(t)$ itself, namely,
a quadratic $\ds y=at^2+bt+c$. Substituting this guess into the
differential equation we get
$$
y''-y'-6y = 2a-(2at+b)-6(at^2+bt+c) = -6at^2+(-2a-6b)t+(2a-b-6c).
$$
We want this to equal $18t^2+5$, so we need 
\begin{eqnarray*}
-6a&=&18\cr
-2a-6b&=&0\cr
2a-b-6c&=&5
\end{eqnarray*}
This is a system of three equations in three unknowns and is not hard
to solve: $a=-3$, $b=1$, $c=-2$. Thus the general solution to the
differential equation is $\ds Ae^{3t}+Be^{-2t}-3t^2+t-2$.
\end{solution}

So the ``judicious guess'' is a function with the same form as $f(t)$
but with undetermined (or better, yet to be determined)
coefficients. This works whenever $f(t)$ is a polynomial.

\begin{example}{Mass-Spring System with No Damping}{Mass-Spring System with No Damping}\label{Mass-Spring System with No Damping}
Analyze the initial value problem $\ds my'' +ky=-mg$,
$y(0)=2$, $\ds y'(0)=50$.
\end{example}

\begin{solution}
The left hand side represents a mass-spring
system with no damping, i.e., $b=0$. Unlike the homogeneous case, we
now consider the force due to gravity, $-mg$, assuming the spring is
vertical at the surface of the earth, so that $g=980$. To be specific,
let us take $m=1$ and $k=100$. The general solution to the homogeneous
equation is $\ds A\cos(10t)+B\sin(10t)$. For the solution to the 
non-homogeneous equation we guess simply a constant $y=a$, since $-mg=-980$
is a constant. Then $\ds y''+100y= 100a$ so $a=-980/100=-9.8$. The
desired general solution is then $\ds A\cos(10t)+B\sin(10t)-9.8$.
Substituting the initial conditions we get
\begin{eqnarray*}
2&=&A-9.8\cr
50&=&10B
\end{eqnarray*}
so $A=11.8$ and $B=5$ and the solution is $\ds 11.8\cos(10t)+5\sin(10t)-9.8$.
\end{solution}

More generally, this method can be used when a function similar to
$f(t)$ has derivatives that are also similar to $f(t)$; in the
examples so far, since $f(t)$ was a polynomial, so were its derivatives.
The method will work if $f(t)$ has the form $p(t)e^{\alpha t}\cos(\beta t)+
q(t)e^{\alpha t}\sin(\beta t)$, where $p(t)$ and $q(t)$ are
polynomials; when $\alpha=\beta=0$ this is simply $p(t)$, a
polynomial. In the most general form it is not simple to describe the
appropriate judicious guess; we content ourselves with some examples
to illustrate the process.

\begin{example}{Solving a Second Order Linear Equation}{Solving a Second Order Linear Equation 2}\label{Solving a Second Order Linear Equation 2}
 Find the general solution to $\ds y''+7y'+10y=e^{3t}.$
\end{example}

\begin{solution}
The characteristic equation is $r^2+7r+10=(r+5)(r+2)$,
so the solution to the homogeneous equation is
$Ae^{-5t}+Be^{-2t}$. For a particular solution to the inhomogeneous
equation we guess $Ce^{3t}$. Substituting we get
$$
9Ce^{3t}+21Ce^{3t}+10Ce^{3t}=e^{3t}40C.
$$
When $C=1/40$ this is equal to $f(t)=e^{3t}$, so the solution is
$Ae^{-5t}+Be^{-2t}+(1/40)e^{3t}$.
\end{solution}

\begin{example}{Solving a Second Order Linear Equation}{Solving a Second Order Linear Equation 3}\label{Solving a Second Order Linear Equation 3}
 Find the general solution to 
$\ds y''+7y'+10y=e^{-2t}.$
\end{example}

\begin{solution}
Following the last example we might guess
$Ce^{-2t}$, but since this is a solution to the homogeneous equation
it cannot work. Instead we guess $Cte^{-2t}$. Then
$$
(-2Ce^{-2t}-2Ce^{-2t}+4Cte^{-2t})+7(Ce^{-2t}-2Cte^{-2t})+10Cte^{-2t}
=e^{-2t}(-3C).
$$
Then $C=-1/3$ and the solution is $Ae^{-5t}+Be^{-2t}-(1/3)te^{-2t}$.
\end{solution}

In general, if $f(t)=e^{kt}$ and $k$ is one of the roots of the
characteristic equation, then we guess $Cte^{kt}$ instead of
$Ce^{kt}$. If $k$ is the only root of the characteristic equation,
then $Cte^{kt}$ will also not work, so we must guess $Ct^2e^{kt}$.

\begin{example}{Solving a Second Order Linear Equation}{Solving a Second Order Linear Equation 4}\label{Solving a Second Order Linear Equation 4}
 Find the general solution to 
$\ds y''-6y'+9y=e^{3t}.$
\end{example}

\begin{solution}
The characteristic equation is 
$\ds r^2-6r+9=(r-3)^2$, so the general solution to the homogeneous
equation is $Ae^{3t}+Bte^{3t}$. Guessing $Ct^2e^{3t}$ for the
particular solution, we get
$$
(9Ct^2e^{3t}+6Cte^{3t}+6Cte^{3t}+2Ce^{3t})-6(3Ct^2e^{3t}+2Cte^{3t})+9Ct^2e^{3t}
=e^{3t}2C.
$$
Thus, the solution is $\ds Ae^{3t}+Bte^{3t}+(1/2)t^2e^{3t}$.
\end{solution}

It is common in various physical systems to encounter an $f(t)$ of the
form $\ds a\cos(\omega t)+b\sin(\omega t)$.

\begin{example}{Solving a Second Order Linear Equation}{Solving a Second Order Linear Equation 5}\label{Solving a Second Order Linear Equation 5}
 Find the general solution to 
$\ds y''+6y'+25y=\cos(4t).$
\end{example}

\begin{solution}
The roots of the characteristic equation are
$-3\pm 4i$, so the solution to the homogeneous equation is
$\ds e^{-3t}(A\cos(4t)+B\sin(4t))$. For a particular solution, we
guess $C\cos(4t)+D\sin(4t)$. Substituting as usual:
$$(-16C\cos(4t)+-16D\sin(4t))+6(-4C\sin(4t)+4D\cos(4t))+25(C\cos(4t)+D\sin(4t))$$
$$=(24D+9C)\cos(4t)+(-24C+9D)\sin(4t).$$
To make this equal to $\cos(4t)$ we need
\begin{eqnarray*}
24D+9C&=&1\cr
9D-24C&=&0
\end{eqnarray*}
which gives $C=1/73$ and $D=8/219$. The full solution is then
$\ds e^{-3t}(A\cos(4t)+B\sin(4t))+(1/73)\cos(4t)+(8/219)\sin(4t)$.

The function $\ds e^{-3t}(A\cos(4t)+B\sin(4t))$ is a damped
oscillation as in example~\ref{Mass-Spring System with No Damping},
while $\ds(1/73)\cos(4t)+(8/219)\sin(4t)$ is a simple undamped
oscillation. As $t$ increases, the sum $\ds
e^{-3t}(A\cos(4t)+B\sin(4t))$ approaches zero, so the solution
$$e^{-3t}(A\cos(4t)+B\sin(4t))+(1/73)\cos(4t)+(8/219)\sin(4t)$$
becomes more and more like the simple oscillation
$\ds(1/73)\cos(4t)+(8/219)\sin(4t)$---notice that the initial
conditions don't matter to this long term behavior. The damped portion
is called the 
\dfont{transient} [part of the]
\dfont{solution}, and the simple oscillation is called the 
\dfont{steady state} [part of the] \dfont{solution}. 
A physical example is a mass-spring system. If the only force on the
mass is due to the spring, then the behavior of the system is a damped
oscillation. If in addition an external force is applied to the mass,
and if the force varies according to a function of the form
 $\ds a\cos(\omega t)+b\sin(\omega t)$, then the long term behavior
will be a simple oscillation determined by the steady state portion of the
general solution; the initial position of the mass will not matter.
\end{solution}

As with the exponential form, such a simple guess may not work.

\begin{example}{Solving a Second Order Linear Equation}{Solving a Second Order Linear Equation 6}\label{Solving a Second Order Linear Equation 6}
 Find the general solution to $\ds y''+16y=-\sin(4t).$
\end{example}

\begin{solution}
The roots of the characteristic equation are $\pm4i$, so the
solution to the homogeneous equation is $A\cos(4t)+B\sin(4t)$. Since
both $\cos(4t)$ and $\sin(4t)$ are solutions to the homogeneous
equation,  $C\cos(4t)+D\sin(4t)$ is also, so it cannot be a solution
to the non-homogeneous equation. Instead, we guess
$Ct\cos(4t)+Dt\sin(4t)$. Then substituting:
$$(-16Ct\cos(4t)-16D\sin(4t)+8D\cos(4t)-8C\sin(4t)))+16(Ct\cos(4t)+Dt\sin(4t))$$
$$=8D\cos(4t)-8C\sin(4t).$$
Thus $C=1/8$, $D=0$, and the solution is
$\ds C\cos(4t)+D\sin(4t)+(1/8)t\cos(4t)$.
\end{solution}

In general, if $f(t)=a\cos(\omega t)+b\sin(\omega t)$, and $\pm \omega
i$ are the roots of the characteristic equation, then instead of 
$C\cos(\omega t)+D\sin(\omega t)$ we guess $Ct\cos(\omega t)+Dt\sin(\omega t)$.


%%%%%%%%%%%%%%%%%%%%%%%%%%%%%%%%%%%%%%%%%%%%
\Opensolutionfile{solutions}[ex]
\section*{Exercises for \ref{sec:second order linear equations}}

\begin{enumialphparenastyle}

Find the general solution to the differential equation.

%%%%%%%%%%
\begin{ex}
 $\ds y'' -10y'+25y=\cos t$
\begin{sol}
 $Ae^{5t}+Bte^{5t}+(6/169)\cos t-(5/338)\sin t$
\end{sol}
\end{ex}


%%%%%%%%%%
\begin{ex}
 $\ds y''+2\sqrt2y'+2y=10$
\begin{sol}
 $\ds Ae^{-\sqrt2t}+Bte^{-\sqrt2t}+5$
\end{sol}
\end{ex}


%%%%%%%%%%
\begin{ex}
 $\ds y''+16y=8t^2+3t-4$
\begin{sol}
 $\ds A\cos(4t)+B\sin(4t)+ (1/2)t^2+(3/16)t-5/16$
\end{sol}
\end{ex}


%%%%%%%%%%
\begin{ex}
 $\ds y''+2y=\cos(5t)+\sin(5t)$
\begin{sol}
 $\ds A\cos(\sqrt2t)+B\sin(\sqrt2t)-(\cos(5t)+\sin(5t))/23$
\end{sol}
\end{ex}


%%%%%%%%%%
\begin{ex}
 $\ds y''-2y'+2y=e^{2t}$
\begin{sol}
 $\ds e^{t}(A\cos t+B\sin t)+e^{2t}/2$
\end{sol}
\end{ex}


%%%%%%%%%%
\begin{ex}
 $\ds y''-6y+13=1+2t+e^{-t}$
\begin{sol}
 $\ds Ae^{\sqrt6t}+Be^{-\sqrt6t}+2-t/3-e^{-t}/5$
\end{sol}
\end{ex}


%%%%%%%%%%
\begin{ex}
 $\ds y''+y'-6y=e^{-3t}$
\begin{sol}
 $\ds Ae^{-3t}+Be^{2t}-(1/5)te^{-3t}$
\end{sol}
\end{ex}


%%%%%%%%%%
\begin{ex}
 $\ds y''-4y'+3y=e^{3t}$
\begin{sol}
 $\ds Ae^t+Be^{3t}+(1/2)te^{3t}$
\end{sol}
\end{ex}


%%%%%%%%%%
\begin{ex}
 $\ds y''+16y=\cos(4t)$
\begin{sol}
 $\ds A\cos(4t)+B\sin(4t)+(1/8)t\sin(4t)$
\end{sol}
\end{ex}


%%%%%%%%%%
\begin{ex}
 $\ds y'' +9y=3\sin(3t)$
\begin{sol}
 $\ds A\cos(3t)+B\sin(3t)-(1/2)t\cos(3t)$
\end{sol}
\end{ex}


%%%%%%%%%%
\begin{ex}
 $\ds y''+12y'+36y=6e^{-6t}$
\begin{sol}
 $\ds Ae^{-6t}+Bte^{-6t}+3t^2e^{-6t}$
\end{sol}
\end{ex}


%%%%%%%%%%
\begin{ex}
 $\ds y''-8y'+16y=-2e^{4t}$
\begin{sol}
 $\ds Ae^{4t}+Bte^{4t}-t^2e^{4t}$
\end{sol}
\end{ex}


%%%%%%%%%%
\begin{ex}
 $\ds y''+6y'+5y=4$
\begin{sol}
 $\ds Ae^{-t}+Be^{-5t}+(4/5)$
\end{sol}
\end{ex}


%%%%%%%%%%
\begin{ex}
 $\ds y''-y'-12y=t$
\begin{sol}
 $\ds Ae^{4t}+Be^{-3t}+(1/144)-(t/12)$
\end{sol}
\end{ex}


%%%%%%%%%%
\begin{ex}
 $\ds y''+5y=8\sin(2t)$
\begin{sol}
 $\ds A\cos(\sqrt5t)+B\sin(\sqrt5t)+8\sin(2t)$
\end{sol}
\end{ex}


%%%%%%%%%%
\begin{ex}
 $\ds y''-4y=4e^{2t}$
\begin{sol}
 $\ds Ae^{2t}+Be^{-2t}+te^{2t}$
\end{sol}
\end{ex}


\noindent Solve the initial value problem.

%%%%%%%%%%
\begin{ex}
 $\ds y''-y=3t+5$, $y(0)=0$, $\ds y'(0)=0$
\begin{sol}
 $\ds 4e^{t}+e^{-t}-3t-5$
\end{sol}
\end{ex}


%%%%%%%%%%
\begin{ex}
 $\ds y''+9y=4t$, $y(0)=0$, $\ds y'(0)=0$
\begin{sol}
 $\ds -(4/27)\sin(3t)+(4/9)t$
\end{sol}
\end{ex}


%%%%%%%%%%
\begin{ex}
 $\ds y'' +12y' +37y=10e^{-4t}$, $y(0)=4$, $\ds y'(0)=0$
\begin{sol}
 $\ds e^{-6t}(2\cos t+20\sin t)+2e^{-4t}$
\end{sol}
\end{ex}


%%%%%%%%%%
\begin{ex}
 $\ds y''+6y'+18y=\cos t-\sin t$, $y(0)=0$, $\ds y'(0)=2$ 
\begin{sol}
 $\ds
\left(-{23\over 325}\cos(3t)+{592\over 975}\sin(3t)\right)+
{23\over325}\cos t-{11\over325}\sin t$
\end{sol}
\end{ex}


%%%%%%%%%%
\begin{ex}
 Find the solution for the mass-spring equation
$\ds y''+4y'+29y=689\cos(2t)$.
\begin{sol}
 $\ds e^{-2t}(A\sin(5t)+B\cos(5t))+8\sin(2t)+25\cos(2t)$
\end{sol}
\end{ex}


%%%%%%%%%%
\begin{ex}
 Find the solution for the mass-spring equation
$\ds3y''+12y'+24y=2\sin t$.
\begin{sol}
 $\ds e^{-2t}(A\sin(2t)+B\cos(2t))+(14/195)\sin t-(8/195)\cos t$
\end{sol}
\end{ex}


%%%%%%%%%%
\begin{ex}
 Consider the differential 
equation $\ds my''+by'+ky=\cos(\omega t)$,
with $m$, $b$, and $k$ all positive and $\ds b^2<2mk$; this equation
is a model for a damped mass-spring system with external 
driving force $\cos(\omega t)$.
Show that the steady state part of the solution has amplitude
$${1\over \sqrt{(k-m\omega^2)^2+\omega^2b^2}}.$$
Show that this amplitude is largest when 
$\ds \omega={\sqrt{4mk-2b^2}\over 2m}$. This is the 
\dfont{resonant frequency} of the system.
\end{ex}

\end{enumialphparenastyle}