\section{First Order Linear Equations}{}{}\label{sec:first order linear}
As you might guess, a first order linear differential equation has the form 
$\ds y' + p(t)y = f(t)$. Not only is this closely related in form
to the first order homogeneous linear equation, we can use what we
know about solving homogeneous equations to solve the general linear
equation. 

Suppose that $y_1(t)$ and $y_2(t)$ are solutions to 
$\ds y' + p(t)y = f(t)$. Let $\ds g(t)=y_1-y_2$. Then
\begin{eqnarray*}
 g'(t)+p(t)g(t)&=&y_1'-y_2'+p(t)(y_1-y_2)\cr
&=&(y_1'+p(t)y_1)-(y_2'+p(t)y_2)\cr
&=&f(t)-f(t)=0.
\end{eqnarray*}
In other words, $\ds g(t)=y_1-y_2$ is a solution to the homogeneous
equation $\ds y' + p(t)y = 0$. Turning this around, any solution
to the linear equation $\ds y' + p(t)y = f(t)$, call it $y_1$, can
be written as $y_2+g(t)$, for some particular $y_2$ and some solution
$g(t)$ of the homogeneous equation $\ds y' + p(t)y = 0$. Since we
already know how to find all solutions of the homogeneous equation,
finding just one solution to the equation $\ds y' + p(t)y = f(t)$
will give us all of them.

How might we find that one particular solution to $\ds y' + p(t)y
= f(t)$? Again, it turns out that what we already know helps. We know
that the general solution to the homogeneous equation
$\ds y' + p(t)y = 0$ looks like $\ds Ae^{P(t)}$. We now make an
inspired guess: Consider the function $\ds v(t)e^{P(t)}$, in which we
have replaced the constant parameter $A$ with the function
$v(t)$. This technique is called 
\dfont{variation of parameters}.
For
convenience write this as $s(t)=v(t)h(t)$, where $\ds h(t)=e^{P(t)}$ 
is a solution to the
homogeneous equation. Now let's compute a bit with $s(t)$:
\begin{eqnarray*}
s'(t)+p(t)s(t)&=&v(t)h'(t)+v'(t)h(t)+p(t)v(t)h(t)\cr
&=&v(t)(h'(t)+p(t)h(t)) + v'(t)h(t)\cr
&=&v'(t)h(t).\end{eqnarray*}
The last equality is true because $\ds h'(t)+p(t)h(t)=0$. Since $h(t)$
is a solution to the homogeneous equation. We are hoping to find a
function $s(t)$ so that $\ds s'(t)+p(t)s(t)=f(t)$; we will have such a
function if we can arrange to have $\ds v'(t)h(t)=f(t)$, that is,
$\ds v'(t)=f(t)/h(t)$. But this is as easy (or hard) as finding an
anti-derivative of $\ds f(t)/h(t)$. Putting this all together, the
general solution to $\ds y' + p(t)y = f(t)$ is
$$v(t)h(t)+Ae^{P(t)} = v(t)e^{P(t)}+Ae^{P(t)}.$$
\begin{example}{Solving an IVP}{Solving an IVP}\label{Solving an IVP}
 Find the solution of the initial value problem
$\ds y'+3y/t=t^2$, $y(1)=1/2$. 
\end{example}

\begin{solution}
First we find the general solution;
since we are interested in a solution with a given condition at $t=1$,
we may assume $t>0$.
We start by solving the homogeneous equation as usual; call the
solution $g$:
$$g=Ae^{-\int (3/t)\,dt}=Ae^{-3\ln t}=At^{-3}.$$
Then as in the discussion, $\ds h(t)=t^{-3}$ and
$\ds v'(t)=t^2/t^{-3}=t^5$, so $\ds v(t)=t^6/6$. We know that
every solution to the equation looks like
$$v(t)t^{-3}+At^{-3}={t^6\over6}t^{-3}+At^{-3}={t^3\over6}+At^{-3}.$$
Finally we substitute to find $A$:
\begin{eqnarray*}
{1\over 2}&=&{(1)^3\over6}+A(1)^{-3}={1\over6}+A\cr
A&=&{1\over 2}-{1\over6}={1\over3}.
\end{eqnarray*}
The solution is then
$$y={t^3\over6}+{1\over3}t^{-3}.$$
\end{solution}

Another common method for solving such a differential equation is by means of an
\dfont{integrating factor}. 

\subsection*{Using an Integrating Factor}
\label{sec:integrating-factor}

Linear equations of the form $ y'+p(t)y=q(t) $ can always be solved by multiplying both sides of the equation
with a specially chosen function called the \emph{integrating factor,} $ I(t)$.  It is
defined by
\begin{equation}
I (t) = e^{\int p(t)\; dt}.
  \label{eq:integrating-factor-defined}
\end{equation}
 It looks like we just pulled this definition
of $I(t)$ out of a hat.  

Multiply the equation by the integrating factor
$I(t)$ to get
\[
I(t)\frac{d y}{d t}+p(t)I(t)y = I(t)q(t).
\]
By the chain rule the integrating factor satisfies
\[
\frac{ d }{d t}I(t) = \frac{d} {d t} e^{\int p(t)\; dt}
= \underbrace{\frac{d} {d t} \left( \int p(t)\; dt\right)}_{=p(t)} ~\underbrace{e^{\int p(t)\; dt}_{}}_{=I(t)}
= p(t)I(t).
\]
Therefore one has
\begin{align*}
  \frac{d }{d t}I(t)y
  &= I(t)\frac{\; d }{\; d t}y +p(t)I(t)y \\
  &= I(t)\Bigl\{\frac{d}{d t} y+p(t)y \Bigr\}\\
  &= I(t)q(t).
\end{align*}
Integrating and then dividing by the integrating factor gives the solution
\[
y=\frac1{I(t)}\left(\int I(t)q(t)\,d t+C\right).
\]
In this derivation we have to divide by $I(t)$, but since $I(t)=e^{\int p(t)\; dt}$ and since
exponentials never vanish we know that $I (t)\neq0$, so we can always divide by $I
(t)$.


% In the differential equation
%$\ds y'+p(t)y=f(t)$, we note that if we multiply through by a function
%$I(t)$ to get $\ds I(t)y'+I(t)p(t)y=I(t)f(t)$, the left hand side
%looks like it could be a derivative computed by the product rule:
%$${d\over dt}(I(t)y)=I(t)y'+I'(t)y.$$
%Now if we could choose $I(t)$ so that $I'(t)=I(t)p(t)$, this would be
%exactly the left hand side of the differential equation. But this is
%just a first order homogeneous linear equation, and we know a solution
%is $\ds I(t)=e^{Q(t)}$, where $\ds Q(t)=\int p\,dt$; note that 
%$Q(t)=-P(t)$, where $P(t)$ appears in the variation of parameters
%method and $P'(t)=-p$. Now the modified differential equation is 
%\begin{eqnarray*}
%e^{-P(t)}y'+e^{-P(t)}p(t)y&=&e^{-P(t)}f(t)\cr
%{d\over dt}(e^{-P(t)}y)&=&e^{-P(t)}f(t).\cr
%\end{eqnarray*}
%Integrating both sides gives
%\begin{eqnarray*}
%e^{-P(t)}y&=&\int e^{-P(t)}f(t)\,dt\cr
%y&=&e^{P(t)}\int e^{-P(t)}f(t)\,dt.\cr
%\end{eqnarray*}
If you look carefully, you will see that this is exactly the same
solution we found by variation of parameters, because
$\ds e^{-P(t)}q(t)=q(t)/h(t)$.

Some people find it easier to remember how to use the integrating
factor method, rather than variation of parameters. Since ultimately they
require the same calculation, you should use whichever of the two methods
appeals to you more strongly. 





\begin{example}{}{sec:an-example}
Find the general solution to the differential equation
\[
\frac{\; d y} {\; d x} = y + x.
\]
Then find the solution that satisfies \upshape
\begin{equation}
  y(2)=0.
  \label{eq:example-linear-initial-condition}
\end{equation}
\end{example}


\begin{solution}
We first write the equation in the standard linear form
\begin{equation}
  \label{eq:diffeq-linear-example}
  \frac{\; d y} {\; d x} - y = x,
\end{equation}
and then multiply by the integrating factor $I(x)$.  We could of course memorize
the formula 
\[
I(x) = e^{\int p(x)\;dx}
\]
but the following procedure will always give us the integrating factor.

Assuming that $I(x)$ is as yet unknown we multiply the differential
equation~\eqref{eq:diffeq-linear-example} by $I$,
\begin{equation}
  I(x)\frac{\; d y} {\; d x} - I(x) y = I(x)x.
  \label{eq:example-multiplied-with-m}
\end{equation}
If $I(x)$ is such that
\begin{equation}
  -I(x) = \frac{\; d I(x)} {\; d x},
  \label{eq:integrating-factor-condition}
\end{equation}
then equation~\eqref{eq:example-multiplied-with-m} implies
\[
  I(x)\frac{\; d y} {\; d x} + \frac{\; d I(x)} {\; d x} y = I(x)x.
\]
The expression on the left is exactly what comes out of the product rule -- this is
the point of multiplying with $I(x)$ and then insisting
on~\eqref{eq:integrating-factor-condition}.  So, if $I(x)$
satisfies~\eqref{eq:integrating-factor-condition}, then the differential equation for
$y$ is equivalent with
\[
\frac{\; d I(x) y} {\; d x} = I(x) x.
\]
We can integrate this equation,
\[
I(x) y = \int I(x) x \;\; d x,
\]
and thus find the solution
\begin{equation}
  y(x) =  \frac{1} {I(x)}  \int I(x) x \;\; d x.
  \label{eq:example-linear-almost-solved}
\end{equation}
All we have to do is find the integrating factor $I$.  This factor can be any
function that satisfies~\eqref{eq:integrating-factor-condition}.
Equation~\eqref{eq:integrating-factor-condition} is a differential equation for $I$,
but it is separable, and we can easily solve it:
\[
\frac{\; d I} {\; d x} = -I \iff
\frac{1} {I} \; d I = -\; d x \iff
\ln|I| = -x +C.
\]
Since \textit{we only need one integrating factor} $I$ we are not interested in
finding all solutions of~\eqref{eq:integrating-factor-condition}, and therefore we
can choose the constant $C$.  The simplest choice is $C=0$, which leads to
\[
\ln |I| = -x \iff |I| = e^{-x} \iff I = \pm e^{-x}. 
\]
Again, we only need one integrating factor, so we may choose the $\pm$~sign: the simplest
choice for $I$ here is
\[
m(x) = e^{-x}.
\]
With this choice of integrating factor we can now complete the calculation that led
to~\eqref{eq:example-linear-almost-solved}.  The solution to the differential
equation is
\begin{align*}
  y(x)
  &= \frac{1} {I(x)}  \int I(x) x \;\; d x \\
  &= \frac{1} {e^{-x}} \int e^{-x}x \;\; d x
  &\text{\color{red}\sffamily\footnotesize%
    (integrate by parts)}\\
  &= e^{x} \Bigl\{-e^{-x}x - e^{-x} + C\Bigr\} \\
  &= -x-1+Ce^x.
\end{align*}
This is the general solution.

To find the solution that satisfies not just the differential equation, but also the
``initial condition''~\eqref{eq:example-linear-initial-condition}, i.e.~$y(2)=0$, we
compute $y(2)$ for the general solution,
\[
y(2) = -2-1+Ce^2 = -3 + Ce^2.
\]
The requirement $y(2) = 0$ then tells us that $C=3e^{-2}$.  The solution of the
differential equation that satisfies the prescribed initial condition is therefore
\[
y(x) = -x-1+3e^{x-2}.
\]
\end{solution}



\begin{example}
Use the Integrating Factor method to solve the IVP in Example \ref{exa:Solving an IVP}
\end{example}


\begin{solution}
Given $\ds y'+3y/t=t^2$, we have $ p(t) = \frac{3}{t} $, $ q(t) =  t^2 $, the integrating factor is $ I(t) = e^{\int(p(t)\; dt} $, and the solution to the differential equation is
\[
y=\frac{1}{I(t)}\int I(t)q(t)\; dt
\] 
where 
\[
I(t) = \ds e^{\int 3/t}=e^{3\ln |t|}=|t^3|=\pm t^3
\]

Again, we only need one integrating factor, so we may choose $ I(t)=t^3 $. So the general solution is

\[
y=\frac{1}{t^3}\int t^3\cdot t^2\; dt = \frac{1}{t^3}\int t^5\; dt = \frac{1}{t^3}\left(\frac{t^6}{6}+C\right) =\frac{t^3}{6}+\frac{C}{t^3} 
\] 

The initial value $ y(1)=\frac12 $ then gives $ C=\frac13 $, giving the same answer as before.
 
\end{solution}

%Using this method to solve the  the solution of the previous
%example would look just a bit different: Starting with
%$\ds y'+3y/t=t^2$, we recall that the integrating factor is
%$I(t) = \ds e^{\int 3/t}=e^{3\ln t}=t^3$. Then we multiply through by the
%integrating factor and solve:
%\begin{eqnarray*}
%t^3y'+t^3 3y/t&=&t^3t^2\cr
%t^3y'+t^2 3y&=&t^5\cr
%{d\over dt}(t^3 y)&=&t^5\cr
%t^3 y&=&t^6/6\cr
%y&=&t^3/6.
%\end{eqnarray*}
%This is the same answer, of course, and the problem is then finished
%just as before.




%%%%%%%%%%%%%%%%%%%%%%%%%%%%%%%%%%%%%%%%%%%%
\Opensolutionfile{solutions}[ex]
\section*{Exercises for \ref{sec:first order linear}}

\begin{enumialphparenastyle}

In the following exercises, find the general solution of the equation.

%%%%%%%%%%
\begin{ex}
 $\ds y' +4y=8$
\begin{sol}
 $\ds y=Ae^{-4t}+2$
\end{sol}
\end{ex}


%%%%%%%%%%
\begin{ex}
 $\ds y'-2y=6$
\begin{sol}
 $\ds y=Ae^{2t}-3$
\end{sol}
\end{ex}


%%%%%%%%%%
\begin{ex}
 $\ds y' +ty=5t$
\begin{sol}
 $\ds y=Ae^{-(1/2)t^2}+5$
\end{sol}
\end{ex}


%%%%%%%%%%
\begin{ex}
 $\ds y'+e^ty=-2e^t$
\begin{sol}
 $\ds y=Ae^{-e^t}-2$
\end{sol}
\end{ex}


%%%%%%%%%%
\begin{ex}
 $\ds y'-y=t^2$
\begin{sol}
 $\ds y=Ae^{t}-t^2-2t-2$
\end{sol}
\end{ex}


%%%%%%%%%%
\begin{ex}
 $\ds 2y' +y=t$
\begin{sol}
 $\ds y=Ae^{-t/2}+t-2$
\end{sol}
\end{ex}


%%%%%%%%%%
\begin{ex}
 $\ds ty' -2y=1/t$, $t>0$
\begin{sol}
 $\ds y=At^2-{1\over3t}$
\end{sol}
\end{ex}


%%%%%%%%%%
\begin{ex}
 $\ds ty'+y=\sqrt{t}$, $t>0$
\begin{sol}
 $\ds y={c\over t}+{2\over3}\sqrt t$
\end{sol}
\end{ex}


%%%%%%%%%%
\begin{ex}
 $\ds y'\cos t+y\sin t=1$, $-\pi/2<t<\pi/2$
\begin{sol}
 $\ds y= A\cos t+\sin t$
\end{sol}
\end{ex}


%%%%%%%%%%
\begin{ex}
 $\ds y' + y\sec t=\tan t$, $-\pi/2<t<\pi/2$
\begin{sol}
 $\ds y= {A\over\sec t+\tan t}+1-{t\over\sec t+\tan t}$
\end{sol}
\end{ex}

\end{enumialphparenastyle}