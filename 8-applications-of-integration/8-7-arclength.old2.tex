\section{Arc Length}{}{}\label{sec:Arc Length}
Here is another geometric application of the integral: Find the length
of a portion of a curve. As usual, we need to think about how we might
approximate the length, and turn the approximation into an integral.

We already know how to compute one simple arc length, that of a line
segment. If the endpoints are $\ds P_0(x_0,y_0)$ and $\ds P_1(x_1,y_1)$
then the length of the segment is the distance between the points,
$\ds \sqrt{(x_1-x_0)^2+(y_1-y_0)^2}$, from the Pythagorean theorem, as
illustrated in Figure~\ref{fig:length of a line segment}.

\figure[H]
%\texonly
\centerline{\vbox{\beginpicture
\normalgraphs
%\sevenpoint
\setcoordinatesystem units <1.5truecm,1.5truecm>
\setplotarea x from 0 to 5, y from 0 to 3
\axis bottom /
\axis left /
\putrule from 2 1 to 4.5 1
\putrule from 4.5 1 to 4.5 2.5
\plot 2 1 4.5 2.5 /
\put {$(x_1,y_1)$} [bl] <3pt,3pt> at 4.5 2.5
\put {$(x_0,y_0)$} [tr] <-3pt,-3pt> at 2 1
\put {$x_1-x_0$} [t] <0pt,-3pt> at 3.25 1
\put {$y_1-y_0$} [l] <3pt,0pt> at 4.5 1.75
\put {$\sqrt{(x_1-x_0)^2+(y_1-y_0)^2}$} [br] <-3pt,3pt> at 3.25 1.75
\endpicture}}
%\endtexonly
%\figrdef{fig:length of a line segment}
%\htmlfigure{Integration_applications-arc_length.html}
\caption{\label{fig:length of a line segment}
The length of a line segment.}
%\endcaption
\endfigure

Now if the graph of $f$ is ``nice'' (say, differentiable) it appears
that we can approximate the length of a portion of the curve with line
segments, and that as the number of segments increases, and their
lengths decrease, the sum of the lengths of the line segments will
approach the true arc length; see 
Figure~\ref{fig:approximating arc length}.

\figure[H]
%\texonly
\centerline{\vbox{\beginpicture
\normalgraphs
%\sevenpoint
\setcoordinatesystem units <1.5truecm,0.8truecm>
\setplotarea x from 0 to 8, y from 0 to 5
\axis bottom /
\axis left /
\setquadratic\plot 
1.000 1.000 1.150 1.953 1.300 2.694 1.450 3.246 1.600 3.636 
1.750 3.884 1.900 4.012 2.050 4.041 2.200 3.990 2.350 3.874 
2.500 3.711 2.650 3.515 2.800 3.299 2.950 3.075 3.100 2.853 
3.250 2.644 3.400 2.454 3.550 2.290 3.700 2.157 3.850 2.060 
4.000 2.000 4.150 1.979 4.300 1.996 4.450 2.050 4.600 2.138 
4.750 2.255 4.900 2.396 5.050 2.554 5.200 2.721 5.350 2.886 
5.500 3.039 5.650 3.168 5.800 3.258 5.950 3.294 6.100 3.261 
6.250 3.140 6.400 2.912 6.550 2.556 6.700 2.051 6.850 1.374 
7.000 0.500 /
\multiput {$\bullet$} at 1 1 2.050 4.041 3.250 2.644
4.300 1.996 5.500 3.039 6.250 3.140 7 0.5 /
\setlinear\plot
1 1 2.050 4.041 3.250 2.644
4.300 1.996 5.500 3.039 6.250 3.140 7 0.5 /
\endpicture}}
%\endtexonly
%\figrdef{fig:approximating arc length}
%\htmlfigure{Integration_applications-arc_length_line_segments.html}
\caption{\label{fig:approximating arc length}
Approximating arc length with line segments.}
%\endcaption
\endfigure

Now we need to write a formula for the sum of the lengths of the line
segments, in a form that we know becomes an integral in the limit.  So
we suppose we have divided the interval $[a,b]$ into $n$ subintervals
as usual, each with length $\Delta x =(b-a)/n$, and endpoints $\ds
a=x_0$, $\ds x_1$, $\ds x_2$, \dots, $\ds x_n=b$.  The length of a
typical line segment, joining $\ds (x_i,f(x_i))$ to $\ds
(x_{i+1},f(x_{i+1}))$, is $\ds\sqrt{(\Delta x )^2
  +(f(x_{i+1})-f(x_i))^2}$.  By the Mean Value Theorem, %(\xrefn{thm:mvt}), 
there is a number $\ds t_i$ in $\ds (x_i,x_{i+1})$
such that $\ds f'(t_i)\Delta x=f(x_{i+1})-f(x_i)$, so the length of
the line segment can be written as
$$
  \sqrt{(\Delta x)^2 + (f'(t_i))^2\Delta x^2}=
  \sqrt{1+(f'(t_i))^2}\,\Delta x.
$$
Then arc length is:
$$
  \lim_{n\to\infty}\sum_{i=0}^{n-1} \sqrt{1+(f'(t_i))^2}\,\Delta x=
  \int_a^b \sqrt{1+(f'(x))^2}\,dx.
$$
Note that the sum looks a bit different than others we have
encountered, because the approximation contains a $\ds t_i$ instead of an
$\ds x_i$. In the past we have always used left endpoints (namely, $\ds x_i$)
to get a representative value of $f$ on $\ds [x_i,x_{i+1}]$; now we are
using a different point, but the principle is the same.

To summarize, to compute the length of a curve on the interval
$[a,b]$, we compute the integral
$$\int_a^b \sqrt{1+(f'(x))^2}\,dx.$$ 
Unfortunately, integrals of this form are typically difficult or
impossible to compute exactly, because usually none of our methods for
finding antiderivatives will work. In practice this means that the
integral will usually have to be approximated.

\begin{example}{Circumference of a Circle}{Circumference of a Circle}\label{Circumference of a Circle} 
Let $\ds f(x) = \sqrt{r^2-x^2}$, the upper half circle of radius
$r$. The length of this curve is half the circumference, namely $\pi
r$. Compute this with the arc length formula.
\end{example}

\begin{solution}
The derivative $f'$ is $\ds \ds -x/\sqrt{r^2-x^2}$ so the integral is
$$
  \int_{-r}^r \sqrt{1+{x^2\over r^2-x^2}}\,dx
  =\int_{-r}^r \sqrt{r^2\over r^2-x^2}\,dx
  =r\int_{-r}^r \sqrt{1\over r^2-x^2}\,dx.
$$
Using a trigonometric substitution, we find the antiderivative, namely
$\ds \arcsin(x/r)$. Notice that the integral is improper at both
endpoints, as the function $\ds \sqrt{1/(r^2-x^2)}$ is undefined when
$x=\pm r$. So we need to compute
$$
  \lim_{D\to-r^+}\int_D^0  \sqrt{1\over r^2-x^2}\,dx +
  \lim_{D\to r^-}\int_0^D  \sqrt{1\over r^2-x^2}\,dx.
$$
This is not difficult, and has value $\pi$, so the original integral,
with the extra $r$ in front, has value $\pi r$ as expected.
\end{solution}


%%%%%%%%%%%%%%%%%%%%%%%%%%%%%%%%%%%%%%%%%%%%
\Opensolutionfile{solutions}[ex]
\section*{Exercises for \ref{sec:Arc Length}}

\begin{enumialphparenastyle}

%%%%%%%%%%
\begin{ex}
 Find the arc length of $\ds f(x)=x^{3/2}$ on $[0,2]$.
\begin{sol}
 $\ds (22\sqrt{22}-8)/27$
\end{sol}
\end{ex}

%%%%%%%%%%
\begin{ex}
 Find the arc length of $\ds f(x) = x^2/8-\ln x$
on $[1,2]$.
\begin{sol}
 $\ln(2)+3/8$
\end{sol}
\end{ex}

%%%%%%%%%%
\begin{ex}

Find the arc length of $\ds f(x) = (1/3)(x^2 +2)^{3/2}$
on the interval $[0,a]$.
\begin{sol}
 $\ds a+a^3/3$
\end{sol}
\end{ex}

%%%%%%%%%%
\begin{ex}
 Find the arc length of $f(x)=\ln(\sin x)$ on the
interval $[\pi/4,\pi/3]$.
\begin{sol}
 $\ds \ln((\sqrt2+1)/\sqrt3)$
\end{sol}
\end{ex}

%%%%%%%%%%
\begin{ex}
 Let $a>0$. Show that the length of $y=\cosh x$ on
$[0,a]$ is equal to $\ds \int _0 ^a \cosh x\,dx$.
\end{ex}

%%%%%%%%%%
\begin{ex}
 Find the arc length of $f(x)=\cosh x$ on $[0, \ln 2]$.
\begin{sol}
 $3/4$
\end{sol}
\end{ex}

%%%%%%%%%%
\begin{ex}
 Set up the integral to find the arc length of $\sin x$ 
on the interval $[0,\pi]$; do not evaluate the integral. If you have
access to appropriate software, approximate the value of the integral.
\begin{sol}
 $\approx 3.82$
\end{sol}
\end{ex}

%%%%%%%%%%
\begin{ex}
 Set up the integral to find the arc length of $\ds y=xe^{-x}$
on the interval $[2,3]$; do not evaluate the integral. If you have
access to appropriate software, approximate the value of the integral.
\begin{sol}
 $\approx 1.01$
\end{sol}
\end{ex}

%%%%%%%%%%
\begin{ex}
 Find the arc length of $\ds y=e^x$ on the interval $[0,1]$.
(This can be done exactly; it is a bit tricky and a bit long.)
\begin{sol}
 $\ds \sqrt{1+e^2}-\sqrt2+
{1\over2}\ln\left({\sqrt{1+e^2}-1\over\sqrt{1+e^2}+1}\right)+
{1\over2}\ln(3+2\sqrt2)$
\end{sol}
\end{ex}

\end{enumialphparenastyle}