\section{Volume}{}{}\label{sec:volume}
Now that we have seen how to compute certain areas by using integration; we will now look into how some
volumes may also be computed by evaluating an integral. Generally, the
volumes that we can compute this way have cross-sections that are easy
to describe.














%\begin{figure}
%\centering
%\includegraphics[width=0.2\textwidth]{/figures/figcross1_3D}
%\caption{The volume of a general right cylinder}
%\label{fig:cross1}
%\end{figure}

			

\figure[H]
%\texonly
\vbox{\beginpicture
\normalgraphs
%\ninepoint
\setcoordinatesystem units <0.3truecm,0.3truecm>
\setplotarea x from -10 to 10, y from -3 to 20
\axis bottom shiftedto y=0 ticks withvalues {$x_i$} / at 7 / /
\put {\hbox{\epsfxsize9truecm\epsfbox{images/pyramid_steps.eps}}} at 30 10
\put {$y_i\rightarrow$} [l] <-5pt,-1truept> at -10 6
\plot -10 0 0 20 10 0 /
\putrule from -10 0 to -10 1
\putrule from -10 1 to 10 1
\putrule from 10 1 to 10 0
\putrule from -9.5 1 to -9.5 2
\putrule from -9.5 2 to 9.5 2
\putrule from 9.5 2 to 9.5 1
\putrule from -9 2 to -9 3
\putrule from -9 3 to 9 3
\putrule from 9 3 to 9 2
\putrule from -8.5 3 to -8.5 4
\putrule from -8.5 4 to 8.5 4
\putrule from 8.5 4 to 8.5 3
\putrule from -8 4 to -8 5
\putrule from -8 5 to 8 5
\putrule from 8 5 to 8 4
\putrule from -7.5 5 to -7.5 6
\putrule from -7.5 6 to 7.5 6
\putrule from 7.5 6 to 7.5 5
\putrule from -7 6 to -7 7
\putrule from -7 7 to 7 7
\putrule from 7 7 to 7 6
\putrule from -6.5 7 to -6.5 8
\putrule from -6.5 8 to 6.5 8
\putrule from 6.5 8 to 6.5 7
\putrule from -6 8 to -6 9
\putrule from -6 9 to 6 9
\putrule from 6 9 to 6 8
\put {$\vdots$} at 0 13
\endpicture}
%\caption{
%Volume of a pyramid approximated by rectangular prisms.
%(\expandafter\url\expandafter{\liveurl pyramid.html}%
%AP\endurl)
%\endcaption
%\endtexonly
%\figrdef{fig:pyramid}
%\htmlfigure{Integration_applications-volume_pyramid.html}
%\htmlonly
\caption{\label{fig:pyramid}
Volume of a pyramid approximated by rectangular prisms.}
%\endhtmlonly
\endfigure

\begin{example}{Volume of a Pyramid}{Volume of a Pyramid}\label{Volume of a Pyramid}
Find the volume of a pyramid with a square base that is 20 meters tall
and 20 meters on a side at the base. 
\end{example}

\begin{solution}
As with most of our applications
of integration, we begin by asking how we might approximate the
volume. Since we can easily compute the volume of a rectangular prism
(that is, a ``box''), we will use some boxes to approximate the volume of
the pyramid, as shown in figure~\ref{fig:pyramid}: On the left is a
cross-sectional view, on the right is a 3D view of part of the pyramid
with some of the boxes used to approximate the volume.

Each box has volume of the form $\ds (2x_i)(2x_i)\Delta y$. Unfortunately,
there are two variables here; fortunately, we can write $x$ in terms
of $y$: From the cross-sectional view we see that a height of 20 is achieved at the midpoint of the base. We will also position the cross-sectional view symmetrically about the $y$-axis. Thus at $x=0$, $y=20$, and we have a slope of $m=-2$. So
\begin{align*}
y&=-2x+b	\\
20&=-2(0)+b	\\
20&=b.
\end{align*}

Therefore, $y=20-2x$, and in the terms of $x$: $x=10-y/2$ or $\ds x_i=10-y_i/2$. Then the total volume is
approximately
$$\sum_{i=0}^{n-1} 4(10-y_i/2)^2\Delta y$$
and in the limit we get the volume as the value of an integral:
$$
  \int_0^{20} 4(10-y/2)^2\,dy=\int_0^{20} (20-y)^2\,dy=
  \left.-{(20-y)^3\over3}\right|_0^{20}=
  -{0^3\over3}--{20^3\over3}={8000\over3}.
$$
As you may know, the volume of a pyramid is 
$(1/3)(\hbox{height})(\hbox{area of base})=(1/3)(20)(400)$, which
agrees with our answer.
\end{solution}

\begin{example}{Volume of an Object}{Volume of an Object}\label{Volume of an Object}
The base of a solid is the region between $\ds f(x)=x^2-1$ and
$\ds g(x)=-x^2+1$, and its cross-sections perpendicular to the $x$-axis 
are equilateral triangles, as indicated in
Figure~\xrefn{fig:triangular cross-sections}.
%\texonly
The solid has been truncated to show a triangular
cross-section above $x=1/2$.
%\endtexonly
Find the volume of the solid.
\end{example}

\figure[H]
%\texonly
\centerline{\vbox{\beginpicture
\normalgraphs
%\ninepoint
\setcoordinatesystem units <3truecm,3truecm>
\setplotarea x from -1.1 to 1.1, y from -1.1 to 1.1
\put {\hbox{\epsfxsize6truecm\epsfbox{images/triangular_solid.eps}}} at 3 0
\axis bottom shiftedto y=0 ticks withvalues {$-1\quad$} {$1$} / at -1 1 / /
\axis left shiftedto x=0 ticks withvalues {$-1\quad$} {$1\quad$} / at -1 1 / /
\plot
-1.000 0.000 -0.900 -0.190 -0.800 -0.360 -0.700 -0.510 -0.600 -0.640 
-0.500 -0.750 -0.400 -0.840 -0.300 -0.910 -0.200 -0.960 -0.100 -0.990 
0.000 -1.000 0.100 -0.990 0.200 -0.960 0.300 -0.910 0.400 -0.840 
0.500 -0.750 0.600 -0.640 0.700 -0.510 0.800 -0.360 0.900 -0.190 
1.000 0.000 /
\plot
-1.000 0.000 -0.900 0.190 -0.800 0.360 -0.700 0.510 -0.600 0.640 
-0.500 0.750 -0.400 0.840 -0.300 0.910 -0.200 0.960 -0.100 0.990 
0.000 1.000 0.100 0.990 0.200 0.960 0.300 0.910 0.400 0.840 
0.500 0.750 0.600 0.640 0.700 0.510 0.800 0.360 0.900 0.190 
1.000 0.000 /
\endpicture}}
%\begincaption
%{Solid with equilateral triangles as cross-sections.
%(\expandafter\url\expandafter{\liveurl jmol_triangular_x_sections}%
%AP\endurl)}
%%(\expandafter\url\expandafter{\sageurl solid_with_triangular_x-sections}%
%%AP\endurl)}
%\endcaption
%\endtexonly
%\figrdef{fig:triangular cross-sections}
%\htmlfigure{Integration_applications-volume_equilateral_x_sections.html}
%\htmlonly
\caption
{Solid with equilateral triangles as cross-sections.\label{fig:triangular cross-sections}}
%You can download the <a href="http://www.whitman.edu/mathematics/calculus/live/sage/solid_with_triangular_x-sections/solid_with_triangular_x-sections.sws">Sage worksheet</a>
%for this plot and upload it to your own sage account.
%\endcaption
%\endhtmlonly
\endfigure

\begin{solution}
A cross-section at a value $\ds x_i$ on the $x$-axis is a triangle with
base $\ds 2(1-x_i^2)$ and height $\ds \sqrt3(1-x_i^2)$, so the area of the
cross-section is 
$$
  {1\over2}(\hbox{base})(\hbox{height})=
  (1-x_i^2)\sqrt3(1-x_i^2),
$$
and the volume of a thin ``slab'' is then
$$(1-x_i^2)\sqrt3(1-x_i^2)\Delta x.$$
Thus the total volume is 
$$\int_{-1}^1 \sqrt3(1-x^2)^2\,dx={16\over15}\sqrt3.$$
\vskip-10pt
\end{solution}

One easy way to get ``nice'' cross-sections is by rotating a plane
figure around a line. For example, in Figure~\ref{fig:solid of rotation} 
we see a plane region under a curve and between two
vertical lines; then the result of rotating this around the $\ds x$-axis, and
a typical circular cross-section is a circle.
 
\figure[H]
%\texonly
\centerline{
\vbox{\hbox{\hfill\raise53pt\vbox{
\beginpicture
\normalgraphs
%\ninepoint
\setcoordinatesystem units <5.5truemm,5.5truemm>
\setplotarea x from 0 to 8, y from -4 to 4
\axis left /
\axis bottom shiftedto y=0 / 
\putrule from 1 0 to 1 4 
\putrule from 6 0 to 6 3
\plot 1 4 1.150 3.546 
1.312 3.125 1.475 2.772 1.638 2.484 1.800 2.255 1.962 2.082 
2.125 1.961 2.288 1.886 2.450 1.854 2.612 1.859 2.775 1.898 
2.938 1.967 3.100 2.060 3.262 2.174 3.425 2.303 3.588 2.444 
3.750 2.593 3.912 2.744 4.075 2.894 4.238 3.037 4.400 3.170 
4.562 3.289 4.725 3.388 4.888 3.464 5.050 3.511 5.212 3.527 
5.375 3.505 5.538 3.443 5.700 3.334 5.862 3.176 6.025 2.964 /
\endpicture}
\quad\epsfxsize3.8cm\epsfbox{images/rotated_surface.eps}
\quad\epsfxsize3.8cm\epsfbox{images/one_disk.eps}
\hfill}\vglue-0pt}}
\caption{\label{fig:solid of rotation} A solid of rotation.}
\endfigure

Of course a real ``slice'' of this figure will not have straight
sides, but we can approximate the volume of the slice by a cylinder or
disk with circular top and bottom and straight sides; the volume of
this disk will have the form $\ds \pi r^2\Delta x$. As long as we can
write $r$ in terms of $x$ we can compute the volume by an integral.

\begin{example}{Volume of a Right Circular Cone}{Volume of a Right Circular Cone}\label{Volume of a Right Circular Cone}
Find the volume of a right circular cone with base radius 10 and
height 20. (A right circular cone is one with a circular base and with
the tip of the cone directly over the center of the base.)
\end{example}

\begin{solution}
We can view this cone as produced by the rotation of the line
$y=x/2$ rotated about the $x$-axis, as indicated in
figure~\ref{fig:line to cone}.

\figure[H]
%\texonly
\centerline{\vbox{\beginpicture
\normalgraphs
%\ninepoint
\setcoordinatesystem units <0.25truecm,0.25truecm>
\setplotarea x from 0 to 20, y from 0 to 10
\axis bottom shiftedto y=0 ticks withvalues {$0$} {$20$} / at 0 20 / /
\put {\hbox{\epsfxsize6cm\epsfbox{images/cone.eps}}} at 40 5
\plot 0 0 20 10 /
\putrule from 20 0 to 20 10
\endpicture}}
\caption{\label{fig:line to cone}
A region that generates a cone; approximating the volume
by circular disks.}
%(\expandafter\url\expandafter{\liveurl cone.html}%
%AP\endurl)
%\endcaption
%\endtexonly
%\figrdef{fig:line to cone}
%\htmlfigure{Integration_applications-volume_of_cone.html}
%\begincaption
%Approximating the volume of a cone
%by circular disks.
%\endcaption
\endfigure

At a particular point on the $x$-axis, say $\ds x_i$, the radius of the
resulting cone is the $y$-coordinate of the corresponding point on the
line, namely $\ds y_i=x_i/2$. Thus the total volume is approximately
$$\sum_{i=0}^{n-1} \pi (x_i/2)^2\,dx$$
and the exact volume is
$$
  \int_0^{20} \pi
  {x^2\over4}\,dx={\pi\over4}{20^3\over3}={2000\pi\over3}.
$$ 
Note that we can instead do the calculation with a generic height and
radius: 
$$
  \int_0^{h} \pi{r^2\over h^2}x^2\,dx
  ={\pi r^2\over h^2}{h^3\over3}={\pi r^2h\over3},
$$ 
giving us the usual formula for the volume of a cone.
\end{solution}

\begin{example}{Volume of an Object with a Hole}{Volume of an Object with a Hole}\label{Volume of an Object with a Hole}
Find the volume of the object generated when the area between
$\ds y=x^2$ and $y=x$ is rotated around the $x$-axis. 
\end{example}

\begin{solution}
This solid has a
``hole'' in the middle; we can compute the volume by subtracting the
volume of the hole from the volume enclosed by the outer surface of
the solid. In figure~\ref{fig:solid with hole} we show the region
that is rotated, the resulting solid with the front half cut away,
the cone that forms the outer surface, the
horn-shaped hole, and a cross-section perpendicular to the $x$-axis.

\figure[H]
%\texonly
\centerline{\vbox{\beginpicture
\normalgraphs
%\ninepoint
\setcoordinatesystem units <3truecm,3truecm>
\setplotarea x from 0 to 1.1, y from 0 to 1.1
\axis bottom shiftedto y=0 ticks withvalues {$0$} {$1$} / at 0 1 / /
\axis left shiftedto x=0 ticks withvalues {$0$} {$1$} / at 0 1 / /
\put {\hbox{\epsfxsize3cm\epsfbox{images/cutaway_horn.eps}}} at 3 0
\put {\hbox{\epsfxsize3cm\epsfbox{images/outer_cone.eps}}} at 0 -2
\put {\hbox{\epsfxsize3cm\epsfbox{images/horn.eps}}} at 2 -2
\put {\hbox{\epsfxsize3cm\epsfbox{images/washer_section.eps}}} at 4 -2
\plot 0 0 1 1 /
\setquadratic
\plot
0.000 0.000 0.100 0.010 0.200 0.040 0.300 0.090 0.400 0.160 
0.500 0.250 0.600 0.360 0.700 0.490 0.800 0.640 0.900 0.810 
1.000 1.000 /
\endpicture}}
\caption{\label{fig:solid with hole}
Solid with a hole, showing the outer cone and the shape to
be removed to form the hole.}
%(\expandafter\url\expandafter{\liveurl solid_with_hole.html}%
%AP\endurl)
%\endcaption
%\endtexonly
%\figrdef{fig:solid with hole}
%\htmlfigure{Integration_applications-volume_with_hole.html}
%\htmlonly
%\begincaption
%Solid with a hole. You can download the <a href="http://www.whitman.edu/mathematics/calculus/live/jmol_solid_of_rotation_with_hole/solid_of_rotation_with_hole.sws">Sage
%worksheet</a>
%for this plot and upload it to your own sage account.
%\endcaption
%\endhtmlonly
\endfigure

We have already computed the volume of a cone; in this case it is
$\pi/3$. At a particular value of $x$, say $\ds x_i$, the cross-section of
the horn is a circle with radius $\ds x_i^2$, so the volume of the horn is
$$\int_0^1 \pi(x^2)^2\,dx=\int_0^1 \pi x^4\,dx=\pi{1\over 5},$$
so the desired volume is $\pi/3-\pi/5=2\pi/15$.

As with the area between curves, there is an alternate approach that
computes the desired volume ``all at once'' by approximating the
volume of the actual solid. We can approximate the volume of a slice
of the solid with a washer-shaped volume, as indicated in
Figure~\ref{fig:solid with hole}.

The volume of such a washer is the area of the face times the
thickness. The thickness, as usual, is $\Delta x$, while the area of
the face is the area of the outer circle minus the area of the inner
circle, say $\ds \pi R^2-\pi r^2$, or $\pi(\text{TOP})^2-\pi(\text{BOTTOM})^2$. In the present example, at a particular $\ds x_i$,
the radius $R$ (The ``TOP'' function) is $\ds x_i$ and $r$ (The ``BOTTOM'' function) $\ds x_i^2$. Hence, the whole volume is
$$
  \int_0^1 \pi\left(\text{TOP}^2-\text{BOTTOM}^2\right)\,dx=
  \int_0^1 \pi x^2-\pi x^4\,dx=
  \left.\pi\left({x^3\over3}-{x^5\over5}\right)\right|_0^1=
  \pi\left({1\over3}-{1\over5}\right)={2\pi\over15}.
$$
Of course, what we have done here is exactly the same calculation as
before, except we have in effect recomputed the volume of the outer cone.
\end{solution}

Suppose the region between $f(x)=x+1$ and $\ds g(x)=(x-1)^2$ is rotated around
the $y$-axis; see Figure~\ref{fig:shell method}. It is possible, but
inconvenient, to compute the  volume of the resulting solid by the
method we have used so far. The problem is that there are two
``kinds'' of typical rectangles: Those that go from the line to the
parabola and those that touch the parabola on both ends. To compute
the volume using this approach, we need to break the problem into two
parts and compute two integrals:
$$
  \pi\int_0^1 (1+\sqrt{y})^2-(1-\sqrt{y})^2\,dy+
  \pi\int_1^4  (1+\sqrt{y})^2-(y-1)^2\,dy={8\over3}\pi + {65\over6}\pi
  ={27\over2}\pi.
$$
If instead we consider a typical vertical rectangle, {but still rotate
around the $y$-axis,} we get a thin ``shell'' instead of a thin
``washer''. Note that ``washers'' are related to the area of a circle, $\pi r^2$, whereas ``shells'' are related to the surface area of a cylinder, $2\pi rh$. If we add up the volume of such thin shells we will get an
approximation to the true volume. What is the volume of such a shell?
Consider the shell at $\ds x_i$.
Imagine that we cut the shell vertically in one place and ``unroll''
it into a thin, flat sheet, namely the surface of a cylinder. This sheet will be almost a rectangular
prism that is $\Delta x$ thick, $\ds f(x_i)-g(x_i)$ (TOP$-$BOTTOM) tall, and $\ds 2\pi x_i$
wide. The volume will then be approximately the volume of a rectangular
prism with these dimensions: $\ds 2\pi x_i(f(x_i)-g(x_i))\Delta x$. If we
add these up and take the limit as usual, we get the integral
$$
  \int_0^3 2\pi x(f(x)-g(x))\,dx=
  \int_0^3 2\pi x\left(\text{TOP}-\text{BOTTOM}\right)\,dx=
  \int_0^3 2\pi x(x+1-(x-1)^2)\,dx={27\over2}\pi.
$$
Not only does this accomplish the task with only one integral, the
integral is somewhat easier than those in the previous
calculation. Things are not always so neat, but it is often the case
that one of the two methods will be simpler than the other, so it is
worth considering both before starting to do calculations.

\figure[H]
%\texonly
\centerline{\vbox{\beginpicture
\normalgraphs
%\ninepoint
\setcoordinatesystem units <1truecm,1truecm>
\setplotarea x from 0 to 3.1, y from 0 to 4.1
\axis bottom shiftedto y=0 ticks numbered from 0 to 3 by 1 /
\axis left shiftedto x=0 ticks numbered from 0 to 4 by 1 /
\putrule from 1 2 to 2.4142 2
\putrule from 1 1.8 to 2.4142 1.8
\putrule from 1 1.8 to 1 2
\putrule from 2.4142 1.8 to 2.4142 2
\putrule from 0.25 .5625 to 1.75 .5625
\putrule from 0.25 .3625 to 1.75 .3625
\putrule from 0.25 .3625 to 0.25 .5625
\putrule from 1.75 .3625 to 1.75 .5625
\plot 0 1 3 4 /
\setquadratic
\plot
0.000 1.000 0.150 0.722 0.300 0.490 0.450 0.302 0.600 0.160 
0.750 0.062 0.900 0.010 1.050 0.002 1.200 0.040 1.350 0.122 
1.500 0.250 1.650 0.422 1.800 0.640 1.950 0.902 2.100 1.210 
2.250 1.562 2.400 1.960 2.550 2.402 2.700 2.890 2.850 3.422 
3.000 4.000 /
\setcoordinatesystem units <1truecm,1truecm> point at -5 0
\setplotarea x from 0 to 3.1, y from 0 to 4.1
\axis bottom shiftedto y=0 ticks numbered from 0 to 3 by 1 /
\axis left shiftedto x=0 ticks numbered from 0 to 4 by 1 /
\put {\hbox{\epsfxsize4cm\epsfbox{images/shell.eps}}} at 7 2
\putrule from 1.5 0.25 to 1.5 2.5
\putrule from 1.7 0.25 to 1.7 2.5
\putrule from 1.5 0.25 to 1.7 0.25
\putrule from 1.5 2.5 to 1.7 2.5
\setlinear
\plot 0 1 3 4 /
\setquadratic
\plot
0.000 1.000 0.150 0.722 0.300 0.490 0.450 0.302 0.600 0.160 
0.750 0.062 0.900 0.010 1.050 0.002 1.200 0.040 1.350 0.122 
1.500 0.250 1.650 0.422 1.800 0.640 1.950 0.902 2.100 1.210 
2.250 1.562 2.400 1.960 2.550 2.402 2.700 2.890 2.850 3.422 
3.000 4.000 /
\endpicture}}
\caption{\label{fig:shell method}
Computing volumes with ``shells''.}
%(\url{http://www.whitman.edu/mathematics/calculus/live/shell.html}%
%AP\endurl)}
%(\expandafter\url\expandafter{\liveurl shell.html}%
%AP\endurl)
%\endcaption
%\endtexonly
%\figrdef{fig:shell method}
%\htmlfigure{Integration_applications-volume_shell_method.html}
%\htmlonly
%\begincaption
%Computing volumes with ``shells''.
%\endcaption
%\endhtmlonly
\endfigure

\begin{example}{}{}\label{}
Suppose the area under $\ds y=-x^2+1$ between $x=0$ and $x=1$ is
rotated around the $x$-axis. Find the volume by both methods.
\end{example}

\begin{solution}
Using the disk method we obtain:
$$\ds \int_0^1 \pi(1-x^2)^2\,dx={8\over15}\pi.$$
Using the shell method we obtain:
$$\ds \int_0^1 2\pi y \sqrt{1-y}\,dy={8\over15}\pi.$$
\end{solution}


%%%%%%%%%%%%%%%%%%%%%%%%%%%%%%%%%%%%%%%%%%%%
\Opensolutionfile{solutions}[ex]
\section*{Exercises for \ref{sec:volume}}

\begin{enumialphparenastyle}

%%%%%%%%%%
\begin{ex}
Verify that $\ds\pi\int_0^1 (1+\sqrt{y})^2-(1-\sqrt{y})^2\,dy+
\pi\int_1^4  (1+\sqrt{y})^2-(y-1)^2={8\over3}\pi + {65\over6}\pi
={27\over2}\pi$.
\end{ex}

%%%%%%%%%%
\begin{ex}
 Verify that $\ds\int_0^3 2\pi x(x+1-(x-1)^2)\,dx={27\over2}\pi$.
\end{ex}

%%%%%%%%%%
\begin{ex}
 Verify that $\ds \int_0^1 \pi(1-x^2)^2\,dx={8\over15}\pi$.
\end{ex}

%%%%%%%%%%
\begin{ex}
 Verify that $\ds \int_0^1 2\pi y \sqrt{1-y}\,dy={8\over15}\pi$.
\end{ex}

%%%%%%%%%%
\begin{ex}
Use integration to find the volume of the solid obtained by revolving 
the region bounded by $x+y=2$ and the $x$ and $y$ axes around the
$x$-axis. 
\begin{sol}
 $8\pi/3$
\end{sol}
\end{ex}

%%%%%%%%%%
\begin{ex}
Find the volume of the solid obtained by revolving 
the region bounded by $\ds y=x-x^2$
and the $x$-axis around the
$x$-axis. 
\begin{sol}
 $\pi/30$
\end{sol}
\end{ex}

%%%%%%%%%%
\begin{ex}
Find the volume of the solid obtained by revolving 
the region bounded by $\ds y=\sqrt{\sin x}$ between $x=0$ and
$x=\pi/2$, the $y$-axis, and the line
$y=1$ around the
$x$-axis. 
\begin{sol}
 $\pi(\pi/2-1)$
\end{sol}
\end{ex}

%%%%%%%%%%
\begin{ex}
Let $S$ be the region of the $xy$-plane bounded above by the curve
$\ds x^3y=64$, below by the line $y=1$, on the left by  the line $x=2$, and
on the right by the line $x=4$.  Find
the volume of the solid obtained by rotating $S$ around:
\begin{multicols}{2}
\begin{enumerate}
	\item	the $x$-axis;
	\item	the line $y=1$;
	\item	the $y$-axis; and
	\item	the line $x=2$.
\end{enumerate}
\end{multicols}
\begin{sol}
\begin{multicols}{2}
\begin{enumerate}
	\item	$114\pi/5$
	\item	$74\pi/5$
	\item	$20\pi$
	\item	$4\pi$
\end{enumerate}
\end{multicols}
\end{sol}
\end{ex}

%%%%%%%%%%
\begin{ex}
 The equation $\ds x^2/9+y^2/4=1$ describes an ellipse.  Find the
volume of the solid obtained by rotating the ellipse around the
$x$-axis and also around the $y$-axis. These solids are
called \dfont{ellipsoids}; one is vaguely rugby-ball shaped, one is
sort of flying-saucer shaped, or perhaps squished-beach-ball-shaped.
\begin{sol}
 $16\pi$, $24\pi$
\end{sol}
\end{ex}


\figure[H]
%\texonly
\centerline{\vbox{\beginpicture
\normalgraphs
%\ninepoint
\setcoordinatesystem units <3truecm,3truecm>
\setplotarea x from 0 to 1.1, y from 0 to 0.5
\put {\hbox{\epsfxsize3cm\epsfbox{images/rugby.eps}}} at 0 0
\put {\hbox{\epsfxsize3cm\epsfbox{images/ufo.eps}}} at 2 0
\endpicture}}
\caption{Ellipsoids.\label{fig:ellipsoids}}
%(\url{http://www.whitman.edu/mathematics/calculus/live/ellipsoid.html}%
%AP\endurl)}
%(\expandafter\url\expandafter{\liveurl ellipsoid.html}%
%AP\endurl)}
%\endcaption
%\endtexonly
%\figrdef{fig:ellipsoids}
%\htmlfigure{Integration_applications-ellipsoids.html}
%\htmlonly
%\begincaption
%Ellipsoids.
%\endcaption
%\endhtmlonly
\endfigure


%%%%%%%%%%
\begin{ex}
 Use integration to compute the volume of a sphere of radius
$r$. You should of course get the well-known formula $\ds 4\pi r^3/3$.
\end{ex}

%%%%%%%%%%
\begin{ex}
A hemispheric bowl of radius $r$ contains water to a depth $h$.  Find
the volume of water in the bowl.
\begin{sol}
 $\ds \pi h^2(3r-h)/3$
\end{sol}
\end{ex}

%%%%%%%%%%
\begin{ex}
 The base of a tetrahedron (a triangular pyramid) of height $h$
is an equilateral triangle of side $s$.  Its cross-sections
perpendicular to an altitude are equilateral triangles.  Express its
volume $V$ as an integral, and find a formula for $V$ in terms of $h$
and $s$. Verify that your answer is $(1/3)(\hbox{area of base})(\hbox{height})$. 
%% fixme: include picture? see exercise_9.3.13.mw
\end{ex}

%%%%%%%%%%
\begin{ex}
The base of a solid is the region between $f(x)=\cos x$ and
$g(x)=-\cos x$, $-\pi/2\le x\le\pi/2$,
and its cross-sections perpendicular to the $x$-axis 
are squares.
Find the volume of the solid.
\begin{sol}
 $2\pi$
\end{sol}
\end{ex}

\end{enumialphparenastyle}