\section{Series}\label{sec:series}

While much more can be said about sequences, we now turn to our
principal interest, series. Recall that a series, roughly speaking, is
the sum of a sequence: If $\ds\{a_n\}_{n=0}^\infty$ is a sequence then the
associated series is
\[\sum_{i=0}^\infty a_n=a_0+a_1+a_2+\cdots\]
Associated with a series is a second sequence, called the {\dfont sequence of
  partial sums} $\ds\{s_n\}_{n=0}^\infty$:
\[s_n=\sum_{i=0}^n a_i.\]
So
\[s_0=a_0,\quad s_1=a_0+a_1,\quad s_2=a_0+a_1+a_2,\quad \ldots\]
A series converges\index{series!convergent}\index{convergent series} 
if the sequence of partial sums converges, and otherwise the series 
diverges\index{series!divergent}\index{divergent series}.

If $\{kx^n\}_{n=0}^{\infty}$ is a geometric sequence, then the associated series $\sum_{i=0}^{\infty}kx^i$ is called a geometric series.

\begin{theorem}{Geometric Series Convergence}{GeoSeriesConvergence}
If $|x|<1$, the geometric series $\ds\sum_i kx^i$ converges to $\ds\frac{k}{1-x}$, otherwise the series diverges (unless $k=0$).
\end{theorem}
\begin{proof}
If $\ds a_n=kx^n$, $\ds\sum_{n=0}^\infty a_n$ is called a 
\dfont{geometric series}. A typical partial sum is
\[s_n=k+kx+kx^2+kx^3+\cdots+kx^n=k(1+x+x^2+x^3+\cdots+x^n).\]
We note that
\begin{align*}
  s_n(1-x)&=k(1+x+x^2+x^3+\cdots+x^n)(1-x) \\
  &=k(1+x+x^2+x^3+\cdots+x^n)1-k(1+x+x^2+x^3+\cdots+x^{n-1}+x^n)x \\
  &=k(1+x+x^2+x^3+\cdots+x^n-x-x^2-x^3-\cdots-x^n-x^{n+1}) \\
  &=k(1-x^{n+1}) \\
\end{align*}
so
\begin{align*}
  s_n(1-x)&=k(1-x^{n+1}) \\
  s_n&=k{1-x^{n+1}\over 1-x}. \\
\end{align*}
If $|x|<1$, $\ds\lim_{n\to\infty}x^n=0$ so
\[
  \lim_{n\to\infty}s_n=\lim_{n\to\infty}k{1-x^{n+1}\over 1-x}=
  k{1\over 1-x}.
\]
Thus, when $|x|<1$ the geometric series converges to $k/(1-x)$.
\end{proof}

When, for example, $k=1$ and $x=1/2$:
\[
  s_n={1-(1/2)^{n+1}\over 1-1/2}={2^{n+1}-1\over 2^n}=2-{1\over 2^n}
  \quad\hbox{and}\quad \sum_{n=0}^\infty {1\over 2^n} = 
  {1\over 1-1/2} = 2.
\]
We began the chapter with the series
\[\sum_{n=1}^\infty {1\over 2^n},\]
namely, the geometric series without the first term $1$. Each partial
sum of this series is 1 less than the corresponding partial sum for 
the geometric series, so of course the limit is also one less than the
value of the geometric series, that is,
\[\sum_{n=1}^\infty {1\over 2^n}=1.\]

It is not hard to see that the following theorem follows from
Theorem~\ref{thm:SequenceProperties}. 

\begin{theorem}{Series are Linear}{SeriesLinear}
Suppose that $\ds\sum a_n$ and $\ds\sum b_n$ are convergent series,
and $c$ is a constant. Then

\begin{enumerate}
\item $\ds\sum ca_n$ is convergent and $\ds\sum ca_n=c\sum a_n$
\item $\ds\sum (a_n+b_n)$ is convergent and $\ds\sum (a_n+b_n)=\sum a_n+\sum b_n$.
\end{enumerate}
\end{theorem}

Note that when $c$ is non-zero, the converse of the first part of this theorem is also true. That is, if $\sum ca_n$ is convergent, then $\sum a_n$ is also convergent; if $\sum ca_n$ converges then $\frac{1}{c}\sum ca_n$ must converge.

On the other hand, the converse of the second part of the theorem is not true. For example, if $a_n=1$ and $b_n=-1$, then $\sum a_n+\sum b_n=\sum 0=0$ converges, but each of $\sum a_n$ and $\sum b_n$ diverges.

%The two parts of this theorem are subtly different. Suppose that $\sum
%a_n$ diverges; does $\sum ca_n$ also diverge if $c$ is non-zero? Yes:
%suppose instead that $\sum ca_n$ converges; then by the theorem, $\sum
%(1/c)ca_n$ converges, but this is the same as $\sum a_n$, which by
%assumption diverges. Hence $\sum ca_n$ also diverges. Note that we are
%applying the theorem with $a_n$ replaced by $ca_n$ and $c$ replaced by
%$(1/c)$.
%
%Now suppose that $\sum a_n$ and $\sum b_n$ diverge; does
%$\sum (a_n+b_n)$ also diverge? Now the answer is no: Let $a_n=1$ and
%$b_n=-1$, so certainly $\sum a_n$ and $\sum b_n$ diverge. But
%$\sum (a_n+b_n)=\sum(1+-1)=\sum 0 = 0$. Of course, sometimes 
%$\sum (a_n+b_n)$ will also diverge, for example, if $a_n=b_n=1$, then
%$\sum (a_n+b_n)=\sum(1+1)=\sum 2$ diverges.

In general, the sequence of partial sums $\ds s_n$ is harder to understand
and analyze than the sequence of terms $\ds a_n$, and it is difficult
to determine whether series converge and if so to what. The following result
will let us deal with some simple cases easily.

\begin{theorem}{Divergence Test}{thm:DivergenceTest}
If $\ds\sum a_n$ converges then $\ds\lim_{n\to\infty}a_n=0$.
\end{theorem}
\begin{proof}
Since $\sum a_n$ converges, $\ds\lim_{n\to\infty}s_n=L$ and 
$\ds\lim_{n\to\infty}s_{n-1}=L$, because this really says the same
thing but ``renumbers'' the terms. By Theorem~\ref{thm:SequenceProperties}, 
\[
  \lim_{n\to\infty} (s_{n}-s_{n-1})=
  \lim_{n\to\infty} s_{n}-\lim_{n\to\infty}s_{n-1}=L-L=0.
\]
But
\[
  s_{n}-s_{n-1}=(a_0+a_1+a_2+\cdots+a_n)-(a_0+a_1+a_2+\cdots+a_{n-1})
  =a_n,
\]
so as desired $\ds\lim_{n\to\infty}a_n=0$.
\end{proof}

This theorem presents an easy divergence test: Given a series $\sum
a_n$, if the limit $\ds\lim_{n\to\infty}a_n$ does not exist or has a value
other than zero, the series diverges. Note well that the converse is
\emph{not} true: If $\ds\lim_{n\to\infty}a_n=0$ then the series does
not necessarily converge.

\begin{theorem}{The $n$-th Term Test}{nthTermTestTheorem}
If $\ds\lim_{n\to\infty}a_n\neq 0$ or if the limit does not exist, then $\ds\sum a_n$ diverges.
\end{theorem}
\begin{proof}
Consider the statement of the theorem in contrapositive form:
\[\textbf{If }\ds\sum_{n=1}^{\infty}a_n\text{ converges, then }\lim_{n\to\infty}a_n=0.\]
If $s_n$ are the partial sums of the series, then the assumption that the series converges gives us
\[\ds\lim_{n\to\infty}s_n=s\]
for some number $s$. Then
\[\ds\lim_{n\to\infty}a_n=\lim_{n\to\infty}(s_n-s_{n-1})=\lim_{n\to\infty}s_n-\lim_{n\to\infty}s_{n-1}=s-s=0.\]
\end{proof}

\begin{example}{}{}
Show that $\ds\sum_{n=1}^\infty {n\over n+1}$ diverges.
\end{example}\
\begin{solution}
We compute the limit:
$$\lim _{n\to\infty}{n\over n+1}=1\not=0.$$
Looking at the first few terms perhaps makes it clear that the series
has no chance of converging:
$${1\over2}+{2\over3}+{3\over4}+{4\over5}+\cdots$$
will just get larger and larger; indeed, after a bit longer the series
starts to look very much like $\cdots+1+1+1+1+\cdots$, and of course
if we add up enough 1's we can make the sum as large as we desire.
\end{solution}

\begin{example}{Harmonic Series}{HarmonicSeries}
Show that $\ds\sum_{n=1}^\infty {1\over n}$ diverges.
\end{example}
\begin{solution}
Here the theorem does not apply: $\ds\lim _{n\to\infty} 1/n=0$, so it
looks like perhaps the series converges. Indeed, if you have the
fortitude (or the software) to add up the first 1000 terms you will find that
$$\sum_{n=1}^{1000} {1\over n}\approx 7.49,$$
so it might be reasonable to speculate that the series converges to
something in the neighborhood of 10. But in fact the partial sums do go
to infinity; they just get big very, very slowly. Consider the
following:

\hbox to \hsize{$\ds 1+{1\over 2}+{1\over 3}+{1\over 4} > 
1+{1\over 2}+{1\over 4}+{1\over
  4} = 1+{1\over 2}+{1\over 2}$\hfill}

\hbox to \hsize{$\ds 1+{1\over 2}+{1\over 3}+{1\over 4}+
{1\over 5}+{1\over 6}+{1\over
    7}+{1\over 8} > 
1+{1\over 2}+{1\over 4}+{1\over 4}+{1\over 8}+{1\over 8}+{1\over
    8}+{1\over 8} = 1+{1\over 2}+{1\over 2}+{1\over 2}$\hfill}

\hbox to \hsize{$\ds 1+{1\over 2}+{1\over 3}+\cdots+{1\over16}>
1+{1\over 2}+{1\over 4}+{1\over 4}+{1\over 8}+\cdots+{1\over
  8}+{1\over16}+\cdots +{1\over16} =1+{1\over 2}+{1\over 2}+{1\over
  2}+{1\over 2}$\hfill}
and so on. By swallowing up more and more terms we can always manage
to add at least another $1/2$ to the sum, and by adding enough of
these we can make the partial sums as big as we like. In fact, it's
not hard to see from this pattern that
$$1+{1\over 2}+{1\over 3}+\cdots+{1\over 2^n} > 1+{n\over 2},$$
so to make sure the sum is over 100, for example, we'd add
up terms until we get to around $\ds 1/2^{198}$, that is,
about $\ds 4\cdot 10^{59}$ terms. This series, $\sum (1/n)$, is called the
\dfont{harmonic series}.
\end{solution}

We will often make use of the fact that the first few (e.g. any finite number of) terms in a series are irrelevant when determining whether it will converge. In other words, $\sum_{n=0}^{\infty}a_n$ converges if and only if $\sum_{n=N}^{\infty}a_n$ converges for some $N\geq 1$.

%%%%%%%%%%%%%%%%%%%%%%%%%%%%%%%%%%%%%%%%%%%%
\Opensolutionfile{solutions}[ex]
\section*{Exercises for \ref{sec:series}}

\begin{enumialphparenastyle}

\begin{ex}
Explain why $\ds\sum_{n=1}^\infty {n^2\over 2n^2+1}$
diverges.
\begin{sol}
$\ds\lim_{n\to\infty} n^2/(2n^2+1)=1/2$
\end{sol}
\end{ex}

\begin{ex}
Explain why $\ds\sum_{n=1}^\infty {5\over 2^{1/n}+14}$
diverges.
\begin{sol}
$\ds\lim_{n\to\infty} 5/(2^{1/n}+14)=1/3$
\end{sol}
\end{ex}

\begin{ex}
Explain why $\ds\sum_{n=1}^\infty {3\over n}$
diverges.
\begin{sol}
$\sum_{n=1}^\infty {1\over n}$ diverges, so $\ds\sum_{n=1}^\infty 3{1\over n}$ diverges
\end{sol}
\end{ex}

\begin{ex}
Compute $\ds\sum_{n=0}^\infty {4\over (-3)^n}- {3\over 3^n}$. 
\begin{sol}
$-3/2$
\end{sol}
\end{ex}

\begin{ex}
Compute $\ds\sum_{n=0}^\infty {3\over 2^n}+ {4\over 5^n}$. 
\begin{sol}
$11$
\end{sol}
\end{ex}

\begin{ex}
Compute $\ds\sum_{n=0}^\infty {4^{n+1}\over 5^n}$.
\begin{sol}
$20$
\end{sol}
\end{ex}

\begin{ex}
Compute $\ds\sum_{n=0}^\infty {3^{n+1}\over 7^{n+1}}$.
\begin{sol}
$3/4$
\end{sol}
\end{ex}

\begin{ex}
Compute $\ds\sum_{n=1}^\infty \left({3\over 5}\right)^n$.
\begin{sol}
$3/2$
\end{sol}
\end{ex}

\begin{ex}
Compute $\ds\sum_{n=1}^\infty {3^n\over 5^{n+1}}$.
\begin{sol}
$3/10$
\end{sol}
\end{ex}

\end{enumialphparenastyle}