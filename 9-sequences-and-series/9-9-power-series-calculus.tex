\section{Calculus with Power Series}\label{sec:CalculuswithPowerSeries}

We now know that some functions can be expressed as power series,
which look like infinite polynomials. Since it is easy to find derivatives
and integrals of polynomials, we might hope that we can take derivatives
and integrals of power series in an analogous way. In fact we can, as stated
in the following theorem, which we will not prove here.

\begin{theorem}{}{}
Suppose the power series $f(x)=\ds\sum_{n=0}^\infty a_n(x-a)^n$ has
radius of convergence $R$. Then
\begin{align*}
  f'(x)&=\sum_{n=0}^\infty na_n(x-a)^{n-1},	\\
  \int f(x)\,dx &= C+\sum_{n=0}^\infty {a_n\over n+1}(x-a)^{n+1},
\end{align*}
and these two series have radius of convergence $R$.
\end{theorem}

\begin{example}{}{}
Find a power series representation of $\ln|1-x|$.
\end{example}
\begin{solution}
Starting with the geometric series:
\begin{align*}
  {1\over 1-x} &= \sum_{n=0}^\infty x^n	\\
  \int{1\over 1-x}\,dx &= -\ln|1-x| = \sum_{n=0}^\infty {1\over n+1}x^{n+1}	\\
  \ln|1-x| &= \sum_{n=0}^\infty -{1\over n+1}x^{n+1}
\end{align*}
when $|x|<1$. The series does not converge when $x=1$ but does
converge when $x=-1$ or $1-x=2$. The interval of convergence is
$[-1,1)$, or $0<1-x\le2$.
We can use this series to express $\ln(a)$ as a series
when $0<a\le2$ by setting $x-1=a$. For example
$$
  \ln(3/2)=\ln\left(1-(-1/2)\right)=
  \sum_{n=0}^\infty (-1)^n{1\over n+1}{1\over 2^{n+1}}.
$$
We can use this in turn to approximate $\ln(3/2)$:
$$
  \ln(3/2)\approx {1\over 2}-{1\over 8}+{1\over 24}-{1\over 64}
  +{1\over 160}-{1\over 384}+{1\over 896}
  ={909\over 2240}\approx 0.406
.$$
Because this is an alternating series with decreasing terms,
we know that the true value is between $909/2240$ and
$909/2240-1/2048=29053/71680\approx .4053$, so $0.4053\leq\ln(3/2)\leq 0.406$.
\end{solution}

With a bit of arithmetic, we can approximate values outside of the interval of convergence:

\begin{example}{}{}
Find an approximation for $\ln(9/4)$.
\end{example}
\begin{solution}
We can use the approximation we just computed, plus some rules for logarithms:
$$\ln(9/4)=\ln((3/2)^2)=2\ln(3/2)\approx 0.812,$$
and using our bounds above,
$$0.8106\leq \ln(9/4)\leq 0.812.$$
\end{solution}


%%%%%%%%%%%%%%%%%%%%%%%%%%%%%%%%%%%%%%%%%%%%
\Opensolutionfile{solutions}[ex]
\section*{Exercises for \ref{sec:CalculuswithPowerSeries}}

\begin{enumialphparenastyle}

\begin{ex}
Find a series representation for $\ln 2$.
\begin{sol}
the alternating harmonic series
\end{sol}
\end{ex}

\begin{ex}
Find a power series representation for $\ds 1/(1-x)^2$.
\begin{sol}
$\ds\sum_{n=0}^\infty (n+1)x^n$
\end{sol}
\end{ex}

\begin{ex}
Find a power series representation for $\ds 2/(1-x)^3$.
\begin{sol}
$\ds\sum_{n=0}^\infty (n+1)(n+2)x^n$
\end{sol}
\end{ex}

\begin{ex}
Find a power series representation for $\ds 1/(1-x)^3$.
What is the radius of convergence?
\begin{sol}
$\ds\sum_{n=0}^\infty {(n+1)(n+2)\over 2}x^n$, $R=1$
\end{sol}
\end{ex}

\begin{ex}
Find a power series representation for $\ds\int\ln(1-x)\,dx$.
\begin{sol}
$\ds C+\sum_{n=0}^\infty {-1\over (n+1)(n+2)}x^{n+2}$ 
\end{sol}
\end{ex}

\end{enumialphparenastyle}
\clearpage