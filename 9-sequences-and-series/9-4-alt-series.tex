\section{Alternating Series}\label{sec:AlternatingSeries}

Next we consider series with both positive and negative terms, but in
a regular pattern: they alternate, as in the \dfont{alternating
  harmonic series}:
$$
  \sum_{n=1}^\infty {(-1)^{n-1}\over n}=
  {1\over1}+{-1\over2}+{1\over3}+{-1\over4}+\cdots=
  {1\over1}-{1\over2}+{1\over3}-{1\over4}+\cdots.
$$

In this example the magnitude of the terms decrease, that is, 
$\ds |a_n|$ forms a decreasing sequence, although this is not required in an
alternating series. Recall that for a series with positive terms, if the limit
of the terms is not zero, the series cannot converge; but even if the limit
of the terms is zero, the series still may not converge. It turns out that for
alternating series, the series converges exactly when the limit of the terms is zero.
In Figure~\ref{fig:alternating harmonic series}, we illustrate what happens to
the partial sums of the alternating harmonic series. Because the sizes of
the terms $\ds a_n$ are decreasing, the odd partial sums $\ds s_1$, $\ds s_3$, $\ds s_5$,
and so on, form a decreasing sequence that is bounded below by
$\ds s_2$, so this sequence must converge.
Likewise, the even partial sums $\ds s_2$, $\ds s_4$, $\ds s_6$,
and so on, form an increasing sequence that is bounded above by
$\ds s_1$, so this sequence also converges. Since all the even numbered
partial sums are less than all the odd numbered ones, and since the
``jumps'' (that is, the $\ds a_i$ terms) are getting smaller and smaller,
the two sequences must converge to the same value, meaning the entire
sequence of partial sums $\ds s_1,s_2,s_3,\ldots$ converges as well.

\begin{figure}[H]
\centerline{
%\texonly
\vbox{\beginpicture
\normalgraphs
%\ninepoint
\setcoordinatesystem units <15truecm,1.5truecm>
\setplotarea x from 0.25 to 1.1, y from -1 to 1
\axis bottom shiftedto y=0 ticks withvalues {$1\over4$} / at 0.25 / /
\axis bottom shiftedto y=0 ticks from 1 to 1 by 1 /
\put {$1=s_1=a_1$} [tl] <-2pt,-9pt> at 1 0
\put {$a_2=-{1\over 2}$} [b] <0pt,3pt> at 0.75 1
\put {$s_2={1\over 2}$} [tr] <-3pt,-3pt> at 0.5 0
\put {$a_3$} [t] <0pt,-3pt> at 0.667 -1
\put {$s_3$} [tl] <3pt,-3pt> at 0.833 0
\put {$a_4$} [b] <0pt,3pt> at 0.708 0.667
\put {$s_4$} [tr] <-3pt,-3pt> at 0.583 0
\put {$a_5$} [b] <0pt,2pt> at 0.683 -0.667
\put {$s_5$} [tl] <3pt,-3pt> at 0.783 0
\put {$a_6$} [b] <0pt,3pt> at 0.7 0.333
\put {$s_6$} [tl] <1pt,-3pt> at 0.6167 0
\putrule from 1 0 to 1 1
%\putrule from 1 1 to 0.5 1
\arrow <4pt> [0.35, 1] from 1 1 to 0.5 1
\putrule from 0.5 1 to 0.5 -1
%\putrule from 0.5 -1 to 0.833 -1
\arrow <4pt> [0.35, 1] from 0.5 -1 to 0.833 -1
\putrule from 0.833 -1 to 0.833 0.67
%\putrule from 0.833 0.67 to 0.583 0.67
\arrow <4pt> [0.35, 1] from 0.833 0.67 to 0.583 0.67
\putrule from 0.583 0.67 to 0.583 -0.67
%\putrule from 0.583 -0.67 to 0.783 -0.67
\arrow <4pt> [0.35, 1] from 0.583 -0.67 to 0.783 -0.67
\putrule from  0.783 -0.67 to  0.783 0.33
%\putrule from  0.783 0.33 to 0.6167 0.33
\arrow <4pt> [0.35, 1] from 0.783 0.33 to 0.6167 0.33
\putrule from  0.6167 0.33 to  0.6167 0
\endpicture}}
%\endtexonly
%\figrdef{fig:alternating harmonic series}
%\htmlfigure{Sequences_series-alternating_harmonic.html}
\caption{The alternating harmonic series.}
\label{fig:alternating harmonic series}
\end{figure}

The same argument works for any alternating sequence with terms that decrease in absolute value.
The alternating series test is worth calling a theorem.

\begin{theorem}{Alternating Series Test}{AltSeriesTest}
Suppose that $\ds\{a_n\}_{n=1}^\infty$ is a non-increasing
sequence of positive numbers and $\ds\lim_{n\to\infty}a_n=0$. Then the
alternating series $\ds\sum_{n=1}^\infty (-1)^{n-1} a_n$ converges.
\end{theorem}
\begin{proof}
The odd-numbered partial sums, $\ds s_1, s_3, s_5,\ldots, s_{2k+1},\ldots$,
form a non-increasing sequence, because
$\ds s_{2k+3}=s_{2k+1}-a_{2k+2}+a_{2k+3}\le s_{2k+1}$, since
$\ds a_{2k+2}\ge a_{2k+3}$. This sequence is bounded below by
$\ds s_2$, so it must converge, to some value $L$.
Likewise, the partial sums $\ds s_2, s_4, s_6,\ldots,s_{2k},\ldots$,
form a non-decreasing sequence that is bounded above by
$\ds s_1$, so this sequence also converges, to some value $M$.
Since $\ds\lim_{n\to\infty} a_n=0$ and
$\ds s_{2k+1}= s_{2k}+a_{2k+1}$,
$$
  L=\lim_{k\to\infty}s_{2k+1}=\lim_{k\to\infty}(s_{2k}+a_{2k+1})=
  \lim_{k\to\infty}s_{2k}+\lim_{k\to\infty}a_{2k+1}=M+0=M,
$$
so $L=M$; the two sequences of partial sums converge to the same
limit, and this means the entire sequence of partial sums also
converges to $L$.
\end{proof}

Another useful fact is implicit in this discussion. Suppose that 
$$L=\sum_{n=1}^\infty (-1)^{n-1} a_n$$
and that we approximate $L$ by a finite part of this sum, say
$$L\approx \sum_{n=1}^N (-1)^{n-1} a_n.$$
Because the terms are decreasing in size, we know that the true value
of $L$ must be between this approximation and the next one, that is,
between 
$$
  \sum_{n=1}^N (-1)^{n-1} a_n \quad \hbox{and}\quad
  \sum_{n=1}^{N+1} (-1)^{n-1} a_n.
$$
Depending on whether $N$ is odd or even, the second will be larger or
smaller than the first.

\begin{example}{}{}
Approximate the sum of the alternating harmonic series to within 0.05.
\end{example}
\begin{solution}
We need to go to the point at which the next term to be added
or subtracted is $1/10$. Adding up the first nine and the first ten
terms we get approximately $0.746$ and $0.646$. These are $1/10$
apart, so the value halfway between them, 0.696, is within 0.05 of the correct value.
\end{solution}

We have considered alternating series with first index 1, and in which
the first term is positive, but a little thought shows this is not
crucial. The same test applies to any similar series, such as
$\ds\sum_{n=0}^\infty (-1)^n a_n$, $\ds\sum_{n=1}^\infty (-1)^n a_n$, 
$\ds\sum_{n=17}^\infty (-1)^n a_n$, etc.

%%%%%%%%%%%%%%%%%%%%%%%%%%%%%%%%%%%%%%%%%%%%
\Opensolutionfile{solutions}[ex]
\section*{Exercises for \ref{sec:AlternatingSeries}}

\begin{enumialphparenastyle}

Determine whether the following series converge or diverge.

\begin{multicols}{2}
\begin{ex}
$\ds\sum_{n=1}^\infty {(-1)^{n-1}\over 2n+5}$
\begin{sol}
converges
\end{sol}
\end{ex}

\begin{ex}
$\ds\sum_{n=4}^\infty {(-1)^{n-1}\over \sqrt{n-3}}$
\begin{sol}
converges
\end{sol}
\end{ex}

\begin{ex}
$\ds\sum_{n=1}^\infty (-1)^{n-1}{n\over 3n-2}$
\begin{sol}
diverges
\end{sol}
\end{ex}

\begin{ex}
$\ds\sum_{n=1}^\infty (-1)^{n-1}{\ln n\over n}$
\begin{sol}
converges
\end{sol}
\end{ex}

\end{multicols}

\begin{ex}
Approximate $\ds\sum_{n=1}^\infty (-1)^{n-1}{1\over n^3}$ to within 0.005.
\begin{sol}
$0.90$
\end{sol}
\end{ex}

\begin{ex}
Approximate $\ds\sum_{n=1}^\infty (-1)^{n-1}{1\over n^4}$ to within 0.005. 
\begin{sol}
$0.95$
\end{sol}
\end{ex}

\end{enumialphparenastyle}