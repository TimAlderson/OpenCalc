\section{Absolute Convergence}\label{sec:AbsoluteConvergence}

Roughly speaking there are two ways for a series to converge: As in
the case of $\sum 1/n^2$, the individual terms get small very quickly,
so that the sum of all of them stays finite, or, as in the case of
$\ds \sum (-1)^{n-1}/n$, the terms don't get small fast enough ($\sum 1/n$
diverges), but a mixture of positive and negative terms provides
enough cancellation to keep the sum finite. You might guess from what
we've seen that if the terms get small fast enough that the sum of their
absolute values converges, then the series will still converge regardless
of which terms are actually positive or negative. 

\begin{theorem}{Absolute Convergence}{AbsConvTheorem}
If $\ds\sum_{n=0}^\infty |a_n|$ converges, then 
$\ds\sum_{n=0}^\infty a_n$ converges.
\end{theorem}
\begin{proof}
Note that $\ds 0\le a_n+|a_n|\le 2|a_n|$ so by the comparison test
$\ds\sum_{n=0}^\infty (a_n+|a_n|)$ converges. Now
$$
  \sum_{n=0}^\infty (a_n+|a_n|) -\sum_{n=0}^\infty |a_n|
  = \sum_{n=0}^\infty a_n+|a_n|-|a_n| = \sum_{n=0}^\infty a_n 
$$
converges by Theorem~\ref{thm:SeriesLinear}.
\end{proof}

So given a series $\sum a_n$ with both positive and negative terms,
you should first ask whether $\sum |a_n|$ converges. This may be an
easier question to answer, because we have tests that apply
specifically to terms with non-negative terms. If $\sum |a_n|$
converges then you know that $\sum a_n$ converges as well. If $\sum
|a_n|$ diverges then it still may be true that $\sum a_n$
converges, but you will need to use other techniques to decide.
Intuitively this results says that it is (potentially) easier for
$\sum a_n$ to converge than for $\sum |a_n|$ to converge, because terms
may partially cancel in the first series. 

If $\sum |a_n|$ converges we say that $\sum a_n$ converges \dfont{absolutely};
to say that $\sum a_n$ converges absolutely is to say that the terms of the series
get small (in absolute value) quickly enough to guarantee that the series converges,
regardless of whether any of the terms cancel each other. If $\sum a_n$
converges but $\sum |a_n|$ does not, we say that $\sum a_n$ converges
\dfont{conditionally}. For
example $\ds\sum_{n=1}^\infty (-1)^{n-1} {1\over n^2}$ converges
absolutely, while $\ds\sum_{n=1}^\infty (-1)^{n-1} {1\over n}$
converges conditionally.

\begin{example}{}{}
Does $\ds\sum_{n=2}^\infty {\sin n\over n^2}$ converge?
\end{example}
\begin{solution}
In Example~\ref{exa:AbsSineOverNSquared} we saw that 
$\ds\sum_{n=2}^\infty {|\sin n|\over n^2}$ converges, so the given
series converges absolutely.
\end{solution}

\begin{example}{}{}
Does $\ds\sum_{n=0}^\infty (-1)^{n}{3n+4\over 2n^2+3n+5}$ converge?
\end{example}
\begin{solution}
Taking the absolute value, $\ds\sum_{n=0}^\infty {3n+4\over 2n^2+3n+5}$
diverges by comparison to $\ds\sum_{n=1}^\infty {3\over 10n}$, so if
the series converges it does so conditionally. It is true that
$\ds\lim_{n\to\infty}(3n+4)/(2n^2+3n+5)=0$, so to apply the
alternating series test we need to know whether the terms are
decreasing.
If we let $\ds f(x)=(3x+4)/(2x^2+3x+5)$ then 
$\ds f'(x)=-(6x^2+16x-3)/(2x^2+3x+5)^2$, and it is not hard to see that
this is negative for $x\ge1$, so the series is decreasing and by the
alternating series test it converges.
\end{solution}

%%%%%%%%%%%%%%%%%%%%%%%%%%%%%%%%%%%%%%%%%%%%
\Opensolutionfile{solutions}[ex]
\section*{Exercises for \ref{sec:AbsoluteConvergence}}

\begin{enumialphparenastyle}

Determine whether each series converges absolutely, converges
conditionally, or diverges.

\begin{multicols}{2}

\begin{ex}
$\ds\sum_{n=1}^\infty (-1)^{n-1}{1\over 2n^2+3n+5}$
\begin{sol}
converges absolutely
\end{sol}
\end{ex}

\begin{ex}
$\ds\sum_{n=1}^\infty (-1)^{n-1}{3n^2+4\over 2n^2+3n+5}$
\begin{sol}
diverges
\end{sol}
\end{ex}

\begin{ex}
$\ds\sum_{n=1}^\infty (-1)^{n-1}{\ln n\over n}$
\begin{sol}
converges conditionally
\end{sol}
\end{ex}

\begin{ex}
$\ds\sum_{n=1}^\infty (-1)^{n-1} {\ln n\over n^3}$
\begin{sol}
converges absolutely
\end{sol}
\end{ex}

\begin{ex}
$\ds\sum_{n=2}^\infty (-1)^n{1\over \ln n}$
\begin{sol}
converges conditionally
\end{sol}
\end{ex}

\begin{ex}
$\ds\sum_{n=0}^\infty (-1)^{n} {3^n\over 2^n+5^n}$
\begin{sol}
converges absolutely
\end{sol}
\end{ex}

\begin{ex}
$\ds\sum_{n=0}^\infty (-1)^{n} {3^n\over 2^n+3^n}$
\begin{sol}
diverges
\end{sol}
\end{ex}

\begin{ex}
$\ds\sum_{n=1}^\infty (-1)^{n-1} {\arctan n\over n}$
\begin{sol}
converges conditionally
\end{sol}
\end{ex}

\end{multicols}

\end{enumialphparenastyle}